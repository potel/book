 \chapter*{Preface}
The elementary modes of nuclear excitation are vibrations and rotations, single-particle  motion, and pairing vibrations and rotations. The specific reactions probing these modes are inelastic and Coulomb excitation,  single- and two- particle transfer processes respectively. 

The interweaving of the elementary modes of excitation lead to renormalization of the energy, wavefunction and particle content of the single-particles, as well as of the energy, width and collectivity of vibrations. This implies renormalization of the formfactors and transition densities, $Q$-value and effective deformation parameters both in 3D- and in gauge-space, and state and mass number dependence of the optical potential. As a consequence, the emergence of long range correlations. Also of resonance phenomena as a function of the bombarding energy of the projectile inducing the anelastic and/or transfer processes, implying also the need to go beyond lowest order distorted wave Born approximation (DWBA).

Within this context one can posit that nuclear structure (bound) and reactions (continuum) are but two aspects of the same physics.  Even more so concerning the study of light exotic halo nuclei, in which case the distinction between bound and continuum states is almost completely blurred. This is also the reason why these two aspects of nuclear physics are treated in the present monograph on equal footing,  within the framework of the unified nuclear field theory of structure and reactions (NFT)$_\text{(s+r)}$. 


This theory provides the (graphical) rules to diagonalize in a compact and economic way the nuclear Hamiltonian for both bound and continuum states. It does so in terms of Feynman diagrams which describe the coupling of elementary modes of excitation, correcting for the overcompletness of the basis  (structure) and for the  non-orthogonality of the scattering states (reaction), as well as for Pauli principle violation. The outcome connects directly with observables: absolute reaction cross sections and decay probabilities. 


In other words (NFT)$_\text{(s+r)}$ focuses on the scattering amplitudes which determine the absolute cross sections for the variety of physical processes, involving also those in which bosons and fermions are created or annihilated, connecting such processes to formfactors and transition densities. Processes where one set of particles with given energies, momenta angular momenta, etc. go in and another group (or the same), comes out. That is, as it happens in the laboratory, let alone in nature.

Pairing vibrations and rotations, closely connected with nuclear superfluidity are  paradigms of quantal nuclear phenomena. They thus play an important  role within the field of nuclear structure. It is only natural that two-nucleon transfer plays a similar role concerning the probing of the nucleus.

 
At the basis of fermionic pairing phenomena one finds Cooper pairs, weakly bound, extended, strongly overlapping (quasi-) bosonic entities, made out of pairs of nucleons dressed by collective vibrations and interacting through the exchange of these vibrations as well as through the bare $NN$-interaction, eventually corrected by $3N$ contributions.
Cooper pairs not only change the statistics of the nuclear stuff around the Fermi surface and, condensing, the properties of nuclei close to their ground state. They also display a rather remarkable mechanism of tunnelling between  target and projectile in  direct two-nucleon transfer reaction.


Cooper pair partners being weakly bound ($\ll \epsilon_F$, Fermi energy), are correlated over distances (correlation length) much larger than nuclear dimensions ( $\gg R$, nuclear radius). On the other hand, Cooper pairs are forced to be confined within regions in which normal density is present and thus, within nuclear dimensions. Within this context the mean field acts as a strong external field,  distorting its spatial structure.

 Nonetheless, the correlation length paradigm comes into evidence, for example, when two nuclei are set into weak contact in a direct reaction. In this case,  the partner nucleons of a Cooper pair have a finite probability to be confined each within the mean field of a different nucleus. It is then natural that a Cooper pair can tunnel between target and projectile, equally well correlated, through simultaneous than through successive transfer processes.
 
 
  Although one does not expects supercurrents in nuclei, one can study long-range pairing correlations in terms of individual quantal states and of the tunneling of single Cooper pairs. Such weak coupling Cooper pair transfer reminds  the tunneling mechanism of electronic Cooper pairs across a barrier (e.g. a dioxide layer of dimensions much smaller than the correlation length) separating two superconductors, known as a Josephson junction\footnote{\cite{Josephson:62,Anderson:64b}.}. The main difference is that, as a rule, in the nuclear time dependent junction efimerely established in  direct two-nucleon transfer process, only one or even none of the two weakly interacting nuclei are superfluid. 
  On the other hand in nuclei, paradigmatic example of fermionic quantum  finite many-body system, zero point fluctuations  (ZPF) in general, and those associated with pair addition and pair substraction modes known as pairing vibrations in particular, are much stronger than in condensed matter. Thus, pairing correlations based on even  a single Cooper pair can lead to distinct pairing correlation effects in two-nucleon transfer processes. 
  
  
 Nucleonic Cooper pair tunneling has played and is playing a central role in the probing of these subtle quantal phenomena, both in the case of  light exotic nuclei as well as of medium and heavy nuclei lying along the stability valley. They  have been instrumental in shedding light on the subject of pairing in nuclei at large, and on nuclear superfluidity in particular. Consequently, and as said before, the subject of two-nucleon transfer occupies  a central place in the present monograph. Both concerning the conceptual and the computational aspects of the description of nuclear pairing, as well as regarding the quantitative confrontation of the results  with the experimental findings in terms of absolute differential cross sections.

Concerning exotic nuclei, recent experiments carried out at TRIUMF (Canada) have provided, through the magnifying glass of (NFT)$_\text{(s+r)}$, a microscopic view of what can be considered a unique embodiment of Copper's pair model\footnote{\cite{Cooper:56}.}: a pair of fermions (neutrons) moving in time reversal states on top of a quiescent Fermi surface and interacting through the exchange of a long wavelength vibration (phonon)\footnote{\cite{Frohlich:52,Bardeen:55,Bardeen:57a,Bardeen:57b}.}, leading to a barely bound system. The two neutrons give rise to an isotropic halo. Because the vibration (phonon) results from the sloshing back and forth of the neutron halo against the core nucleons, one is in presence of a realization of Nambu's tumbling\footnote{\cite{Nambu:91}.} or, more precisely, symbiotic mechanism of spontaneously broken symmetry in gauge space.

Regarding the case of medium heavy nuclei lying along the stability valley, recent studies of heavy ion reactions between superfluid nuclei carried out at energies below the Coulomb barrier at the National Laboratory of Legnaro (Italy) have provided a measure of the neutron Cooper pair correlation length. Within this context, in the present monograph interdisciplinarity is used as a tool to attack concrete nuclear problems. But also, making use of the unique laboratory provided by the finite quantum many-body system of which the atomic nucleus is a paradigmatic example\footnote{\cite{Bohr:19} (Overview).}, to shed light on condensed matter results, in terms of analogies involving individual, quantal single-particle states, let alone tunneling of single Cooper pairs.


Because of the central role the interweaving of the variety of elementary modes of nuclear excitation, namely single-particle motion and collective vibrations play in nuclear superfluidity, the study of Cooper pair tunneling in nuclei requires  a consistent description of nuclear structure in terms of dressed quasiparticles and, making use of the resulting renormalized wavefunctions (formfactors),  of one-nucleon transfer processes\footnote{Within this context one recognizes the difficulties of extracting spectroscopic factors from experiment, in terms of single-particle transfer cross sections calculated making use of mean field wavefunctions.}. This is similar to the situation encountered in superconductors, in connection with strongly coupled systems, and experimentally studied through one- and two-electron tunneling experiments\footnote{\cite{Giaver:73}.} 

Thus, in the present monograph the general physical arguments and technical computational details concerning the   calculation of  absolute one-and two nucleon  transfer differential cross sections, making use of state of the art NFT structure input, are discussed in detail. 


As a result of this approach, theoretical and experimental nuclear practitioners, as well as fourth year and PhD students can use the present monograph at profit. To help this use, the basic nuclear structure formalism, in particular that associated with pairing and with collectives modes in nuclei, is economically introduced through general physical arguments. This is also in keeping with the availability in the current literature, of detailed discussions of the corresponding material. Within this context, the monographs \emph{Nuclear Superfluidity} by Brink and Broglia and \emph{Oscillations in Finite Quantum Systems}  by Bertsch and Broglia, published also by Cambridge University Press can be considered companion volumes to the present one. Volume which shares with those a similar aim: to provide a broad physical view of central issues in the study of finite quantal many-body nuclear systems accessible to motivated students and practitioners. However, neither the present one, nor the other two are introductory texts. In particular the present one in which an attempt at unifying structure and reactions as it happens in nature, is made. On the other hand, unifying discrete (mainly structure) and continuum (reactions) configuration spaces, implies that we will be dealing with those structure results which can be tested by means of experiment. A fact which makes the subject of the present monograph a chapter of quantum mechanics, and thus open to a wide range of practitioners\footnote{Within this context let us mention the intimately correlated subjects of Random Phase Approximation (RPA) and Particle Vibration Coupling (PVC) not found in a fourth year curriculum. They are explained and refer to in a number of places throughout the present monograph, starting from a pedestrian level and for both surface (particle-hole) and pairing (particle-particle and hole-hole) vibrations (Sects. \ref{Sect1.2} and \ref{Sect1.3}), and then extended to include further details and facets (see Sects. \ref{Sect2.6}, \ref{Sect1.7}, \ref{appintroD} Fig. (\ref{fig:4.1}), Sect. \ref{Sect2.3.2} (Fig. \ref{fig:4.10}), App. \ref{App1.D} (Fig. \ref{fig1.D.1}), Sect. \ref{C1S1} (Fig. \ref{fig2.1.5}). Furthermore, in the case of RPA of pairing vibrations around the closed shell system $^{208}$Pb, one provides in Sect. \ref{App1E}, detailed documentation of the numerical calculation of the associated wavefunctions ($X$- and $Y$-amplitudes) at the level of an exercise in a fourth year course. A similar situation is encountered in connection with the subject of the Distorted Wave Born Approximation (DWBA), again a subject not found in fourth year curricula. It is treated at the pedestrian level in Apps. \ref{appintroE}, \ref{C6AppE} and \ref{C7AppB} in connection with inelastic, one-particle and two-particle transfer reactions respectively. And once more in full detail, without eschewing complexities, but again in the style of an example-exercise (``Sometimes one has to say difficult things, but one ought to say them as simply as one know-how'' (G. H. Hardy, A mathematician's apology, Cambridge University Press, Cambridge (1969) )) in Sects. \ref{C4S1}, \ref{Sect6.2.1}, \ref{C7SS722} and \ref{csc} for one- and two-particle transfer reactions respectively.}.

Concerning the notation, we have divided each chapter into sections. Each section may, in turn, be broken down into subsections. Equations and Figures are identified by the number of the chapter and that of the section. Thus (5.1.33) labels the thirtythird equation of section 1 of chapter 5. Similarly, Fig. 5.1.2 labels the second figure of section 1 of chapter 5. Concerning the Appendices, they are labelled by the chapter number and by a Latin letter in alphabetical order, e.g. App. 2.A, App. 2.B, etc. Concerning equations and Figures, a sequential number is added. Thus (2.B.2) labels the second equation of Appendix B of chapter 2, while Fig. 2.B.1 labels the first figure of Appendix B of Chapter 2. References are called  in terms of the author's surname and publication year and are found in alphabetic order in the bibliography at the end of each Chapter, as well as in the complete bibliography at the end of the monograph.

A methodological approach used in the present monograph concerns a certain degree of repetition. Similar, but not the same issues are dealt with more than once using different but equatable terminologies. This approach reflects the fact that useful concepts like reaction channels, or correlation length, let alone elementary modes of excitation, are easy to understand but difficult to define\footnote{``Gentagelsen den er virkeligheden, og Tilverelsens Alvor'' (Repetition is reality, and life's seriousness: S. Kierkegaard Gjentagelsen (1843)) {\cyrrm{povtorenie maty obuqenie}} : repetition is learning's mother.}. This is because their validity is not exhausted in a single perspective. But even more important, because their power in helping at connecting\footnote{``The concepts and propositions get ``meaning'' viz. ``content'', only through their connection with sense-experience\dots The degree of certainty with which this connection, viz., intuitive combination, can be undertaken, and nothing else, differentiates empty fantasy from scientific ``truth''\dots A correct proposition borrows its ``truth'' from the truth-content of the system to which it belongs'' (A. Einstein, Autobiographical notes, in Albert Einstein, Ed. P. A. Schlipp, Harper, New York (1951)) p.1, Vol I.} seemingly unrelated results and phenomena is difficult to be fully appreciated the first time around, spontaneous symmetry breaking and associated emergent properties providing an example of this fact.

\newpage
Throughout, a number of footnotes are found. This is in keeping with the fact that footnotes can play a special role within the framework of an elaborated presentation. In particular, they are useful to emphasize relevant issues in an economic way. Being outside the main text, they give the possibility of stating eventual important results, without the need of elaborating on the proof, but referring to the corresponding sources.
 Within this context, and keeping the natural distances, one can mention that in the paper  in which Born\footnote{\cite{Born:26}. Within this context, it is of notice that the extension of Born probabilistic interpretation to the case of many-particle systems is also found in a footnote (\cite{Pauli:27}, footnote on p. 83 of the paper).} introduces the probabilistic interpretation of Schr\"odinger's  wavefunction, the fact that this probability is connected with its modulus squared and not with the wavefunction itself, is only referred to in a footnote.



  Most of the material contained in this monograph have been the subject of lectures of the four year course ``Nuclear Structure Theory'' which RAB delivered throughout the years at the Department of Physics of the University of Milan, as well as at the Niels Bohr Institute and at Stony Brook (State University of New York). It was also presented by the authors in the course Nuclear Reactions held at the PhD School of Physics of the University of Milan.

GP wants to thank the tutoring of  Ben Bayman concerning specific aspects of two-particle transfer reactions. Discussions with Ian Thompson and Filomena Nunes on a variety of reaction subjects are gratefully acknowledged. 
RAB  acknowledges the essential role the collaboration with Francisco Barranco and Enrico Vigezzi has played concerning  nuclear structure aspects of the present monograph. Its debt with the late Aage Winther regarding the reaction aspects of it is difficult to express in words. The overall contributions of Daniel B\`{e}s, Ben Bayman and Pier Francesco Bortignon\footnote{Deceased August 27, 2018.} are only too explicitly evident throughout the text and constitute a daily source of inspiration.  G. P. and R. A. B. have received important suggestions and comments regarding concrete points and the overall presentation of the material discussed below from Ben Bayman, Pier Francesco Bortignon, David Brink, Willem Dickhoff and Vladimir Zelevinsky and are here gratefully acknowledged. We are specially beholden to Elena Litvinova and Horst Lenske for much constructive criticism and suggestions.
\begin{flushleft}
Gregory Potel Aguilar\\
 East Lansing
\end{flushleft}
\vspace{-1.7cm}
\begin{flushright}
Ricardo A. Broglia\\
 Copenhagen
\end{flushright}

%\renewcommand{\bibname}{Bibliography}
%\bibliographystyle{abbrvnat}
%\bibliography{../nuclear_bib.bib}
