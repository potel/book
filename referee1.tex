\documentclass[a4paper,12pt]{book}
\usepackage{latexsym}
\usepackage{amssymb}
\usepackage{amsmath}
%\usepackage [latin1]{inputenc}
\usepackage{verbatim}
\usepackage{array}
\usepackage{color}
\pagestyle{plain}
\usepackage{graphicx}

\begin{document}

\begin{flushleft}
Answer to referee 1
\end{flushleft}
We have followed all of the referee recommendations and implemented all of his/her suggestions. In the process we have modified the manuscript in a major way. We believe that  the paper has become more focused, the message conveyed, clearer. We thank the referee for having helped us at assessing some basic issues which where only barely touch upon in the previous version. In what follows we answer to the different points raised by the referee in the same order he/she did it. 
\begin{enumerate}
\item We have modified the abstract and the conclusions and added a long paragraph on p. 5 starting right after the citation to p. 486 of ref (Bohr, A. and Mottelson, 1975).
\item We now dedicate many lines to analize and discuss the experimental results of Cappuzzello et al. We also make clear that we consider that the evidence presented needs further confirmation, and suggest a number of experiments which can help at obtaining it. We now use many lines discussing the issue why is that light closed shell nuclei are, arguably, better than heavy systems like e.g. $^{208}$Pb, regarding the observation of GPV.
\item The answer is quite simple. We were not aware of it. We not only quote it (after having studied it), but use some of the main results of the paper concerning why $^{14}$C yes and $^{210}$Pb no. In fact, the PRC of Laskin et al. has allowed us quite naturally to relate GPV with other types of GR (e.g. the $(ph)$ GQR) and associated damping processes. In the process, the advantage of light nuclei is strengthened. The possibility of a novel mechanism is also mentioned.
\item We have taken care of all the concrete linguistic issues and of the more general ones to the best of our abilities.
\end{enumerate}
\end{document}













