\chapter{Collective pairing modes}\label{chapter1}
\epigraph{I wrote down the trial ground state as a product of operators --one for each pair state-- acting on the vacuum\dots \mbox{$\Psi_0=\prod_{\mathbf k}(U_{\mathbf k}+V_{\mathbf k}b^\dagger_{\mathbf k})\ket{0}$}\dots Since the pair creation operators $b^\dagger_{\mathbf k}$ commute for different $\mathbf k$'s\dots $\Psi_0$ represents uncorrelated occupancy of the various pair states\dots. The operators $b^\dagger_{\mathbf k}=c^\dagger_{\mathbf k\uparrow}c^\dagger_{\mathbf k\downarrow}$, being a product of two fermions (quasiparticle) creation operators, do not satisfy Bose statistics, since $b^{\dagger 2}_{\mathbf k}=0$.}{J. R. Schrieffer}
\section{Structure and reactions}\label{C1S1}
The low-energy properties of the finite, quantal, many-body nuclear system, in which nucleons interact through the strong force of strength $v_0(\approx -100$ MeV) and range $a(\approx 0.9$ fm) are controlled, in first approximation, by independent particle motion. This is a consequence of the fact that nucleons display a sizable value of the quantal zero point (kinetic) energy of localization ($\hbar^2/ma^2\approx 50$ MeV) as compared to the absolute value of the strength of the $NN$-potential\footnote{\label{f1C3} The corresponding ratio $q=\left(\frac{\hbar^2}{ma^2}\right)\frac{1}{|v_0|}$ \idx{Quantality parameter}
	 is known as the quantality parameter and was first used in connection with the study of condensed matter (\cite{deBoer:48,deBoer:57,deBoer:48b,Nosanow:76}). In connection with  nuclear physics it is discussed in \cite{Mottelson:02} who notes that its value $q=0.5$ testifies to the validity of independent particle motion. Questions like the one posed in connection with localization and long mean free path were already discussed by \cite{Lindemann:10} in connection with the study of the stability or less of crystals. The generalization to aperiodic crystals, like e.g. proteins (\cite{Schrodinger:44}) was carried out in \cite{Stillinger:90}. Its possible application to the atomic nucleus is discussed in App. \ref{C2AppC} } $|v_0|=100$ MeV. 
The corresponding ground state $\ket{HF}=\Pi_ia^\dagger_i\ket{0}$ describes a step function in the probability of the occupied ($\epsilon_i\leq \epsilon_F$) and empty ($\epsilon_k>\epsilon_F$) states, displaying a sharp discontinuity at the Fermi surface. Pushing the system it reacts with an inertia $Am$, sum of the nucleon masses (App. \ref{App1.D}). Setting it into rotation, assuming the density $\rho(r)=\sum_i|\braket{\mathbf r|i}|^2$ ($\ket{i}=a^\dagger_i\ket{0}$) to be spatially deformed, it responds with the rigid moment of inertia. This is because the single-particle orbitals are solidly anchored to the mean field (Fig. \ref{fig1A3}).





Pairing acting on nucleons moving in time reversal states $\nu,\bar\nu$ ($\nu\equiv(nlj)$), in configurations of the type ($(l)^2_{L=0},(s)^2_{S=0}$), and lying close to the Fermi energy, alter the above picture in a conspicuous way\footnote{\cite{Bohr:58}; see also \cite{Broglia:13} and references therein.}. Within an energy range of the order of the absolute value of the pair  \idx{Cooper pair!binding energy}
 correlation energy\footnote{In BCS, $E_{corr}\approx-\frac{N(0)}{2}\Delta^2$, where $N(0)=\frac{g}{2}$ is the density of nuclear states at the Fermi energy and for one spin orientation, $g_i=i/16$ MeV$^{-1}$ ($i=N,Z$) being the result of an empirical estimate which takes surface effects into account (\cite{Bohr:75} Eq. (2-47); \cite{Bortignon:98} Eq. (7.15)), while $\Delta$ is the pairing gap, $2\Delta$ being the binding energy of the Cooper pair.} $|E_{corr}|(\approx 3 $MeV) centered around $\epsilon_F$ ($|E_{corr}|/\epsilon_F\ll1$), the role of independent particles is taken over by  pairs of nucleons, correlated over distances $\xi\approx\hbar v_F/(\pi\Delta)\,(\approx 14$ fm; $\Delta=1.2$ MeV), and being equally phased (see Sect. \ref{App1D}). In other words, and bringing the argument to a previous step, one can posit that since the creation operators $a^\dagger_i$ commute for different $i$'s, $\ket{HF}$ represents uncorrelated occupancy of the different single-particle states. Similarly, because the pair operators $P^\dagger_\nu$ commute for different $\nu$'s, $\ket{BCS}$ represents uncorrelated occupancy of the various pair states. The independent-particle (normal density) and, phase correlated, independent Cooper\footnote{\cite{Cooper:56}.} pair (abnormal density), parallel  suggested above, is summarized in Fig. \ref{fig1D2}.




For intrinsic\footnote{As opposed to collective excitations which do not alter the temperature of the system, as all the energy is concentrated in a single mode.} nuclear excitation energies and rotational frequencies\footnote{Coriolis force acts oppositely on each member of a Cooper pair. When the difference in rotational energy between superfluid and normal rotation becomes about equal to the correlation energy, the nucleon moving opposite to the collective rotation becomes so much retarded in its revolution period with respect to the partner nucleon, that eventually it cannot correlate any more with it and ``align'', its motion (and spin) with the rotational motion, becoming again a pair of fermions and not participating in the BCS  condensate. This happens for a (critical) angular momentum $I_c\approx(120\times|E_{corr}|)^{1/2}\approx 20\hbar$, corresponding to a rotational frequency $\hbar\omega_c\approx 0.5$ MeV (see \cite{Bohr:75}, \cite{Broglia:13} and references therein).} sensibly smaller than $|E_{corr}/2|$ and than $\hbar\omega_{rot}\approx0.5$ MeV respectively, as well as distances of closest approach \idx{Cooper pair!quasi-bosonic character (conditions)}
 ($D_0$) in Cooper pair transfer processes $D_0\lesssim\xi$,  \idx{Cooper pair!--stability (bosonic) requirements}
 the system can be described in terms of gauge phase correlated,  independent pair motion, each pair contributing a phase $\phi$ (see Eq. (\ref{eq3.4.2})) This is a consequence of the fact that the kinetic energy of (Cooper) pair confinement ($\hbar^2/(2m\xi^2)\approx 10^{-2}$ MeV), is much smaller than the  value of the pair binding energy $|E_{corr}|$, \idx{BCS!pairing condensation (correlation energy)} leading essentially to binding by \idx{Halo (single Cooper pair) nucleus $^{11}$Li!binding through deconfinement}
   \idx{Cooper pair!deconfinement (and binding energy)} deconfinement\footnote{See also footnote \ref{f40C4} Ch. \ref{chapter2}.}, and implying that each pair of entangled nucleon partners \idx{Josephson effect!entanglement between Cooper pair partners}
    behaves as an entity\footnote{\idx{Generalized quantality parameter}
    	 The ratio $q_\xi=\frac{\hbar^2}{2m\xi^2}\frac{1}{|E_{corr}|}\approx 0.02$ provides a generalized quantality parameter. It testifies to the stability of nuclear Cooper pairs in superfluid nuclei. \idx{Cooper pair!generalized quantality parameter}} of mass $2m$ and intrinsic structure $L=S=0$. That is, under these circumstances Cooper pairs behave like (quasi) \idx{Cooper pair!quasi-bosonic character (conditions)}
  bosons\footnote{See footnote \ref{f45C3} Ch. \ref{chapter1}.}, the single-particle orbits on which they correlate become dynamically detached from the mean field, leading to a BCS condensate. It is however very different from a standard condensate of real bosons. Within this context see  Sect. \ref{App1D} (Figs. \ref{fig1A4} and \ref{fig1A5}) and App. \ref{App3E}.

\begin{figure}
\centerline{\includegraphics*[width=\textwidth,angle=0]{nutshell/figs/Excited0Pb206tp.pdf}}
\caption[$^{206}$Pb$(t,p)^{208}$Pb.]{(a) Ratio of the absolute $L=0$ differential cross sections $d\sigma(E_x,\theta=59^{\circ})/d\sigma(gs,\theta=59^{\circ})$ (=(0.05 mb/sr)/(0.12 mb/sr)) below 5 MeV  for the reaction $^{206}$Pb$(t,p)^{208}$Pb at the second maximum ($\theta=59^{\circ}$; \cite{Bjerregaard:66b}). It is of notice the large experimental errors of the corresponding angular distributions associated with the poor statistics of the cross section at the first maximum $\theta=5^{\circ}$. This is the reason why the maximum at $59^\circ$ was preferred to report the ratio of the cross sections. (b) Schematic representation of the pairing vibrational spectrum around $^{208}$Pb (see also Fig. \ref{fig0.3.2}). Also shown is a cartoon representation of the softening of the sharp mean field Fermi surface due to the ZPF of the pairing vibrational modes. The label $a$ and $r$ indicate the pair addition and pair removal modes. It is to be noted that a linear term in $N$ has been added to the binding energy to make the  values associated with $^{206}$Pb ($N=124$) and $^{210}$Pb ($N=128$), equal, in an attempt to emphasize a harmonic picture for the two-phonon state. Concerning the anharmonicities of the modes see last paragraph Sect. \ref{App1E}. After \cite{Potel:13}.}\label{fig1.1}
\end{figure}



 Cooper pairs exist also in situations in which the environmental conditions are above critical. For example, in metals at room temperature, in closed shell nuclei as well as in deformed open shell ones at high values of the angular momentum\footnote{See e.g. \cite{Shimizu:90} and references therein.}. However, in such circumstances, they break essentially as soon as they are generated (pairing vibrations). While these pair addition and substraction fluctuations have little effect in condensed matter systems with the exception of at\footnote{See \cite{Schmidt:68}, \cite{Schmid:69} \cite{Abrahams:68}; concerning superfluid $^3$He see \cite{Wolfe:78}. We make reference to systems of dimensions much larger than the correlation length. The situation being different for zero-dimensional superconducting particles of dimensions small compared to $\xi$ (see e.g. \cite{Tinkham:96}, Ch. 7, in particular Sect. 7-2), connotation which can also be ascribed to the nucleus.} $T\approx T_c$ (critical temperature), they play an important role in normal (non-superfluid) nuclei. In particular around  closed shells\footnote{See Sects. \ref{App1E} and \ref{App1AF}; \cite{Bohr:64}, \cite{Bes:66}, \cite{Hogassen:61}, \cite{Schmidt:72}, \cite{Schmidt:68}, \cite{Barranco:01}, \cite{Potel:13}, \cite{Potel:14}, \cite{Schmidt:64}.} (Fig. \ref{fig1.1}), and specially in the case of light, highly polarizable, exotic halo nuclei\footnote{See \cite{Potel:13} and refs. therein. Also \cite{Potel:13b} in connection with the closed shell system $^{132}$Sn.}. From this vantage point one can posit that it is not so much, or, at least not only, the superfluid phase which is abnormal in the nuclear case, but the normal state around the closed shell systems, displaying a consistent component of dynamical abnormal density. The importance of pairing vibrations is also found in connection with the self-energy of nucleons moving around closed shells\footnote{See e.g. \cite{Bes:71,Flynn:71}.} as well as concerning the binding energy of the closed shell systems\footnote{\cite{Baroni:04}.}.
 
 
  Furthermore, it is of notice the role pairing vibrations play in the  transition between superfluid and normal nuclear phases (cf. Fig. \ref{fig1.2}) as a function of the rotational frequency (angular momentum) as emerged from the experimental studies of high spin states carried out by, among others, Garrett and collaborators\footnote{\cite{Garrett:85,Garrett:86}, see also \cite{Shimizu:89}, \cite{Barranco:87b} and Ch. 6 of \cite{Brink:05}.}.
 While the (dynamic) pairing gap associated with pairing vibrations leads to  $\approx$ 20\% increase of the static pairing gap for low rotational frequencies, it becomes the overwhelming contribution above the critical frequency\footnote{\cite{Shimizu:89}, \cite{Shimizu:90}, \cite{Shimizu:13},  \cite{Donau:99}, \cite{Shimizu:00}.}.
 
 
 
  \textit{The central role played by pairing vibrations \idx{Pairing vibrations!transition between superfluid and normal phases} within the present circumstances is that to restore particle-number conservation, another example after that provided by the quantality parameter and by its generalization to pair motion, of the fact that potential functionals are, as a rule, best profited by special arrangements of particles (spontaneous symmetry breaking), while fluctuations \idx{Fluctuations (symmetry restoration)} favor symmetry}\footnote{\cite{Anderson:84,Anderson:76}.}. Within this context see Eqs. (\ref{eq0.1.91}) and (\ref{eq0.1.95}).
  
  

  
  
  There are a number of methods which allows one to go beyond BCS mean-field approximation, or of its generalization known as the Hartree-Fock-Bogolyubov approximation (HFB). Generally referred to as number projection methods\footnote{See \cite{Allaart:74,Ring:80}, \cite{Egido:13}, \cite{Vaquero:13}, \cite{Robledo:13}; see also \cite{Frauendorf:13}, \cite{Ring:13}, \cite{Heenen:13}, and references therein.}(NP), they make use of a variety of techniques like Generator Coordinate Method, etc. as well as protocols like variation after projection, etc. The advantages of NP methods over the RPA is to lead to smooth functions for both the correlation energy and the pairing gap at the pairing phase transition between normal and superfluid phases. 
  

  In Fig. \ref{fig1.3} we display the excitation function associated with the reaction $^{122}$Sn$(p,t)^{120}$Sn($J^\pi$), populating the low-energy states of the single open shell  superfluid nucleus $^{120}$Sn. The angle selected to report the value of the absolute differential cross sections, that is $5^\circ$, corresponds to the first, and largest, peak of the absolute $L=0$ differential two-nucleon transfer cross section. Essentially all the strength in the $L=0$ channel is concentrated in the ground state, the strongest $0^+$--excited state carrying a cross section of the order of 3\% of that of the ground state. Within this context, the difference with the results displayed in Fig. \ref{fig1.1} is apparent\footnote{\label{f19} While this ``distortion'' of the $(t,p)$ excitation function is useful to emphasize the parallels between vibrational and rotational bands in 3D-and in gauge-spaces, it has to be used with care concerning the parallel with Cooper pair tunneling between weakly coupled superconductors (Sect. \ref{C3AppC} and in particular, Sect. \ref{S7.3}). This is also in keeping with the fact that at the basis of nuclear BCS one finds an  interdisciplinary connection (see \cite{Bohr:58}).}.
  
  In the inset to Fig. \ref{fig1.3} a quantity closely related to the Sn--isotopes binding energy is displayed (bold face levels). Namely $B$($^{50+N}$Sn$_N$)-8.124$N$ MeV+46.33 MeV, resulting from the substraction of the contribution of the single nucleon addition to the nuclear binding energy (linear function in $N$)  obtained by a linear fitting of the binding energies of all the Sn-isotopes. 
  Also displayed is the parabolic fit to these energies, a quantity to be compared with $E_N=(\hbar^2/2\mathcal I)(N-N_0)^2$, namely the energy associated with the members of the pairing rotational band. The difference with the spectrum of pair addition and substraction modes displayed in Fig. \ref{fig1.1} b) (see also Fig. \ref{fig0.3.2}) is again evident.  Concerning the parallel one can draw between the 3D- and pairing-vibrations and -rotations, see Fig. \ref{fig1D1}.
  
  \begin{figure}[h!]
  \centerline{\includegraphics*[width=8cm,angle=0]{nutshell/figs/fig1_1_2x}}
  \caption[Pairing gap as a function of $\hbar\omega_{rot}$.]{\idx{Pairing vibrations!transition between superfluid and normal phases} Pairing gap calculated taking into account the correlation associated with pair vibrations in the RPA approximation $(\Delta=(\Delta^2_{BCS}+\tfrac{1}{2}G^2S_0(RPA))^{1/2})$ (upper panel) and RPA correlation energy (lower panel) for neutrons in $^{164}$Er as a function of the rotational frequency (\cite{Shimizu:89,Shimizu:13}). Both quantities are in MeV (\textit{dashed-dotted curves}). The value of the static (mean-field) pairing gap $\Delta$, which vanishes at $\hbar \omega_{rot}=0.34 $ MeV, is also displayed in the upper panel (continuous curve). The results of the number-projection (NP) calculations are shown as dotted curves.   The non-energy weighted sum rule 
$S_0$ (RPA)= $\sum_{n \neq AGN} \left[\langle n|P |0\rangle  + \langle n|P^{\dagger} |0\rangle \right]^2_{RPA}$ , $P^\dagger=\sum_{\nu> 0}a^\dagger_\nu a^\dagger_{\bar \nu}$.    describes the contribution of pairing fluctuations to the effective (RPA) gap,
    and is intimately associated with projection in particle number. It is of notice
    that $\sum_{n \neq AGN}$ means that the divergent contribution from the zero energy mode
    (Anderson, Goldstone, Nambu mode, see e.g. \cite{Broglia:00} and references therein), associated with the lowest ($\hbar \omega_0$) solution
    of the $H=H_{p}+H''_p$ is to be excluded (see Sect. \ref{S1.4.3},  as well as \cite{Brink:05} App. J).}\label{fig1.2}
  \end{figure}
  \FloatBarrier
  \subsection[Pairing rotational and vibrational bands with transfer]{Pairing rotational and vibrational bands probed with one- and two-nucleon transfer}\label{S3.1.1}
  \idx{Pairing rotational bands}
  A simple estimate of the pairing rotational band moment of inertia is provided by the single $j$-shell model\footnote{See e.g. \cite{Brink:05} App. H and refs. therein.}, namely ($\hbar^2/2\mathcal I$)=$G/4\approx25/(4N_0)$ MeV. This estimate turns out to be rather accurate (Figs. \ref{fig1.3} and \ref{fig1.4}). On the other hand, one is reminded of the fact that we are discussing properties which specifically characterize a coherent state\footnote{See  Sect. \ref{C8S2}; see also \cite{Potel:17}.}, namely $\ket{BCS}$ (see Eq. (\ref{eq0.1.74})).
  
  

  Also reported in  Fig. \ref{fig1.3} (inset), are the integrated values of the measured absolute two-neutron transfer cross sections, \idx{Cooper pair!--absolute transfer cross sections ($\sigma_{2n}$)}
   quantities which are reproduced by the theoretical predictions within experimental errors (Fig. \ref{fig1.5} and Fig. \ref{fig8_2_4}). In principle, one could have expected a sensible constancy of these cross sections (transitions) as the pairing rotational model implies a common intrinsic deformed state in gauge space, namely $\ket{BCS (N_0=82)}$ (see Eq. (\ref{eq0.1.101})). On the other hand, the number of Cooper pairs $\alpha'_0$ \idx{Cooper pair!number of}
   which defines deformation in gauge space is rather small ($\approx6$) and thus subject to conspicuous fractional fluctuations  ($\Delta\alpha_0'/\alpha'_0\approx\sqrt{6}/6\approx 0.4$).  Because $\sigma\sim\alpha'^{2}_0$, fluctuations in $\sigma$ of the order of 100\%, can be expected.  
  
  In keeping with the analogies discussed in connection with Figs. \ref{fig1D1} and \ref{fig1A3}  between pairing and quadrupole rotational bands, we note that in the electromagnetic decay of these last bands one expects, in the case of heavy nuclei, fractional fluctuations of the order of $\sqrt{200}/200$, in keeping with the magnitude of the associated $B(E2)$ values, when measured in terms of single-particle units. Within this context, the average value of the absolute experimental cross sections displayed in the inset of Fig. \ref{fig1.3} is 1762 $\mu$b, while the average difference between experimental and predicted values\footnote{\cite{Potel:13b}.} is 94 $\mu$b. Thus, the discrepancies between theory and experiment are bound in the interval $0\leq|1-\sigma_{th}(i\to f)/\sigma_{exp}(i\to f)|\leq 0.09$, the average discrepancy being 5\%.
  
  
  In Fig. \ref{fig1.4} the excited pairing rotational bands based on $0^+$ pairing vibrational modes are displayed as a single band, resulting from the average  value of the $0^+$ excited states with energy $\leq$3 MeV. The best parabolic fit is shown. Also given are the relative $(p,t)$ absolute integrated cross sections normalized to the corresponding values of the ground state rotational band. The cross talk between bands is in all cases $\leq$ 8\%, the single $j$-shell value estimate being\footnote{\cite{Brink:05} App. H.} 6\%.
  
The above results underscore the fact that, at the basis of an operative coarse grained approximation to the nuclear many-body problem, one finds a judicious choice of the collective coordinates\footnote{In this connection, we quote from S. Weinberg: ``You can use any degrees of freedom that you like to describe a physical system, but if you choose the wrong ones, you will be sorry'' \citep{Weinberg:83}.} which in turn is closely connected with the probe used to study the system. In other words, pairing vibrations are elementary modes of excitation containing the right physics to restore gauge invariance through their interweaving with  quasiparticle states (Eqs. (\ref{eq0.1.95})--(\ref{eq0.1.98})). They project from the deformed state ($\alpha_0'\neq0$), the members (Eq. (\ref{eq0.1.101})) of pairing rotational bands which, similarly to pairing vibrational bands, are specifically excited in two-nucleon transfer reactions. Examples are provided by Cooper pair tunneling in heavy ion collisions between superfluid nuclei, as well as ($t,p$), ($p,t$), etc reactions\footnote{See e.g. \cite{Yoshida:62}, \cite{Glendenning:65}, \cite{Bohr:64}, \cite{Bayman:71},  \cite{Broglia:73},  \cite{Hansen:12} and \cite{Potel:13} and references therein.}. From other fields of physics, the Josephson\footnote{\cite{Josephson:62}.} effect between weakly coupled metallic superconductors \idx{Josephson effect!parallel with tunneling among members of pairing rotational bands}
 is rightly considered the paradigmatic example.  
  \begin{figure}
  \centerline{\includegraphics*[width=\textwidth,angle=0]{nutshell/figs/ExcitedSn122pt.pdf}}
  \caption[$^{122}$Sn$(p,t)^{120}$Sn$(J^\pi)$.]{\idx{Pairing rotational bands} Excitation function associated with the reaction$^{122}$Sn$(p,t)^{120}$Sn$(J^\pi)$. The absolute experimental values of $d\sigma(J^\pi)/d\Omega|_{5^\circ}$ are given as a function of the excitation energy $E_x$ (after \cite{Guazzoni:11}). In the inset the  neutron pairing rotational band between magic numbers $N=50$ and $N=82$ is  displayed (see also caption to Fig. \ref{fig0.4.5}), the absolute $^{A+2}_{\;\;\;50}$Sn$_{N+2}$ ($p,t$) $^{A}_{50}$Sn$_{N}$(gs) experimental cross sections taken from \cite{Guazzoni:99}, \cite{Guazzoni:04}, \cite{Guazzoni:06}, \cite{Guazzoni:08}, \cite{Guazzoni:11}, \cite{Guazzoni:12}, are also given in $\mu$b (number connected by lines to the corresponding levels of the pairing rotational band). After \cite{Potel:13}.}\label{fig1.3}
  \end{figure}
  \begin{figure}
  \centerline{\includegraphics*[width=\textwidth,angle=0]{nutshell/figs/fig2_1_4.pdf}}
  \caption[Sn pairing rotational band.]{\idx{Pairing rotational bands} The weighted average energies ($E_{exc}=\sum_i E_i \sigma_i/\sum_i \sigma_i$) of the excited $0^+$ states below 3 MeV in the Sn isotopic chain are shown on top of the ground state pairing rotational band. Also indicated is the percentage of cross section for two-neutron transfer to excited states, normalized to the cross sections populating the ground states. The estimate of the ratio of cross sections displayed on top of the figure was obtained making use of the single $j$-shell model (\cite{Brink:05} App. H). After \cite{Potel:13b}.}\label{fig1.4}
  \end{figure}
\begin{figure}
	\centerline{\includegraphics*[width=0.9\textwidth,angle=0]{nutshell/figs/fig1D1_v2.pdf}}
	\caption[Parallel between dynamic and static deformations in 3D- and in gauge-space.]{Parallel between dynamic and static deformations in 3D- and in gauge-space for the nuclear finite many-body system. In the first case, the angular momentum $\mathbf{I}$ and the Euler angles are conjugate variables. In the second, particle number $N$ and gauge angle. While the fingerprint of static (quadrupole and gauge) deformations are quadrupole and pairing rotational bands, vibrational bands are the expression of such phenomena in non deformed systems.}\label{fig1D1}
\end{figure}

Within this context one can now take the basic consequence of pairing condensation in nuclei regarding reaction mechanisms. For this purpose let us consider a \textit{gedankenexperiment} in which the superfluid target and the projectile can at best come in  weak contact.  Because $\left(\hbar^2/2m\xi^2\right)/|E_{corr}|\approx10^{-2}$, Cooper pairs in superfluid nuclei behave as (potentially) very extended entities of mean square radius $(\xi\gg R_0)$ and mass $2m$,  even in the case in which the $^1S_0$, $NN$-potential vanishes in the zone between the weakly overlapping (normal) densities of the two interacting nuclei. One then expects Cooper pair transfer to be observed. Not only. One also expects  the associated absolute differential cross section to be of the same order of magnitude than one-nucleon transfer ones, \idx{Josephson effect!transfer (tunneling) probability} \idx{Josephson effect!parallel with tunneling among members of pairing rotational bands}
 and to be dominated by successive transfer. These expectations have been confirmed experimentally\footnote{\label{f28} Both in heavy ion reaction between superfluid nuclei (\cite{Montanari:14}, see also Fig. \ref{fig4.6.1} and Sect. \ref{S7.3}) and light ions ($d,p$) and $(t,p)$ reaction. See e.g. \cite{Cavallaro:17} $d\sigma(^9\text{Li} (d,p)^{10}\text{Li} (1/2^-))/d\Omega|_{\theta_{max}}\approx0.8$ mb/sr, as compared to \cite{Tanihata:08}  $d\sigma(^{11}\text{Li} (p,t)^{9}\text{Li} (1/2^-))/d\Omega|_{\theta_{max}}\approx1 $ mb/sr, \cite{Fortune:94} $^{10}$Be$(t,p)^{12}$Be(gs) $(\sigma=1.9\pm0.5$ mb, $4.4^\circ\leq\theta_{cm}\leq54.4^\circ)$ as compared to \cite{Schmitt:13} 
 	$^{10}$Be$(d,p)^{11}$Be($1/2^+$) $(\sigma=2.4\pm0.013$ mb, $5^\circ\leq\theta_{cm}\leq39^\circ)$ in the case of light nuclei around closed $(N=6)$ shell, and \cite{Bassani:65} $^{120}$Sn$(p,t)^{118}$Sn(gs) $(\sigma=3.024\pm0.907$ mb, $5^\circ\leq\theta_{cm}\leq40^\circ)$ as compared to \cite{Bechara:75} $^{120}$Sn$(d,p)^{121}$Sn($7/2^+$) $(\sigma=5.2\pm0.6$ mb, $2^\circ\leq\theta_{cm}\leq58^\circ)$. \idx{Cooper pair!comparison between $\sigma_{2n}$ with $\sigma_{1n}$}
 }.
 The above parlance, being at the basis of the Josephson effect, reflects both one of the most solidly established results in the study of BCS pairing, and explains the workings of a paradigmatic probe of spontaneous symmetry breaking phenomena.
 
 
 
Due to the fact that, away from the Fermi energy ($\gtrsim\epsilon_F\pm\Delta$) pair  motion becomes essentially independent particle motion, one-particle transfer reactions like e.g. ($d,p$) and ($p,d$) can be used together with ($t,p$) and ($p,t$) processes, as  valid tools to cross check pair correlation predictions. In particular, to shed light on the origin of pairing in nuclei. In particular, concerning the relative importance of the bare- and the induced-pairing interaction.  For this purpose, one should be able to reproduce the absolute differential cross sections within a 10\% error. 

While the calculation of two-nucleon transfer spectroscopic amplitudes and differential cross sections are, a priori, more involved to be worked out than those associated with one-nucleon transfer reactions, the former are, as a rule, more ``intrinsically'' accurate than the latter ones. This is because, in the case of two-nucleon transfer reactions, the quantity (order parameter $\alpha'_0$) which expresses the collectivity of the members of a pairing rotational band, reflects the properties of a coherent state ($|BCS\rangle$). In other words, it results from the sum over many two-nucleon transfer spectroscopic amplitudes ($\sqrt{j_{\nu}+1/2}\,U'_\nu V'_\nu$), all phased in the same way (phase correlated).

There is a further reason which confers $\alpha'_0=\sum_j(j+1/2)U'_jV'_j$ a privileged position with respect to the single-particle spectroscopic amplitudes  $(U'_j, V'_j)$. It is the fact that $\alpha'_0=e^{2i\phi}\sum_j(j+1/2)U_jV_j=e^{2i\phi}\alpha_0$ defines a privileged orientation in gauge space, $\alpha_0$ being the order parameter referred to the laboratory system which makes an angle $\phi$ in gauge space with respect to the intrinsic system to which $\alpha'_0$ is referred\footnote{It is of notice that the square root of $\alpha_0=e^{-2i\phi}\alpha_0'$, normalized to a proper volume element $V$, that is, $\Psi=e^{-i\phi}\sqrt{n_S}$, where $n_S=\alpha'_0/V$ is the density of superconducting electrons, constitutes the order parameter at the basis of the Ginzburg-Landau theory of superconductivity (\cite{Ginzburg:50}), called by Ginzburg the $\Psi$-theory of superconductivity (\cite{Ginzburg:04}), published seven years before than BCS.}. In other words, the quantities $\alpha'_0$  measures the deformation of the superfluid nuclear system in gauge space. 

Similar arguments can be used regarding the excitation of pairing vibrations in terms of Cooper pair transfer around closed shells as compared to one-particle transfer. As seen from Fig. \ref{fig2.1.5} (c)--(b), the random phase approximation (RPA) amplitudes $X_\nu^a$ and $Y^a_\nu$ are coherently summed over pairs of time reversal states to give rise to the spectroscopic amplitudes associated with the direct excitation of the pair addition mode displayed in (d). Because of the (dispersion) relation (b)+(c)$\equiv$(d), the $X_\nu$- and $Y_\nu$-amplitudes are, in the Cooper pair transfer process, correlated, the situation being not so in the case of one-particle transfer (see e.g. Fig. \ref{fig1.4.2} (e), (f); see also Fig. \ref{fig2.1.5} (g)). In other words, while the (renormalized) spectroscopic amplitude and (renormalized) radial form factor associated with a one-nucleon transfer process depend on a single renormalized energy and associated wavefunction, the corresponding quantities for Cooper pair transfer depend on a distribution of (renormalized) states and  wavefunctions of levels around the Fermi energy, leading to a coherent state\footnote{We note that, in spite of the fact that one is dealing with the connection between structure and direct transfer reactions, no mention has been made of spectroscopic factors in relation with one-particle transfer processes, let alone when discussing two-particle transfer processes.}. 
\begin{figure}[h!]
\centerline {
\includegraphics*[width=8cm]{nutshell/figs/fig2_1_5}
}
\caption[NFT diagrams associated with one- and two-particle transfer from closed shell associated with ZPF of pair addition modes.]{NFT diagrams associated with one- and two-particle transfer from closed shell associated with ZPF of pair addition modes. \textbf{(a)} ZPF associated with the virtual excitation of a pair addition mode and two uncorrelated holes. \textbf{(b)} two-particle transfer filling a two hole state associated with the ZPF, \textbf{(c)} two-particle transfer to time-reversal states lying above the Fermi energy. These processes receive contribution from all $(\nu,\bar\nu)$ pairs (sum over $\nu>0$), leading to \textbf{(d)}, the direct excitation of the collective pair addition mode. The relation \textbf{(b)}+\textbf{(c)}$\equiv$\textbf{(d)} is the NFT graphical representation of the random phase approximation (RPA) dispersion relation used to calculate the properties of the pair addition mode in the harmonic approximation (Section \ref{App1E}). The backwards and forwards going RPA amplitudes are displayed in  \textbf{(e)} and \textbf{(f)} respectively. \textbf{(g)} one-particle transfer proceeding through the filling of a hole associated with the ZPF.}
\label{fig2.1.5}
\end{figure}

We write now, in somewhat more detail, the spectroscopic amplitudes ($B$-coefficients) for the two-nucleon transfer processes mentioned above. Namely, one for the case in which $A$ and $B(=A+2)$ are members of a pairing rotational band.\idx{BCS!pair transfer amplitudes} A second one, for the case in which they are members of a pairing vibrational band. That is, 
\begin{equation}\label{eqC21.8}
\begin{split}
\mathbf{1)}, B((nlj)^2(0))=&\langle BCS(N+2)|\frac{[a^\dagger_{nlj}a^\dagger_{nlj}]^0_0}{\sqrt{2}}|BCS(N)\rangle\\
&=\sqrt{j+1/2}\,U_{nlj}(N)V_{nlj}(N+2), 
\end{split}\idx{Pairing rotational bands!two-nucleon transfer spectroscopic amplitudes}
\end{equation}
and
\begin{equation}\label{eqC21.9}
\begin{split}
\mathbf{2)}, B((nlj)^2(0))=&\langle (N_0+2)(gs)|\frac{[a^\dagger_{nlj}a^\dagger_{nlj}]^0_0}{\sqrt{2}}|N_0(gs)\rangle \\
&\\
&=\left\{\begin{array}{c}
\sqrt{j_k+1/2}\;\;X^a(n_kl_kj_k)\quad (\epsilon_{j_k}>\epsilon_F) \\ 
\sqrt{j_i+1/2}\;\;Y^a(n_il_ij_i)\quad (\epsilon_{j_i}\leq\epsilon_F).
\end{array} \right.
\end{split}
\end{equation}
\idx{Pairing vibrations!two-nucleon transfer spectroscopic amplitudes}
Where the $X$ and $Y$ coefficients are the forwardsgoing and backwardsgoing RPA amplitudes of the pair addition mode.
For actual numerical values, see Tables \ref{tab1D1},   \ref{tab1E2}--\ref{tab1E5}. Making use of these spectroscopic amplitudes to calculate the successive and simultaneous transfer amplitudes, correcting both of them for non-orthogonality contributions, makes the above picture the quantitative probe of Cooper pair correlations in nuclei\footnote{\label{f29C3} \cite{Bayman:82}, \cite{Thompson:88,Thompson:09,Broglia:04a} and \cite{Potel:13}, see also \cite{Gotz:75} and \cite{Broglia:77b}.}, as can be seen  from the variety of examples shown in the present monograph. In particular those shown in  Figs. \ref{fig1.5},  \ref{fig2A4} and \ref{fig8_2_1}. In the first one, we display the absolute cross sections associated with the $^{A}_{50}$Sn$_{N}$($p,t$)$^{A-2}_{50}$Sn$_{N-2}$(gs) reactions studied between the double closed shell systems $^{132}_{50}$Sn$_{82}$ and $^{100}_{50}$Sn$_{50}$. That is, from pairing vibrational through pairing rotational to pairing vibrational nuclei again.  Examples of the population of  pairing vibrational states are displayed in Figs. \ref{fig2A4} and \ref{fig8_2_1}.

Summing up, one will use throughout the present monograph, exception made when explicitly mentioned, absolute cross sections as the  link between spectroscopic amplitudes and experimental observations\footnote{It is of notice that concerning nuclear structure information, it is also contained in the optical potential used in the calculation of the corresponding cross sections. See footnote \ref{f21c1} Ch. \ref{introduction}.}.
 \begin{table}[h!]
	{\begin{tabular}{|c|c|c|c|c|}
			\cline{2-5} 
			\multicolumn{1}{c|}{ }& \multicolumn{2}{|c|}{ $^{112}$Sn($p,t)^{110}$Sn(gs)}&\multicolumn{2}{|c|}{$^{124}$Sn($p,t)^{122}$Sn(gs)} \\
			\hline
			$nlj^{\,\mathbf a)}$ & BCS$^{\mathbf b)}$ & $V_{low-k}^{\mathbf c)}$ & BCS$^{\mathbf d)}$& NuShell$^{\mathbf e)}$  \\
			\hline
			$1g_{7/2}$ & 0.96 &-1.1073 & 0.44 & 0.63  \\
			$2d_{5/2}$ & 0.66 & -0.7556& 0.35 & 0.60  \\
			$2d_{3/2}$ & 0.54 &  -0.4825& 0.58 & 0.72  \\
			$3s_{1/2}$ & 0.45 &  -0.3663&  0.36 & 0.52  \\
			$1h_{11/2}$ & 0.69 & -0.6647 & 1.22 & -1.24  \\
			\hline 
	\end{tabular}}
	\caption{\idx{Pairing rotational bands!two-nucleon transfer spectroscopic amplitudes} Two-nucleon transfer spectroscopic amplitudes associated with the reactions $^{112}$Sn$(p,t)^{110}$Sn(gs) and $^{124}$Sn$(p,t)^{122}$Sn(gs). \textbf{a}) quantum numbers of the two--particle configurations $(nlj)^2_{J=0}$ coupled to angular momentum $J=0$. \textbf{b}) and \textbf{d}) $\langle BCS|T_\nu|BCS\rangle=\sqrt{(2j_\nu+1)/2}\,U_\nu(A) V_\nu(A+2)\;(A+2=112$ and $ 124$ respectively), where $T_\nu=[a^\dagger_{\nu}a^\dagger_\nu]^0/\sqrt{2} \,(\nu\equiv nlj)$ (cf. \cite{Potel:11,Potel:13,Potel:13b}) \textbf{c}) two-nucleon transfer spectroscopic amplitudes calculated making use of initial and final state wavefunctions obtained by diagonalizing a $v_{low-k}$, that is a renormalized, low--momentum interaction derived from the CD--Bonn nucleon--nucleon potential (see \cite{Guazzoni:06} and references therein). \textbf{e}) Two--neutron overlap functions obtained making use of the shell--model wavefunctions for the ground state of $^{122}$Sn and $^{124}$Sn calculated with the code NuShell \citep{Brown:07}. The wavefunctions were obtained starting with a $G$--matrix derived from the $CD$--Bonn nucleon--nucleon interaction \cite{Machleidt:96}. These amplitudes were used in the calculation of $^{124}$Sn$(p,t)^{122}$Sn absolute cross sections carried out by I.J. Thompson \citep{Thompson:13}. \idx{BCS!pair transfer amplitudes}}\label{tab1D1}
\end{table}


  \begin{figure}
  \centerline{\includegraphics*[width=10cm,angle=0]{nutshell/figs/cross_strength.pdf}}
  \caption[Two-nucleon transfer cross section $^A_{50}$Sn$_{N}$$(p,t)^{A-2}_{\;\;\;50}$Sn$_{N-2}$(gs).]{\idx{Pairing rotational bands} Absolute value of the  two-nucleon transfer cross section $^A_{50}$Sn$_{N}$$(p,t)^{A-2}_{\;\;\;50}$Sn$_{N-2}$(gs) ($102\leq A\leq 132$, i.e. from (final ($A-2$)) closed shell $^{100}_{50}$Sn$_{50}$ to (initial $A$) closed shell $^{132}_{50}$Sn$_{82}$ isotopes, see   \cite{Potel:13}, \cite{Potel:13b}, see also Sect. \ref{C8S2}) calculated making use of RPA and BCS spectroscopic amplitudes (Eqs. (\ref{eqC21.8}) and (\ref{eqC21.9}) respectively; in connection with this last one see Tables \ref{tab1D1} and \ref{tab8_2_1}), and taking into account successive and simultaneous transfer in second order DWBA, properly corrected for non-orthogonality contributions, in comparison with the experimental data (\cite{Guazzoni:99}, \cite{Guazzoni:04}, \cite{Guazzoni:06}, \cite{Guazzoni:08}, \cite{Guazzoni:11}, \cite{Guazzoni:12}). \idx{Cooper pair!--absolute transfer cross sections ($\sigma_{2n}$)}
  }\label{fig1.5}
  \end{figure}
\FloatBarrier




\section{Renormalization and spectroscopic amplitudes}\label{S3.2}
\idx{Particle-vibration coupling!single-particle renormalization}
As a result of the interweaving of single-particle and collective motion, the nucleons acquire a state dependent self energy $\Delta E_j(\omega)$.   Consequently, the single-particle potential which was already non-local in space (exchange potential, related to the Pauli principle) becomes also non-local in time. A possible technique to make it local is that of the effective mass approximation. In it  one  describes the single-particle motion in terms of a local potential  given by $U'(r)=(m/m^*)U(r)$. The quantity $m^*=m_km_\omega/m$ is the effective nucleon mass, $m_k$ being the so-called $k$-mass associated with the Fock potential, and $m_\omega=m(1+\lambda)$ being the $\omega$-mass (non-locality in time, as implied by the relation $\Delta \omega\Delta t\geq1$). The quantity  $\lambda=-\partial \Delta E(\omega)/\partial \hbar \omega$ is the so-called mass enhancement factor. It reflects the ability with which vibrations dress single-particles. In other words, it measures the probability with which a nucleon moving at  $t=-\infty$ in a ``pure'' orbital $j$ can be found at a later time in a 2$p-1h$ like (doorway state) $|j'L;j\rangle$, $L$ being the multipolarity of a vibrational state. Within this context, the discontinuity taking place at the Fermi energy in the dressed particle picture ($Z_\omega=(m/m_\omega)$) reflects the associated single-particle occupancy probability\footnote{See e.g. \cite{Brink:05} Ch. 9, and references therein.}.


It is of notice that dressed particles  imply also an induced pairing interaction $v_p^{ind}$ which renormalizes the bare $NN$-$^1S_0$ interaction $v_p^{bare}$, and results  from the exchange of the dressing vibrations between pairs of nucleons moving in time reversal states close to the Fermi energy\footnote{See e.g. Figs. \ref{fig3_A_3},  \ref{fig6G1} (II) (b),(d),(e).}. \textit{In other words, fluctuations in the normal density $\delta \rho$    and associated particle-vibration coupling vertices (Time-Dependent HF $\delta U$), lead to abnormal (pairing) density}. Whether this results in  a dynamic or static phenomenon, depends on whether the parameter (see Fig. \ref{fig1_E8}\footnote{\cite{Brink:05} App. H. Sect. H4 and refs. therein; \cite{Barranco:05}.}) 
\begin{align}\label{eq2_1_10}
x'=G'N'(0),  
\end{align}
product of the effective pairing strength, 
\begin{align}\label{eq2_2_2}
G'=Z_\omega^2(v_p^{bare}+v_p^{ind}),
\end{align}
and of the renormalized density of levels $N'(0)$ is considerable smaller  (larger) than $\approx1/2$. The quantity $G'$ is the sum of the bare and induced pairing interaction, renormalized by the degree of single-particle content of the levels in which nucleons correlate. The quantity 
\begin{align}\label{eq3.2.3}
N'(0)=Z_\omega^{-1}N(0)=(1+\lambda)N(0) \idx{Mass enhancement factor!--and single-particle content}
\end{align}
is the similarly renormalized density of levels at the Fermi energy. From the above relations one obtains 
\begin{align}
x'=Z_\omega(v_p^{bare}+v_p^{ind})N(0).
\end{align}
Typical values of $Z_\omega\approx0.7$, while for nuclei along the stability valley the bare and the induced pairing contributions are about equal. Thus, according to Eq. (\ref{eq2_2_2})  $G'\approx G=v^{bare}_p$, as in the case of a non-renormalized situation. On the other hand the physics is radically different, particles being a consistent fraction of the time in excited states coupled to collective vibrations, pairing acquiring a state dependence. 
 
All of the above many-body, $\omega$-dependent effects which imply in many cases a coherent sum of amplitudes, together with the corresponding renormalizations of the single-particle radial wavefunctions (formfactors) not discussed within the present framework, are not simple to capture in a spectroscopic factor\footnote{In keeping with the fact that $m_k\approx 0.6-0.7 m$ and that $m^*\approx m$, as testified by the satisfactory fitting standard Saxon-Woods potentials provides for the valence orbitals of nucleons of mass $m$ around closed shells, one obtains $m_\omega\approx 1.4-1.7 m$. Thus $Z_\omega \approx 0.6-0.7$. It is still an open question how much of the observed single-particle depopulation can be ascribed to hard core effects, which shifts the associated strength to high momentum levels (see \cite{Dickhoff:05}, \cite{Jenning:11}, \cite{Kramer:01}, \cite{Barbieri:09}, \cite{Schiffer:12}, \cite{Duguet:12}, \cite{Furnstahl:10}).  An estimate of such an effect of about 20\% will not qualitatively alter the long wavelength estimate of $Z_\omega$ given above. Arguably, a much larger depopulation through hard core effects remains an open problem within the overall picture of elementary modes of nuclear excitation and of medium polarization effects. Within this context it remains an open question  the role   the renormalization of the radial dependence of the single-particle wavefunctions (form factors), due to many-body effects, can play in the determination of spectroscopic factors. Discussions with Augusto Macchiavelli on this subject are gratefully acknowledged.} neither in connection with one-particle transfer, nor regarding two-nucleon transfer processes\footnote{See \cite{Barranco:05,Barranco:99}.}. 






\section{Quantality Parameter}\label{App1A}
\idx{Quantality parameter}
\begin{table}
 \begin{tabular}{|c|c|c|c|c|c|}
 \hline \rule[-2ex]{0pt}{5.5ex}   constituents& $M/M_n$  & $a$(cm) &$v_0$(eV)  &q&phase($T=0$)    \\ 
 \hline \rule[-2ex]{0pt}{5.5ex}   $^{3}$He &3& 2.9(-8)  &8.6(-4)  &0.19  &liquid$^{a)}$    \\ 
 \hline \rule[-2ex]{0pt}{5.5ex}  $^{4}$He  &4&  2.9(-8)&  8.6(-4)&  0.14& liquid$^{a)}$   \\ 
 \hline \rule[-2ex]{0pt}{5.5ex}    H$_2$&2&  3.3(-8)&  32(-4)&  0.06&solid$^{b)}$   \\ 
 \hline \rule[-2ex]{0pt}{5.5ex}    $^{20}$Ne&20& 3.1(-8) &  31(-4)&  0.007&solid$^b)$    \\ 
 \hline \rule[-2ex]{0pt}{5.5ex}    nucleons&1&  9(-14)& 100(+6) &  0.5&liquid$^{a),c),d)}$  \\ 
 \hline 
 \end{tabular}
 \caption{Zero temperature phase for a number of systems of mass $M$ ($M_n$: nucleon mass), the first four depending on atomic interactions (range \AA, strength eV), the last one referring to the atomic nucleus. a) delocalized (condensed), b) localized, c) non--Newtonian solid (cf. e.g. \cite{Bertsch:88b}, \cite{DeGennes:94}), that is, systems which react elastically to sudden solicitations and plastically under prolonged strain, d) paradigm of quantal, strongly fluctuating, finite many-body  systems. Delocalization or less does not seem to depend much on whether one is dealing with fermions or bosons (\cite{Mottelson:02} and refs. therein; cf also \cite{Ebran:14}, \cite{Ebran:14b}, \cite{Ebran:13}, \cite{Ebran:12}).}\label{tab1A1}
 \end{table}
 \begin{figure}
 \centerline{\includegraphics*[width=13cm,angle=0]{nutshell/figs/potential.pdf}}
 \caption[Bare $NN$-interaction.]{Schematic representation of the bare $NN$-interaction acting among nucleons displayed as a function of the relative coordinate $r=|\mathbf{r}_1-\mathbf{r}_2|$,used to estimate the quantality parameter $q$, ratio of the zero point fluctuations (ZPF) of confinement and the potential energy.}\label{fig1A1}
 \end{figure}
The quantality parameter\footnote{\cite{Nosanow:76}, \cite{deBoer:57}, \cite{deBoer:48}, \cite{deBoer:48b},\cite{Mottelson:02}.} is defined as the ratio of the quantal kinetic energy of localization (confinement) and the potential energy, (cf. Fig. \ref{fig1A1} and Table \ref{tab1A1}).
 Fluctuations, quantal or classical, favor symmetry: gases and liquids are homogeneous. Potential energy on the other hand prefers special arrangements: atoms like to be at specific distances and orientations from each other (spontaneous breaking of translational and of rotational symmetry reflecting the homogeneity and isotropy of empty space\footnote{As already stated in footnote \ref{f1C3} of this Chapter, within such a  general context, the physics embodied in the quantality parameter is closely related to that which is at the basis of the classical Lindemann criterion (\cite{Lindemann:10}) to measure whether a system is ordered (e.g. a crystal) or disordered (e.g. a melted system) (\cite{Bilgram:87}, \cite{Lowen:94}, \cite{Stillinger:90,Stillinger:95}). The above statement concerning the competition between potential energy and fluctuations,  is also valid for the generalized Lindemann parameter (\cite{Stillinger:90}, \cite{Zhou:99}), used to provide similar insight into inhomogeneous finite systems like e.g. proteins (aperiodic crystals \cite{Schrodinger:44}, see also Ehrenfest's theorem (\cite{Basdevant:05} pag. 138 see also App. \ref{C2AppC}).}).
 
 
  When $q$ is small, quantal effects are small and the lower state for $T<T_c$ will be a solid  (crystalline structure), $T_c$ denoting the critical temperature.  For sufficiently large values of $q\, (>0.15$) the system will display particle delocalization and,  likely, be  amenable, within some approximation, to a mean field description (Figs. \ref{fig1.0.6} and   \ref{fig1A3}). In fact, the step \textit{delocalization $\rightarrow$ mean field} is  not automatic, neither guaranteed. In any case, not for all properties neither for all levels of the system.
  
  As already stated, independent particle motion can be viewed as the most collective of all nuclear properties, reflecting the effect of all nucleons on a given one resulting in a macroscopic effect. Consequently, it should be possible to calculate the mean field in an accurate manner. Arguably, as accurately as one can calculate collective vibrations, e.g. quadrupole vibrations. But this does not mean that one knows how to correctly calculate the energy and associated dynamical deformation parameter of each single state of the quadrupole response function. Within this context one may find, through mean field approximation a good description for the energy of the valence orbitals of a nucleus but for  specific levels (e.g. the $d_{5/2}$ level\footnote{\cite{Idini:15}.} of the isotopes $^{119-120}$Sn, see Fig \ref{fig6.2.3}). It is not said that  including  particle-vibration coupling corrections, a process which in average makes theory come closer to experiment\footnote{See e.g. \cite{Bohr:75}, \cite{Bohr:77}, \cite{Hamamoto:77}, \cite{Bes:77c}, \cite{Barnes:72}, \cite{Reich:74}, \cite{Hamamoto:76}, \cite{Bortignon:77}, \cite{Bernard:81}, \cite{Mahaux:85}, \cite{Bes:71,Flynn:71}, \cite{Bortignon:76}, \cite{Bes:88}, \cite{Barranco:87b}, \cite{Vinh:95}, \cite{Barranco:01,Orrigo:09} and references therein.}, single specific quasiparticle energies will satisfactorily agree  with the data\footnote{See \cite{Tarpanov:14}.}. Cases like this one constitute a fruitful experience concerning the intricacies of the many-body problem in general, and of the nuclear one (finite many-body system, FMBS) in particular, where spatial quantization plays a central role. In other words, one is dealing with a self-bound, strongly interacting, finite many-body system generated from collisions originally associated  with a variety of astrophysical events and thus with  the coupling and interweaving of different scattering channels and resonances, a little bit as e.g. the Hoyle monopole resonance ($\alpha+\alpha+\alpha\rightarrow^{12}$C). Within the anthropomorphic  scenario such phenomena are found in the evolution of the Universe to eventually allow for the presence of organic matter and, arguably, life on earth\footnote{See e.g. \cite{Rees:00}, \cite{Meissner:15} and references therein.} more likely than to make mean field approximation, also the renormalized one, an ``exact'' description of nuclear structure and reactions for every single nuclear level.
 

 
\begin{figure}
\centerline{\includegraphics*[width=\textwidth,angle=0]{nutshell/figs/fig1A3.pdf}}
\caption[Independent particle motion vs. independent pair motion.]{\textbf{(I)} \textbf{(a)} Schematic representation of ``normal'' (independent-particle) motion of nucleons in  two-fold degenerate (Kramers, time-reversal degeneracy) orbits solidly anchored to the mean field (solid dots at the ends of the single-particle levels) and  displaying a sharp, step-function-like, discontinuity in the occupancy at the Fermi energy can lead to a deformed  (\cite{Nilsson:55}) rotating nucleus with a rigid moment of inertia $\mathcal{I}_r$ \textbf{(b)}. \textbf{(II)} Schematic representation of, phase correlated, independent nucleon Cooper pair  motion in which few (of the order of 5-8) pairs lead to \textbf{(c)} a sigmoidal occupation function at the Fermi energy and, having mainly uncoupled themselves from the fermionic mean field (no solid dot at the end of the Kramers invariant single-particle levels),    contribute in a reduced fashion to \textbf{(d)} the moment of inertia of quadrupole rotational bands leading to $\mathcal{I}\approx\mathcal{I}_r/2$ (cf. \cite{Belyaev:13}, \cite{Belyaev:59}, \cite{Bohr:75} and references therein), \textbf{(e)} pairing rotational bands in gauge space, an example of which is  provided by the ground states of the superfluid Sn-isotopes.}\label{fig1A3}
\end{figure}
\FloatBarrier
 \section{Generalized quantality parameter}\label{App1D}
 \idx{Generalized quantality parameter}
 Within the framework of independent particle motion, but allowing for the dressing of particles one obtains a sigmoidal distribution around the Fermi energy reflecting the $\omega$-dependence of $Z_\nu(\omega)$. In the case of the Sn-isotopes where the density of levels is $N(0)\approx 4$MeV$^{-1}$ and the energy of the lowest $\beta=0$ collective vibration ($\lambda^\pi=2^+$) is $\hbar\omega_2\approx1.3$ MeV, one expects the width of the distribution to be $\approx2-3$ MeV around the Fermi energy. Allowing nucleons not only to emit and reabsorb $\beta=0$ vibrational modes, but also to exchange them with the other nucleons, as well as to interact through $v_{p}^{bare}$, one is confronted with a many-body problem like the one discussed in Sect. \ref{S3.2}  for $x\gg1/2$,  which can be solved at profit within the BCS framework. It leads, again for Sn-isotopes, to a sigmoidal occupation probability extending over an energy range also of $\approx 3$ MeV ($\approx 2\Delta$) around the Fermi energy\footnote{Within this context see discussion following Eq. (\ref{eq1.4.57}) (see also the discussion in connection with Fig. 2-1 of \cite{Tinkham:96}).}. 
 
 
\begin{figure}
\centerline{\includegraphics*[width=8cm,angle=0]{nutshell/figs/fig1A4.pdf}}
\caption[A system of localized pairs in which the size of the pair bound state is small compared to the average spacing between pairs.]{A system of localized pairs (Schafroth pairs; \cite{Schafroth:58}, \cite{Schafroth:57}; see also \cite{Schrieffer:64}), in which the size of the pair bound state is small compared to the average spacing between pairs.}\label{fig1A4}
\end{figure}
\begin{figure}
\centerline{\includegraphics*[width=8cm,angle=0]{nutshell/figs/fig1A5.pdf}}
\caption[BCS condensation.]{At the basis of BCS superconductivity one finds condensation of very extended ($\approx10^{4}$ \AA), strongly overlapping (within the volume of a given pair the centers of $\approx10^6$ other pairs is found), weakly bound ($\approx2\Delta\approx$ meV) Cooper pairs corresponding to ordering in occupying momentum space, and not space-like condensation of strongly bound pairs which undergo Bose condensation (cf. \cite{Schrieffer:64}).}\label{fig1A5}
\end{figure}
 
 
 In spite of the similitudes in occupancy of levels lying around the Fermi energy between the two physical situations discussed above, the associated $L=0$ (gs)$\to$(gs) two-nucleon transfer cross sections are quite different. In the first case, it is proportional to $(Z_{\nu_1}\sqrt{\sigma_{\nu_1}})^2\approx(1/2)\times\sigma_{\nu_1}$ and to $(\sum_{\nu>0}U_{\nu}'V_{\nu}'\sqrt{\sigma_\nu})^2\approx(\alpha_0')^2\bar\sigma\approx50\times\bar\sigma$ in the second. The quantity $\sigma_{\nu_1}$ is the cross section associated with the lowest $ j_{\nu_1}^2(0)$ two-particle configuration. The summation $\sum_{\nu>0}$ extends over the orbitals around $\epsilon_F$ within a range $\approx2\Delta$, and $\bar\sigma$ is the average of the cross sections $\sigma_\nu$ associated with the valence configurations $j^2_\nu(0)$.
 
 In connection with the Cooper pair (BCS solution) one should,
  more correctly, use the wavefunction (\ref{eq0.1.101}) to calculate the two-nucleon transfer cross section. In keeping with the definition $c'_\nu=V'_\nu/U'_\nu$ ad the phasing (\ref{eq0.1.76}), it is found that
\begin{align}\label{eq3.4.1}
	\sum_{\nu>0}c'_\nu P_\nu^\dagger\ket{0}_F=e^{2i\phi}\sum_{\nu>0}c_\nu P_\nu^\dagger\ket{0}_F,
\end{align}
leading to, \idx{BCS!--condensation (phase coherence)}
 \begin{align}\label{eq3.4.2}
\ket{\Psi_{N_1}}\sim\left(e^{2i\phi}\sum_{\nu>0}c_\nu P^\dagger_\nu\right)^{N_1/2}\ket{0}_F. \idx{Cooper pair!BCS projected-many pair state}
 \end{align}
 \textit{That is, independent pair motion, each Cooper pair contributing a phase $\phi$ to $\ket{N}$}. 
 
 
 Note that in a metallic superconductor as described by BCS, there are of the order of $10^6$ Cooper pairs around the Fermi surface. Because the phase $\phi$ is common to so many pairs, its effects can become felt on the macroscopic scale. \textit{By the same token, the quantal behaviour of the BCS condensate must also be the same as the quantal behaviour of a single Cooper pair}. A feature which reflects itself in superfluid nuclei, although in this case one is talking of only 6--8 Cooper pairs.  
 
 
 If one finds, as one does\footnote{\cite{Josephson:62,Anderson:63}.}, that the tunneling energy is a function of the relative phase, it will adjust so as to minimize the energy through tunneling of electrons.  The generalized quantality parameter, \idx{Cooper pair!generalized quantality parameter} ratio of the kinetic energy of confinement $(T_\xi=\hbar^2/(2m_e\xi^2)\approx4\times10^{-5}$meV; $\xi\approx10^4$\AA) of the electrons forming the Cooper pairs undergoing tunneling, and the Cooper pair correlation (binding) energy $E_{corr}(\approx3$ meV, $\Delta\approx1.4$ meV, Pb), is much smaller than 1 $ (q_\xi\approx10^{-5})$. A result which implies that for $T<T_c\,(T_c\approx7.1$ K, critical temperature of Pb), the electron Cooper pair partners are solidly anchored to each other. Said it differently, the intrinsic Cooper pair motion is frozen \idx{Cooper pair!--stability (bosonic) requirements}
  because of the pairing gap. Consequently, if under such conditions ($T<T_c$) one observes a current across the unbiased junction between the two superconductors, it must be a current of carriers of mass $2m_e$ and charge $2e$. If one electron partner of a Cooper pair tunnels, the other entangled electron does it  also.\idx{Josephson effect!entanglement between Cooper pair tunneling partners}
   \idx{Josephson effect!transfer (tunneling) probability}
   \idx{BCS!--(generalized) quantality parameter}
 \textit{It is of notice that in the present case this result emerges not because of the large role played by potential effects, but because of the very large dimensions, low relative intrinsic momentum of the Cooper pair partners.}
\begin{figure}
\centerline{\includegraphics*[width=15cm,angle=0]{nutshell/figs/resumevec.pdf}}
\caption{Classical localization and zero point fluctuations, associated with independent-particle (normal density) and, phase correlated, independent-pair  (abnormal density) motion. 
}\label{fig1D2}
\end{figure}

 BCS condensation, that is condensation of strongly overlapping, very extended, weakly bound Cooper pairs corresponds to ordering in occupying momentum space, and not space-like condensation of strongly bound clusters which undergo Bose condensation (Figs. \ref{fig1A4} and \ref{fig1A5} respectively). In BCS condensation, the inner, intrinsic structure of the pair, that is, the fact that it is made out of fermions entangled in time reversal states, is the characterizing feature, at the basis of ODLRO. \idx{Off-diagonal-long-range-order (ODLRO)}
 Cooper pairs (see Eqs. (\ref{eq0.1.82}) and (\ref{eq3.4.1})) are not bosons, as $(P_\nu^{\dagger})^2=0$ \idx{Cooper pair!quasi-bosonic character (conditions)}
 testifies (see also App. \ref{App3E}). However, it is true that, under certain circumstances, namely $T<T_c$, $q<1/\xi$ and $d\approx D_0\leq\xi$ in the case of heavy ion reactions between superfluid nuclei, where $\xi$ is the correlation length and $d$ is the width of the  Josephson junction ($D_0$ being the distance of closest approach), Cooper pairs show properties which strongly resemble those of bosons\footnote{\label{f45C3} \cite{terHaar:95b}, \cite{TerHaar:95} Ch. 4.9 and \cite{Ehrenfest:31}.}.\idx{Cooper pair!quasi-bosonic character (conditions)}
 

We conclude this Section by summarizing in\footnote{In connection with  Fig. \ref{fig1D2}, the estimate $R=8/k_F$ was carried out with the help of the Fermi gas model (see e.g. \cite{Bohr:69}). The Fermi momentum is written as $k_F\approx (3\pi^2 A/2V)^{1/3}\approx (\frac{3\pi^2}{2}\rho(0))^{1/3}$. It is of notice that this expression leads for $\rho(0)\approx 0.17$ fm $^{-3}$ and to the Fermi momentum $k_F\approx 1.36 $ fm$^{-1}$.} Fig. \ref{fig1D2}, the central aspect of the parallel between independent-particle and independent pair motion discussed above. 





\section[Pair vibration spectroscopic amplitudes]{Two-nucleon spectroscopic amplitudes associated with pairing vibrational modes in closed shell systems: the $^{208}$Pb case.}\label{App1E}
The solution of the pairing Hamiltonian
leads, in the case of closed shell systems and within the harmonic approximation (RPA), to pair addition $(a)$ pair removal $(r)$ two-particle, two-hole correlated modes. The associated creation  operator have been defined in Eqs. (\ref{eq1.0.34})--(\ref{eq1.0.36})
for the pair addition mode. Similarly, in the case of the pair removal mode, 
\begin{equation*}
\Gamma_r^\dagger(n)=\sum_i X_n^r(i)\Gamma_i^\dagger+\sum_kY^r_n(k)\Gamma_k,
\end{equation*}
$X,Y$ fulfilling
\begin{equation*}
\sum X^2-Y^2=1,
\end{equation*}
and
\begin{equation*}
\Gamma_k^\dagger=a_k^\dagger a_{\tilde k}^\dagger,\quad (\epsilon_k>\epsilon_F).
\end{equation*}
Similarly,
\begin{equation*}
\Gamma_i^\dagger=a_{\tilde i} a_i ,\quad (\epsilon_i\leq\epsilon_F).
\end{equation*}
The relations
\begin{equation*}
[H,\Gamma_a^\dagger(n)]=\hbar W_n (\beta=+2),
\end{equation*}
and
\begin{equation*}
[H,\Gamma_r^\dagger(n)]=\hbar W_n (\beta=-2),
\end{equation*}
where $\beta$ is the transfer quantum number, lead to the dispersion relations, 
\begin{equation}\label{eq3.5.1}
\frac{1}{G(\pm2)}=\sum_k \frac{(\Omega_k/2)}{2\epsilon_k\mp W_n(\pm2)}+\sum_i \frac{(\Omega_i/2)}{2\epsilon_i\pm W_n(\pm2)}, \idx{Pairing vibrations!RPA relations}
\end{equation}
$n$ labeling the corresponding solutions in increasing order of energy. In the above equation, $\Omega_j=j+1/2$ is the pair degeneracy of the orbital with total angular momentum $j$.





For the case of the (neutron) pair addition and pair substraction modes of $^{208}$Pb the above equations are  graphically solved in Fig \ref{fig1E1} (see also Table \ref{tab1E1}). The minimum of the dispersion relation defines the Fermi energy of the system under study. This is in keeping with the fact that \textit{in the case in which $W_1 (\beta=+2)=W_1(\beta=-2)=0$, situation corresponding to the  transition between normal and superfluid phases\footnote{See footnote \ref{f25Ch1} Ch. \ref{introduction}.}}, the energy value at which the dispersion relation touches for the first time the energy axis, coincides with the BCS $\lambda$ variational parameter. It is of notice that, as a rule, the Fermi energy of closed shell nuclei is empirically defined as half the energy difference between the last occupied and the first empty single particle state\footnote{cf. e.g. \cite{Mahaux:85}.}. Making use of the values\footnote{\label{f50C3} It is of notice that in the present section, the quantities $E_{corr}$ are defined in such a way that their value is positive.} 
  \begin{figure}
  \centerline{\includegraphics*[width=\textwidth,angle=0]{nutshell/figs/dispersion.pdf}}
  \caption[Dispersion relation for $^{208}$Pb.]{The right hand side of the RPA pairing vibrational dispersion relation for neutrons in the case of the closed shell system $^{208}$Pb in the region between the two neighboring shells ($p_{1/2}$ and $g_{9/2}$). All quantities are in MeV. For each value of $G$ there is a straight horizontal line, which is divided by the the curve in three sections. The first one from the left corresponds to the pairing correlation energy of the nucleus $^{206}$Pb (two correlated neutron hole states) while the last segment to the right measures the pairing correlation energy of $^{210}$Pb (two correlated neutrons above closed shell) the intermediate segment measures the energy of the two phonon (correlated ($2p-2h$)) pairing vibrational state  of $^{208}$Pb.}\label{fig1E1}
  \end{figure}
\begin{table}
	\begin{center}
\begin{tabular}{|c|c|c|}
\hline \rule[-2ex]{0pt}{5.5ex}   orbit& $\epsilon_j$  & $\epsilon_{p_{1/2}}-\epsilon_k\equiv|\epsilon_k|-|\epsilon_{p_{1/2}}|$  \\ 
\hline
$0h_{9/2}$&-10.62 &3.47\\
$1f_{7/2}$& -9.50&2.35\\
$0i_{13/2}$& -8.79&1.64\\
$2p_{3/2}$& -8.05&0.90\\
$1f_{5/2}$& -7.72&0.57\\
$2p_{1/2}$& -7.15&0\\
\hline \rule[-2ex]{0pt}{5.5ex}   $\epsilon_F=-5.825$ MeV&   & $\epsilon_k-\epsilon_{g_{9/2}}\equiv|\epsilon_{g_{9/2}}|-|\epsilon_k|$  \\ 
\hline
$1g_{9/2}$&-3.74 &0.\\
$0i_{11/2}$& -2.97&0.77\\
$0j_{15/2}$& -2.33&1.41\\
$2d_{5/2}$& -2.18&1.56\\
$3s_{1/2}$& -1.71&2.03\\
$1g_{7/2}$& -1.27&2.47\\
$2d_{3/2}$& -1.23&2.51\\
\hline
\end{tabular}
\end{center}
\caption{Empirical energies of the valence single-particle levels of $^{208}$Pb. In the upper part the occupied levels ($\epsilon_i\leq\epsilon_F$) are displayed while in the lower part the empty levels are shown ($\epsilon_k>\epsilon_F$). Of notice that $\epsilon_{p_{1/2}}-\epsilon_{g_{9/2}}=3.41$ MeV, is the single-particle gap associated with $N=126$ shell closure.}\label{tab1E1}
 \end{table}
 
 \begin{equation}\label{eq3.5.2}
 \left\{
 \begin{array}{c}
  E_{corr}(+2)=BE(208)+BE(210)-2BE(209)=1.248\,\text{MeV},\\ 
  E_{corr}(-2)=BE(208)+BE(206)-2BE(207)=0.640\,\text{MeV},
 \end{array}
 \right.
 \end{equation}
 one obtains
 $W_1(-2)+W_1(+2)=(BE(208)-BE(206))-(BE(210)-BE(208))$
 $=14.11-9.115=4.995 \,\text{MeV}$. \textit{One notes that in the above calculations all energies differences are positive}. In particular (see Table \ref{tab1E1})  
\begin{equation*}
\epsilon_i<\epsilon_F\Rightarrow \epsilon_F-\epsilon_i=-|\epsilon_F|+|\epsilon_i|=|\epsilon_i|-|\epsilon_F|>0,
\end{equation*}
and
\begin{equation*}
\epsilon_k>\epsilon_F\Rightarrow \epsilon_k-\epsilon_F=-|\epsilon_k|+|\epsilon_F|=|\epsilon_F|-|\epsilon_k|>0.
\end{equation*}
Thus,
\begin{equation*}
\left\{
\begin{array}{c}
 2(\epsilon_F-\epsilon_{p_{1/2}})=W_1(-2)+E_{corr}(-2)>0,\\ 
 2(\epsilon_{g_{9/2}}-\epsilon_F)=W_1(+2)+E_{corr}(+2)>0.
\end{array}
\right.
\end{equation*}
From Fig. \ref{fig1E1} and Table \ref{tab1E1} one can then write,
\begin{equation*} 
2\times(-5.825-(-7.15))\,\text{MeV}=2.650\,\text{MeV}=W_1(-2)+0.640\,\text{MeV}
\end{equation*}
and 
\begin{equation*} 
2\times(-3.74\, \text{MeV}-(-5.825)\,\text{MeV})=4.17\, \text{MeV}=W_1(+2)+1.248\, \text{MeV}.
\end{equation*}
Consequently,
\begin{equation*} 
W_1(-2)=2.01\,\text{MeV}\quad \text{and} \quad W_1(+2)=2.92\,\text{MeV},
\end{equation*}
leading to,
\begin{equation}\label{eq2.5.1} 
W_1(+2)+W_1(-2)=4.93\,\text{MeV}.
\end{equation}

\subsection{Pair removal mode}\label{S3.5.1}
\idx{Pairing vibrations!two-nucleon transfer spectroscopic amplitudes}
In Fig. \ref{fig1E2}, a graphical representation of the forwards going RPA amplitude of the pair removal mode is shown. Its expression for $n=1$, is 
\begin{equation*}
X_1^r(i)=\frac{\frac{1}{2}\Omega_i^{1/2}\Lambda_1(-2)}{2(\epsilon_F-\epsilon_i)-W_1(-2)},
\end{equation*}
where
\begin{equation*}
\begin{split}
2\times(\epsilon_F-\epsilon_i)-W_1(-2)&=2\times(\epsilon_F-\epsilon_i)-2\times(\epsilon_F-\epsilon_{p_{1/2}})+E_{corr}(-2)\\
&=2\times(\epsilon_{p_{1/2}}-\epsilon_i)+E_{corr}(-2)=2\times(|\epsilon_i|-|\epsilon_{p_{1/2}}|)+E_{corr}(-2).
\end{split}
\end{equation*}
Thus,
\begin{equation*}
X_1^r(i)=\frac{\frac{1}{2}\Omega_i^{1/2}\Lambda_1(-2)}{2(|\epsilon_i|-|\epsilon_{p_{1/2}}|)+E_{corr}(-2)}.
\end{equation*}
Making use of the empirical value of $E_{corr}(-2)$ worked out above one obtains, 
  \begin{figure}
  \centerline{\includegraphics*[width=0.1\textwidth,angle=0]{nutshell/figs/removal_forward.pdf}}
  \caption[NFT representation of the forwards going RPA amplitude of the pair removal mode  describing a two correlated hole state.]{\idx{Pairing vibrations!two-nucleon transfer spectroscopic amplitudes}NFT representation of the forwards going RPA amplitude of the pair removal mode (double downward going arrowed line) describing a two correlated hole state (single downward going arrowed line for each hole with quantum numbers collectively labeled $i$).}\label{fig1E2}
  \end{figure}
\begin{equation}\label{eq3.5.4}
X_1^r(i)=\frac{\frac{1}{2}\Omega_i^{1/2}\Lambda_1(-2)}{2(|\epsilon_i|-|\epsilon_{p_{1/2}}|)+0.640\,\text{MeV}}.
\end{equation}
In Fig. \ref{fig1E3} we display the graphical process associated with the backwards going ($n=1$) RPA amplitude,
\begin{equation*}
Y_1^r(k)=\frac{\frac{1}{2}\Omega_k^{1/2}\Lambda_1(-2)}{2(\epsilon_k-\epsilon_F)+W_1(-2)}.
\end{equation*}
  \begin{figure}
  \centerline{\includegraphics*[width=0.2\textwidth,angle=0]{nutshell/figs/removal_backward.pdf}}
  \caption[Pairing vibration backwards going amplitudes.]{\idx{Pairing vibrations!two-nucleon transfer spectroscopic amplitudes} Same as Fig. \ref{fig1E2} but for the backwards going amplitudes.}\label{fig1E3}
  \end{figure}
  Making use of

\begin{equation*}
\begin{split}
2\times(\epsilon_F&-\epsilon_{p_{1/2}})-E_{corr}(-2)=W_1(-2),
\end{split}
\end{equation*}
one can write
\begin{equation*}
\begin{split}
2\times(\epsilon_F-\epsilon_{p_{1/2}})+2\times(\epsilon_k-\epsilon_{F})-E_{corr}(-2)=2\times(\epsilon_k-\epsilon_{F})+W_1(-2),
\end{split}
\end{equation*}
leading to 
\begin{equation*}
\begin{split}
2\times(|\epsilon_{p_{1/2}}|-|\epsilon_k|)-E_{corr}(-2)=2\times(|\epsilon_{p_{1/2}}|-|\epsilon_{g_{9/2}}|)+2\times(|\epsilon_{g_{9/2}}|-|\epsilon_k|)-E_{corr}(-2).
\end{split}
\end{equation*}
Thus, 
\begin{equation*}
Y_1^r(k)=\frac{\frac{1}{2}\Omega_k^{1/2}\Lambda_1(-2)}{2(|\epsilon_{g_{9/2}}|-|\epsilon_k|)+2(|\epsilon_{p_{1/2}}|-|\epsilon_{g_{9/2}}|)-E_{corr}(-2)}.
\end{equation*}
With the help of  $2\times(|\epsilon_{p_{1/2}}|-|\epsilon_{g_{9/2}}|)-E_{corr}(-2)=6.82 \text{MeV}-0.640 \text{MeV}=6.18 \text{MeV}$, one obtains,
\begin{equation}\label{eq3.5.5}
Y_1^r(k)=\frac{\frac{1}{2}\Omega_k^{1/2}\Lambda_1(-2)}{2(|\epsilon_{g_{9/2}}|-|\epsilon_k|)+6.18\,\text{MeV}}.
\end{equation}
The above  expressions of $X_1^r(i)$ and $Y_1^r(k)$ contain the experimental values of the $2$-hole correlation energies (0.640 MeV). \textit{Because (see Fig. \ref{fig1E1}) the associated values of $G$ does not lead to the observed correlation energy of the pair addition mode (1.248 MeV), we prefer to choose a single value of $G$ and use the resulting $E_{corr}(-2)$ (=0.5 MeV) and $E_{corr}(+2)$ (=1.5 MeV), correlation energies, to calculate the corresponding $X,Y$ amplitudes} for both the lowest removal and lowest addition pairing modes. Making use of, 
\begin{equation*}
\begin{split}
2\times(|\epsilon_{p_{1/2}}|-|\epsilon_{g_{9/2}}|)=6.82\,\text{MeV}\quad\text{and}\quad 2&\times(|\epsilon_{p_{1/2}}|-|\epsilon_{g_{9/2}}|)-E_{corr}(-2)\\
&=(6.82-0.5)\,\text{MeV}=6.32\,\text{MeV},
\end{split}
\end{equation*}
one can write
\begin{align}\label{eq3.5.6}
X_1^r(i)&=\frac{\frac{1}{2}\Omega_i^{1/2}\Lambda_1(-2)}{2(|\epsilon_i|-|\epsilon_{p_{1/2}}|)+0.5\,\text{MeV}},\\ Y_1^r(k)&=\frac{\frac{1}{2}\Omega_k^{1/2}\Lambda_1(-2)}{2(|\epsilon_{g_{9/2}}|-|\epsilon_k|)+6.32\,\text{MeV}}.\label{eq3.5.7}
\end{align}
Tables \ref{tab1E2} and \ref{tab1E3} contain the amplitudes of the pair removal mode of $^{208}$Pb ($\Gamma^\dagger_r(1)=\sum X^r_{1}(i)\Gamma^\dagger_i+\sum Y^r_{1}(k)\Gamma_k$), that is of the two neutron  correlated hole state describing $|^{206}\text{Pb (gs)}\rangle=\Gamma^\dagger_r(1)|0\rangle$. 


It is of notice that the coupling strength $\Lambda_1 (-2)$ with which the pair removal mode couples to the two single-particles (-holes) states is calculated by normalizing the amplitudes: \textbf{1}) Tamm Dancoff, TD implies no ground state correlations\footnote{See footnote \ref{f25Ch1} Ch. \ref{introduction}.}, i.e. to assume that $Y\equiv0$) $\sum_iA^2(i)=1.5549 \,\text{MeV}^{-2}$ and thus $\Lambda_1 (-2)=0.802$ MeV, ($\sum X(i)^2_{TD}=1$); \textbf{2}) RPA, $\Lambda_1^2 (-2)\times(\sum_i A^2(i)-\sum_k B^2(k))=\Lambda_1^2(-2)\times1.45073=1$. Thus $\Lambda_1(-2)=0.830$ MeV.


 The above results show that there is a few percentage difference between the two values of $\Lambda$ (TD and RPA), as well as for the corresponding $X$ amplitudes (Table \ref{tab1E2}). Nonetheless, ground state correlations as expressed by the $Y$ amplitudes (Table \ref{tab1E3}), gives rise to a 53\% increase in the $^{206}$Pb$(t,p)^{208}$Pb(gs) absolute cross section, from 0.34 mb to 0.52 mb to be compared with experimental data $\sigma=0.68\pm 0.24$ mb (see Fig. \ref{fig2A4}).


\subsection{Pair addition mode}\label{S3.5.2}
\idx{Pairing vibrations!two-nucleon transfer spectroscopic amplitudes}
\begin{table}
\begin{tabular}{|c|c|c|c|c|c|}
\hline units &  &MeV  &MeV$^{-1}$  & RPA & TD \\ 
\hline  $nlj$&$\Omega_i$  &$|\epsilon_i|-|\epsilon_{p_{1/2}}|$  &$A(i)=\frac{\frac{1}{2}\Omega_i^{1/2}}{2(|\epsilon_i|-|\epsilon_{p_{1/2}}|)+0.5\,\text{MeV}}$  & $X_1^r(i)$ & $X_1^r(i)$ \\ 
\hline  $2p_{1/2}$& 1 &  0& 1 &  0.83&  0.80\\ 
\hline $1f_{5/2}$ & 3 &  0.57&0.528  & 0.44 & 0.42 \\ 
\hline  $2p_{3/2}$& 2 &  0.90&  0.307&  0.25&  0.25\\ 
\hline  $0i_{13/2}$& 7 & 1.64 & 0.350 & 0.29 &  0.28\\ 
\hline  $1f_{7/2}$&  4& 2.35 &  0.192& 0.16 &  0.15\\ 
\hline  $0h_{9/2}$&  5& 3.47 & 0.150 &  0.12&0.12  \\ 
\hline 
\end{tabular}\caption{\idx{Pairing vibrations!two-nucleon transfer spectroscopic amplitudes} Forwards going RPA amplitudes $X_1^r(i)$ of the lowest pair removal mode of $^{208}$Pb (i.e. $|gs\left(^{206}\text{Pb}\right)\rangle$ state), cf. Table XVI \cite{Broglia:73}.}\label{tab1E2}
\end{table}
\begin{table}
\begin{tabular}{|c|c|c|c|c|}
\hline units &  &MeV  &MeV$^{-1}$  &  RPA  \\ 
\hline  $nlj$&$\Omega_k$  &$|\epsilon_{g_{9/2}}|-|\epsilon_k|$  &$B(k)=\frac{\frac{1}{2}\Omega_i^{1/2}}{2(|\epsilon_{g_{9/2}}|-|\epsilon_k|)+6.23\,\text{MeV}}$  & $Y_1^r(i)$  \\ 
\hline  $1g_{9/2}$& 5 &  0& 0.179 &  0.15\\ 
\hline $0i_{11/2}$ & 6 &  0.77&0.158  & 0.13  \\ 
\hline  $0j_{15/2}$& 8 &  1.41&  0.156&  0.13\\ 
\hline  $2d_{5/2}$& 3 & 1.56 & 0.093 & 0.08 \\ 
\hline  $3s_{1/2}$&  1& 2.03 &  0.046& 0.04\\ 
\hline  $1g_{7/2}$&  4& 2.47 & 0.090 &  0.07  \\ 
\hline  $2d_{3/2}$&  2& 2.51 & 0.063 &  0.05 \\ 
\hline 
\end{tabular}\caption{\idx{Pairing vibrations!two-nucleon transfer spectroscopic amplitudes} Same as Table \ref{tab1E2} but for the backwards amplitudes $Y_1^r(k)$ of the lowest energy pair removal mode.}\label{tab1E3}
\end{table}
In Fig. \ref{fig1E4} the $X$-amplitude of the pair addition mode is shown (NFT diagram). The associated expression ($n=1$),
\begin{equation*}
X_1^a(k)=\frac{\frac{1}{2}\Omega_k^{1/2}\Lambda_1(+2)}{2(\epsilon_k-\epsilon_F)-W_1(+2)},
\end{equation*}
can, making use of the relation
  \begin{figure}
  \centerline{\includegraphics*[width=0.1\textwidth,angle=0]{nutshell/figs/addition_forward.pdf}}
  \caption[Pair addition mode.]{\idx{Pairing vibrations!two-nucleon transfer spectroscopic amplitudes} Same as Fig. \ref{fig1E2} but for the pair addition mode}\label{fig1E4}
  \end{figure}
  
\begin{equation*}
\begin{split}
2\times(\epsilon_k-\epsilon_F)-W_1(+2)&=2\times(\epsilon_k-\epsilon_F)-2\times(\epsilon_{g_{9/2}}-\epsilon_F)+E_{corr}(+2)\\
&=2\times(\epsilon_k-\epsilon_{g_{9/2}})+E_{corr}(+2)=2\times(|\epsilon_{g_{9/2}}|-|\epsilon_k|)+E_{corr}(+2),
\end{split}
\end{equation*}
be expressed as
\begin{equation*}
X_1^a(k)=\frac{\frac{1}{2}\Omega_k^{1/2}\Lambda_1(+2)}{2(|\epsilon_{g_{9/2}}|-|\epsilon_k|)+E_{corr}(+2)}.
\end{equation*}
Similarly (Fig. \ref{fig1E5}),
\begin{equation*}
Y_1^a(i)=\frac{\frac{1}{2}\Omega_i^{1/2}\Lambda_1(+2)}{2(\epsilon_F-\epsilon_i)+W_1(+2)},
\end{equation*}
can be written, with the help of the relation
\begin{equation*}
\begin{split}
2\times(\epsilon_F-\epsilon_i)+W_1(+2)&=2\times(\epsilon_F-\epsilon_i)-2\times(\epsilon_{g_{9/2}}-\epsilon_F)-E_{corr}(+2)\\
&=2\times(\epsilon_{p_{1/2}}-\epsilon_i)+2\times(\epsilon_{g_{9/2}}-\epsilon_{p_{1/2}})-E_{corr}(+2)\\
&=2\times(|\epsilon_i|-|\epsilon_{p_{1/2}}|)+2\times(|\epsilon_{p_{1/2}}|-|\epsilon_{g_{9/2}}|)-E_{corr}(+2),
\end{split}
\end{equation*}
as
  \begin{figure}
  \centerline{\includegraphics*[width=0.2\textwidth,angle=0]{nutshell/figs/addition_backward.pdf}}
  \caption{\idx{Pairing vibrations!two-nucleon transfer spectroscopic amplitudes} Same as Fig. \ref{fig1E3} but for the pair addition mode}\label{fig1E5}
  \end{figure}

\begin{equation}\label{eq3.5.8}
Y_1^a(i)=\frac{\frac{1}{2}\Omega_i^{1/2}\Lambda_1(+2)}{2(|\epsilon_i|-|\epsilon_{p_{1/2}}|)+2\Delta\epsilon_{sp}-E_{corr}(+2)}.
\end{equation}
Making use of $E_{corr}(+2)=1.5$ MeV (cf. Fig. \ref{fig1E1}) and 
\begin{equation}\label{eq3.5.9}
2\times\Delta\epsilon_{sp}=2\times(|\epsilon_{p_{1/2}}|-|\epsilon_{g_{9/2}}|)=6.82\,\text{MeV},
\end{equation}
one can write $2\Delta\epsilon_{sp}-E_{corr}(+2)=(6.82-1.5)$ MeV=5.32 MeV, leading to
\begin{equation}\label{eq3.5.10}
\left\{\begin{array}{c}
X_1^a(k)=\frac{\frac{1}{2}\Omega_k^{1/2}\Lambda(-2)}{2(|\epsilon_{g_{9/2}}|-|\epsilon_k|)+1.5\,\text{MeV}}, \\ 
Y_1^a(i)=-\frac{\frac{1}{2}\Omega_i^{1/2}\Lambda(+2)}{2(|\epsilon_i|-|\epsilon_{p_{1/2}}|)+5.32\,\text{MeV}}.
\end{array}\right.
\end{equation}
The corresponding numerical values are displayed in Tables \ref{tab1E4} and \ref{tab1E5}, while in Fig. \ref{fig1E6} we display a schematic summary of the graphical solution of the dispersion relations.

  \begin{figure}
  \centerline{\includegraphics*[width=\textwidth,angle=0]{nutshell/figs/fig1E6.pdf}}
  \caption[Quantal phase transition taking place as a function of the pairing coupling
  constant in a  closed shell nucleus.]{Schematic representation of the quantal phase transition taking place as a function of the pairing coupling
  constant in a  closed shell nucleus. (a)
  dispersion relation associated with the RPA diagonalization of the Hamiltonian
  $H = H_{sp} + H_p$ for the pair addition and pair removal modes. In the insets are
  shown the two-particle transfer processes exciting these modes, which testify to
  the fact that the associated zero point fluctuations (ZPF) which diverge at
  $G = G_{crit}$, blur the distinction between occupied and empty states typical of
  closed shell nuclei. (b) occupation number associated with the single-particle
  levels. For $G < G_{crit}$ there is a dynamical depopulation (population) of levels
  $ i(k)$ below (above) the Fermi energy. For $G > G_{crit}$,
  the deformation of the Fermi
  surface becomes stable, although with a non-vanishing dynamic component (see Fig. \ref{fig1.2}).}\label{fig1E6}
  \end{figure}
\begin{table}
\begin{tabular}{|c|c|c|c|c|}
\hline units &  &MeV  &MeV$^{-1}$  &    \\ 
\hline  $nlj$&$\Omega_k$  &$|\epsilon_{g_{9/2}}|-|\epsilon_k|$  &$C(k)=\frac{\frac{1}{2}\Omega_k^{1/2}}{2(|\epsilon_{g_{9/2}}|-|\epsilon_k|)+1.5\,\text{MeV}}$  $\vphantom{\prod}^{a)}$& $X_1^a(k)$  \\ 
\hline  $1g_{9/2}$& 5 &  0& 0.745 &  0.82\\ 
\hline $0i_{11/2}$ & 6 &  0.77&0.403  & 0.44  \\ 
\hline  $0j_{15/2}$& 8 &  1.41&  0.327&  0.36\\ 
\hline  $2d_{5/2}$& 3 & 1.56 & 0.187 & 0.21 \\ 
\hline  $3s_{1/2}$&  1& 2.03 &  0.090& 0.10\\ 
\hline  $1g_{7/2}$&  4& 2.47 & 0.155 &  0.17  \\ 
\hline  $2d_{3/2}$&  2& 2.51 & 0.108 &  0.12 \\ 
\hline 
\end{tabular}\caption{\idx{Pairing vibrations!two-nucleon transfer spectroscopic amplitudes} Forwards going RPA amplitudes associated with the pair addition mode of $^{208}$Pb (cf.  Table XVI \cite{Broglia:73}). a), in the TD approximation (amplitudes not shown) the particle-pair addition coupling constant is calculated from the normalization of the amplitudes $C(k)$. That is,  $\sum_{k}C^2(k)=0.903$ MeV$^{-2}$ and $\Lambda_1(+2)=(0.903)^{-1/2}$ MeV =1.052 MeV.}\label{tab1E4}
\end{table}
\begin{table}
\begin{tabular}{|c|c|c|c|c|}
\hline units &  &MeV  &MeV$^{-1}$  &    \\ 
\hline  $nlj$&$\Omega_i$  &$|\epsilon_i|-|\epsilon_{p_{1/2}}|$  &$D(i)=\frac{\frac{1}{2}\Omega_i^{1/2}}{2(|\epsilon_i|-|\epsilon_{p_{1/2}}|)+5.32\,\text{MeV}}$ $^{a)}$ & $Y_1^a(i)$  \\ 
\hline  $2p_{1/2}$& 1 &  0& 0.094 &  -0.1\\ 
\hline $1f_{5/2}$ & 3 &  0.57& 0.134  & -0.15 \\ 
\hline  $2p_{3/2}$& 2 &  0.90&  0.099&  -0.11\\ 
\hline  $0i_{13/2}$& 7 & 1.64 & 0.154 & -0.17 \\ 
\hline  $1f_{7/2}$&  4& 2.35 &  0.100& -0.11 \\ 
\hline  $0h_{9/2}$&  5& 3.47 & 0.091 &  -0.10  \\ 
\hline 
\end{tabular}\caption{\idx{Pairing vibrations!two-nucleon transfer spectroscopic amplitudes} Same as Table \ref{tab1E4} but for the backwards going amplitude. a) $\sum_i D^2(i)=0.079$ and $\Lambda_1^2(+2) (\sum_k C^2(k)-D^2(i))=\Lambda_1^2(+2) (0.903-0.079)$ MeV$^{-2}=0.824$MeV$^{-2}$; $\Lambda_1(+2)=(0.824)^{-1/2}$MeV, thus $\Lambda_1(+2)=1.102$ MeV.}\label{tab1E5}
\end{table}

Let us conclude this Section by noting that while the harmonic (RPA) description of the pair vibrational modes of $^{208}$Pb provides a fair picture of the two neutron transfer spectroscopic amplitudes, in keeping with the collective character of these (coherent) states, conspicuous anharmonicities in the multi--phonon spectrum have been observed and calculated\footnote{Cf. for example \cite{Flynn:72}, \cite{Lanford:73}, \cite{Bortignon:78}; \cite{Clark:06}.}. Within the framework of Fig. \ref{fig1D1}, we schematically emphasize in Fig. \ref{fig1_E8} the relative importance of dynamic and static pairing distortions, in comparison with the corresponding quantities in the case of quadrupole surface distortions in 3D--space\footnote{For details cf. \cite{Bes:77}, \cite{Broglia:68}, \cite{Bes:88},\cite{Barranco:87a} \cite{Shimizu:89}, \cite{Shimizu:13}, \cite{Vaquero:13} and references therein.}  These results underscore the major role pairing vibrations play in nuclei around closed shells, while those shown in Fig. \ref{fig1.2} emphasize their importance in gauge invariance restoration in systems far away from closed shells.
  \begin{figure}
  \centerline{\includegraphics*[width=10cm,angle=0]{nutshell/figs/fig1E8.pdf}}
  \caption[Dynamic and static pairing.]{Relative importance of dynamic and static pairing distortion ($\alpha_{dyn}$ and $\alpha_0$ respectively) associated with closed shell and open shell  nuclei, calculated in terms of a two level model, as compared with similar quantities for the case of quadrupole surface degrees of freedom ($\beta_2$--values). The parameter $x'$ (product of the effective pairing strength $G'=Z_\omega^2(v_p^{bare}+v_p^{ind})$ and of the effective density of levels at the Fermi energy $N'(0)=Z_\omega^{-1}N(0)=Z^{-1}_\omega(2\Omega/D)=2\Omega'/D=2\Omega/D';\Omega'=Z_\omega^{-1}\Omega,D'=Z_\omega D$), measures the relative importance of the single--particle gap $D'=Z_\omega D$ and of the pair correlation $G'\Omega$ (see Sect. \ref{S3.2};  see also \cite{Brink:05} App. H, Sect. H.4). It is of notice that in referring to $\alpha_0$, one has actually in mind $\alpha_0'$ ($\alpha_0=e^{-2i\phi}\alpha_0'$, $\alpha_0'$ being a real quantity). We nonetheless avoid the use of $\alpha_0'$ in keeping with the use of primes to refer to renormalized quantities, both here and in Sect. \ref{S3.2}. }\label{fig1_E8}
  \end{figure}
\section[Halo pair addition mode and pygmy]{Halo pair addition and pygmy dipole modes: a new mechanism to break gauge invariance}\label{App1AF}
Pairing correlations are  intimately connected with particle number violation and thus spontaneous breaking of gauge invariance, as testified by the order parameter\\ \mbox{$\langle BCS|P^{\dagger}|BCS\rangle=\alpha_0$}.  In the nuclear case,  dynamical breaking of gauge symmetry is similarly important to that associated with static distortions. The fact that the average single-particle potential acts as an external field  is one of the reasons for the existence of a critical value $G_c$ of the pairing strength $G$ to bind Cooper pairs in nuclei. Spatial quantization in finite systems at large and in nuclei in particular, is intimately connected with the paramount role the surface plays in these systems\footnote{See \cite{Bohr:75}; see also \cite{Broglia:02d} and references therein.}. Another consequence of this role is  the fact that in nuclei an important fraction ($\approx$50\% in the case of nuclei lying along the stability valley and even more, up to 80\%, for light halo nuclei) of Cooper pair binding is due to the exchange of collective vibrations between the  pair partners\footnote{\label{f53C3} See \cite{Barranco:99}, \cite{Terasaki:02a}, \cite{Brink:05}, \cite{Saperstein:12}, \cite{Avdenkov:12}, \cite{Lombardo:12};  \cite{Barranco:01}, \cite{Potel:10}, \cite{Pankratov:11}, \citet{Barranco:99}, \cite{Idini:15}; \cite{Barranco:05}.}, the rest being associated with the bare $NN$-interaction in the $^1S_0$ channel.  
%   \begin{figure}
% 	\centerline{\includegraphics*[width=0.7\textwidth,angle=0]{nutshell/figs/fig1F1.pdf}}\caption[Density-dependent pairing interaction.]{(top) Nuclear density $\rho$ in units of fm$^{-3}$, plotted as a function of the distance $r$ (in units of fm) from
% 		the center \textit{of a nucleus lying along the stability valley} (for comparison with a bound unstable nucleus lying at the neutron drip line see Fig. \ref{fig3.2.2}) . Saturation density correspond to $\approx$0.17 fm$^{-3}$, equivalent to $2.8\times 10^{14}$ g/cm$^3$. Because of the short range of
% 		the nuclear force, the nuclear density changes from 90\% of saturation density to 10\% within 0.65 fm, i.e. within the
% 		nuclear diffusivity. (bottom) Phase shift  parameter associated with the elastic scattering of two nucleons moving in states of time reversal, so
% 		called $^1S_0$ phase shift, in keeping with the fact that the system is in a singlet state of spin zero. Positive values of $\delta$ implies an attractive interaction, negative a repulsive one. For low relative velocities
% 		(kinetic energies $E_L$), i.e. for a situation similar to that found for nucleons moving around the nuclear surface where the density is low, the $^1S_0$ phase shift arizing from the exchange of mesons (e.g. pions, represented by an horizontal dotted  line) between nucleons (represented by upward pointing arrowed lines)
% 		is attractive. This mechanism provides about half of the glue to nucleons moving in time reversal states to form Cooper pairs. Cooper pair formation is
% 		further assisted by the exchange of collective surface vibrations (wavy curve in the scattering process) between the members of the
% 		pair.}\label{fig1F1}
% \end{figure}




The study of light exotic nuclei lying along the neutron drip line have revealed a novel aspect of the interplay between shell effects and induced pairing interaction. It has been found  that there are situations in which spatial quantization screens, essentially completely, the bare nucleon-nucleon paring interaction. This happens in the case in which the nuclear valence orbitals are $s,p$-states at threshold\footnote{Pairing anti-halo effect; \cite{Bennaceur:00} 
, \cite{Hamamoto:03}, \cite{Hamamoto:04}.}. An example of situations of this type is provided by some of the $N=7$ isotones, in particular $^{10}_3$Li and, to some extent, $^{11}_4$Be. Nuclei which display ``\textit{parity inversion}'' in the sequence of single-particle levels $1p_{1/2}$ and $2s_{1/2}$ as compared to the Mayer-Jensen prediction (Fig. \ref{fig1.0.3}).
 In what follows we discuss the (unbound) nucleus $^{10}$Li (see also Sect. \ref{S5.2.4}), in connection with the (bound) two-neutron halo system $^{11}$Li.
 
  The $N=7$ isotone $^{10}$Li displays a virtual $s_{1/2}$ and a resonant $p_{1/2}$ state\footnote{See  \cite{Barranco:20}, \cite{Moro:19} and refs. therein, as well as footnotes \ref{f5.28},   \ref{f5.29}  and \ref{f29C5} of Ch. \ref{C6}.}. 
The binding provided by a contact pairing interaction $V_\delta (|\mathbf{r}-\mathbf{r}'|)$ ($\delta$-force) to a pair of fermions moving in time--reversal states in a single $j$-shell\footnote{cf. e.g. Eq. (2.12) \cite{Brink:05}.} is given by the matrix element,
\begin{equation}
M_j=\langle j^2(0)|V_\delta|j^2(0)\rangle=-\frac{(2j+1)}{2} V_0 I(j)\approx -\frac{(2j+1)}{2}V_0\frac{3}{R^3}.
\end{equation} 
It is of notice that $G=V_0I(j)$ ($\approx 25/A$ MeV$\approx 2.3$ MeV ($A=11$)). 
The ratio of the above matrix element associated with the two halo neutrons of $^{11}$Li and with an hypothetical normal nucleus of mass $A=11$ is
\begin{equation}\label{eq3.6.2}
r=\frac{(M_j)_{halo}}{(M_j)_{core}}=\frac{2}{(2j+1)}\left(\frac{R_0}{R}\right)^3.
\end{equation}
The quantities $R_0=1.2 A^{1/3}$fm$=2.7$fm ($A=11$), and $R=\sqrt{\frac{5}{3}}\langle r^2\rangle^{1/2}_{^{11}\text{Li}}=\sqrt{\frac{5}{3}}(3.55\pm0.1)$ fm =$(4.58\pm 0.13)$ fm are the radius of a  nucleus of mass $A=11$ (systematics), and  the measured (mean square) radius\footnote{In other words, a contact interaction has an effective strength which scales with $R^{-3}\sim A^{-1}$. In the case of halo nuclei, the radius is larger as a rule than that expected from systematics. In particular, in the case of $^{11}$Li, the value of the radius corresponds, assuming $r_0=1.2$ fm, to an effective mass number $A_{eff}\approx56$.} of $^{11}$Li, respectively\footnote{See footnote \ref{f64C3} of this Chapter.} . The quantity $j$ is the effective angular momentum of a single $j$-shell which can accommodate 8 neutrons $(j\sim k_F R_0\approx 1.36 \text{ fm}^{-1}\times2.7\text{ fm}\approx3.7)$. One thus obtains
\begin{align}\label{eq2.6.3}
r=0.048.
\end{align}
 Consequently, the bare $NN$-nucleon pairing interaction is expected to become strongly screened, the resulting effective $G$-value ($A\approx11$) 
\begin{equation}\label{eq1C2AppF}
G_{scr}=r\times G=0.048\times 25 \text{MeV}/A\approx 1 \text{MeV}/A\approx 0.1\,\text{MeV},
\end{equation}
becoming subcritical and thus unable to bind the halo Cooper pair ($2\tilde \epsilon_{s_{1/2}}=0.3$ MeV, see Fig. \ref{fig1.9.1}) to the $^9$Li core.




 Further insight into this question can be shed making use of the multipole expansion of a general interaction
\begin{equation}
v(|\mathbf{r}_1-\mathbf r_2|)=\sum_{\lambda}V_{\lambda}(r_1,r_2)P_\lambda(\cos\theta_{12}).
\end{equation}
Because the function $P_\lambda$ drops from its maximum at $\theta_{12}=0$ in an angular distance $1/\lambda$, particles 1 and 2 interact through the component $\lambda$ of the force, only if $r_{12}=|\mathbf{r}_1-\mathbf{r}_2|<R/\lambda$, where $R$ is the mean value of the radii $\mathbf{r}_1$ and $\mathbf{r}_2$. Thus, as $\lambda$ increases, the effective force range decreases. For a force of range much greater than the nuclear size, only the lowest $\lambda$ (long wavelength) terms are important. At the other extreme, a $\delta$--function force has coefficients $V_\lambda(r_1,r_2)\left(=\tfrac{(2\lambda+1)}{4\pi r_1^2}\delta(r_1-r_2)\right)$ that increase with $\lambda$. In the case of $^{11}$Li(gs) one is thus confronted with the need to accept  a long range, low $\lambda$ pairing interaction, as responsible for the binding of the dineutron, halo Cooper pair to the $^9$Li core. This is equivalent to saying, an induced pairing interaction arising from the exchange of vibrations with low $\lambda$-value.
\subsection{Cooper pair binding: a novel embodiment of the Axel-Brink hypothesis.}\label{sect1F1}
In what follows we discuss a possible novel test of the Axel-Brink hypothesis\footnote{The color of an object can be determined in two ways: by illuminating it with white light and see which wavelength it absorbs, or by heating it up and see the  wavelength it emits. In both cases one is talking about dipole radiation. To describe the de-excitation process of hot nuclei requires the knowledge of the photon interactions with excited states. The common assumption, known as the Axel-Brink hypothesis, has been that each excited state of a nucleus carries a giant dipole resonance (GDR) on top of it, and that the properties of such resonances are unaffected by any excitation of the nucleus (\cite{Brink:55}, \cite{Lynn:68} pag. 321, \cite{Axel:62}; cf. also \cite{Bertsch:86}, \cite{Bortignon:98}, \cite{Martin:17} and refs. therein).}. Within the $s,p$ subspace, the most natural low multipolarity, long-wavelength vibration is the dipole mode. From systematics, the centroid of these vibrations is found at $\hbar \omega_{GDR}\approx 100$ MeV$/R$, $R$ being the nuclear radius\footnote{See \cite{Bohr:75} \cite{Bortignon:98} and \cite{Bertsch:05} and refs. therein.}. Thus, in the case of $^{11}$Li, one expects the centroid of the \textit{Giant Dipole Resonance} (GDR) carrying $\approx$100\% of the energy weighted sum rule (EWSR) at $\hbar \omega_{GDR}\approx 100$ MeV$/4.6\approx 22$ MeV. Such a high frequency mode can hardly be expected to give rise to anything, but polarization effects. On the other hand, there exists experimental evidence which testifies to the presence in $^{11}$Li, of a well defined dipole resonance with centroid at $\lessapprox1$ MeV and carrying $\approx 8$\% of the EWSR\footnote{See footnote \ref{f119C2} Ch. \ref{intro}.}. The existence of this \textit{``pygmy (dipole) resonance''} (PDR) \idx{Halo (single Cooper pair) nucleus $^{11}$Li!soft $E1$-mode (pygmy dipole resonance)}
 which can be viewed as a  consequence of the existence of a low-lying particle-hole state associated with the transition $s_{1/2}\rightarrow p_{1/2}$ testifies, arguably, to the \textit{coexistence}\footnote{Within this context one can mention similar situations concerning the coexistence of spherical and quadrupole deformed states (cf. e.g. \cite{Wimmer:10}, \cite{Federman:65}, \cite{Federman:66}, \cite{Donau:67} and refs. therein; cf. also \cite{Bohr:63}), typically of nuclei with $N\approx Z$. Surface quadrupole inhomogeneous damping has modest consequences in bringing dipole strength at low energies as compared with radial (isotropic) deformations in $^{11}$Li. This is understood in terms of the plasticity   displayed by the atomic nucleus regarding quadrupole deformations (low-lying collective $2^+$ surface vibrations, fission, exotic decay (cf. \cite{Barranco:88}, \cite{Barranco:89,Bertsch:88b}, \cite{Bertsch:87})), and of the little tolerance to both compressibility and rarification displayed by the same system and connected with saturation properties (see also \cite{Broglia:19} and \cite{Broglia:19b}).} 
of two states with rather different radii in the ground state. One, closely connected with the $^{9}$Li core, ($\approx 2.5$ fm), the second with the diffuse halo ($\approx 4.6$ fm), namely displaying a large radial (isotropic) deformation (neutron skin) \idx{Halo (single Cooper pair) nucleus $^{11}$Li!isotropic radial deformation}
, and thus able to induce a conspicuous (radial, isotropic)\footnote{See Sect. \ref{S4.10.1}, in particular footnote \ref{f155C4} of Ch. \ref{chapter2}.} inhomogeneous damping to the dipole mode. 


The importance of this mechanism is underscored by the fact that  $^{11}$Li, displaying a neutron excess $(N-Z)/A\approx0.45$ as compared to the value of 0.21 in the case of $^{208}$Pb, is able to bring down by tens of MeV a consistent fraction ($\approx 8$\%) of the TRK-sum rule. A consequence of the fact that the nucleus is most sensitive to changes in density (saturation phenomena). In the case of the halo of $^{11}$Li we are confronted with nuclear structure phenomena in a medium displaying a density of $\approx 4$\% of saturation density (Sect. \ref{App3C}, see also Eq. (\ref{eq3.2.24})). 

Before proceeding, let us estimate the overlap $\mathcal{O}$ between the two ``ground states''. Making use of a schematic expression for the single-particle radial wavefunctions\footnote{\cite{Bohr:69}.}
\begin{equation}
\mathcal{R}=\sqrt{3/R_0^3}\;\Theta(r-R_0),
\end{equation}
where 
\begin{equation*}
\Theta=1 \quad (r\leq R_0);\quad 0 \quad (r>R_0),
\end{equation*}
leading to,
\begin{equation}
\int_0^{\infty}dr r^2 \mathcal{R}^2(r)=\frac{3}{R_0^3}\int_0^{R_0}dr^3/3=1,
\end{equation}
one can work out the overlap $\mathcal{O}$ between the two halo neutrons and the core nucleons. That is, 
\begin{equation}\label{eq2.6.4}
\begin{split}
\mathcal{O}&=|\langle\mathcal{R}_{halo}|\mathcal{R}_{core}\rangle|^2=\left(\sqrt{\frac{3}{R_0^3}}\sqrt{\frac{3}{R^3}}\int_0^{\infty}dr\,r^2\Theta(r-R)\Theta(r-R_0)\right)^2\\
&=\left(\sqrt{\frac{3}{R_0^3}}\sqrt{\frac{3}{R^3}}\int_0^{R_0}dr^3/3\,\right)^2=(R_0/R)^3=0.20,
\end{split}
\end{equation}
where use has been made of $\Theta(r-R)\Theta(r-R_0)=\Theta(r-R_0), R_0=1.2A^{1/3} \text{ fm}=2.7 \text{ fm} (A=11)$ and $R=(4.58\pm 0.013)$ fm.
Because of the small value of this overlap, one can posit that the $E_x\lesssim1$ MeV ($\Gamma \approx 0.5$ MeV) soft $E1$-mode of $^{11}$Li is a \textit{bona fide} dipole  resonance based on an exotic, unusually extended $\ket{0_\nu}$ state of radial dimensions equivalent, according to systematics, to  a system of effective $A$-mass number about 5 times that of the actual system $(A\approx (4.6/1.2)^3\approx 60)$ .
 Thus consistent with the connotation of PDR.

\textit{It is of notice that the small values of $r$ and of $\mathcal{O}$ have essentially the same origin}. On the other hand, they have apparently, rather different physical consequences. In fact, the first makes the bare pairing interaction strength $G$ subcritical, while the second one screens the repulsive symmetry potential $V_1(\approx +25 $ MeV)\footnote{See e.g. \cite{Bortignon:98} Eq. (3.48) and refs. therein.}, that is, the energy price one has to pay to separate protons from neutrons. This effect allows for a consistent fraction of the dipole Thomas--Reiche--Kuhn sum rule, that is of the $J^{\pi}=1^-$ energy weighted sum rule (EWSR), to come low in energy ($p_{1/2}-s_{1/2}$ transition) from the value $E_{GDR}\approx(100/R)$ MeV and, acting as an intermediate boson between the two halo neutrons, glue them to the $^{9}$Li core. \textit{Summing up, the halo anti-pairing effect $G_{scr}=r\times G\ll G<G_{crit}$ also triggers ($\mathcal{O}V_1\ll V_1$) the virtual presence of a ``gas'' of dipole (pygmy) bosons which, exchanged between the two halo neutrons (cf. Fig. \ref{pigmy}), overcompensates the reduction of the bare pairing interaction, leading to the binding of the halo Cooper \idx{Halo (single Cooper pair) nucleus $^{11}$Li!symbiotic (bootstrap) mechanism to break gauge invariance}
	 pair to the core (anti-(halo anti-pairing effect)). It can thus be stated that the halo of $^{11}$Li and the pygmy dipole resonance\footnote{In which the two halo neutrons oscillate out of phase with respect to nucleons of the core $^9$Li moving in phase (see \cite{Broglia:19}).} built on top of it constitute a pair of symbiotic states (see also Chapter \ref{C8}).}

Let us further elaborate on these issues. Making use of the relation $\langle r^2\rangle^{1/2}\approx (3/5)^{1/2}R$ between mean square radius and the radius, one may write
\begin{equation}\label{eq2.6.9}
\langle r^2\rangle_{^{11}\text{Li}}\approx \frac{3}{5}R_{eff}^2(^{11}\text{Li}).
\end{equation}
 with
\begin{equation}\label{eq2.6.10}
R_{eff}^2(^{11}\text{Li})=\left(\frac{9}{11}R_0^2(^9\text{Li})+\frac{2}{11}\left(\frac{\xi}{2}\right)^2\right),
\end{equation}
where
\begin{equation}
R_0(^9\text{Li})=2.5 \text{fm},
\end{equation}
is the $^9$Li radius ($R_0=r_0A^{1/3}, r_0=1.2$fm), while $\xi$ is the correlation length of the Cooper pair neutron halo. An estimate of this quantity is provided by the relation\footnote{The associated generalized quantality parameter being $q_\xi=\frac{\hbar^2}{2m\xi^2}\frac{1}{|E_{corr}|}\approx0.1$ (see also App. \ref{App6H}). \idx{Halo (single Cooper pair) nucleus $^{11}$Li!generalized quantality parameter}}\idx{Halo (single Cooper pair) nucleus $^{11}$Li!correlation length}
 \begin{equation}\label{eq3.6.12}
\xi=\frac{\hbar v_F}{\pi |E_{corr}|}\approx 20 \, \text{fm}, 
 \end{equation}
in keeping with the fact that in $^{11}$Li, $(v_F/c)\approx 0.16$ and $E_{corr}\approx-0.5$ MeV (see App. \ref{App6H}). Consequently, 
\begin{align}\label{eq2.F.5}
R_{eff}\,(^{11}\text{Li})\approx 4.8 \;\text{fm},
\end{align} 
and  $\langle r^2\rangle_{^{11}\text{Li}}^{1/2}\approx 3.7$ fm, to be compared with the experimental value\footnote{\label{f64C3} \cite{Kobayashi:89}.} $\langle r^2\rangle_{^{11}\text{Li}}^{1/2}= 3.55\pm0.1$fm. It is of notice that this experimental value implies  the radius $R$($^{11}$Li)$=\sqrt{5/3\langle r^2\rangle_{^{11}\text{Li}}}=4.58\pm 0.13$ fm.


We now proceed to the calculation of the centroid of the dipole pygmy resonance of $^{11}$Li in  RPA making use of the separable interaction\footnote{For details we refer to \cite{Bortignon:98} Sect. 3.2.3.}
 \begin{equation}\label{eq2.F.6}
H_D=-\kappa_1\vec D\cdot\vec D,
 \end{equation}
where $\vec D=\vec r$ and
 \begin{equation}
\kappa_1=\frac{-5V_1}{AR^2}.
 \end{equation}
  $V_1=25$ MeV being the symmetry potential energy.
The  RPA dispersion relation is\footnote{See (3.30) p.55 of \cite{Bortignon:98}.}
\begin{equation}\label{eq3.6.16}
W(E)=\sum_{k,i}\frac{2(\epsilon_k-\epsilon_i)|\langle \tilde i|F|k\rangle|^2}{(\epsilon_k-\epsilon_i)^2-E^2}=\frac{1}{\kappa_1}.
\end{equation}
 Making use of this relation as well as (see Fig. \ref{fig1.9.1}) $\tilde\epsilon_{p_{1/2}}-\tilde\epsilon_{s_{1/2}}\approx 0.45 $MeV for the value of 
 $\epsilon_{\nu_k}-\epsilon_{\nu_i}$, and that the EWSR associated with the $^{11}$Li pygmy resonance is $\approx 8$\% of the total Thomas-Reiche-Kuhn sum rule\footnote{\label{f73C3} The Thomas-Reiche-Kuhn sum rule (\cite{Bohr:75,Bortignon:98}) TRK=$\frac{9}{4\pi} \frac{\hbar^2e^2}{2m} \frac{NZ}{A}=14.8 \frac{NZ}{A} e^2\text{ fm}^2 \text{ MeV}$ has a value of 32.3 $e^2$ MeV fm$^2$ for $^{11}_3$Li$_8$. Assuming a systematic behaviour, the centroid of the giant dipole resonance is expected at $\hbar\omega_D\approx80/A^{1/3}$ MeV $\approx 36$ MeV, leading to the \textit{ratio} (\equiv(32.3 $e^2$ MeV fm$^2)/(36\text{ MeV})\approx 0.9 e^2$ fm$^2$). The $E1$--single--particle (Weisskopf) unit can be written as (\cite{Bohr:69} p. 389, Eq. (3C-38)) $B_W(E1)\approx ((1.2)^2/4\pi) (3/4)^2 A^{2/3} (e_{E1})^2$ fm$^2=0.32 (e_{E1})^2$ fm$^2$, $e_{E1}$ being the effective dipole charge equal to $(N/A)e=0.73 e$ for the protons of $^{11}$Li, and $-(Z/A)e=0.27$ for the neutrons. Making use of the average value one can write $\bar B_W(E1)\approx0.1 e^2$ fm$^2$. Thus 8\%$\times$ \textit{ratio} /$\bar B_W(E1)\approx 0.072/0.1\approx 0.7$. In other words, about 1 single--particle unit is associated with the eventual $\gamma$--decay of the PDR of $^{11}$Li.}
 \begin{equation}
\sum_n |\langle0|F|n\rangle|^2(E_n-E_0)=\frac{\hbar^2}{2m}\int d\mathbf r |\vec\nabla F|^2 \rho(r),
 \end{equation}
 which, for $F=r$ has the value\footnote{cf. \cite{Bertsch:05} pag. 53.} $\hbar^2 A/2m$, one can write the numerator of Eq. (\ref{eq3.6.16}) as, 
\begin{equation}
2\times 0.08\times \frac{\hbar^2A}{2m}=\frac{1}{\kappa_1}[(0.45\text{MeV})^2-(\hbar \omega_{pygmy})^2],
\end{equation}
 and thus
\begin{equation}
 (\hbar\omega_{pygmy})^2=(0.45 \text{MeV})^2-2\times 0.08\times\frac{\hbar^2A}{2m}\kappa_1,
\end{equation}
 where\footnote{see \cite{Bortignon:98}.}
\begin{equation}\label{eq2.6.14}
\kappa_1=-\frac{5V_1}{A(\xi/2)^2}\left(\frac{2}{11}\right)=-\frac{125\text{MeV}}{A 100 \text{ fm}^2}\left(\frac{2}{11}\right)\approx -0.021\text{ fm}^{-2}\text{ MeV}.
\end{equation}
The ratio in parenthesis reflects the fact that only 2 out of 11 nucleons, slosh back and forth in an extended configuration with little overlap with the other nucleons. The quantity,
\begin{equation}\label{eq2.6.21}
\kappa_1^0=-\frac{5V_1}{AR^2_{eff}(^{11}\text{Li})}\approx 0.49\; \text{MeV fm}^{-2},
\end{equation}
is the ``standard'' self consistent dipole strength\footnote{cf. \cite{Bohr:75}.}. The screening factor $s=(\kappa_1/\kappa_1^0)=(R^2_{eff}\times(2/11))/(\xi/2)^2=0.043$ is very close in magnitude to the ratio $r(=0.048$, Eq. (\ref{eq2.6.3})) and has a similar physical origin. It is of notice that $(V_1)_{scr}=sV_1\approx 1$ MeV. Making use of (\ref{eq2.6.14})  one obtains,
\begin{equation}
-2\times 0.08\frac{\hbar^2A}{2m}\kappa_1\approx 0.74 \text{MeV}^2\approx (0.86\text{MeV})^2.
\end{equation} 
 Consequently
\begin{equation}\label{eq3.6.23}
\hbar \omega_{pygmy}=\sqrt{(0.45)^2+(0.86)^2}\text{MeV}\approx 1\, \text{MeV},
\end{equation}  
 in overall agreement with the experimental findings\footnote{See footnote \ref{f119C2} Ch. \ref{intro}.}. It is of notice that the centroid of the pygmy dipole resonance \idx{Halo (single Cooper pair) nucleus $^{11}$Li!soft $E1$-mode (pygmy dipole resonance)}
  (microscopically) calculated in  QRPA with the help of a separable dipole interaction is\footnote{See \cite{Barranco:01}; in particular  Fig. 2a where the doubled peaked ($\approx0.6$ MeV and $\approx1.6$ MeV) $dB(E1)/dE$ strength function is displayed.} $\approx (0.6\,\text{MeV}+ 1.6\, \text{MeV})/2\approx 1.1\, \text{MeV}$.
 \begin{figure}
 \centerline{\includegraphics*[width=0.5\textwidth,angle=0]{nutshell/figs/pigmy.pdf}}
 \caption[Exchange of pygmy resonance.]{Diagrammatic representation of the exchange of a collective $1^-$ pygmy resonance between pairs of nucleons moving in the time-reversal configurations $s_{1/2}^2(0)$ and $p_{1/2}^2(0)$. It is of notice that both these configurations can act as initial states  the figure showing only one of the two possibilities. Consequently, the energy denominator to be used in the simple estimate (\ref{eq2.F.10}) is the average value $DEN=(DEN_1+DEN_2)/2=-\hbar\omega_{pygmy}$ where $DEN_1=\Delta \epsilon-\hbar\omega_{pygmy}$ and $DEN_2=-\Delta\epsilon-\hbar\omega_{pygmy}$, while $\Delta\epsilon=\epsilon_{s_{1/2}}-\epsilon_{p_{1/2}}$.}\label{pigmy}
 \end{figure}

 Let us now estimate the binding energy which the exchange of the pygmy resonance between the two neutrons of the halo Cooper pair  of $^{11}$Li can provide.
The associated particle--vibration coupling\footnote{cf. e.g. \cite{Brink:05} Eq. (8.42) p.189.} is $\Lambda= \left(\partial W(E)/\partial E|_{\hbar\omega_{pygmy}}\right)^{-1/2}$. Note the use in what follows of a dimensionless dipole single--particle field $F'=F/R_{eff}(^{11}\text{Li})$). This is in keeping with the fact that one wants to obtain a quantity with energy dimensions ($[\Lambda]=$ MeV), and that $\kappa_1$ has been introduced through the Hamiltonian $H_D$ with the self consistent value normalized in terms of $R_{eff}^2(^{11}$Li) (Eq. (\ref{eq2.6.21})). 
 One then obtains
\begin{equation}
\begin{split}
\Lambda&=\left\{2\hbar \omega_{pygmy}\frac{2\times 0.08(\frac{\hbar^2A}{2m})/R_{eff}^2}{\left[(\tilde\epsilon_{p_{1/2}}-\tilde\epsilon_{s_{1/2}})^2-(\hbar\omega_{pygmy})^2\right]^2}\right\}^{-1/2},\\
&=\left\{2\text{MeV}\frac{0.16(\hbar^2A/2m)(1/4.8)^2\,\text{fm}^2}{\left[(0.45\text{ MeV})^2-(1\text{MeV})^2\right]^2}\right\}^{-1/2},\\
&=\left(\frac{3\,\text{MeV}^2}{(0.8)^2\,\text{MeV}^4}\right)^{-1/2}\approx 0.5\,\text{MeV}.
\end{split}
\end{equation}   
The value of the induced interaction matrix elements is then given by (Fig. \ref{pigmy}),
 \begin{equation}\label{eq2.F.10}
M_{ind}=\frac{2\Lambda^2}{DEN}\approx-\frac{2\Lambda^2}{\hbar\omega_{pygmy}}\approx-0.5\,\text{MeV},
 \end{equation}
 the factor of two arising from the two time ordering contributions. The resulting correlation energy is thus $E_{corr}=2\tilde\epsilon_{s_{1/2}}-G_{scr}+M_{ind}=(0.3-0.1-0.5)\text{ MeV}\approx- 0.3$ MeV, in overall agreement with the experimental\footnote{\cite{Bachelet:08}, \cite{Smith:08}.} findings ($-0.380$ MeV). It is of notice that in this estimate the (subcritical) effect of the screened bare pairing interaction has also been used (see Eq. (\ref{eq1C2AppF}))\footnote{That new physics, namely a novel mechanism to (dynamically) violate gauge invariance finds, to express itself, a scenario of a barely bound Cooper pair at the drip line (half life 8.75 ms), seems to confirm a recurrent expectation. That truly new complex phenomena appear at the border between rigid order and randomness (see \cite{DeGennes:94}).}, as well as  the theoretical value\footnote{See footnotes \ref{f5.28} and \ref{f5.29} of Ch. \ref{C6}.} $\tilde\epsilon_{1/2}=0.15$ MeV (Fig. \ref{fig1.9.1}).
 
 
 
 This schematic model\footnote{See also \cite{Broglia:19b}.} has been implemented with microscopic detail\footnote{cf. \cite{Barranco:01}; see also \cite{Potel:10}.} within the framework of a field theoretical description of the interweaving of collective vibrations and single--particle motion, and is also discussed  within the context of single-particle (Chapter \ref{C6}) and two-particle (Chapter \ref{C8}) transfer processes. Here we provide a summary of the theoretical findings. 
 
 
 
 
 In Fig. \ref{fig1F3} \textbf{(I)}, the lowest single-particle neutron virtual and resonant states of  $^{10}$Li are indicated\footnote{\cite{Barranco:01}. For a more detailed (NFT) study of $^{10}$Li see \cite{Barranco:20} (see also Sect. \ref{S5.2.4}). See also \cite{Moro:19}.}. The 
  position of the levels $s_{1/2}$ and $p_{1/2}$ determined making use
 of mean-field theory is shown (left hatched area and thin horizontal
 line, respectively). The coupling of a single--neutron (upward
 pointing arrowed line) to a vibration (wavy line) calculated
 making use of NFT Feynman diagrams 
 (schematically depicted also in terms of either solid dots (neutron)
 or open circles (neutron hole) moving in a single-particle
 level around or in the $^9$Li core (gray circle)), leads to conspicuous
 shifts in the energy centroid of the $s_{1/2}$ and $p_{1/2}$ resonances
 (shown by thick horizontal lines to the right) and eventually to
 an inversion in their sequence. In Fig. \ref{fig1F3} \textbf{(II)} the  processes contributing to  binding the  halo neutron system $^{11}$Li are displayed. One starts with the dressed mean  field
 picture in which two neutrons (solid dots) coupled to angular momentum and parity $0^+$ move in
 time-reversal states around the core $^{9}$Li (hatched area) in the
 $s_{1/2}$ virtual state leading to an unbound $s^2
 _{1/2}(0)$ configuration.  The associated spatial structure of the uncorrelated pair is shown in \textbf{a)}. The exchange
 of vibrations between the two neutrons displayed in the upper
 part of the figure leads to an induced pairing interaction
 that, added to the subcritical bare nucleon-nucleon Argonne potential\footnote{\cite{Wiringa:84}.}  (see boxed inset), leads to a bound state $|\tilde 0\rangle$. The corresponding   wavefunction is  displayed in \textbf{b)}, together with the spatial structure of the resulting Cooper pair. It is of notice that a large fraction of the induced interaction arises from the exchange of the pygmy dipole resonance between the two halo neutrons.  Within this scenario one can posit that the $^{11}$Li PDR  can hardly be viewed but in symbiosis with the $^{9}$Li halo neutron pair addition mode and vice versa. Furthermore, that the two halo neutrons of $^{11}$Li, provide the first example of a Van der Waals Cooper pair, the first of its type in nuclei (Fig. \ref{fig2.A.1}). For further details see Chapter \ref{C8} as well as\footnote{\cite{Barranco:01}.}.
  \begin{figure}
  \centerline{\includegraphics*[width=\textwidth,angle=0]{nutshell/figs/fig1F3.pdf}}
  \caption[NFT processes renormalizing the single-particle motion in $^{10}$Li.]{(Color online) In (I) and (II) the NFT processes renormalizing the single-particle motion ($^{10}$Li) and leading to the effective interaction, sum of the bare (horizontal dotted line, inset) and induced (wavy line) interactions which bind the two-neutron halo to the core of $^{9}$Li  thus leading to the $|^{11}$Li$\rangle$ ground state are displayed. It is of notice that the odd $p_{3/2}(\pi)$ proton, considered as a spectator, is not shown. In a) and b)  the  spatial structure of the pure $|s_{1/2}^2(0)\rangle$ configuration and that of the two-neutron halo $|\tilde 0\rangle$ Cooper pair is displayed. The modulus squared of the wave function $|\Psi_0(\mathbf{r}_1,\mathbf{r}_2)|^2=|\langle \mathbf{r}_1,\mathbf{r}_1|0^+\rangle|^2$ describing the motion of the two halo neutrons around the $^9$Li core 
  is shown as a function of the cartesian coordinates of particle 2, for fixed values of the
  position of particle 1 ($r_1 = 5$ fm) represented  by a solid dot, while the core radius ($^9$Li), is shown as a solid
  circle. The numbers appearing on the $z$-axis of the three-dimensional plots displayed on the left side of the figure are in units of fm$^{-2}$.}\label{fig1F3}
  \end{figure}
\FloatBarrier 
 
 

 \section{Nuclear van der Waals Cooper pair}\label{C2SG2}
 \idx{Cooper pair!van der Waals}
 The atomic van der Waals (dispersive, retarded) interaction which, like gravitation,  acts between all atoms and molecules, also non-polar, can be written for two systems placed at a distance $R$ as (see App. \ref{C2AppD}), 
 \begin{align}\label{eq1C2AppG}
 \Delta E=-\frac{6\times e^2 \times a_0^5}{R^6}=-\frac{16}{(R/a_0)^6}\left(\frac{e^2}{2a_0}\times 0.75\right),
 \end{align}
 where $a_0$ is the Bohr radius. One recognizes in the term in parenthesis the transition energy $2P\to1S$ of the hydrogen atom. A possible qualitative nuclear parallel (see Fig. \ref{fig2.A.1})  can be established making the  correspondences  of the term in parenthesis with $\tilde\epsilon_{p_{1/2}}-\tilde\epsilon_{s_{1/2}}\approx0.45$ MeV, $R$ with $R_{eff}\approx4.6$ fm and $a_0$ with $R_0(^{11}\text{Li})\approx3$ fm, leading to $\Delta E\approx-0.6$ MeV (=$M_{ind}$).
  Thus,
 \begin{align*}
E_{corr}=2\tilde{\epsilon}_{s_{1/2}}-G_{scr}+\Delta E=0.3\,\text{MeV}-0.1\,\text{MeV}-0.6\,\text{MeV}\approx -0.4\text{ MeV},
 \end{align*} 
not inconsistent with the experimental finding ($-0.380$ MeV).

   \begin{figure}
   \centerline{\includegraphics*[width=17cm,angle=0]{nutshell/figs/VdW.pdf}}\caption[NFT Feynman diagrams describing the binding of the halo Cooper pair through pygmy.]{NFT Feynman diagrams describing the binding of the halo Cooper pair through pygmy. That is, producing the symbiotic mode involving the pair addition mode and the PDR. The single-particle states $s_{1/2}$ and $p_{1/2}$ are labeled in \textbf{(a)} $s$ and $p$ for simplicity. The different particle-vibration coupling vertices (either with the quadrupole ($2^+$) or with the pygmy ($1^-$) modes drawn as  wavy lines) are denoted by a solid dot, and numbered in increasing time sequence so as to show that diagram \textbf{(b)} emerges from \textbf{(a)} through time ordering. The motion of the neutrons are drawn in terms of continuous solid curves. In keeping with the fact that the occupation of the single-particle states is neither 1 nor 0 we treat, for simplicity, these states  as quasiparticle states. Thus no arrow is drawn on them. Diagram (\textbf{a}) emphasizes the self-energy renormalization of the state $s_{1/2}$ lying in the continuum and which   through its clothing with the quadrupole mode is brought down becoming a virtual ($\widetilde\epsilon_{s_{1/2}}\approx0.2$ MeV) state (see (I) and (II)), while (III) contributes to the induced pairing interaction through pygmy (see also Fig. \ref{pigmy}). The ``eagle'' diagram (\textbf{b}) contains ((IV) and (V)) Pauli corrections which push the bound state $p_{1/2}$ into a resonant state in the continuum ($\widetilde\epsilon_{p_{1/2}}\approx0.5$ MeV). In other words, processes (I), (II), (III), (IV) and (V) are at the basis of parity inversion, and of the appearance of the new magic number $N=6$. Processes (VI) and (VII) are associated with the pygmy ZPF, while (VIII) contributes to the induced pairing interaction through pygmy (van der Waals-like process) \idx{Cooper pair!van der Waals}
   	.}\label{fig2.A.1}
   \end{figure}

 \section{Renormalized coupling constants $^{11}$Li: resum\'e}\label{C2SF2}
The fact that the screening factors   $r$ and $s$ (see Eqs. (\ref{eq2.6.3}),  (\ref{eq3.6.2}); also (\ref{eq2.6.21}) and subsequent paragraph), essentially coincide within numerical approximations is in keeping with the fact that both quantities are closely related to the overlap\footnote{One can equivalently use $\left(R_0/R_{eff}\right)^3\approx\left( 2.7\text{ fm}/(4.8\text{ fm})\right)^3\approx0.18$.}
  \begin{align}
\mathcal{O}\approx\left(\frac{R_0}{R}\right)^3\approx\left(\frac{2.7\,\text{fm}}{4.6\,\text{fm}}\right)^3\approx0.2,
  \end{align}
quantity that has two main effects concerning the mechanism which is at the basis of much of the nuclear structure of the halo exotic nucleus $^{11}$Li: \textbf{1)} it makes subcritical the screened bare $NN$-pairing interaction $G_{scr}=r\times G<G_c$ ($G_{scr}=1\,\text{MeV}/A$); \textbf{2)} it screens the symmetry potential drastically, reducing the energy price one has to pay to separate protons from delocalized neutrons, permitting a consistent chunk ($\approx 8$\%) of the TRK sum rule\footnote{See footnote \ref{f119C2} Ch. \ref{intro}.} to  become essentially degenerate with the  ground state ($(V_1)_{scr}=1 \,\text{MeV}$). In other words,  allowing for the first nuclear example of a van der Waals Cooper pair and a novel mechanism to break dynamically gauge invariance. Namely, dipole-dipole fluctuating fields associated with the exchange of the pygmy dipole resonance between the halo neutrons of $^{11}$Li. As a result, we are in presence of a new, (composite) elementary mode of nuclear excitation: a halo pair addition mode carrying on top of it, a low-lying collective soft $E1$-vibration. This symbiotic mode can be studied through two-particle transfer reactions, eventually in coincidence with $\gamma$-decay. In particular, making use of the reaction $^{9}$Li$(t,p)^{11}$Li$(f)$ for\\
\vspace{0.1cm}
%\centerline{$^{9}$Li$(t,p)^{11}$Li$(f)$,}
\vspace{0.2cm}
\centerline{$|f\rangle$: ground state ($L=0$) and pygmy ($L=1$; $E_x\lesssim$ 1 MeV).}
Similar, but in this case in connection with the reaction $^{10}$Be$(t,p)^{12}$Be$(f)$ for\\
%\vspace{0.2cm}
% \centerline{$^{7}$Li$(t,p)^{9}$Li$(f)$,}
%\vspace{0.3cm}
%\centerline{$^{10}$Be$(t,p)^{12}$Be$(f)$,}
%\vspace{0.2cm}
\centerline{$|f\rangle$: first excited $0^+$ state $(E_x=2.24$ MeV),}
\vspace{0.2cm}
as well as pygmy $(L=1)$ on top of it, arguably the $1^-$ state at $E_x=2.70$ MeV being part of this mode\footnote{\cite{Iwasaki:00}.}.
  \begin{subappendices}
\section[Lindemann criterion and  quantality parameter]{Lindemann criterion and connection with quantality parameter}\label{C2AppC}
The original Lindemann criterion\footnote{\cite{Lindemann:10}.} compares the atomic fluctuation amplitude $\langle\Delta r^2\rangle^{1/2}$ with the lattice constant $a$ of a crystal. If this ratio, which is defined as the disorder parameter $\Delta_L$, reaches a certain value, fluctuations cannot increase without damaging or destroying the crystal lattice. The results of experiments and simulations show that the critical value of $\Delta_L$ for simple solids is in the range of 0.10 to 0.15, relatively independent of the type of substance, the nature of the interaction potential, and the crystal structure\footnote{\cite{Bilgram:87,Lowen:94,Stillinger:95}.}. Applications of this criterion to an inhomogeneous finite system like a protein in its native state (aperiodic crystal)\footnote{\cite{Schrodinger:44}.}, requires evaluation of the generalized Lindemann parameter\footnote{\cite{Stillinger:90}.}
\begin{align}
\Delta_L=\frac{\sqrt{\sum_i\langle \Delta r_i^2\rangle/N}}{a'},
\end{align}  
where $N$ is the number of atoms and $a'$ the most probable non-bonded near-neighbor distance, $\mathbf r_i$ being the position of atom $i$, $\Delta r_i^2=(\mathbf r_i-\langle \mathbf r_i\rangle)^2$, and $\langle\rangle$ denotes configurational averages at the conditions of measurement or simulations (e.g. biological, in which case $T\approx 310$ K, PH$\approx 7$, etc.\footnote{Fluctuations, classical (thermal) or quantal imply a probabilistic description. While one can only predict the odds for a given outcome of an experiment in quantum mechanics, probabilities themselves evolve in a deterministic fashion  (\cite{Born:48}).}). The dynamics as a function of the distance from the geometric center of the protein is characterized by defining an interior ($int$) Lindemann parameter, 
\begin{align}
\Delta^{int}_L(r_{cut})=\frac{\sqrt{\sum_{i,r_i<r_{cut}}\langle \Delta r_i^2\rangle/N}}{a'},
\end{align}  
which is obtained by averaging over the atoms that are within a chosen cutoff distance, $r_{cut}$, from the center of mass of the protein.

Simulations and experimental data for a number of proteins, in particular Barnase, Myoglobin, Crambin and Ribonuclease A indicate 0.14 as the critical value distinguishing between solid--like and liquid--like behaviour, and $r_{cut}\approx 6$ \AA. As can be seen from Table \ref{tab2C1}, the interior of a protein, under physiological conditions, is solid--like  ($\Delta_L<0.14$), while its surface is liquid--like ($\Delta_L>0.14$). The beginning of thermal denaturation in the simulations appears to be related to the melting of its interior (i.e. $\Delta^{int}_L>0.14$), so that the entire protein becomes liquid--like. This is also the situation of the denatured state of a protein under physiological conditions\footnote{see e.g. \cite{Rosner:17}.} 



\begin{table}[h]
 \begin{tabular}{|c|c|c|c|c|}
 \hline
 &\multicolumn{4}{|c|}{$\Delta_L(\Delta_L^{int}(6\;\text{\AA}))(300$ K)}\\
 \cline{2-5}
 &\multicolumn{3}{|c|}{MD simulations}&X--ray data\\
 \hline
 Proteins&Barnase&Myoglobin&Crambin&Ribonuclease A\\
 \hline
 all atoms&0.21(0.12)&0.16(0.11)&0.16(0.09)&0.16(0.12)\\
 backbone atoms only&0.16(0.10)&0.12(0.09)&0.12(0.08)&0.13(0.10)\\
 side--chain atoms only&0.25(0.14)&0.18(0.12)&0.19(0.10)&0.19(0.13)\\
 \hline
 \end{tabular}
 \caption{The heavy-atom $\Delta_L(\Delta_L^{int})$ value, for four proteins at 300 K. After \cite{Zhou:99}.}\label{tab2C1}
 \end{table}

\subsection{Lindemann (``disorder'') parameter for a nucleus}
An estimate of  $\sqrt{\sum_i\langle \Delta r_i^2\rangle/A}$ in the case of nuclei considered as a sphere of nuclear matter of radius $R_0$, is provided by the ``spill out'' of nucleons due to quantal effects. That is\footnote{\cite{Bertsch:05}, see e.g. Ch. 5. See also paragraph following Eq. (\ref{eq3App3E}).} $\sqrt{\vphantom{\Sigma}\quad}\approx 0.69\times a_0$, where $a_0$ is of the order of the nuclear diffusivity  ($\approx 0.65$ fm).
The average internucleon distance can be estimated to be $2\times r_0\approx 2.4$ fm. Thus,
\begin{align}
\Delta_L=\frac{0.69 a_0}{2.4\;\text{fm}}\approx0.19.
\end{align} 
While it is difficult to compare among them crystals, aperiodic finite crystals and atomic nuclei, arguably, the above value indicates that a nucleus is liquid-like. 
\section{The van der Waals interaction}\label{C2AppD}
% \begin{figure}
%  \centerline{\includegraphics*[width=15cm,angle=0]{nutshell/figs/fig2D1.pdf}}
%  \caption{}\label{fig2.D.1}
%  \end{figure}
Historically one can distinguish two contributions to the van der Waals interaction\footnote{Let us mainly think of non polar (NP) molecules.}: 
\begin{enumerate}
\item \textbf{dispersive} retarded contribution\footnote{Dispersion: variation of a quantity, e.g. spatial separation of white light (rainbow), as a function of frequency (c.f. e.g. \cite{Israelachvili:85}, p.65).}, emerging from the dynamical dipole-dipole, as well as from higher multipolarities, interaction associated with the quantum mechanical zero point fluctuations (ZPF) of the ground state of the two interacting atoms or molecules\footnote{These forces act between all atoms and molecules, even non-polar, totally neutral ones.}.
\item \textbf{inductive} implying the polarization of one molecule in the permanent dipole or quadrupole field of the other molecule\footnote{\cite{Debye:20,Debye:21}.} 
\end{enumerate}

%  \begin{figure}
%   \centerline{\includegraphics*[width=15cm,angle=0]{nutshell/figs/fig2D2.pdf}}
%   \caption{Schematic representation concerning the response to photons of both nuclei and metal clusters (after \cite{Broglia:92}).}\label{fig2.D.2}
%   \end{figure}
It is only the first one which is a \textit{bona fide} van der Waals interaction. In fact with the advent of quantum mechanics it was very early  recognized\footnote{\cite{London:30}.} that for most molecules, interactions of type 2 are small compared with interactions of type 1. That is the interaction corresponding to the mutual polarization of one molecule in the rapidly changing field --due to the instantaneous configuration of electrons and nuclei-- of the other molecule\footnote{\cite{Pauling:63} p. 384, \cite{Born:69} p. 471.}.



%In connection with theories of systems with many degrees of freedom (i.e. fields-- and (many--body)--systems) developed in particular by Anderson, Nambu and Goldstone, it has been recognized that the phenomenon of spontaneous symmetry breaking is at the basis of physical emergent properties (see App. \ref{C2AppE}).
%
%
%Within the present context, an atom violates translational invariance as its center of mass (CM) occupies a definite position in space defining a privileged origin for a reference frame. Setting both ions and electrons in uniform motion through a Galilean transformation restores symmetry, the inertia being the total mass (App. \ref{App1.D}). Thus, when one pushes the system  on one end it starts moving as a rigid body (generalized rigidity), without the need  of propagation of information through it.
%
%
%Such a motion (isoscalar in the case of the atomic nuclei, where $N$ and $Z$ move in phase, equivalent to electrons and ions doing so in the case of condensed or soft matter) display zero restoring force. Thus, the associated ZPF diverge requiring, quantum mechanically, a state orthogonal to it in which the two types of constituents particles (electrons and ions, neutrons and protons), move out of phase. Such a state is, in the nuclear case\footnote{Note however the  pygmy halo resonance, soft $E1$--mode in the neutron halo nuclei like $^{11}$Li, which essentially forces a permanent dipole in the $\ket{^{11}\text{Li (gs)}}$.} the GDR and corresponds to a mode in which protons and neutrons slosh back and forth out of phase (isovector mode), a situation which is similar to that of atomic clusters (Mie resonance). In the atomic or molecular case these states (dipole vibration of electrons against the positive ions) are rather directly related to the single--electron atomic shell physics ($1s\to2p$ transition in the case of the H atom). Two atoms displaying the above ZPF will interact through van der Waals (dispersive, retarded) interaction.
\subsection{van der Waals interaction between two hydrogen atoms}
For large values of the internuclear distance $r_{AB}=R$, the exchange phenomenon is unimportant (Pauli principle) and one can take as the unperturbed wavefunction for a system of two hydrogen atoms  the simple product of two hydrogenlike wavefunctions,
\begin{align}\label{eq2.D.1}
\Psi^0=u_{1sA}(1)u_{1sB}(2).
\end{align}
%   \begin{figure}
%    \centerline{\includegraphics*[width=5cm,angle=0]{nutshell/figs/fig3D1_v2.pdf}}
%    \caption{Schematic representation of the lowest single--particle levels in which the electron can move in a hydrogen atom.}\label{fig2.D.3}
%    \end{figure}
The perturbation for this wavefunction arises from the potential energy terms
\begin{align}\label{eq2.D.2}
H'=\frac{e^2}{r_{12}}+\frac{e^2}{r_{AB}}-\frac{e^2}{r_{A2}}-\frac{e^2}{r_{B1}},
\end{align}
corresponding to the variety of Coulomb interactions involving electrons and protons. Let us assume for simplicity that we are dealing with a one-dimensional problem, in which case one can write (Fig. \ref{fig2.D.4})
\begin{align}\label{eq2.D.3}
\nonumber \mathbf r_{12}&=\left(R+z_1+z_2\right)\hat {\mathbf{z}},\\
\nonumber \mathbf r_{AB}&=R\,\hat {\mathbf{z}},\\
\nonumber \mathbf r_{A2}&=\left(R+z_2\right)\hat {\mathbf{z}},\\
\mathbf r_{B1}&=\left(R+z_1\right)\hat {\mathbf{z}}.
\end{align}

    \begin{figure}
     \centerline{\includegraphics*[width=5cm,angle=0]{nutshell/figs/fig2D4_v2.pdf}}
     \caption{Planar configuration assumed for two hydrogen atoms at a relative distance $R$, where $p$ stands for proton and $e$ for electron.}\label{fig2.D.4}
     \end{figure}
Because all these distances are much larger than the radius of the atom ($a_0\approx0.529$ \AA, Bohr radius) the expression (\ref{eq2.D.2}) can be calculated making use of a Taylor expansion. One obtains
\begin{align}\label{eq2.D.4}
r_{12}^2=(R+z_1+z_2)^2=R^2\left[1+2\frac{(z_1+z_2)}{R}+\frac{(z_1+z_2)^2}{R^2}\right],
\end{align}
which leads to 
\begin{align}\label{eq2.D.5}
\frac{e^2}{r_{12}}=\frac{e^2}{R\left[1+\frac{2(z_1+z_2)}{R}+\frac{(z_1+z_2)^2}{R^2}\right]^{1/2}}\approx\frac{e^2}{R}\left[1-\frac{(z_1+z_2)}{R}-\frac{(z_1+z_2)^2}{2R^2}\right].
\end{align}
 Similarly
\begin{align}\label{eq2.D.6}
r^2_{A2}=\left(R^2+2Rz_2+z_2^2\right)=R^2\left(1+2\frac{z_2}{R}+\frac{z_2^2}{R^2}\right),
\end{align}
and
\begin{align}\label{eq2.D.7}
r^2_{B1}=R^2\left(1+2\frac{z_1}{R}+\frac{z_1^2}{R^2}\right),
\end{align}
leading to
\begin{align}\label{eq2.D.8}
-\frac{e^2}{r_{A2}}=-\frac{e^2}{R}\left(1-\frac{z_2}{R}-\frac{z_2^2}{2R^2}\right),
\end{align}
and
\begin{align}\label{eq2.D.9}
-\frac{e^2}{r_{B1}}=-\frac{e^2}{R}\left(1-\frac{z_1}{R}-\frac{z_1^2}{2R^2}\right).
\end{align}
Finally
\begin{align}\label{eq2.D.10}
\frac{e^2}{r_{AB}}=\frac{e^2}{R}.
\end{align}
With the exception of the cross term of (\ref{eq2.D.5}) there is complete cancellation between the different contributions to (\ref{eq2.D.2}). Thus
\begin{align}\label{eq2.D.11}
H'=-\frac{\mathbf D_1\cdot\mathbf D_2}{R^3},
\end{align} 
where
\begin{align}\label{eq2.D.12}
\mathbf D_i=ez_i\,\hat {\mathbf{z}}
\end{align} 
is the dipole moment operator associated with electron $i$. Because $R\gg z_i$, one can diagonalize the interaction Hamiltonian (\ref{eq2.D.11}) perturbatively. In keeping with the fact that   a single--particle quantum state displaying a given parity (and in the present case angular momentum $(-1)^\ell=\pi$) cannot sustain a permanent dipole moment, in particular
\begin{align}\label{eq2.D.13}
\int d\tau u_{1s}(z)\times z\times u_{1s}(z)=\int d\tau (u_{1s}(z))^2z=\int d\tau \rho(z)z=0,
\end{align} 
 the lowest perturbative correction to (\ref{eq2.D.1}) is of second order. The associated energy correction is given by the relation,
 \begin{align}\label{eq2.D.14}
\Delta E_z^{(2)}=-\sum_{int}\frac{\braket{0|H'|int}\braket{int|H'|0}}{E_{int}-E_0}.
 \end{align} 
 Because we are concerned with the $1s\to 2p$ transition (Fig. \ref{fig2.D.5}), the intermediate state is
  \begin{align}\label{eq2.D.15}
\Psi^{int}=u_{2pA}(1)u_{2pB}(2),
  \end{align} 
 and thus
  \begin{align}\label{eq2.D.16}
D_{en}=E_{int}-E_0.
   \end{align} 
   
        \begin{figure}[h]
         \centerline{\includegraphics*[width=4cm,angle=0]{nutshell/figs/fig2D5_v2.pdf}}
         \caption[Schematic representation of the virtual process associated with (\ref{eq2.D.14}), the intermediate, virtual state being the $2p$ state.]{Schematic representation of the virtual process associated with (\ref{eq2.D.14}), the intermediate, virtual state being the $2p$ state.}\label{fig2.D.5}
         \end{figure}
  One can then write
  \begin{align}\label{eq2.D.17}
  \nonumber \Delta E_z^{(2)}=&-\frac{|\braket{0|H'|0}|^2}{D_{en}}=-\frac{e^4}{R^6}\frac{|\braket{0|z_1^2z_2^2|0}|^2}{D_{en}},\\
\nonumber  &\approx-\frac{e^4}{R^6}\frac{\int d\tau_1\,d\tau_2u^2_{1s}(1)u^2_{1s}(2)z_1^2z_2^2}{D_{en}},\\
\nonumber  &=-\frac{e^4}{R^6}\frac{\int d\tau_1\rho(z_1)z_1^2\,\int d\tau_2\rho(z_2)z_2^2}{D_{en}},\\
&=-\frac{e^4}{R^6}\frac{\bar z_1^2 \bar z_2^2}{D_{en}}\approx-\frac{e^4}{R^6}\frac{a_0^2a_0^2}{\frac{e^2}{2a_0}}=-\frac{2e^2a_0^5}{R^6}.  
   \end{align} 
This result corresponds to the $z$--degree of freedom of the system (two H atoms at a distance $R\gg a_0$). One has then to multiply the above result by 3 to take into account the $x$ and $y$ degrees of freedom. Thus   
   \begin{align}\label{eq2.D.18}
\Delta E^{(2)}=-\frac{6e^2a_0^5}{R^6}.
 \end{align}
Let us now calculate the van der Waals interaction between two H--atoms at a distance of the order of ten times the summed radii of the two atoms ($\approx2a_0\approx1$\AA), that is for $R\approx10$\AA, 
  \begin{align}\label{eq2.D.19}
  \nonumber \Delta E_{H-H}^{(2)}(10\text{ \AA})\approx&-\frac{6\times14.4\text{ eV \AA}(0.529\text{ \AA})^5}{(10\text{ \AA})^6}\\
  &\approx-3.6\times10^{-6}\text{ eV}=-3.6\,\mu\text{eV}  
   \end{align}  
Making use of the relation
\begin{align}\label{eq2.D.20}
1\text{ eV}=2.42\times 10^{14}\text{ Hz},\quad(1\text{Hz=s}^{-1})  
   \end{align}  
 one obtains
\begin{align}\label{eq2.D.21}
|\Delta E_{H-H}^{(2)}(10\text{ \AA})|\approx3.6\times 10^{-6}\times 2.42\times 10^{14}\text{ Hz}\approx9\times10^8\text{ Hz}\approx10^3\text{ MHz},      
\end{align} 
 a quantity which can be compared with the Lamb shift (1058 MHz; see Fig. \ref{fig6.2.1x}). It is of notice that $|\Delta E_{H-H}^{(2)}(2.5\text{ \AA})|\approx15\text{ meV}/$part$\approx0.35$ kcal/mole, (1meV/part$\approx0.02306$ kcal/mole), a value of the order of $kT/2$. That is, one half of the thermal energy under biological conditions ($T\approx$ 300 K, $kT\approx 0.6$ kcal/mole)\footnote{\cite{Huang:05}.}, in keeping with the role played by the Van der Waals interaction in the folding and stability of proteins.










\end{subappendices}










%\renewcommand{\bibname}{Bibliography Ch 3}
%\bibliographystyle{abbrvnat}
%\bibliography{../nuclear_bib.bib}


%%% Local Variables:
%%% mode: latex
%%% TeX-master: "../main_libro_CUP"
%%% End:
