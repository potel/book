\documentclass[a4paper,14pt]{book}
%\documentclass[a4paper]{book}
% \linespread{2.}
%\documentclass[12pt]{article}
%\documentclass[12pt]{cmmp}

%%\usepackage{psfig}
%\usepackage{harvard}
\usepackage{epsfig}
%%\usepackage{amsmath}
\usepackage{amsfonts}
%%\usepackage{amssymb}
%%\usepackage{graphicx}
%%
%%\usepackage{txfonts}
%%%\usepackage{mathrsfs}
%
%\usepackage{feynmf}     %<------------ Obbligatorio
\unitlength=1mm         %<------------ Obbligatorio
%
\newcommand{\braket}[1]{\langle {#1} \rangle }
\newcommand{\ket}[1]{|{#1} \rangle }
\newcommand{\bra}[1]{\langle {#1}|}
\usepackage{latexsym}
\usepackage{amssymb}
\usepackage{amsmath}
\usepackage[varg]{txfonts}
\usepackage{mathrsfs}
\usepackage{upgreek}
%\usepackage [latin1]{inputenc}
\usepackage{verbatim}
\usepackage{array}
\usepackage{color}
%\pagestyle{plain}
\usepackage{graphicx}
\DeclareMathAlphabet{\mathpzc}{OT1}{pzc}{m}{it}



\begin{document}
 \chapter{Preface}
The elementary modes of nuclear excitation are vibrations and rotations, single--particle (quasiparticle)  motion, and pairing vibrations and rotations. The specific reactions probing these modes are inelastic,  single-- and two-- particle transfer processes respectively.

Pairing vibrations and rotations, closely connected with nuclear superfluidity are, arguably, a paradigm of quantal nuclear phenomena. They thus play a central role within the field of nuclear structure. It is only natural that two--nucleon transfer plays a similar role concerning direct nuclear reactions. In fact, this is the central subject of the present monograph.


At the basis of pairing phenomena one finds Cooper pairs, weakly bound, extended, strongly overlapping bosonic entities, made out of pairs of nucleons dressed by collective vibrations and interacting through the exchange of these vibrations as well as through the bare $NN$--interaction.
Cooper pairs not only change the statistics of the nuclear stuff around the Fermi surface and, condensing, the properties of nuclei close to their ground state. They also display a rather remarkable mechanism of tunnelling between  target and projectile in  direct two--nucleon transfer reaction. In fact, displaying correlations over distances (correlation length) much larger than nuclear dimensions, Cooper pairs are forced to be confined within such dimensions by the action of the average potential, which can be viewed as an external field as far as these pairs are concerned.


The correlation length paradigm comes into evidence, for example, when two nuclei are set into weak contact in a direct reaction. In this case, each of the partner nucleons has a finite probability to be confined within the mean field of each of the two nuclei. It is then natural that a Cooper pair can tunnel, equally well correlated, between target and projectile, through simultaneous than through successive transfer processes. In particular, in this last case, making use of virtual states which, if forced to become real by intervening the reaction with an external mean field, will lead to single--nucleon transfer events. The above mentioned weak coupling Cooper pair tunnelling reminds  the tunnelling mechanism of electronic Cooper pairs across a barrier (e.g. a dioxide layer) separating two superconductors, known as Josephson junction. The main difference is that, as a rule, in the nuclear time dependent junction provided by a direct two--nucleon transfer process, only one or even none of the two weakly interacting nuclei are superfluid (or superconducting). Now, in nuclei, paradigmatic example of fermionic  finite many--body system, zero point fluctuations  (ZPF) in general, and those associated with pair addition and pair substraction modes known as pairing vibrations in particular, are much stronger than in condensed matter. Consequently, and in keeping with the fact that pairing vibrations are the nuclear embodiment of Cooper pairs in nuclei,   pairing correlations based on even  a single Cooper pair can lead to clearly observable effects. In  some cases, like for example in connection with the exotic nucleus $^{11}$Li, to a rather unexpected phenomenon, namely the presence of a tenuous halo extending much beyond standard nuclear dimensions.


 Cooper pair tunnelling has played and is playing a central role in the probing of these subtle quantal phenomena, both in the case of  exotic nuclei as well as of nuclei lying along the stability valley, and have been instrumental in shedding light on the subject of pairing in nuclei at large, and on nuclear superfluidity in particular. Consequently, the subject of two--nucleon transfer occupies  a central place in the present monograph both concerning the conceptual and the computational aspects of the description of nuclear pairing, as well as regarding the quantitative confrontation of the results and predictions with the experimental findings.

Because the interweaving of the variety of elementary modes of nuclear excitation, the study of Cooper pair tunnelling in nuclei aside from requiring a consistent description of nuclear structure in terms of dressed quasiparticles and vibrations, resulting from both bare and induced interactions, involves also the description of one--nucleon transfer as well as knock out processes, let alone inelastic and Coulomb excitation processes.
The corresponding softwares \textsc{cooper, one, knock, inelastic} and \textsc{coulomb} are briefly presented, referring to the enclosed CD for the corresponding files and input--output examples.


Summing up, general physical arguments and technical computational details, as well as the software used in the description and calculation of the absolute two--nucleon transfer cross sections, making use of state of the art nuclear structure information, are provided. As a consequence, theoretical and experimental practitioners, as well as PhD students could use the present monograph at profit. 


Concerning the notation, we have divided each chapter into sections. Each subsection may in turn be broken down into subsections. Equations and Figures are identified by the number of the chapter and that of the section. Thus (6.1.33) labels the thirtythird equation of section 1 of chapter 6. Similarly, Fig. 6.1.2 labels the second figure of section 1 of chapter 6. Concerning the Appendices, they are labelled by the chapter number and by a Latin letter in alphabetical order, e.g. App. 6.A, App. 6.B, etc. Concerning equations and Figures, a sequential number is added. Thus (6.E.15) labels the fifteenth equation of Appendix E of chapter 6, while Fig. 6.F.1 labels the first figure of Appendix F of chapter 6.




















\end{document} 