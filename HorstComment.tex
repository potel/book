\documentclass[prl,aps,12pt]{revtex4}
\usepackage{latexsym}
\usepackage{amssymb}
\usepackage{amsmath}
\usepackage[varg]{txfonts}
\usepackage{mathrsfs}
\usepackage{upgreek}
\usepackage{verbatim}
\usepackage{graphicx}
\DeclareMathAlphabet{\mathpzc}{OT1}{pzc}{m}{it}
\begin{document}
	Ch.1
1. Eq.(1.1.9): spend a few words on the classical LD-picture and the quantization (maybe by ref. to BM)\\
2. Fig.1.1.1: explain the dashed lines, shown in the inset; explain the doube-parabola\\
3. Fig.1.1.3 , 1.1.4: poor quality\\
4. p.20: unclear sentences:\\
- "This behaviour is to be found..." that sentence mixes up a variety of pnenomena whhci better should be discussed in sperate sentences
- "...which change in one unit.." the same is true for this sentence. 
Such highly condensed statements appear repeatedly also at other isntances of the text. For those readers using the book as an 
introductory textg to and /or a study text for nuclear theory, this might pose a severe problems.
5. Eq.(1.2.3) etc. : explain kappa, explain $\hat{\alpha}$\\
6. Eq.(1.2.6): point out that the proper in-medium interaction has to be used to. e.g. $v=v(\vec{x},\rho)$\\
7. Fig. 1.2.1: poor quality\\
8. Footnote: RPA sums the bubble or ring diagrams (Fig. 1.2.3), ladder diagrams are contributing to fig.1.2.2\\     
9. Fig. 1.2.2: left column is of poor quality; what is the rational to show (3) and /4) in additon to (1b) and (2)?;\\ 
(8) might be attached to (5) and (6); the phrase "bare NN-interaction" is misleading: I guess, it's the ladder-summed (Breuckner) 
in-medium interaction what is meant
10. Caption Fig.(1.2.3): The equations ($Y, 1/kappa$) deserve a deeper discussions in the text\\
11. Fig.1.3.1: the appearance of bold "=" and "+" and $\Sigma$ disatracts the reader form the true content, namely the  diagrams.\\
12. Eq.(1.3.4)Ö: a coupling constatn might be in place.\\
13. Fig.(1.3.2): very busy figure, many new and unexplained symbols and notations, e.g. what is $\sigma$, what is meant by the various "Q"'s? 
Someone who is not yet an expert, might get lost. As a rule: Help the potential reader, otherwise people may give up reading the book!\\
14. p. 30: The phrase "...is thus gauge invariant." might be controversial in it's generality and therefore should be explained in view of that 
QFT/QED-people have a diffferent understanding of gauge invariance, so do our friends from chiral EFT etc.  \\
15. Eq.(1.8.4): although the notation $a^\dag(\vec r)$ is used frequently, I'd prefer the QFT notation $\psi(\vec r)$ as used in many other 
text books of quantum many-body theory - but that mighrt be a matter of personal taste\\
16. p.50, First paragraph: it migth be mentioned that $m_\omega$ is related to the (energy-)slope of the dispersive self-energy\\ 
17. Eq.(1.8.7) adn (1.8.8): definition of $R_{v_i}(r)$?\\
18. A conclusion on Chpt.1: many interesting and important topics are addressed but the rather condensed presentation might 
bring the non-expert reader to a feeling of "getting lost". Many of the NFT aspects, you are discussing later 
anyway in due detail, so there is no urgent need to "squeeze" them into the introduction. As I wrote before, it might be more helpful 
to the reader to spending instead more space on the origin and structure of NN-interactions in the nuclear medium and how they lead to 
self-binding and other cooperative dynamical effects in nuclei.	
	
	
	
	\newpage
Chpt.2:
1. p.67: the last paragraph, "An essential test..." is hardly understandable, 
esp. the second sentence seems to be a fragment, the same for ""Coupligns which are.."\\
2. As a general reamrk: in many cases the figure captions are quite extensive, containing 
important material whcih would better be discussed in the running text.\\
3. Fig.(2.4.1): much to busy!! No one will understand the meaning of all the lines, coming without labelling.
Is that fig. really needed?\\
4. Fig.(2.4.2): the lines are a bit faint.\\
5. Sect. 2.7.4: Mayn of the figure are of bad quality - maybe it's a probelms of the proof printing?\\
6. Sect.2.9.2 is in danger to "trivialize" the optical potential problem. I'd suggest to refer to te vast literature on that subject, 
ew.g. the books of Satchler, Feshbach, Hogdson and the reviews of Mahaux et al.   \\
7. p. 134: You should reconsider that busy figure: It is worth at least 5 separate figures and the the eqautions must be 
discussed in the runnign text. That figure migth work in a talk where you lead by discussion the audience through the
displayed material. But it's horrible in printed form, probably mainly due to the reduction from screen to page format.\\  
8. p.135/136: the numbering looks a bit odd - spend a period to the numbers, i.e. use 1. 2. 3.  
9. Fig(2.9.2) -> see remarks 2. and 7.\\
10. Fig,(2.9.3), upper part -> see remarks 2 . and 7.  
11. Fig.(2.9.4) -> see remark 2.\\
12. Fig.(2.9.5) -> see remark 2.\\
13. Fig.(2.10.1) another very busy figure which would deserve disassembling into a couple of separate figures. 
I uderstand the intention to illustrated the interpay of various observables but is that also obvious to a 
broader readership?\\
14. Conclusions on Chpt.2: Overall, it's well done, also the material in the appendices! The connection of microscopic aspects of nuclear many-body dynamics and nuclear phenomenology is elucidated
by NFT. Why don't you start the chapter with intriducing NFT in the first place? I'd consider such a rearrangemnt an 
interesting alternative, putting more weight on "hard" many-body theory.  
\end{document} 