
\chapter{First Lecture: Spectroscopy with direct nuclear reactions}\label{C2}


\section{Introduction}

A very direct information about the nature of force between nucleons and of the structure of nuclei is obtained from the study of collisions. In a scattering experiment, a beam of particles is directed at a target containing the scattering material, and the energy and angular distributions of the deflected beam or of the recoil particles is measured.

There are two basic lengths governing the nuclear reactions, namely the radius of the nucleus ($\approx 10^{-13}$cm) and the distance from the target nucleus to the detector ($\approx 10^2$cm). Because of the difference of these two characteristic magnitudes, one can divide the scattering process in two separate parts, namely,

\begin{enumerate}
\item{the analysis of the outgoing beam properties in terms of reduced widths, average potentials, density of levels and other parameters determining the nuclear spectrum, and}
\item{the relation between these parameters and the motion of nucleons in the nuclei}
\end{enumerate}

During the first part of the lectures the necessary techniques to deal with problem 1) will be worked out, as far as needed to deal with problem 2), which would be the central subject.
In our consideration of scattering, we shall assume that the interaction between the scattered systems can be represented by a potential $V(r)$, where $r$ is the relative coordinate of the projectile and target.


\section{Reaction channel}

Let us consider the case in which $^{18}O$ is the target and the projectile is a proton. The following processes can take place, among others

\begin{equation}
{\rm entrance\;channel} \left\{
\begin{array}{rl}
p+^{18}O \rightarrow & \;p + ^{18}O \;({\rm gs})\;\; (Q=0) \;\;\; {\rm Elastic\; scattering} \\
p+^{18}O \rightarrow & \left. \begin{array}{ll}
     p + ^{18}O \;({\rm 6 MeV}) & + Q_1 \\
     d + ^{17}O \;({\rm gs})    & + Q_2 \\
     t + ^{16}O \;({\rm gs})    & + Q_3
		 \end{array} \right\} \;\; {\rm reaction\;channels}
\end{array}
\right.
\end{equation}

\noindent where

\begin{equation}
Q_1 = M_p + M(^{18}O) - (M_p + ^{18}O \; (6 \;{\rm MeV})) = -6 {\rm MeV}
\end{equation}

\begin{equation}
Q_2 = M_p + M(^{18}O) - (M_d + ^{17}O \; ({\rm gs}))
\end{equation}

\begin{equation}
Q_3 = M_p + M(^{18}O) - (M_t + ^{16}O \; ({\rm gs}))
\end{equation}

\noindent If $Q>0$ the reaction can proceed at zero bombarding energy. For $Q<0$ the reaction is not observed below the threshold $E_t$ which is defined, for a general reaction $A(a,b)B$ as

\begin{equation}
E_{CM} = \frac{1}{2} \frac{M_a M_A}{M_a + M_A} V_a^2 = \frac{M_A}{M_a + M_A} E_{lab} = \frac{1}{1+(M_a/M_A)} E_{lab}
\end{equation}

\noindent ($1/2 M_a V_a^2$ is the total energy of the system). If $E=|Q|$ we have

\begin{equation}
E_t^{lab} = \frac{M_a + M_A}{M_A} |Q| \rightarrow E_t^{lab} = \left( 1+ \frac{M_a}{M_A} \right)  |Q|
\end{equation}

\noindent For the particular case

\begin{equation}
\begin{array}{l}
p + ^{18}O \rightarrow p + ^{18}O \;\;\; (6 \;{\rm MeV}) \\
E_t^{lab} = \frac{19}{18} \times 6 \;{\rm MeV}
\end{array}
\end{equation}

A complete specification of the type of two-particle breakup and of the internal states of the two particles, is called a channel and is specified by a product $\Psi_{\alpha} = \Psi_a \Psi_A$ of bound internal wave functions of the two nuclei. Here $A$ and $a$ denote the nuclei into which the system breaks up and also their state of excitation, their angular momenta and the projection of their angular momenta.

The same word channel is understood sometimes to include all the properties already mentioned, together with a definite value of the orbital angular momentum of the relative motion of the centers of mass of the two separating systems.


\section{The reaction cross section}

The initial situation can be described by a plane wave \footnote{This free particle wave function can be normalized in terms of the function or asking the function to obey periodic boundary conditions inside a box (see Appendix)},

\begin{equation}
\Psi_{inc} = \Psi_{\alpha} e^{i k_{\alpha} z}
\label{eqn:psi_inc}
\end{equation}

\noindent which represents a beam of particles of unit density, incident upon the scattering center in the $z$-direction. The price to pay for such a simplification is that momentum and energy are then absolutely defined, and it is no longer possible to follow the scattering process neither in space now in time (from $\Delta x \Delta p \ge \frac{\hbar}{2}$ we loose the localization in space and from $\Delta E \Delta t \ge \frac{\hbar}{2}$ the one in time).

This way of describing the incident beam is of course an idealization, and in most cases there is no problem of principle, as one can approach the situation described by Eq.(\ref{eqn:psi_inc}) as closely as desired.

{\it One must be able, however, to place the detector outside the beam}. It is important to verify that the corresponding spread of the wave packet does not produce any essential limitation. Let us call $W$ the width of the beam (see fig. \ref{fig1st_1}). The velocity perpendicular to the beam can be easily calculated

\begin{equation}
\begin{array}{rl}
\Delta x \Delta p_x \sim \hbar &, p_x = m v_x \\
\Delta p_x = m \Delta v_x &, \Delta X = W \\
W(m \Delta v_x) &\sim \hbar \\
\Delta v_x &\sim \frac{\hbar}{MW}
\end{array}
\label{eqn:10}
\end{equation}
%\begin{figure}
%\centerline{\includegraphics*[width=10cm,angle=0]{C:/Gregory/Broglia/notas_ricardo/Figures/ricardo_091105/1_1}}
%\caption{}\label{fig1st_1}
%\end{figure}

A particle leaving the collimator ($\langle v_x \rangle = 0$) has an uncertainty in $v_x$ given by the equation (\ref{eqn:10}). In the most unfavourable case $v_x = \Delta v_x$. After a time $\Delta t = t$ (we choose $t=0$ when the particle leaves the collimator, namely for $z=0$), the particle has travelled a distance $L$, and the spreading of the wave packet along $x$ is equal to

\begin{equation}
\begin{array}{rl}
\Delta W = v_x t &\approx \Delta v_x t \;\;\;\;\; t = \frac{L}{v} \\
\Delta W = \frac{\Delta v_x}{v} L &\approx \frac{\hbar L}{m W v}
\end{array}
\end{equation}

\noindent The relative increase of the wave packet is then equal to

\begin{equation}
\frac{\Delta W}{W} \approx \frac{\hbar L}{m W^2 v}
\end{equation}

\noindent Let us put some numbers corresponding to the accelerator at Ris\o

\begin{equation}
\begin{array}{rl}
L &\approx 10^2 {\rm cm} \\
v &\approx 10^9 {\rm cm/seg} \\
W &\approx 10^{-1} {\rm cm} \\
\frac{\Delta W}{W} &\approx 10^{-8}
\end{array}
\end{equation}

\noindent which is very small indeed.

The most dangerous case corresponds to reactions with slow neutrons, which usually are done also in large equipments (time of flight techniques). Let us put

\begin{equation}
\begin{array}{rl}
L &\approx 10^4 {\rm cm} \\
v &\approx 2 \times 10^5 {\rm cm/seg} \\
W &\approx 10^{-1} {\rm cm} \\
\frac{\Delta W}{W} &\approx 10^{-2}
\end{array}
\end{equation}

The ratio is still small, but just on the limit of becoming important. Still if this ratio would be of order unit we could use the same concepts, but we should threat with more detail the problem of how the wave packet is constructed and the possible interference at the edge of the beam, with the outgoing wave packet.

{\it The scattered wave (asymptotic region) must be the solution of the free field equation}

\begin{equation}
\begin{array}{rl}
H \Psi_{\rm scatt} &= E \Psi_{\rm scatt} \\
E &= \frac{\hbar^2 k_{\alpha}^2}{2m}
\end{array}
\end{equation}

\noindent where

\begin{equation}
H = \frac{p^2}{2m} = - \frac{\hbar^2}{2m} \nabla^2 = - \frac{\hbar^2}{2m} \left\{ \frac{1}{r^2} \frac{\partial}{\partial r} \left( r^2 \frac{\partial}{\partial r} \right) + \frac{1}{r^2} \hat{L}^2 \right\}
\end{equation}

\noindent where $\hat{L}$ is the angular momentum operator, $m = \frac{M_B M_b}{M_b + M_B}$ the reduced mass, and $k$ and $r$ the relative momentum and coordinate of the nuclei $b$ and $B$.

\noindent At large distances the angular momentum terms drops out as $\frac{1}{r^2}$ and it is easy to verify that the asymptotic solution is

\begin{equation}
\Psi_{\rm scatt} = \frac{e^{ik_{\alpha} r}}{r} f_{\alpha \alpha}(E, \theta, \phi) \Psi_{\alpha}
\end{equation}

\noindent Let us now calculate the incoming current and the scattered current of particles.

For a given wave function $\Psi$, (describing the motion of a particle of mass $m$), the associated current is equal to

\begin{equation}
\begin{array}{rl}
\vec{I} &= \frac{\hbar}{2im} \left( \Psi^* \vec{\nabla} \Psi - \left( \vec{\nabla} \Psi^* \right) \Psi \right) \\
&= \frac{\hbar}{m} {\cal I}_m \left( \Psi^* \vec{\nabla} \Psi \right)
\end{array}
\end{equation}

The incident current is equal to (cf. eq. (9))

\begin{equation}
\begin{array}{rl}
\vec{I}_{\rm inc} &= \frac{\hbar}{m} {\cal I}_m e^{-i k_{\alpha} z} \frac{\rm d}{{\rm d}z} \left( e^{i k_{\alpha} z} \right) \hat{z} \\
&= \frac{\hbar k_{\alpha}}{m} \hat{z} = v_{\infty} \hat{z}
\end{array}
\end{equation}

\noindent where $v_{\infty}$ is the velocity corresponding to the projectile incident energy.

The scattered current is equal to \footnote{In deriving this equation one assumes that $r \rightarrow \infty$ (asymptotic region).}

\begin{equation}
\vec{I}_{\rm scatt} \approx \frac{|f(\theta,\phi)|^2}{r^2} \frac{\hbar k_{\alpha}}{m} \hat{r}
\end{equation}

The differential cross section is defined as the flux of particles going into the solid angle ${\rm d}\Omega$ at angle $(\theta,\phi)$, divided by the incoming flux of incoming particles.

The flux of outgoing particles is given by the projection of $I_{\rm scatt}$ on the unit $\vec{{\rm d}s} = r^2 {\rm d}\Omega \hat{r}$ of solid angle, namely

\begin{equation}
\vec{I}_{\rm scatt} {\rm d}\vec{s} = |f(\theta,\phi)|^2 \frac{\hbar k_{\alpha}}{m} {\rm d}\Omega
\end{equation}

\noindent The incident flux is given by

\begin{equation}
\vec{I}_{\rm inc} \hat{z} = \frac{\hbar k_{\alpha}}{m}
\end{equation}

\noindent The differential cross section is defined as the flux of particles into the solid angle ${\rm d}\Omega$ at angle $(\theta,\phi)$, divided by the incoming flux of incoming particles, namely

\begin{equation}
{\rm d}\sigma = |f_{\alpha \alpha}(E,\theta,\phi)|^2 {\rm d}\Omega
\end{equation}

In other channels there will be no incident wave, and in general, the outgoing waves would have a different value of the wave number, i.e.

%\begin{figure}
%\centerline{\includegraphics*[width=10cm,angle=0]{C:/Gregory/Broglia/notas_ricardo/Figures/ricardo_091105/1_2}}
%\caption{}\label{fig1st_2}
%\end{figure}
\begin{equation}
\Psi_{\rm scatt} = \frac{1}{r} e^{i k_{\beta} r} f_{\alpha \beta} (E',\theta,\phi)
\end{equation}

The symmetries of the problem can produce limitations in the form of $f_{\alpha \beta}(k_{\beta},\theta,\phi)$. In general, using unpolarized particles, and not considering spin, $f_{\alpha \beta}(k_{\beta},\theta,\phi)$ will not depend on the angle $\phi$. This is a consequence of the fact that the incoming beam has zero projection of the angular momentum in the direction of the incident beam. Therefore, in the outgoing channel the angular momentum will mantain its zero projection and therefore the outgoing wave function cannot depend on $\phi$.



\section{Actual evaluation of the spreading of the wave packet}

\begin{equation}
m_p = 1.7 \times 10^{-24} {\rm gr}
\end{equation}

\begin{equation}
1 {\rm MeV} = 1.6 \times 10^{-6} {\rm erg} = 1.6 \times 10^{-6} {\rm gr} \frac{{\rm cm}^2}{{\rm sec}^2}
\end{equation}

\begin{equation}
\begin{array}{rl}
v &= \sqrt{\frac{2E}{m}} = \sqrt{\frac{3.2 E \times 10^{-6}}{1.7 \times 10^{-24} {\rm gr}}{\rm gr} \frac{{\rm cm}^2}{{\rm sec}^2}} \\
  &= 10^9 \sqrt{1.88 E ({\rm MeV})} \\
	&\approx 1.4 \times 10^9 \sqrt{E({\rm MeV})} \frac{{\rm cm}}{{\rm sec}}
\end{array}
\end{equation}

For $20$ MeV protons ($E = 20$ MeV)

\begin{equation}
v \approx 6 \times 10^{9} \frac{{\rm cm}}{{\rm sec}}
\end{equation}

\begin{equation}
\frac{\Delta W}{W} \approx \frac{\hbar L}{m W^2 v}
\end{equation}

\begin{equation}
L \approx 10^2 {\rm cm} \qquad \qquad W \approx 10^{-1} {\rm cm}
\end{equation}

\begin{equation}
\hbar = 1.054 \times 10^{-27} {\rm erg \; sec}
\end{equation}

\begin{equation}
\begin{array}{rl}
\frac{\Delta W}{W} &= \frac{\left( 1.054 \times 10^{-27} {\rm gr} \frac{{\rm cm^2}}{{\rm sec^2}} \right) \left( {\rm cm} \times 10^2 \right)}{1.7 \times 10^{-24} {\rm gr} \times 10^{-2} {\rm cm^2} \times 6 \times 10^9 \frac{{\rm cm}}{{\rm sec}}} \\
&= \frac{1.054 \times 10^{-25}}{1.7 \times 6} \times 10^{17} = \frac{1.054}{1.7 \times 6} 10^{-8} \\
&\approx 10^{-9}
\end{array}
\end{equation}
