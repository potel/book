\documentclass[a4paper,14pt]{article}
%\documentclass[a4paper]{book}
% \linespread{2.}
%\documentclass[12pt]{article}
%\documentclass[12pt]{cmmp}

%%\usepackage{psfig}
%\usepackage{harvard}
\usepackage{epsfig}
%%\usepackage{amsmath}
\usepackage{amsfonts}
%%\usepackage{amssymb}
%%\usepackage{graphicx}
%%
%%\usepackage{txfonts}
%%%\usepackage{mathrsfs}
%
%\usepackage{feynmf}     %<------------ Obbligatorio
\unitlength=1mm         %<------------ Obbligatorio
%
\newcommand{\braket}[1]{\langle {#1} \rangle }
\newcommand{\ket}[1]{|{#1} \rangle }
\newcommand{\bra}[1]{\langle {#1}|}
\usepackage{latexsym}
\usepackage{amssymb}
\usepackage{amsmath}
\usepackage[varg]{txfonts}
\usepackage{mathrsfs}
\usepackage{upgreek}
%\usepackage [latin1]{inputenc}
\usepackage{verbatim}
\usepackage{array}
\usepackage{color}
%\pagestyle{plain}
\usepackage{graphicx}
\DeclareMathAlphabet{\mathpzc}{OT1}{pzc}{m}{it}



\begin{document}
\section{On the relative importance of successive, simultaneous, and pairing induced two--particle transfer}
Let us denote
\begin{equation}\label{eq_est_1}
    H=T+V,
\end{equation}
the total hamiltonian describing the nuclear system, where $V$ is the nuclear two--body interaction.


The fact that the nuclear quantality parameter has a value of $Q\approx 0.4$ testifies to the validity of independent particle motion in nuclei. This is tantamount to saying that there exist a single--particle potential $U$, such that
\begin{equation}\label{eq_est_2}
\langle \Psi_0 | U | \Psi_0 \rangle \ll \langle \Psi_0 |(V- U )| \Psi_0 \rangle,
\end{equation}
where $\Psi_0$ is the exact ground state wavefunction, that is, $H\Psi_0=E_0 \Psi_0$. One can the write \ref{eq_est_1} as
\begin{equation}\label{eq_est_3}
H=T+V_{eff},
\end{equation}
where
\begin{equation}\label{eq_est_4}
V_{eff}=U+(V-U).
\end{equation}
Let us now consider a reaction in which two nucleons are transferred between target and projectile, that is,
\begin{equation}\label{eq_est_5}
a(=b+2)+A\rightarrow b+B(=A+2).
\end{equation}
The transfer cross section is proportional to the square of the amplitude
\begin{equation}\label{eq_est_6}
\sqrt{\sigma}\sim \langle bB | V_{eff} | aA \rangle = \langle bB | U | aA \rangle + \langle bB | (V-U) | aA \rangle.
\end{equation}
Let us assume that the transferred nucleons are e.g. two neutrons moving in time reversal states lying close t othe Fermi energy (Cooper pair). In this case it is natural to assume that the operative component of $(V-U)$ is the pairing interaction
\begin{equation}\label{eq_est_7}
V_p=-GP^\dagger P,
\end{equation}
where
\begin{equation}\label{eq_est_8}
P^\dagger=\sum_{\nu>0}a_\nu^{\dagger}a_{\bar\nu}^{\dagger},
\end{equation}
is the pair operator, and
\begin{equation}\label{eq_est_9}
G\approx\frac{18}{A}\; \text{MeV},
\end{equation}
is the pairing coupling constant for nucleons moving in an extended (2--3 major shell) configuration.


One can then write Eq. \ref{eq_est_6} as
\begin{equation}\label{eq_est_10}
\sqrt{\sigma}=\sqrt{\sigma_1}+\sqrt{\sigma_2},
\end{equation}
where
\begin{equation}\label{eq_est_11}
\sqrt{\sigma_1}\sim \langle Bb | U| aA\rangle \approx 2\left(\frac{|V_0|}{2}\right)\mathcal{O}, \; \text{(SUCC+NO)}
\end{equation}
and
\begin{equation}\label{eq_est_12}
\begin{split}
\sqrt{\sigma_2}\sim & \langle Bb |V-U| aA\rangle = \langle Bb |H_p| aA\rangle\\
&\approx GU(b)V(B)\approx\frac{G}{2}, \; \text{(PAIRING)}
\end{split}
\end{equation}

In the process described by the transfer amplitude $\langle Bb |U| aA\rangle$, one nucleon is transferred under the effect of the single--particle potential of depth $V_0(\approx -50$ MeV ) while, simultaneously, the second nucleon moves over from a single--particle orbit centered around $b$ to one centered around $A$ profiting of the non--orthogonality of the corresponding wavefunctions $\varphi^{(b)}(r_{1b})$ and $\varphi^{(A)}(r_{1A})$. Within this context, it is then natural that $\mathcal{O}$ stands for the overlap between these two  wavefunctions, that is, (see below simple estimate of $\mathcal{O}$),
\begin{equation}\label{eq_est_13}
\mathcal{O}=\langle \varphi^{(b)}|\varphi^{(A)}\rangle \approx 0.3 \times 10^{-2},
\end{equation}
and that (\ref{eq_est_11}) is known as the sum of the simultaneous plus non--orthogonality contributions to the two--nucleon transfer amplitude. Of notice that the prefactor 2 in (\ref{eq_est_11}) is associated with the fact that two nucleons can choose to jump from one system to the other through non--orthogonality while the factor $|V_0|/2$ is associated with the fact that transfer takes mainly place at the surface.


The term (\ref{eq_est_12}) corresponds to the simultaneous $(t,p)$ transfer via the pairing two--body interaction $V_p$ (see Eq. (\ref{eq_est_7})), $U(A)$ and $V(B)$ being the product of two occupation amplitudes: $U(A)$ measures the availability of free single--particle orbitals around the Fermi energy in the target nucleus, while $V(B)$ reflects the degree of occupancy of levels in the residual system. Close to the Fermi energy $U(b)V(B)\approx (1/\sqrt{2})^2=1/2$, leading to the final expression of (\ref{eq_est_12}).


In keeping with the fact that the ratio of transfer amplitudes
\begin{equation}\label{eq_est_14}
\begin{split}
\left(\frac{\sigma_1}{\sigma_2}\right)^{1/2}&\approx 2\frac{|V_0|}{2}\times \mathcal{O}\frac{1}{G/2}\approx 2\times A\times 10^{-2}\\
& \approx 2 (A\approx 100),
\end{split}
\end{equation}
is larger that one, and that the correlation length of nuclear Cooper pairs ($\xi\approx \hbar v_F/2\Delta \approx 30$ fm) is larger than nuclear dimensions, one can expect  that the successive transfer of two nucleons under the influence of the single--particle field, can give an important contribution to the total transfer amplitude $\sqrt{\sigma}$. In other words, we expect the process


\begin{equation}\label{eq_est_15}
a(=b+2)+A \rightarrow f(=b+1)+F(A+1)\rightarrow b+B(=A+2)
\end{equation}
gives a consistent contribution to $\sqrt{\sigma}$. The associated amplitude can be written as

\begin{equation}\label{eq_est_16}
\begin{split}
\sqrt{\sigma_3}\sim & \sum_{fF}\frac{\langle bB|U|fF\rangle \langle fF|U|aA\rangle}{E_{aA}-E_{fF}}\\
& \approx \frac{(V_0/13)(V_0/13)}{\Delta E},
\end{split}
\end{equation}
the factor 1/7 appears in each of the steps (instead of 1/2, see (\ref{eq_est_11})) in keeping with the fact that many other reactions channels and then, absorptive processes will take place at closer distance in two--step channels (of notice that 1/7 corresponds to $r=R_0+2.5 a$).


Typical values of the energy denominator in (\ref{eq_est_16}) are $\Delta E=30$ MeV for medium heavy nuclei lying along the stability valley.

\section{Transfer amplitudes}
Making use of a simplified expression for the elastic scattering amplitude, that is

\begin{equation}\label{eq_est_17}
\sqrt{\sigma_{el}}\sim \langle aA|U|aA\rangle,
\end{equation}
one can calculate the transfer probabilities associated with the different processes discussed above, namely
\begin{equation}\label{eq_est_18}
P_i=\left(\frac{\sigma_i}{\sigma_{el}}\right)=\left\{
\begin{array}{l}
  \left|\frac{\langle bB|U|aA\rangle}{\langle aA|U|aA\rangle}\right|^2\approx \mathcal{O}^2\approx 0.9 \times 10^{-5}\quad (i=1), \\
  \left|\frac{\langle bB|V_p|aA\rangle}{\langle aA|U|aA\rangle}\right|^2\approx \left(\frac{G}{2V_0}\right)^2\approx 1.4 \times 10^{-6}\quad (i=2), \\
    \left|\frac{\langle bB|U|fF\rangle\langle fF|U|aA\rangle}{\Delta E\langle aA|U|aA\rangle}\right|^2\approx \left(\frac{V_0}{170\Delta E}\right)^2\approx 0.96 \times 10^{-4}\quad (i=3).
\end{array}
\right.
\end{equation}
Because all these probabilities are small, one can write
\begin{equation}\label{eq_est_19}
\sigma_i=P_i\sigma_{el},
\end{equation}
where
\begin{equation}\label{eq_est_20}
\begin{split}
\sigma_{el}=&\left(\frac{\mu_\alpha}{2\pi\hbar^2}\right)^2|\langle aA|U|aA\rangle|^2\\
&\approx \left(\left(\frac{\mu_\alpha}{2\pi\hbar^2}\right)(V_0)\right)^2U_0^2\\
&\approx \left( 1.8 \text{MeV}^{-1}\text{fm}\right)^2 \left( 50 \text{MeV}\right)^2 \\
&\approx \left( 90\text{fm}\right)^2 =0.8 \text{b}
\end{split}
\end{equation}
where use has been made of the effective volume associated with the reaction process (see   )
\begin{equation}\label{eq_est_21}
V_{ol}\approx \frac{4\pi}{3}3R^2a\approx 12 A^{2/3} \text{fm}^3\approx 260 \text{fm}^3,
\end{equation}
as well as of $\frac{\mu_\alpha}{2\pi\hbar^2}$ factor of the typical two--nucleon transfer reaction $^{120}$Sn+p$\rightarrow ^{118}$Sn+t, that is (see  ),
\begin{equation}\label{eq_est_22}
\frac{\sqrt{\mu_\alpha\mu_\beta}}{2\pi\hbar^2}\approx \frac{\sqrt{3}M}{2\pi\hbar^2}\approx \frac{1}{145} \text{MeV}^{-1}\text{fm}^{-2}.
\end{equation}
Summing up, one can write
\begin{equation}\label{eq_est_23}
\sigma_i=P_i0.8\text{b}.
\end{equation}
Making use of (\ref{eq_est_18}) one obtains
\begin{equation}\label{eq_est_24}
\sigma_i=\left\{
\begin{array}{l}
  0.7\times 10^{-2} \text{mb}\quad (i=1), \\
   1.1\times 10^{-3} \text{mb}\quad (i=2), \\
    8 \,\text{mb}\quad (i=1),\quad (i=3).
\end{array}
\right.
\end{equation}
These numbers, although worked out for $A$=100 can be rescaled in connection with the reaction $^{11}$Li$(p,t)^9$Li(gs), in which case, microscopic calculation lead to $d\sigma_1(\theta=60^\circ)/d\Omega\approx 0.01$mb/sr and $d\sigma_3(\theta=60^\circ)/d\Omega\approx 5$mb/sr.

\section{Inelastic scattering following two--particle transfer: final state interaction}
This subject is qualitatively discussed in connection with the $^{11}$Li$(p,t)^9$Li$(1/2^-;2.69)$, but of course is a general question, also in connection with the validity of considering perturbation theory instead of coupled channels.



In keeping with the fact that the first excited state of $^9$Li can be viewed as

\begin{equation}\label{eq_est_25}
|^9\text{Li}(1/2^-;2.69 \text{MeV})\rangle\approx|2^+\left(^8\text{Be}\otimes p_{3/2}(\pi)\right)_{1/2^-}\rangle,
\end{equation}
this state can, in principle, be excited in a two--step process, namely
\begin{equation}\label{eq_est_26}
\text{gs ($^{11}$Li)+t$\;\longrightarrow\;$ gs ($^{9}$Li)+p$\;\longrightarrow \;1/2^-$ ($^{9}$Li)+p}.
\end{equation}


Let us calculate the probability associated with the inelastic scattering of the lowest $2^+$ of $^8$Li. In this case, we are interested in the component of $V-U$ corresponding to $\delta U_C= -KF\alpha=-R_0\frac{\partial U}{\partial r}\beta_L$, namely the field associated with the inelastic excitation of multipole vibrations. Making use of the Saxon--Woos potential one obtains
\begin{equation}\label{eq_est_27}
R_0\frac{\partial U}{\partial r}=\frac{R_0}{a}\frac{\exp\left(\frac{(r-R_0)}{a}\right)}{\left(1+\exp\left(\frac{(r-R_0)}{a}\right)\right)^2}.
\end{equation}
In keeping with the fact that
\begin{equation}\label{eq_est_28}
\left\langle R_0 \left.\frac{\partial U}{\partial r}\right|_{r=R_0}\right\rangle \approx \left\langle \frac{R_0 U_0}{a}\right\rangle \approx 1.2 U_0 \text{MeV} \approx -60 \text{MeV}
\end{equation}
and that the main contributions of surface dominated reactions processes is estimated to arize from distances of the order of $r\approx R_0 + 2.5 a$, one obtains
\begin{equation}\label{eq_est_29}
\begin{split}
\left\langle \frac{R_0}{a}\frac{e^{2.5}U_0}{\left(1+\exp 2.5\right)^2}\right\rangle &= \left\langle \frac{R_0 U_0}{a}\right\rangle\frac{e^{2.5}}{\left(1+\exp 2.5\right)^2}\\
&\approx1.2U_0\times 0.7 \times 10^{-1}=0.84\times 10^{-1}U_0.
\end{split}
\end{equation}
Thus
\begin{equation}\label{eq_est_30}
\langle bB^*|\delta U_C|bB\rangle \approx 0.84\times 10^{-1}U_0 \beta_L.
\end{equation}
Consequently

\begin{equation}\label{eq_est_31}
\begin{split}
P_{inel} \approx &\left|\frac{\langle bB^*|\delta U_C|aA\rangle}{\langle aA|U|aA\rangle}\right|^2=\left(0.84\times 10^{-1} \beta_L\right)^2\\
&\approx 0.7\times 10^{-2}\beta_L^2.
\end{split}
\end{equation}
In keeping with the fact that the $\beta_L$ associated with the lowest $2^+$ vibrational states of the Sn--isotopes and of $^8$are $\approx 0.1$ and $\approx 1$ respectively one can write

\begin{equation}\label{eq_est_32}
P_{inel}=\left\{
\begin{array}{l}
  0.7\times 10^{-4} \quad \text{(Sn--isotopes)}, \\
   0.7\times 10^{-2} \quad \text{($^{11}$Li)}. \\
\end{array}
\right.
\end{equation}
Making use of the results collected in (\ref{eq_est_18}),
\begin{equation}\label{eq_est_33}
\begin{split}
\sqrt{P(p,t)}=&\sqrt{P_1}+\sqrt{P_2}+\sqrt{P_3}\\
&=\sqrt{0.9\times 10^{-1}}+\sqrt{1.4\times 10^{-6}}+\sqrt{0.96\times 10^{-4}}\\
&\approx 3\times 10^{-3}+1.2\times 10^{-3}+0.98\times 10^{-2}\\
&\approx 1.4 \times 10^{-2}.
\end{split}
\end{equation}
Thus
\begin{equation}\label{eq_est_34}
P((p,t)\otimes P(\text{inel}))=P(p,t)P(\text{inel})=\left\{
\begin{array}{l}
  2\times 10^{-4}\times 10^{-4}\approx 10^{-8} \quad \text{(Sn)}, \\
   2\times 10^{-4}\times 10^{-2}\approx 10^{-6} \quad \text{($^{11}$Li)}, \\
\end{array}
\right.
\end{equation}
in overall agreement with th result of microscopic calculations (for $^{11}$Li).
\section{Simple estimate $\mathcal{O}$}
The nuclear density can be parametrized according to
\begin{equation}\label{eq_est_35}
\rho(r)=\frac{\rho_0}{1+\exp\left(\frac{r-R_0}{a}\right)}.
\end{equation}
Let us calculate this function for 
\begin{equation}\label{eq_est_36}
r=R_0+3a,
\end{equation}
that is 
\begin{equation}\label{eq_est_37}
\rho(r=R_0+3a)=\frac{\rho_0}{1+\exp 3}=5\times 10^{-2}\rho_0.
\end{equation}
In other words, we assume that the main transfer takes place from densities of the order of 5\% saturation density
\begin{equation}\label{eq_est_38}
\mathcal{O}\approx \frac{\rho_A(R_0^A+3a)\rho_a(R_0^a+3a)}{\rho_0^2}=25\times 10^{-4}\approx 0.3 \times 10^{-2}.
\end{equation}
another estimate 
\begin{equation}\label{eq_est_39}
r=R_0+2.5a,
\end{equation}
for which
\begin{equation}\label{eq_est_40}
\rho(r=R_0+2.5a)=\frac{\rho_0}{1+\exp 2.5}\approx 0.76\times 10^{-1}\rho_0,
\end{equation}
leading to 
\begin{equation}\label{eq_est_41}
\mathcal{O}\approx 0.5 \times 10^{-2}.
\end{equation}
\section{Simple estimate of $\frac{(\mu_\alpha \mu_\beta)^{1/2}}{2\pi\hbar^2}$.}
Let us do it for the case of $^{120}$Sn+p $\longrightarrow ^{118}$Sn+t. In this case
\begin{equation}\label{eq_est_42}
\begin{split}
&\mu_\alpha=\frac{120}{121}M\approx M,\\
&\mu_\beta=\frac{118\times  3}{121} \approx 2.9 M.
\end{split}
\end{equation}
Thus
\begin{equation}\label{eq_est_43}
\begin{split}
\frac{\sqrt{\mu_\alpha\mu_\beta}}{2\pi\hbar^2}&\approx \frac{\sqrt{3}M}{2\pi\hbar^2}=\frac{\sqrt{3}}{2\pi 40 \text{MeV fm}^2}\\
&\approx \frac{1}{145}\times \text{MeV}^{-1}\times \text{fm}^{-2}
\end{split}
\end{equation}
\section{Simple estimate of $V_{ol}$}
In keeping with the assumption that transfer processes are expected to take place at the nuclear surface, the effective volume associate with such processes can be estimated to be
\begin{equation}\label{eq_est_44}
\begin{split}
V_{ol}&=\frac{4\pi}{3}(R^3-(R+a)^3)\\
&\approx \frac{4\pi}{3}3aR^2\approx \frac{4\pi}{3}(2\text{fm})R^2\\
&\approx \frac{8\pi}{3}(1.2A^{1/3})^2\text{fm}^3\\
&\approx 1.2 A^{1/3}\text{fm}^3\approx 260 \text{fm}^3 (A\approx 100)
\end{split}
\end{equation}
\section{Calculation of the (p,t) strength function}
An important component of the interaction which binds the dineutron halo of $^{11}$Li to the core $^9$Li is associated with the exchange, between the two neutrons of dipole (pigmy $1^-$ resonance of $^{11}$Li) and quadrupole ($2^+$ mode of $^9$Li) vibrations. Consequently, it is expected that resonant effects can be observed in the (p,t) strength function in which this neutrons are picked--up from $^{11}$Li. In keeping with the fact that successive transfer plays a central role in the two--particle pick up process, the corresponding transfer amplitude can be written as
\begin{equation}\label{eq_est_45}
\begin{split}
&\left\langle \chi^{(-)}\sum_{fF}\frac{\langle bB|U|fF\rangle\langle fF|U|aA\rangle}{E_{aA}-E_{fF}}\chi^{(+)}\right\rangle\\
&\approx \left\langle \frac{e^{iqr}}{\hbar \omega_L}\right\rangle\sim \left\langle e^{i(qr-\ln \tau/\omega_L )}\right\rangle\\
&\sim \cos\left(q(E_{CM})r-\ln \tau/\omega_L\right).
\end{split}
\end{equation}
Of notice that (p,t) strength function measurements can be viewed as a frequency dependent single Cooper pair transfer, and thus in some way connected to $\omega$--dependent Josephson supercurrent measurement.
\section{Relative importance of successive and simultaneous transfer and non-orthogonality corrections}


In what follows we discuss the relative importance of successive and simultaneous two-neutron transfer and of non-orthogonality 
corrections associated with the reaction 

\begin{equation}
\alpha \equiv  a(=b+2) + A \to b + B(=A+2) \equiv \beta 
\label{A1}
\end{equation}
in the limits of independent particles and of strongly correlated Cooper pairs, making use for simplicity of the semiclassical approximation (for details cf. \cite{Broglia:04a},\cite{Broglia:75}  and refs. therein), in which case the two-particle transfer differential cross section can be written as

\begin{equation}
\frac{d \sigma_{\alpha \to \beta} }{d \Omega} = P_{\alpha \to \beta} (t = +\infty) 
\sqrt{ \left( \frac{d \sigma_{\alpha}}{d \Omega} \right)_{el} }
\sqrt{ \left( \frac{d \sigma_{\beta}}{d \Omega} \right)_{el}}, 
\label{A2}
\end{equation}
where $P$ is the absolute value squared of a quantum mechanical transition amplitude. It gives the probability that the system at $t = + \infty$ is found in the final channel. The quantities $(d \sigma/d\Omega)_{el}$ are the classical elastic cross sections  in the center of mass system, calculated in terms of the deflection function, namely the functional relating the impact parameter and the scattering angle. 

The transfer amplitude can be written as  


\begin{equation}
a(t = + \infty) = a^{(1)}(\infty) - a^{(NO)}(\infty) + \tilde a^{(2)} ( \infty),
\label{A3}
\end{equation}
where $\tilde a^{(2)}(\infty)$ at $t= + \infty$ 
labels  the successive transfer amplitude expressed in the post-prior representation (see below).
The simultaneous transfer amplitude is given by (see Fig. A1(I))

\begin{eqnarray}
a^{(1)} (\infty) = \frac{1}{i \hbar} \int^{\infty}_{-\infty} dt (\psi^b \psi^B, (V_{bB} - <V_{bB}>) \psi^a \psi^A ) \times 
{\rm exp} [\frac{i}{\hbar} (E^{bB} - E^{aA}) t] \nonumber \\
\approx \frac{2}{i \hbar} \int^{\infty}_{- \infty}  dt \left( \phi^{B(A)} (S^B_{(2n)}; \vec r_{1A}, \vec r_{2A}), U(r_{1b}) 
e^{i (\sigma_1 + \sigma_2)}
\phi^{a(b)} (S^a_{(2n)}; \vec r_{1b}, \vec r_{2b}) \right) {\rm exp} [\frac{i}{\hbar} (E^{bB} - E^{aA}) t + \gamma(t)] 
\end{eqnarray}
where 
\begin{equation}
\sigma_1 + \sigma_2 = \frac{1}{\hbar} \frac{m_n}{m_A} ( m_{aA} \vec v_{aA} (t) + m_{bB} v_{bB}(t)) \cdot (\vec r_{1\alpha}
+ \vec r_{2 \alpha}),
\end{equation}
in keeping with the fact that ${\rm exp} ( i (\sigma_1 + \sigma_2))$ takes care of recoil 
effects (Galilean transformation associated with the mismatch between entrance and exit channels). 

The phase $\gamma (t)$ is related  with the effective $Q-$value of the reaction. In the above expression, $\phi$ indicates an antisymmetrized, correlated two-particle (Cooper pair)  wavefunction, $S(2n)$ being the two-neutron separation energy (see Fig. A1), $U(r_{1b})$ being the single particle potential generated by nucleus $b$ ($U(r) = \int d^3 r' \rho^b(r') v(|r-r'|)$). The contribution arising from non-orthogonality effects can be written as (see Fig. A1(II))

\begin{eqnarray}
a^{(NO)} (\infty) = \frac{1}{i \hbar} \int^{\infty}_{-\infty} dt (\psi^b \psi^B, (V_{bB} - <V_{bB}>) \psi^f \psi^F )
(\psi^f\psi^F, \psi^a \psi^A) 
{\rm exp} [\frac{i}{\hbar} (E^{bB} - E^{aA}) t]  \nonumber  \\
\approx \frac{2}{i \hbar} \int^{\infty}_{- \infty} \phi^{B(F)} (S^B_{(n)}, \vec r_{1A}), U(r_{1b}) 
e^{i \sigma_1}
(\phi^{f(b)}(S^f(n), \vec r_{1b})  \phi^{F(A)} (S^F(n),\vec r_{2A}) e^{i \sigma_2} \phi^{a(f)}(S^a(n),\vec r_{2b})) 
{\rm exp} [\frac{i}{\hbar} (E^{bB} - E^{aA}) t + \gamma(t)] ,
\end{eqnarray}
the reaction channel $f= (b+1) + F(=A+1)$ having been introduced, the quantity $S(n)$ being the one-neutron spearation 
energy (see Fig. A1). The summation over $f(\equiv a'_1,a'_2)$ and $F (\equiv a_1,a_2)$ involves a restricted number of states, namely the valence shells in nuclei $B$ and $a$.

The successive transfer amplitude  $\tilde a^{(2)}_{\infty}$ written making use of the post-prior representation is equal to 
(see Fig. A1(III))

\begin{eqnarray}
\tilde a^{(2)} (\infty) = \frac{1}{i \hbar} \int^{\infty}_{-\infty} dt (\psi^b \psi^B, (V_{bB} - <V_{bB}>) e^{i \sigma_1} \psi^f \psi^F ) \times 
{\rm exp} [\frac{i}{\hbar} (E^{bB} - E^{fF}) t + \gamma_1(t)] \nonumber  \\
\times \frac{1}{i \hbar} \int^{t}_{-\infty} dt' (\psi^f \psi^F, (V_{fF} - <V_{fF}>) e^{i \sigma_2} \psi^a \psi^A > \times 
{\rm exp} [\frac{i}{\hbar} (E^{fF} - E^{aA}) t' + \gamma_2(t)].
\end{eqnarray}

To gain insight into the  relative importance of the three terms contributing to Eq. \ref{A3} we discuss two situations, namely,
the independent-particle model and the strong-correlation limits.

\subsection{Independent particle limit}

In the independent particle limit, the two transferred particles do not interact among themselves but for antisymmetrization. 
Thus, the separation energies fulfill the relations (see Fig. A2)
\begin{equation}
S^B(2n) = 2 S^B(n) = 2S^F(n),
\end{equation}
and 
\begin{equation}
S^a(2n) = 2 S^a(n) = 2 S^f(n).
\end{equation}
In this case 
\begin{equation}
\phi^{B(A)} (S^B(2n), \vec r_{1A},\vec r_{2A}) = \sum_{a_1 a_2} \phi_{a_1}^{B(F)} (S^B(n),\vec r_{1A}) 
\phi_{a_{2}}^{F(A)} (S^F(n),\vec r_{2a}),
\end{equation}
and 
\begin{equation}
\phi^{a(b)} (S^a(2n), \vec r_{1b},\vec r_{2b}) = 
\sum_{a^{'}_{1} a^{'}_{2}} \phi_{a^{'}_1}^{a(f)} (S^a(n),\vec r_{2b}) 
\phi_{a^{'}_{2}}^{f(b)} (S^f(n),\vec r_{1b}),
\end{equation}
where $(a_1, a_2) \equiv F$ and $(a'_1, a'_2) \equiv f$ span, as mentioned above, shells in nuclei $B$ and $a$ respectively. 

Inserting (A9) and (A10) in (A4) one can show that 
\begin{equation}
a^{(1)} (\infty) = a^{(NO)}(\infty).
\end{equation}
It can be demonstrated  that within the present approximation, $Im \; \tilde a^{(2)} =0,$ and that 
\begin{eqnarray}
\tilde a^{(2)} (\infty) = \frac{1}{i \hbar} \int^{\infty}_{-\infty} dt (\psi^b \psi^B, (V_{bB} - <V_{bB}>) e^{i \sigma_1} \psi^f \psi^F > \times 
{\rm exp} [\frac{i}{\hbar} (E^{bB} - E^{fF}) t + \gamma_1(t)] \nonumber  \\
\times \frac{1}{i \hbar} \int^{\infty}_{-\infty} dt' (\psi^f \psi^F, (V_{fF} - <V_{fF}>) e^{i \sigma_2} \psi^a \psi^A ) \times 
{\rm exp} [\frac{i}{\hbar} (E^{fF} - E^{aA}) t' + \gamma_2(t)].
\label{A12}
\end{eqnarray}
The total absolute differential cross section \ref{A2}, where $P = |a(\infty)|^2 = |\tilde a^{(2)}|^2$, is then equal to the product of two one-particle transfer cross sections (see Fig. A3), associated with the (virtual) reaction channels
\begin{equation}
\alpha \equiv a+A \to f +F \equiv \gamma,
\end{equation}
and 
\begin{equation}
\gamma \equiv f +F \to b+B \equiv \beta.
\end{equation}

In fact, Eq.(\ref{A12}) involves no time ordering and consequently the two processes above are completely independent of each other. 
This result was expected because being $v_{12}= 0$, the transfer of one nucleon cannot influence, aside form selecting the
initial state for the second step, the behaviour of the other nucleon.


\subsection{Strong correlation (cluster) limit}

The second limit to be considered is the one in which the correlation betwen the two nucleons is so strong that (see Fig. A2)
\begin{equation}
S^a(2n) \approx S^a(n) >> S^f(n),
\label{A15}
\end{equation}
and 
\begin{equation}
S^B(2n) \approx S^B(n) >> S^F(n).
\label{A16}
\end{equation}
That is, the magnitude of the one-nucleon separation energy is strongly modified by the pair breaking.

There is a different , although equivalent way to express (\ref{A3}) which is the more convenient to discuss the strong coupling limit.
In fact, making use of the post-prior representation one can write
\begin{eqnarray}
a^{(2)}(t) = \tilde a^{(2)}(t) - a^{(NO)}(t) = 
%\nonumber \\ 
\frac{1}{i \hbar} \int^{\infty}_{-\infty} dt (\psi^b \psi^B, (V_{bB} - <V_{bB}>) e^{i \sigma_1} \psi^f \psi^F ) \nonumber \\ 
\times {\rm exp} [\frac{i}{\hbar} (E^{bB} - E^{fF}) t + \gamma_1(t)] \times \nonumber  \\
\frac{1}{i \hbar} \int^{t}_{-\infty} dt' (\psi^f \psi^F, (V_{aA} - <V_{aA}>) \psi^a \psi^A ) \times 
{\rm exp} [\frac{i}{\hbar} (E^{fF} - E^{aA}) t' + \gamma_2(t')].
\end{eqnarray}
The relations  (\ref{A15}), (\ref{A16}) imply 

\begin{equation}
E^{fF} - E^{aA} = S^a(n) - S^F(n) >> \frac{\hbar}{\tau},
\end{equation}
where $\tau$ is the collision time. Consequently the real part of $a^{(2)}(\infty)$ vanishes exponentially  with the $Q-$value of the intermediate transition, while the imaginary part  vanishes inversely proportional to this energy.
One can thus write,
\begin{equation}
Re \;  a^{(2)} (\infty) \approx 0,
\end{equation} 
and 
\begin{equation}
a^{(2)}(\infty) \approx \frac{1}{i \hbar} \frac{\tau}{<E^{fF}> - E^{bB}} 
\sum_{fF} (\psi^b \psi^B, 
(V_{bB}- <V_{bB}>) 
\psi^f\psi^F)_{t=0} \times 
(\psi^f\psi^F,(V_{aA} - <V_{aA}) \psi^a \psi^A)_{t=0},
\end{equation} 
where one has utilized the fact that $E^{bB} \approx E^{aA}$. For $v_{12} \to \infty$, $(<E^{fF}> - E^{bB}) \to \infty$
and, consequently, 

\begin{equation}
lim_{v_{12} \to \infty} a^{(2)} (\infty) = 0.
\end{equation} 

Thus the total two-nucleon transfer amplitude is equal, in the strong coupling limit, to the amplitude $a^{(1)} (\infty)$.


Summing up, only when successive transfer and non-orthogonal corrections are included in the description of the two-nucleon 
transfer process, does one obtain a consistent description of the process, which correctly converges to the weak and 
strong correlation limiting values.


\end{document} 