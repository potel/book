
 \chapter*{Preface}
The elementary modes of nuclear excitation are vibrations and rotations, single--particle (quasiparticle)  motion, and pairing vibrations and rotations. The specific reactions probing these modes are inelastic,  single-- and two-- particle transfer processes respectively. Within this context one can posit that nuclear structure (bound) and reactions (continuum) are but two aspects of the same physics. This is the reason why they can be treated on equal footing in terms of elementary modes of excitation, within the framework of nuclear field theory (NFT). This theory provides the rules to diagonalize in a compact and economic way the nuclear Hamiltonian for both bound and continuum states correcting for overcompletness of the basis (particle--vibration coupling (structure), non--orthogonality (reaction)), and for Pauli principle violation. 

Pairing vibrations and rotations, closely connected with nuclear superfluidity are, arguably,  paradigms of quantal nuclear phenomena. They thus play an important  role within the field of nuclear structure. It is only natural that two--nucleon transfer plays a similar role concerning direct nuclear reactions. In fact, this is the central subject of the present monograph.


At the basis of fermionic pairing phenomena one finds Cooper pairs, weakly bound, extended, strongly overlapping (quasi--) bosonic entities, made out of pairs of nucleons dressed by collective vibrations and interacting through the exchange of these vibrations as well as through the bare $NN$--interaction, eventually corrected by $3N$ contributions.
Cooper pairs not only change the statistics of the nuclear stuff around the Fermi surface and, condensing, the properties of nuclei close to their ground state. They also display a rather remarkable mechanism of tunnelling between  target and projectile in  direct two--nucleon transfer reaction. In fact, being weakly bound ($\ll \epsilon_F$, Fermi energy) they display correlations over distances (correlation length) much larger than nuclear dimensions ( $\gg R$, nuclear radius). On the other hand, Cooper pairs are forced to be confined within such dimensions by the action of the average potential, which can be viewed as an external field as far as these pairs are concerned.


The correlation length paradigm comes into evidence, for example, when two nuclei are set into weak contact in a direct reaction. In this case, each of the partner nucleons of a Cooper pair has a finite probability to be confined within the mean field of each of the two nuclei. It is then natural that a Cooper pair can tunnel, equally well correlated, between target and projectile, through simultaneous than through successive transfer processes. Consequently, although one does not expects supercurrents in nuclei, one can study long--range pairing correlations in terms of individual quantal state. The above mentioned weak coupling Cooper pair tunnelling reminds  the tunnelling mechanism of electronic Cooper pairs across a barrier (e.g. a dioxide layer) separating two superconductors, known as Josephson junction. The main difference is that, as a rule, in the nuclear time dependent junction efimerely established in  direct two--nucleon transfer process, only one or even none of the two weakly interacting nuclei are superfluid (or superconducting). Now, in nuclei, paradigmatic example of fermionic  finite many--body system, zero point fluctuations  (ZPF) in general, and those associated with pair addition and pair substraction modes known as pairing vibrations in particular, are much stronger than in condensed matter. Consequently, and in keeping with the fact that pairing vibrations are the nuclear embodiment of Cooper pairs in nuclei,   pairing correlations based on even  a single Cooper pair can lead to clearly observable effects in two--nucleon transfer processes. 


 Nucleonic Cooper pair tunnelling has played and is playing a central role in the probing of these subtle quantal phenomena, both in the case of  exotic nuclei as well as of nuclei lying along the stability valley, and have been instrumental in shedding light on the subject of pairing in nuclei at large, and on nuclear superfluidity in particular. Consequently, the subject of two--nucleon transfer occupies  a central place in the present monograph both concerning the conceptual and the computational aspects of the description of nuclear pairing, as well as regarding the quantitative confrontation of the results and predictions with the experimental findings.

Because of the central role the interweaving of the variety of elementary modes of nuclear excitation, namely single particle motion and collective vibrations play in nuclear superfluidity, the study of Cooper pair tunnelling in nuclei aside from requiring a consistent description of nuclear structure in terms of dressed quasiparticles and vibrations resulting from both bare and induced interactions, also involves  the description of one--nucleon transfer as well as knock out processes. Consequently, in the present monograph the general physical arguments and technical computational details concerning the   calculation of  absolute one--and two nucleon  transfer cross sections, making use of state of the art nuclear structure information, are discussed in detail. As a consequence, theoretical and experimental nuclear practitioners, as well as fourth year and PhD students can use the present monograph at profit. To make simpler this use, the basic nuclear structure formalism, in particular that associated with pairing and with collectives modes in nuclei, is economically introduced through general physical arguments. This is also in keeping with the availability in the current literature, of detailed discussions of such material.


 Within this context, the monographs \emph{Nuclear Superfluidity} by Brink and Broglia and \emph{Oscillations in Finite Quantum Systems}  by Bertsch and Broglia, published also by Cambridge University Press can be considered companion volumes to the present one.


Concerning the notation, we have divided each chapter into sections. Each subsection may in turn be broken down into subsections. Equations and Figures are identified by the number of the chapter and that of the section. Thus (6.1.33) labels the thirtythird equation of section 1 of chapter 6. Similarly, Fig. 6.1.2 labels the second figure of section 1 of chapter 6. Concerning the Appendices, they are labelled by the chapter number and by a Latin letter in alphabetical order, e.g. App. 6.A, App. 6.B, etc. Concerning equations and Figures, a sequential number is added. Thus (6.E.15) labels the fifteenth equation of Appendix E of chapter 6, while Fig. 6.F.1 labels the first figure of Appendix F of Chapter 6. References are referred to in terms of the author's surname and publication year and are found in alphabetic order in the bibliography.

Throughout, a number of footnotes are found. This is in keeping with the fact that footnotes can play a special role within the framework of an elaborated presentation. In particular, they allow one to emphasize relevant issues in an economic way. Being outside the main text, they give the possibility of stating eventual important results, without the need of elaborating on the proof. Within this context, in the paper  \cite{Born:26},  introducing the probabilistic interpretation of Schr\"odinger's  wavefunction, the fact that this probability is connected with its modulus squared and not with the wavefunction itself, is only explained in a footnote.



  Most of the material contained in this monograph have been the subject of lectures of the four year course ``Nuclear Structure Theory'' which RAB delivered throughout the years at the Department of Physics of the University of Milan, as well as at the Niels Bohr Institute and at Stony Brook (State University of New York). It was also presented by the authors in the course Nuclear Reactions held in the academic year 2009 at the PhD School of Physics of the University of Milan.


RAB wants to acknowledge the central role the collaboration with Francisco Barranco and Enrico Vigezzi has played concerning the nuclear structure aspects of the present monograph. Its debt with the late Aage Winther regarding the reaction aspects of the present volume is difficult to express in words. The overall contributions of Daniel B\`{e}s, Ben Bayman and Pier Francesco Bortignon are only too explicitly evident throughout the text and constitute a daily source of inspiration.  
\begin{flushleft}
Gregory Potel Aguilar\\
 Livermore
\end{flushleft}
\vspace{-1.7cm}
\begin{flushright}
Ricardo A. Broglia\\
 Milano
\end{flushright}


\bibliographystyle{abbrvnat}
%\bibliography{C:/Gregory/Broglia/notas_ricardo/nuclear_bib} 
 \bibliography{C:/Gregory/book/nuclear_bib}
