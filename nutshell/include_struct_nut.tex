\chapter{Pairing with transfer}\label{chapter1}
\section{Nuclear Structure in a nutshell}\label{C1S1}
The low--energy properties of the finite, quantal, many--body nuclear system, in which nucleons interact through the strong force of strength $v_0(\approx -100$ MeV) and range $a(\approx 1$ fm) are controlled, in first approximation, by independent particle motion. This is a consequence of the fact that nucleons display a sizable value of the zero point (kinetic) energy of localization ($\hbar^2/Ma\approx 40$ MeV) as compared to the absolute value of the strength of the $NN$--potential\footnote{The corresponding ratio $q=\left(\frac{\hbar^2}{Ma^2}\right)\frac{1}{|v_0|}$ is known as the quantality parameter and was first used in connection with the study of condensed matter (\cite{deBoer:48,deBoer:57,deBoer:48b,Nosanow:76}). It was introduced in nuclear physics in \cite{Mottelson:02} where its value $q=0.4$ testifies to the validity of independent particle motion. It is f notice that questions like the one posed in connection with localization and long mean free path were already discussed by \cite{Lindemann:10} in connection with the study of the stability or less of crystals. The generalization to aperiodic crystals, like e.g. proteins (\cite{Schrodinger:44}) was carried out in \cite{Stillinger:90}. Its possible application to the atomic nucleus is discussed in App. \ref{C2AppC} } $|v_0|=100$ MeV 

The corresponding ground state $\ket{HF}=\Pi_ia^\dagger_i\ket{0}$ describes a step function in the probability of the occupied ($\epsilon_i\leq \epsilon_F$) and empty ($\epsilon_k>\epsilon_F$) states, displaying a sharp discontinuity at the Fermi energy. Pushing the system it reacts with an inertia $AM$, sum of the nucleon masses (App. \ref{App1.D}). Setting it into rotation, assuming the density $\rho(r)=\sum_i|\braket{\mathbf r|i}|^2$ ($\ket{i}=a^\dagger_i\ket{0}$) to be spatially deformed, it responds with the rigid moment of inertia. This is because the single--particle orbitals are solidly anchored to the mean field (Fig. \ref{fig1A3}).

Pairing acting on nucleons moving in time reversal states $\nu,\bar\nu$ ($\nu\equiv(nlj)$), in configurations of the type ($(l)^2_{L=0},(s)^2_{S=0}$), and lying close to the Fermi energy $\epsilon_F(\approx 36$ MeV), alter this picture in a conspicuous way\footnote{\cite{Bohr:58}; for a recent compilation of ongoing research in the field see \cite{Broglia:13}.}. Within an energy range of the order of the absolute value of the pair correlation energy\footnote{In BCS, $E_{corr}\approx\frac{N(0)}{2}\Delta^2$, where $N(0)=\frac{g}{2}$ is the density of states at the Fermi energy and for one spin orientation, $g_i=i/16$ MeV$^{-1}$ ($i=N,Z$) being the result of an empirical estimate which takes surface effects into account (\cite{Bohr:75,Bortignon:98}), while $\Delta$ is the pairing gap. For a typical superfluid, quadrupole deformed nucleus like $^{170}$Yb, $N(0)=5.3$ MeV$^{-1}$, $\Delta\approx1.1$ MeV and $E_{corr}=-3.2$ MeV (\cite{Shimizu:89}).} $|E_{corr}|(\approx 3 $MeV) centered around $\epsilon_F$ ($|E_{corr}|/\epsilon_F\ll1$), the role of independent particles is taken over by independent pairs of nucleons, correlated distances $\xi\approx\hbar v_F/(\pi\Delta)\,(\approx 17$ fm), which flicker in and out of the corresponding $L=0, S=0$ configuration (Cooper pairs\footnote{\cite{Cooper:56}.}$^{,}$\footnote{\cite{Brink:05}.}).


For intrinsic\footnote{As opposed to collective excitations, excitations which do not alter the temperature of the system.} nuclear excitation energies and rotational frequencies\footnote{Coriolis force acts oppositely on each member of a Cooper pair. When the difference in rotational energy between superfluid and normal rotation becomes about equal to the correlation energy, the nucleon moving opposite to the collective rotation becomes so much retarded in its revolution period with respect to the partner nucleon, that eventually it cannot correlate any more with it and ``align'' its motion (and spin) with the rotational motion, becoming again a pair of fermions and not participating any more in the condensate. This happens for a (critical) angular momentum $I_c\approx(120\times|E_{corr}|)^{1/2}\approx 20\hbar$, corresponding to a rotational frequency $\hbar\omega_c\approx 0.5$ MeV (see \cite{Bohr:75}, \cite{Brink:05} and references therein).} sensibly smaller than $|E_{corr}/2|$ and $\hbar\omega_{rot}\approx0.5$ MeV respectively, the system can be described in terms of independent pair motion. This is a consequence of the fact that the kinetic energy of (Cooper) pair confinement ($\hbar^2/(2M\xi^2)\approx 10^{-2}$ MeV), is much smaller than the absolute value of the pair binding energy $|E_{corr}|$, implying that each pair behaves as an entity\footnote{The ratio $q_\xi=\frac{\hbar^2}{2M\xi^2}\frac{1}{|E_{corr}|}\approx 0.007$ provides a generalized quantality parameter. It testifies to the stability of nuclear Cooper pairs in superfluid nuclei.} of mass $2M$ and spin $S=0$. Cooper pairs respect Bose--Einstein statistics, the single--particle orbits on which they correlate become dynamically detached from the mean field, leading to a bosonic--like condensate. This has a number of consequences. In particular, the moment of inertia $\mathcal J$ of quadrupole rotational bands of superfluid nuclei with open shells of both protons and neutrons is found to be smaller than the rigid moment of inertia by a factor of 2. The observed values, however, are a factor of 5 larger than the irrotational moment of inertia\footnote{\cite{Bohr:75,Belyaev:59,Belyaev:13}.}, testifying to a subtle interplay between pairing and shell effects.
\begin{figure}
\centerline{\includegraphics*[width=\textwidth,angle=0]{nutshell/figs/Excited0Pb206tp.pdf}}
\caption{(a) Ratio of the absolute $L=0$ differential cross sections $d\sigma(E_x,\theta=59^{\circ})/d\sigma(gs,\theta=59^{\circ})$ (=(0.05 mb/sr)/(0.12 mb/sr)) below 5 MeV  for the reaction $^{206}$Pb$(t,p)^{208}$Pb at the second minimum ($\theta=59^{\circ}$; \cite{Bjerregaard:66b}). It is of notice the large experimental errors of the corresponding angular distributions associated with the poor statistics of the cross section at the first maximum $\theta=5^{\circ}$. This is the reason why the maximum at $59^\circ$ was preferred to report the ratio of the cross sections. (b) Schematic representation of the pairing vibrational spectrum around $^{208}$Pb. Also shown is a cartoon representation of the softening of the sharp mean field Fermi surface due to the ZPF of thr pairing vibrational modes. The label $a$ and $r$ indicate the pair addition and pair removal modes. It is to be noted that a linear term in $N$ has been added to the binding energy to make the binding energy values associated with $^{206}$Pb ($N=124$) and $^{210}$Pb ($N=128$), equal, in an attempt to emphasize a harmonic picture for the two--phonon state. Concerning the anharmonicities of the modes cf. last paragraph Sect. \ref{App1E}.}\label{fig1.1}
\end{figure}



 Cooper pairs exist also in situations in which the environmental conditions are above critical. For example, in metals at room temperature, in closed shell nuclei as well as in deformed open shell ones at high values of the angular momentum. However, in such circumstances, they break essentially as soon as they are generated (pairing vibrations). While these pair addition and substraction fluctuations have little effect in condensed matter systems with the exception than at\footnote{See \cite{Schmidt:68}, \cite{Schmid:69} \cite{Abrahams:68}; concerning superfluid $^3$He see \cite{Wolfe:78}.} $T\approx T_c$, (critical normal--superconducting temperature) they play an important role in normal (non--superfluid) nuclei. In particular in nuclei around closed shells (Fig. \ref{fig1.1}), and specially in the case of light, highly polarizable, exotic halo nuclei\footnote{See Sects. \ref{App1E} and \ref{App1AF}; \cite{Bohr:75} , \cite{Bes:66}, \cite{Hogassen:61}, \cite{Schmidt:72}, \cite{Schmidt:68}, \cite{Barranco:01}, \cite{Potel:13}, \cite{Potel:14}.}. From this vantage point one can posit that it is not so much, or, at least not only, the superfluid phase which is abnormal in the nuclear case, but the normal state around closed shell systems\footnote{See \cite{Potel:13} and refs. therein. Also \cite{Potel:13b} in connection with the closed shell system $^{132}$Sn.}. In particular in connection with the self--energy of nucleons moving around closed shells\footnote{See e.g. \cite{Bes:71,Bes:71b,Bes:71c}.}.It is of notice nonetheless, the role pairing vibrations play in the  transition between superfluid and normal nuclear phases (cf. Fig. \ref{fig1.2}) as a function of the rotational frequency (angular momentum) as emerged from the experimental studies of high spin states carried out by, among others, Garrett and collaborators\footnote{See \cite{Shimizu:89}; see also \cite{Barranco:87b}, see Ch. 6 of \cite{Brink:05}.}.
 
 
  From Fig. \ref{fig1.2} it is seen that while the dynamic pairing gap associated with pairing vibrations leads to a $\approx$ 20\% increase of the static pairing gap for low rotational frequencies, it becomes the overwhelming contribution above the critical frequency\footnote{\cite{Shimizu:89}, \cite{Shimizu:90}, \cite{Shimizu:13},  \cite{Donau:99} \cite{Shimizu:00}.}. In any case, the central role played by pairing vibrations within the present circumstances is that to restore particle--number conservation, another example after that provided by the quantality parameter and by its generalization to pair motion, of the fact that potential functionals are, as a rule, best profited by special arrangements of fermions (spontaneous symmetry breaking), while fluctuations favour symmetry\footnote{\cite{Anderson:84,Anderson:76}.}. 
  
  
  Within this context, there are a number of methods which allows one to go beyond mean--field approximation (HFB). Generally referred to as number projection methods\footnote{cf. \cite{Ring:80}, \cite{Egido:13}, \cite{Robledo:13}; cf. also \cite{Frauendorf:13}, \cite{Ring:13}, \cite{Heenen:13}, and references therein.}(NP), they make use of a variety of techniques (Generator Coordinate Method, Pfaffians, etc.) as well as protocols (variation after projection, gradient method, etc.). The advantages of NP methods over the RPA is to lead to smooth functions for both the correlation energy and the pairing gap at the pairing phase transition between normal and superfluid phases. That is, between the pairing vibrational and pairing rotational schemes\footnote{Figs. \ref{fig1.1}, \ref{fig1.3}, \ref{fig1.4}, see also Fig. \ref{fig1D1} and Sects. \ref{C1AppDS2} and \ref{App1E}; cf. \cite{Bes:66}, \cite{Bohr:75} and references therein.}.
  \begin{figure}
  \centerline{\includegraphics*[width=\textwidth,angle=0]{nutshell/figs/fig1_1_2.pdf}}
  \caption{Pairing gap calculated taking into account the correlation associated with pair vibrations in the RPA approximation $(\Delta=(\Delta^2_{BCS}+\tfrac{1}{2}G^2S_0(RPA))^{1/2})$ (upper panel) and RPA correlation energy (lower panel) for neutrons in $^{164}$Er as a function of the rotational frequency (\cite{Brink:05}, Sect. 6.6). Both quantities are in MeV (dashed--dotted curves). The value of the static (mean--field) pairing gap $\Delta$, which vanishes at $\hbar \omega_{rot}=0.34 $ MeV, is also displayed in the upper panel (continuous curve). The results of the number--projection (NP) calculations are shown as dotted curves.  $S_0$ (RPA)= $\sum_{n \neq AGN} \left[<n|P |0>  + <n|P^{\dagger} |0>\right]^2_{RPA}$ ,
    where \mbox{$\Delta_{BCS} = G<⟨BCS|P^{\dagger}|BCS>$} is the standard, static BCS pairing gap,
    while $G$ is the pairing force strength. The non-energy weighted sum rule $S_0 (RPA)$
    describes the contribution of pairing fluctuations to the effective (RPA) gap,
    and is intimately associated with projection in particle number. It is of notice
    that $\sum_{n \neq AGN}$ means that the divergent contribution from the zero energy mode
    (Anderson, Goldstone, Nambu mode, see e.g. \cite{Broglia:00} and references therein), associated with the lowest ($\hbar \omega_0$) solution
    of the $H = H'_{p} +H''_p$ is to be excluded (see Sect \ref{C3AppD}, discussion before Eq. (\ref{eqbeta}) as well as \cite{Brink:05} App. J). After \cite{Shimizu:90}.}\label{fig1.2}
  \end{figure}
  
The above results underscore the fact that, at the basis of an operative coarse grained approximation to the nuclear many--body problem (within this context cf. App. \ref{App1.D}, in particular the discussion following Eq. \ref{eq1.D.5x}), one finds a judicious choice of the collective coordinates\footnote{In this connection, we quote allegedly from S. Weinberg: ``In solving a problem you may choose to use the degrees of freedom you like. But if you choose the wrong ones you will be sorry''.}. In other words, pairing vibrations are elementary modes of excitation containing the right physics to restore gauge invariance through their interweaving with  quasiparticle states.
  \begin{figure}
  \centerline{\includegraphics*[width=\textwidth,angle=0]{nutshell/figs/ExcitedSn122pt.pdf}}
  \caption{Excitation function associated with the reaction$^{122}$Sn$(p,t)^{120}$Sn$(J^\pi)$. The absolute experimental values of $d\sigma(J^\pi)/d\Omega|_{5^\circ}$ are given as a function of the excitation energy $E_x$ (after \cite{Guazzoni:11}). In the inset the full neutron pairing rotational band between magic numbers $N=50$ and $N=82$ is also displayed, the absolute $^{A+2}$Sn ($p,t$) $^{A}$Sn experimental cross sections are reported in the abscissa (\cite{Guazzoni:99}, \cite{Guazzoni:04}, \cite{Guazzoni:06}, \cite{Guazzoni:08}, \cite{Guazzoni:11}, \cite{Guazzoni:12}; see also \cite{Potel:11}, \cite{Potel:13b}).}\label{fig1.3}
  \end{figure}
  \begin{figure}
  \centerline{\includegraphics*[width=\textwidth,angle=0]{nutshell/figs/fig2_1_4.pdf}}
  \caption{The weighted average energies ($E_{exc}=\sum_i E_i \sigma_i/\sum_i \sigma_i$) of the excited $0^+$ states below 3 MeV in the Sn isotopic chain are shown on top of the pairing rotational band, already displayed in Fig. \ref{fig1.3}. Also indicated is the percentage of cross section for two--neutron transfer to excited states, normalized to the cross sections populating the ground states (after \cite{Potel:13b}). The estimate of the ratio of cross sections displayed on top of the figure was obtained making use of the single $j$--shell model (see, e.g., \cite{Brink:05} and references therein).}\label{fig1.4}
  \end{figure}
Within the framework of the above picture, one can introduce at profit a collective coordinate $\alpha_0$ (order parameter; see Sect. \ref{C6S2.3}) which measures the number of Cooper pairs participating in the pairing condensate, and define a wavefunction for each pair $\left(U'_\nu+V'_\nu a'^\dagger_\nu a'^\dagger_{\bar\nu}\right)|0\rangle$ (independent pair motion, BCS approximation, see Figs. \ref{fig1D1}, \ref{fig1D2} and \ref{fig1D3}), adjusting the occupation parameters $V_\nu$ and $U_\nu$ (probability amplitudes that the two--fold, Kramer's--degenerate pair state ($\nu,\bar{\nu}$), is either occupied or empty), so as to minimize the energy of the system under the condition that the average number of nucleons is equal to $N_0$ (the Coriolis--like force felt, in the intrinsic system in gauge space by the Cooper pairs, being equal to $-\lambda N_0$). Thus, $|BCS\rangle=\Pi_{\nu>0}\left(U'_\nu+V'_\nu a'^\dagger_\nu a'^\dagger_{\bar\nu}\right)|0\rangle$ provides a valid description of the independent pair mean field ground state, and of the associated order parameter $\alpha'_0=\langle BCS|P'^{\dagger}|BCS\rangle=\sum_{\nu>0}U'_\nu V'_\nu,\; P'^{\dagger}=\sum_{\nu>0}a'^\dagger_\nu a'^\dagger_{\bar \nu}$ being the pair creation operator\footnote{cf. \cite{Bardeen:57a}, \cite{Bardeen:57b}, \cite{Schrieffer:64}, \cite{Schrieffer:73} and references therein.}.
It is then natural to posit that two--particle transfer reactions are specific to probe pairing correlations in many--body fermionic systems. Examples are provided by the Josephson effect\footnote{\cite{Josephson:62}.} between e.g. metallic superconductors, and ($t,p$) and ($p,t$) reactions in atomic nuclei\footnote{cf. e.g. \cite{Yoshida:62}, \cite{Broglia:73}, \cite{Bayman:71}, \cite{Glendenning:65}, \cite{Bohr:64}, \cite{Hansen:12} and \cite{Potel:13} and references therein; cf. also Figs. \ref{fig1.1}, \ref{fig1.3} and \ref{fig1.4}.}.

Within this context we now take the basic consequence of pairing condensation in nuclei regarding reaction mechanisms. For this purpose let us consider a \textit{gedanken experiment} in which the superfluid target and the projectile can at best come in such weak contact that only  single--nucleon transfer leads to a yield falling within the sensitivity range of the measuring setup. Because $\left(\hbar^2/2M\xi^2\right)/|E_{corr}|\approx10^{-2}$, Cooper pairs in superfluid nuclei behave as particles of mass $2M$
 over distances $\xi$, even in the case in which the $NN$--potential vanishes in the zone between the weakly overlapping densities of the two interacting nuclei. One then expects Cooper pair transfer to be observed. Not only. One also expects that the associated absolute differential cross section contains, for the particular choice of mass number made and within the framework of the theory of quantum measurement, all the information needed to work out a comprehensive description of nuclear superfluidity.
 
 Because $\alpha_0\sim N(0)$, cross sections associated with the transfer of Cooper pairs between members of a pairing rotational band, are proportional to the density of single--particle levels quantity squared. As a consequence, absolute two--nucleon transfer cross sections are expected to be of the same order of magnitude than one--nucleon transfer ones, and to be dominated by successive transfer (see Sects. \ref{C3S2} and \ref{C3S3}). These expectations heve been confirmed experimentally and by detailed numerical calculations, respectively.
 The above parlance, being at the basis of the Josephson effect, reflects both one of the most solidly established results in the study of BCS pairing, and explains the workings of a paradigmatic probe of spontaneous symmetry breaking phenomena.
 
 
 
Due to the fact that, away from the Fermi energy pair  motion becomes independent particle motion (see Sect. \ref{App1D}), one--particle transfer reactions like e.g. ($d,p$) and ($p,d$) can be used together with ($t,p$) and ($p,t$) processes, as  valid tools to cross check pair correlation predictions (see Chapter \ref{C6}). In particular, to shed light on the origin of pairing in nuclei: in a nutshell, the relative importance of the bare $NN$--interaction and the induced pairing interaction (within this context see Sect. \ref{App1AF} and Fig. \ref{fig1.9.1}).

While the calculation of two--nucleon transfer spectroscopic amplitudes and differential cross sections are, a priori, more involved to be worked out than those associated with one--nucleon transfer reactions, the former are, as a rule, more ``intrinsically'' accurate than the latter ones. This is because, in the case of two nucleon transfer reactions, the quantity (order parameter $\alpha'_0$) which expresses the collectivity of the members of a pairing rotational band, reflects the properties of a coherent state ($|BCS\rangle$). In other words, it results from the sum over many contributions ($\sqrt{j_{\nu}+1/2}\,U'_\nu V'_\nu$, see Sect. \ref{App1D}, also Sect. \ref{C6S2.3}, all of them having the same phase. Consequently, the relative error decreases as the square root of the number contributions $(\approx N(0)\Delta\approx 4\,\text{MeV}^{-1}\times 1.4\,\text{MeV}\approx 6$ in the case of the superfluid nucleus $^{120}$Sn). 

There is a further reason which confers $\alpha'_0=\sum_j(j+1/2)U'_jV'_j$ a privileged position with respect to the single contributions $(j+1/2)U'_jV'_j$. It is the fact that $\alpha'_0=e^{2i\varphi}\sum_j(j+1/2)U_jV_j=e^{2i\varphi}\alpha_0$ defines a privileged orientation in gauge space, $\alpha_0$ being the order parameter referred to the laboratory system which makes an angle $\varphi$ in gauge space with respect to the intrinsic system to which $\alpha'_0$ is referred\footnote{See Sect. \ref{C1AppDS2}, see \cite{Potel:13b}.}. In other words, the quantities $\alpha'_0$ which measure the deformation of the superfluid nuclear system in gauge space, and the rotational frequency $\lambda=\hbar\dot\varphi$ in this space, and associated Coriolis force $-\lambda N_0$ felt by the nucleons referred to the body fixed frame, are the result of solving selfconsistently the BCS number and gap equations
$N_0=\sum_j(2j+1)\left(1-\frac{(\epsilon_j-\lambda)/\Delta}{\sqrt{1+\left(\frac{\epsilon_j-\lambda}{\Delta}\right)^2}}\right)$ and $\alpha'_0=\sum_j(j+1/2)U'_jV'_j=\sum_j(j+1/2)\left(1-1/\sqrt{1+(\frac{\epsilon_j-\lambda}{\Delta})^2}\right)$ making use as inputs $\epsilon_\nu$ and $ N_0$, that is single--particle energies and the average number of particles.


Similar arguments can be used regarding the excitation of pairing vibrations in terms of Cooper pair transfer from closed shells as compared to one--particle transfer. As seen from Fig. \ref{fig2.1.5} (b)--(c), the random phase approximation (RPA) amplitudes $X_\nu^a$ and $Y^a_\nu$ sum coherently over pairs of time reversal states\footnote{\cite{Brink:05} Ch. 5.} to give rise to the spectroscopic amplitudes associated with the direct excitation of the pair addition mode displayed in (d). Because of the (dispersion) relation (b)+(c)$\equiv$(d), the $X_\nu$-- and $Y_\nu$--amplitudes are correlated, among themselves as well as in phase. As seen from (g) and (h), the situation is quite different in the case of one--particle transfer. 
\begin{figure}[h!]
\centerline {
\includegraphics*[width=15cm]{nutshell/figs/fig2_1_5}
}
\caption{NFT diagrams associated with one-- and two--particle transfer from closed shell. (a) ZPF associated with the virtual excitation of a pair addition mode and two uncorrelated holes. (b) two--particle transfer filling the holes, (c) diagram obtained from the previous one by time ordering. These processes receive contribution from all $(\nu,\bar\nu)$ pairs (sum over $\nu>0$), leading to (d), the direct excitation of the pair addition mode. The relation (b)+(c)$\equiv$(d) is the NFT graphical representation of the random phase approximation (RPA) dispersion relation used to calculate the properties of the pair addition mode in the harmonic approximation (Section \ref{App1E}). The backwards and forwards going RPA amplitudes are displayed in Figs. (e) and (f) respectively. (g) One--particle stripping proceeding through the filling of a hole associated with the ZPF, (h) processes obtained from the previous by time ordering.}
\label{fig2.1.5}
\end{figure}
The soundness of the above parlance  reflects itself in the calculation of the elements resulting from the encounter of structure and reaction, namely one-- and two--nucleon modified transfer formfactors. While it is usually considered that these quantities carry all the structure information associated with the calculation of the corresponding cross sections, a consistent NFT treatment of structure and reaction will posit that equally much is contained in the distorted waves describing the relative motion of the colliding systems. This is because the state dependent components of the optical potential  which determines the scattering waves, emerges from the same elements, eventually including also inelastic transition densities, used in the calculation of the structure properties\footnote{See Sect. \ref{App2B}; cf also \cite{Broglia:81b}, \cite{Pollarolo:83}, \cite{Broglia:04a}, \cite{Fernandez:10}, \cite{Fernandez:10b}, \cite{Dickhoff:05}, \cite{Jenning:11}, \cite{Montanari:14} \cite{Barbieri:05,Dickhoff:17,Rotureau:17}.}. In other words, to describe a two--nucleon transfer reaction like $A+t\rightarrow B(=A+2)+p$, one needs to know what the single--particle states and collective modes of the system $F(=A+1)$ are, equally well than those of nuclei $A$ and $B$. In principle, also  the deuteron wavefunction as one knows the triton wavefunction (see Chapter \ref{chapter2} Sect. \ref{C3S2}, see also Chapter \ref{C7}, Section \ref{C7S1}). Furthermore one needs to take into account the interweaving of different modes and degrees of freedom resulting in   dressed particle states (quasiparticles; fermions) and renormalized normal vibrational modes of excitation (bosons). But these are essentially all the elements needed to calculate the processes leading to the depopulation of e.g. the flux in the incoming channel ($A+t$ in the case under discussion). In particular, and assuming to work with spherical nuclei, one--particle transfer is, as a rule, the main depopulation process\footnote{Again, one is referring to doorway states processes \cite{Feshbach:58}.}. This is a consequence of the long range tail of the associated formfactors as compared to that of other processes, e.g. inelastic processes (see e.g. Fig. \ref{fig_4}).


In keeping with this fact, and because $U$ and $W$ are connected by the Kramers--Kr\"onig generalized dispersion relation\footnote{See e.g. \cite{Mahaux:85} and references therein.}, it is possible to calculate the nuclear dielectric function (optical potential) associated with the elastic channels under discussion (i.e. ($A,t$), ($F,d$) and ($B,p$) in the present case) by first calculating  $W$ --which only involves on--shell contributions-- making use of the above described elements, and from its knowledge work out $U$.

\subsection{Two--nucleon modified formfactors}\label{S2.1.1}
Concerning the modified formfactor associated with e.g. a ($t,p$) process, we shall see in the (Chapter \ref{C7}, Sect. \ref{C7AppB}) that it can be written as
\begin{equation}\label{eq1.1.1}
\begin{split}
u_{LSJ}^{J_iJ_f}(R)&=\sum_{\substack{n_1l_1j_1\\n_2l_2j_2,n}}B(n_1l_1j_1,n_2l_2j_2;JJ_iJ_f)\\
\times\langle SLJ|j_1j_2J\rangle &\times \langle n0, NL,L|n_1l_1,n_2l_2;L\rangle\\
&\times\Omega_n R_{NL}(R),
\end{split}
\end{equation}
where the overlaps
\begin{equation}\label{eq2C2.1}
\begin{split}
B&(n_1l_1j_1,n_2l_2j_2;JJ_iJ_f)\\
=&\langle \Psi^{J_f}(\xi_{A+2})|\left[\phi^J(n_1l_1j_1,n_2l_2j_2),\Psi^{J_i}(\xi_A)\right]^{J_f}\rangle,
\end{split}
\end{equation}
and 
\begin{equation}\label{eq1.1.3}
\Omega_n=\langle \phi_{nlm_l}(\mathbf r)|\phi_{000}(\mathbf r)\rangle,
\end{equation}
encode for the physics of particle--particle (but also, to a large extent, particle--hole) correlations in nuclei, $\langle SLJ|j_1j_2J\rangle$ and $\langle n0,NL,L|n_1 l_1,n_2l_2;L\rangle$ being $LS-jj$ and Moshinsky transformation brackets, keeping track of symmetry and number of degrees of freedom conservation\footnote{\cite{Glendenning:65}, \cite{Broglia:73}.}. In fact, the two--nucleon spectroscopic amplitude (B--coefficient) and the overlap $\Omega_n$ reflect the parentage with which the nucleus $B$ can be written in terms of the system $A$ and a Cooper pair,
\begin{equation}
\Psi_{exit}=\Psi_{M_f}^{J_f}(\xi_{A+2})\times\chi^{S_f}_{M_{S_f}}(\sigma_p),
\end{equation}
where
\begin{equation}
\begin{split}
\Psi_{M_f}^{J_f}(\xi_{A+2})&=\sum_{\substack{n_1l_1j_1\\n_2l_2j_2\\J,J'_i}}B(n_1l_1j_1,n_2l_2j_2;JJ'_iJ_f)\\
&\times\left[\phi^J(n_1l_1j_1,n_2l_2j_2)\Psi^{J'_i}(\xi_A)\right]_{M_f}^{J_f},
\end{split}
\end{equation}
and
\begin{equation}
\Psi_{entrance}=\Psi_{M_i}^{J_i}(\xi_A)\times\phi_t(\mathbf r_{n1},\mathbf r_{n2},r_p;\sigma_{n1},\sigma_{n2},\sigma_p),
\end{equation}
with
\begin{equation}\label{eq1.1.7}
\phi_t=\left[\chi^S(\sigma_{n1},\sigma_{n2})\chi^{S'_f}(\sigma_p)
\right]_{M_{S_i}}^{S_i}\times\phi_t^{L=0}\Big(\sum_{i>j}|\mathbf r_i-\mathbf r_j|\Big).
\end{equation}
Assuming for simplicity a symmetric di--neutron radial wavefunction for the triton  (i.e. neglecting the $d$--component of the corresponding wavefunction) both for the relative and for the center of mass wavefunctions $\phi_{nlm}(\mathbf{r})$ and $\phi_{N\Lambda M}(R)$ ($n=l=m=0, N=\Lambda=M=0$), leads to $\Omega_n$, a quantity which reflects both the non--orthogonality existing between the di--neutron wavefunctions in the final nucleus (Cooper pair) and in the triton as well as their degree of $s$--wave of relative motion. Another way to say the same thing is to state that dineutron correlations in these two systems are different, a fact which underscores the limitations of  light ion reactions to probe specifically pairing correlations in nuclei\footnote{Within this context see  \cite{vonOertzen:01}, \cite{Oertzen:13}.}.


One can then conclude that, provided one makes use of a (sensible) complete single--particle basis (eventually including also the continuum), one can capture through $u_{LSJ}^{J_iJ_f}(R)$ most of the coherence of Cooper pair transfer, as a major fraction of the associated di--neutron non--locality is taken care of by the n--summation appearing in Eq. (\ref{eq1.1.1}), the different contributions being weighted by the non--orthogonality overlaps $\Omega_n$. This is in keeping with the fact that, making use of a more refined triton wavefunction than that employed above, the $n-p$ (deuteron--like) correlations of this particle can be described with reasonable accuracy and thus, the emergence of successive transfer (see Chapter \ref{chapter2}). On the other hand, being the deuteron a bound system, this effective treatment of the associated resonances is not particularly economic. Furthermore, it is of notice that the zero--range approximation ($V(\rho)\phi_{000}(\rho)=D_0\delta(\vec \rho)$) eliminates the above mentioned possibilities (cf. Eq. (\ref{5lec19})).

Anyhow, the fact that one can still work out a detailed and physically insightful picture of two--nucleon transfer reactions in nuclei in terms of absolute cross sections with the help of a single parameter ($D_0^2\approx(31.6\pm 9.3)10^4 \text{MeV}^2\text{fm}^2$) testifies to the fact that the above picture of Cooper pair transfer\footnote{\cite{Glendenning:65}, \cite{Bayman:67}.} is a useful one, as it contains an important fraction of the physics which is at the basis of Cooper pair transfer in nuclei\footnote{\cite{Broglia:73}.}. This is in keeping with the fact that the Cooper pair correlation length is much larger than nuclear dimensions and, consequently, simultaneous and successive transfer feel the same pairing correlations (see Chapter \ref{chapter2}). In other words, treating explicitly the intermediate deuteron channel in terms of successive transfer, correcting both this and the simultaneous transfer channels for non--orthogonality contributions, makes the above picture the quantitative probe of Cooper pair correlations in nuclei\footnote{\cite{Bayman:82} and \cite{Potel:13}.} (Fig. \ref{fig1.5}).


 Within the above context, we provide below two examples of $B$--coefficients associated with coherent states. Namely, one for the case in which $A$ and $B(=A+2)$ are members of a pairing rotational band. A second one, in the case in which they are members of a pairing vibrational band. That is, 
\begin{equation}\label{eqC21.8}
\begin{split}
\mathbf{1)}, B(nlj,nlj;000)=&\langle BCS(N+2)|\frac{[a^\dagger_{nlj}a^\dagger_{nlj}]^0_0}{\sqrt{2}}|BCS(N)\rangle\\
&=\sqrt{j+1/2}\,U_{nlj}(N)V_{nlj}(N+2),
\end{split}
\end{equation}
and
\begin{equation}
\begin{split}
\mathbf{2)}, B(nlj,nlj;000)=&\langle (N_0+2)(gs)|\frac{[a^\dagger_{nlj}a^\dagger_{nlj}]^0_0}{\sqrt{2}}|N_0(gs)\rangle \\
&\\
&=\left\{\begin{array}{c}
 \sqrt{j_k+1/2}\;\;X^a(n_kl_kj_k)\quad (\epsilon_{j_k}>\epsilon_F) \\ 
\sqrt{j_k+1/2}\;\;Y^a(n_il_ij_i)\quad (\epsilon_{j_k}\leq\epsilon_F).
\end{array} \right.
\end{split}
\end{equation}
Where the $X$ and $Y$ coefficients are the forwardsgoing and backwardsgoing RPA amplitudes of the pair addition mode\footnote{\cite{Brink:05}.}.
For actual numerical values see Sect. \ref{App1D}, Table \ref{tab1D1} and Sect. \ref{App1E} Tables \ref{tab1E2}--\ref{tab1E5}.


We conclude this section by remarking that, in spite of the fact that one is dealing with the connection between structure and direct transfer reactions, no mention has been made of spectroscopic factors in relation with one--particle transfer processes, let alone when discussing two--particle transfer. In fact, one will be using throughout the present monograph, exception made when explicitly mentioned, absolute cross sections as the solely link between spectroscopic amplitudes and experimental observations.

\section{Renormalization and spectroscopic amplitudes}
 Elementary modes of nuclear excitation, namely single--particle motion, vibrations and rotations, being tailored to economically describe the nuclear response to external probes, contain a large fraction of the many--body correlations. Consequently, their wavefunctions are non--orthogonal to each other, in keeping with the fact that all the degrees of freedom of the nucleus are exhausted by those of the nucleons (see Chapter \ref{intro}). The corresponding overlaps give a measure of the strength with which the different modes couple to each other. The  resulting particle--vibration coupling Hamiltonian can be diagonalized,  making use of Nuclear Field Theory\footnote{NFT, cf. \cite{Bortignon:77}, \cite{Bortignon:78}.}, and of the  BRST techniques\footnote{cf. \cite{Bes:90} and refs. therein.} in the case of particle--rotor coupling. 
  \begin{figure}
  \centerline{\includegraphics*[width=\textwidth,angle=0]{nutshell/figs/cross_strength.pdf}}
  \caption{Absolute value of the  two--nucleon transfer cross section $^{A+2}$Sn$(p,t)^A$Sn(gs) $(A=112,116,118,120,122,124$ cf. \cite{Potel:13} \cite{Potel:13b}) calculated taking into account successive and simultaneous transfer in second order DWBA, properly corrected for non--orthogonality contributions in comparison with the experimental data (\cite{Guazzoni:99}, \cite{Guazzoni:04}, \cite{Guazzoni:06}, \cite{Guazzoni:08}, \cite{Guazzoni:11}, \cite{Guazzoni:12}).}\label{fig1.5}
  \end{figure}


As a result of the interweaving of single--particle and collective motion, the nucleons acquire a state dependent self energy\footnote{For levels far away from the Fermi energy it can be parametrized at profit by an extension to the complex--energy--plane.} $\Delta E_j(\omega)$.   Consequently, the single--particle potential which was already non--local in space (exchange potential, related to the Pauli principle) becomes also non--local in time (retardation effects; cf. e.g. Fig \ref{fig1F3} (I)). There are a number of techniques to make it local. In particular the Local Density Approximation (LDA) and the effective mass approximation. In this last case one can describe the single--particle motion in terms of a local (complex) potential with a real part given by $U'(r)=(m/m^*)U(r)$, where $m^*=m_km_\omega/m$ is the effective nucleon mass, $m_k$ being the so--called $k$--mass (non--locality in space in keeping with the fact that $\Delta x\Delta k_x\geq1$), and $m_\omega=m(1+\lambda)$ being the $\omega$--mass (non--locality in time, as implied by the relation $\Delta \omega\Delta t\geq1$),  $\lambda=-\partial \Delta E(\omega)/\partial \hbar \omega$ being the so--called mass enhancement factor. It reflects the ability with which vibrations cloth single--particles. In other words, it measures the probability with which a nucleon moving at  $t=-\infty$ in a ``pure'' orbital $j$ can be found at a later time in a 2$p-1h$ like (doorway state) $|j'L;j\rangle$, $L$ being the multipolarity of a vibrational state. Within this context, the discontinuity taking place at the Fermi energy in the dressed particle picture ($Z_\omega=(m/m_\omega)$) is connected with the single--particle occupancy probability\footnote{See e.g. \cite{Brink:05} Ch. 9.}.


It is of notice that dressed particles automatically imply an induced pairing interaction (see e.g. Figs. \ref{fig1F3} (I) and (II)) resulting from the exchange of the clothing vibrations between pairs of nucleons moving in time reversal states close to the Fermi energy (see e.g. Fig. \ref{fig3_A_3}; also Fig. \ref{fig6_A1} (i)). In other words, fluctuations in the normal density $\delta \rho$ and the associated $\delta U$ (Time--Dependent HF) and particle--vibration coupling vertices lead to abnormal (superfluid) density (deformation in gauge space). Whether this is a dynamic or static effect, depends on whether the parameter (cf. Fig. \ref{fig1_E8}\footnote{\cite{Brink:05} App. H. Sect. H4 and refs. therein; \cite{Barranco:05}.}) 
\begin{align}\label{eq2_1_10}
x'=G'N'(0),  
\end{align}
product of the effective pairing strength, 
\begin{align}\label{eq2_2_2}
G'=Z_\omega^2(v_p^{bare}+v_p^{ind}),
\end{align}
and of the renormalized density of levels $N'(0)$ is considerable smaller  (larger) than $\approx1/2$. The quantity $G'$ is the sum of the bare and induced pairing interaction, renormalized by the degree of single--particle content of the levels where nucleons correlate. The quantity 
\begin{align}
N'(0)=Z_\omega^{-1}N(0)=(1+\lambda)N(0)
\end{align}
is the similarly renormalized density of levels at the Fermi energy. From the above relations one obtains 
\begin{align}
x'=Z_\omega(v_p^{bare}+v_p^{ind})N(0).
\end{align}
All of the above many--body, $\omega$--dependent effects which imply in many cases a coherent sum of amplitudes, are not simple to capture in a spectroscopic factor\footnote{In keeping with the fact that $m_k\approx 0.6-0.7 m$ and that $m^*\approx m$, as testified by the satisfactory fitting standard Saxon--Woods potentials provides for the valence orbitals of nucleons of mass $m$ around closed shells, one obtains $m_\omega\approx 1.4-1.7 m$. Thus $Z_\omega \approx 0.6-0.7$. It is still an open question how much of the observed single--particle depopulation can be due to hard core effects, which shifts the associated strength to high momentum levels (see \cite{Dickhoff:05}, \cite{Jenning:11}, \cite{Kramer:01}, \cite{Barbieri:09}, \cite{Schiffer:12}, \cite{Duguet:12}, \cite{Furnstahl:10}).  An estimate of such an effect of about 20\% will not quantitative change the long wavelength estimate of $Z_\omega$ given above. Arguably, a much larger depopulation through hard core effects remains an open problem within the overall picture of elementary modes of nuclear excitation and of medium polarization effects.} in connection with one--particle transfer, let alone two--nucleon transfer processes\footnote{See \cite{Barranco:05,Barranco:99}.}. 






\section{Quantality Parameter}\label{App1A}
\begin{table}
 \begin{tabular}{|c|c|c|c|c|c|}
 \hline \rule[-2ex]{0pt}{5.5ex}   constituents& $M/M_n$  & $a$(cm) &$v_0$(eV)  &q&phase($T=0$)    \\ 
 \hline \rule[-2ex]{0pt}{5.5ex}   $^{3}$He &3& 2.9(-8)  &8.6(-4)  &0.19  &liquid$^{a)}$    \\ 
 \hline \rule[-2ex]{0pt}{5.5ex}  $^{4}$He  &4&  2.9(-8)&  8.6(-4)&  0.14& liquid$^{a)}$   \\ 
 \hline \rule[-2ex]{0pt}{5.5ex}    H$_2$&2&  3.3(-8)&  32(-4)&  0.06&solid$^{b)}$   \\ 
 \hline \rule[-2ex]{0pt}{5.5ex}    $^{20}$Ne&20& 3.1(-8) &  31(-4)&  0.007&solid$^b)$    \\ 
 \hline \rule[-2ex]{0pt}{5.5ex}    nucleons&1&  9(-14)& 100(+6) &  0.4&liquid$^{a),c),d)}$  \\ 
 \hline 
 \end{tabular}
 \caption{Zero temperature phase for a number of systems of mass $M$ ($M_n$: nucleon mass), the first four depending on atomic interactions (range \AA, strength meV), the last one referring to the atomic nucleus. a) delocalized (condensed), b) localized, c) non--Newtonian solid (cf. e.g. \cite{Bertsch:88b}, \cite{DeGennes:94}), that is, systems which react elastically to sudden solicitations and plastically under prolonged strain, d) paradigm of quantal, strongly fluctuating, finite many--body  systems. While delocalization or less does not seem to depend much on whether one is dealing with fermions or bosons (\cite{Mottelson:02} and refs. therein; cf also \cite{Ebran:14}, \cite{Ebran:14b}, \cite{Ebran:13}, \cite{Ebran:12}), the detailed properties of the corresponding single--particle motion are strongly dependent on the statistics obeyed by the associated particle (cf. Sect. \ref{App1E}).}\label{tab1A1}
 \end{table}
 \begin{figure}
 \centerline{\includegraphics*[width=\textwidth,angle=0]{nutshell/figs/potential.pdf}}
 \caption{Schematic representation of the bare $NN$--interaction acting among nucleons displayed as a function of the relative coordinate $r=|\mathbf{r}_1-\mathbf{r}_2|$,used to estimate the quantality parameter $q$, ratio of the zero point fluctuations (ZPF) of confinement and the potential energy.}\label{fig1A1}
 \end{figure}
The quantality parameter\footnote{\cite{Nosanow:76}, \cite{deBoer:57}, \cite{deBoer:48}, \cite{deBoer:48b},\cite{Mottelson:02}.} is defined as the ratio of the quantal kinetic energy of localization and potential energy, (cf. Fig. \ref{fig1A1} and Table \ref{tab1A1}).
 Fluctuations, quantal or classical, favor symmetry: gases and liquids are homogeneous. Potential energy on the other hand prefers special arrangements: atoms like to be at specific distances and orientations from each other (spontaneous breaking of translational and of rotational symmetry reflecting the homogeneity and isotropy of empty space\footnote{Within this general context the physics embodied in the quantality parameter is closely related to that which is at the basis of the classical Lindemann criterion (\cite{Lindemann:10}) to measure whether a system is ordered (e.g. a crystal) or disordered (e.g. a melted system) (\cite{Bilgram:87}, \cite{Lowen:94}, \cite{Stillinger:90,Stillinger:95}). The above statement is also true for the generalized Lindemann parameter (\cite{Stillinger:90}, \cite{Zhou:99}), used to provide similar insight into inhomogeneous finite systems like e.g. proteins (aperiodic crystals \cite{Schrodinger:44}, see also Ehrenfest's theorem (\cite{Basdevant:05} pag. 138) see also App. \ref{C2AppC}).}).
 
 
  When $q$ is small, quantal effects are small and the lower state for $T<T_c$ will have a crystalline structure, $T_c$ denoting the critical temperature.  For sufficiently large values of $q\, (>0.15$) the system will display particle delocalization and,  likely, be  amenable, within some approximation, to a mean field description (Figs. \ref{fig1A2} and   \ref{fig1A3}). In fact, the step delocalization $\rightarrow$ mean field is certainly not automatic, neither guaranteed. In any case, not for all properties neither for all levels of the system. Let us elaborate on these points. Independent particle motion can be viewed as the most collective of all nuclear properties, reflecting the effect of all nucleons on a given one resulting in a macroscopic effect. Namely confinement with long mean free path as compared with nuclear dimensions. Consequently, it should be possible to calculate the mean field in an accurate manner. Arguably, as accurately as one can calculate collective vibrations, e.g. quadrupole vibrations. But this does not mean that one knows how to correctly calculate the energy and associated deformation parameter of each single state of the quadrupole response function. Within this context one may find through mean field approximation a good description for the energy of the valence orbitals of a nucleus but for  specific levels (e.g. the $d_{5/2}$ level of $^{119-120}$Sn, cf. e.g. Fig \ref{fig6.2.3}). It is not said that  including  particle--vibration coupling corrections, a process which in average makes theory come closer to experiment\footnote{See e.g. \cite{Bohr:75}, \cite{Bortignon:77}, \cite{Mahaux:85}, \cite{Bes:71}, \cite{Bes:71b}, \cite{Bes:71c}, \cite{Bortignon:76}, \cite{Bes:88}, \cite{Barranco:87b}, \cite{Barranco:01} and references therein.}, single specific quasiparticle energies will agree better with the data\footnote{cf. also \cite{Tarpanov:14}.}. Cases like this one constitute a sobering experience concerning the intricacies of the many--body problem in general, and the nuclear one (finite many--body system, FMBS) in particular where spatial quantization plays a central role. In other words, one is dealing with a self--confined, strongly interacting, finite many--body system generated from collisions originally associated  with a variety of astrophysical events and thus with  the coupling and interweaving of different scattering channels and resonances, a little bit as e.g. the Hoyle monopole resonance ($\alpha+\alpha+\alpha\rightarrow^{12}$C). Within the anthropomorphic (grand design) scenario such phenomena are found in the evolution of the Universe to eventually allow for the presence of organic matter and, arguably, life on earth\footnote{cf. e.g. \cite{Rees:00}, \cite{Meissner:14} and references therein.} more likely than to make mean field approximation an ``exact'' description of nuclear structure and reactions.
 
 \begin{figure} \centerline{\includegraphics*[width=\textwidth,angle=0]{nutshell/figs/diagrams1.pdf}}
 \caption{Schematic representation of (a) nucleon--nucleon scattering through the bare $NN$--interaction, (b) the associated contribution to the Hartree potential $U(r)$ and, (c) to the Fock (exchange) potential $U_x(r,r')$, $\rho(r)$ being the nucleon density. (d) the Hartree--Fock solution leads to a sharp discontinuity at the Fermi energy $\epsilon_F$. That is, single--particle levels with energy $\epsilon_i\leq \epsilon_F$ are fully occupied. Those with $\epsilon_k\geq \epsilon_F$ empty.}\label{fig1A2}
 \end{figure}
 
 
\begin{figure}
\centerline{\includegraphics*[width=\textwidth,angle=0]{nutshell/figs/fig1A3.pdf}}
\caption{\textbf{(I)} (a) Schematic representation of ``normal'' (independent--particle) motion of nucleons in  two--fold degenerate (Kramers, time--reversal degeneracy) orbits solidly anchored to the mean field and  displaying a sharp, step--function--like, discontinuity in the occupancy at the Fermi energy  lead to a deformed  (\cite{Nilsson:55}) rotating nucleus with a rigid moment of inertia $\mathcal{I}_r$ (b). \textbf{(II)} Schematic representation of independent nucleon Cooper pair  motion in which few (of the order of 5-8) pairs lead to (c) a sigmoidal occupation function at the Fermi energy and, having uncoupled themselves from the fermionic mean field being now (quasi) bosons they essentially do not  contribute to (d) the moment of inertia of quadrupole rotational bands leading to $\mathcal{I}\approx\mathcal{I}_r/2$ (cf. \cite{Belyaev:13}, \cite{Belyaev:59}, \cite{Bohr:75} and references therein), (e) pairing rotational bands in gauge space, an example of which is  provided by the ground states of the superfluid Sn--isotopes (see also Figs. \ref{fig1.3} and \ref{fig1.4}).}\label{fig1A3}
\end{figure}
\begin{figure}
\centerline{\includegraphics*[width=0.7\textwidth,angle=0]{nutshell/figs/fig1A4.pdf}}
\caption{A system of independent Cooper pairs (Schafroth pairs). This situation corresponds to the incoherent solution of the many Cooper pair problem, the so called Fock state. In cold gases it describes the system after the Feschbach resonance leading to BEC (after \cite{Rogovin:76}).}\label{fig1A4}
\end{figure}
\begin{figure}
\centerline{\includegraphics*[width=0.7\textwidth,angle=0]{nutshell/figs/fig1A5.pdf}}
\caption{There are about $10^{18}$ Cooper pairs per cm$^{3}$ in a superconducting metal. A Cooper pair has a spatial extension of about $10^{-4}$ cm. Thus a given Cooper pair will overlap with  $10^{6}$ other Cooper pairs, leading to strong pair--pair correlation, as schematically shown. This picture emerges from  the many Cooper pair solution of the superconducting state of metal  (coherent state), also valid in atomic nuclei (cf. \cite{Schrieffer:64}, \cite{Brink:05}, and references therein). (After \cite{Rogovin:76}).}\label{fig1A5}
\end{figure}
 \section{Cooper pairs}\label{App1D}
Let us assume that the motion of nucleons is described by the Hamiltonian, 
 \begin{equation*}
 H=\sum_{j_1j_2}\langle j_1|T|j_2\rangle a_{j_1}^{\dagger}a_{j_2}+\frac{1}{4}\sum_{\substack{j_1j_2\\j_3j_4}}\langle j_1j_2|v|j_3j_4\rangle a_{j_2}^{\dagger} a_{j1}^{\dagger} a_{j_3} a_{j_4},
 \end{equation*}
 written in second quantization\footnote{cf. e.g. \cite{Brink:05}, App. A.}.
In what follows it will be schematically shown how mean field is extracted from such a Hamiltonian, both in the case of single--particle motion (HF) and of independent pair motion (BCS).  





 \subsection{independent--particle motion}\label{C1S1D1}
In the previous section it was shown that the value of the quantality parameter associated with nuclei ($q\approx 0.4$) leads to particle delocalization and likely makes the system amenable to a mean field description (Fig. \ref{fig1A2}; see however the provisos expressed at the end of Sect. \ref{App1A}). In such a case, Hartree--Fock approximation is tantamount to a selfconsistent relation between density and potential, weighted by the nucleon--nucleon interaction $v$, and
  leading to a  complete separation between occupied ($|i\rangle$) and empty ($|k\rangle$) single--particle states,
 \begin{equation}
\begin{split}
\left(U^2_\nu+V_\nu^2\right)=1;\quad |\varphi_\nu\rangle=&\bar a_\nu^{\dagger}|0\rangle=\left(U_\nu+V_\nu a_\nu^{\dagger}\right)|0\rangle; V^2_\nu=\left\{\begin{array}{c}
1\;\epsilon_\nu\equiv\epsilon_i\leq \epsilon_F, \\ 
0\; \epsilon_\nu\equiv\epsilon_k>\epsilon_F.
\end{array}\right.\\ 
\end{split}
\end{equation}
The Hartree--Fock ground state can then be written as, 
 \begin{equation}
\begin{split}
|HF\rangle&=|\det(\varphi_\nu)\rangle=\Pi_\nu \bar a_\nu^{\dagger}\; |0\rangle=\Pi_i a_i^{\dagger}\; |0\rangle=\Pi_{i>0} a_i^{\dagger}a_{\tilde i}^{\dagger}|0\rangle.\\
\end{split}
\end{equation}
\begin{figure}
\centerline{\includegraphics*[width=0.9\textwidth,angle=0]{nutshell/figs/fig1D1.pdf}}
\caption{Parallel between dynamic and static deformations in 3D-- and in gauge--space for the nuclear finite many body system (FMBS). In the first case, the angular momentum $\mathbf{I}$ and the Euler angles are conjugate variables. In the second, particle number $N$ and gauge angle. While the fingerprint of static (quadrupole and gauge) deformations are quadrupole and pairing rotational bands, vibrational bands are the expression of such phenomena in non deformed systems (after \cite{Broglia:73}).}\label{fig1D1}
\end{figure}
where $|\tilde i\rangle$  is the time reversed state to  $| i\rangle$.


To be solved, the above self--consistent equations have to be given boundary conditions. In particular, make it explicit whether the system has a spherical or, for example, a quadrupole  shape.   That is, whether $\langle HF|Q_2|HF\rangle$ is zero or has a finite value, $Q_{2M}=\sum_{j_1j_2}\langle j_2||r^2Y^2_M||j_1\rangle \left[a^\dagger_{j_1}a_{j_2}\right]^2_M$ being the quadrupole operator which carries particle transfer quantum number $\beta=0$, in keeping with its particle--hole character. In the case in which $\langle Q_{2M}\rangle=0$, the system can display a spectrum of low--lying, large amplitude, collective quadrupole vibrations of frequency $(C/D)^{1/2}$, the associated ZPF $=\left(\hbar^2/(2D\hbar\omega)\right)^{1/2}$ leading to dynamical violations of rotational invariance. In the above relations, $C$ and $D$ stand for the restoring force constant and the inertia of the vibrational mode, respectively. In the case in which $\langle Q_{2M}\rangle\neq0$, the $|HF\rangle$ state is known as the Nilsson state, $|\text{Nilsson}\rangle$, defining a privileged orientation in 3D--space and thus an intrinsic, body--fixed system of reference $\mathcal{K}'$ which makes an angle $\Omega$ (Euler angles) with the laboratory frame\footnote{\cite{Nilsson:55}.} $\mathcal{K}$. Because there is no restoring force associated with the different orientations, fluctuations in $\Omega$ diverge in just the right way to restore rotational invariance, leading to a rotational band displaying a rigid moment of inertia (cf. Fig. \ref{fig1A3} and \ref{fig1D1}), and whose members are the states\footnote{\cite{Bohr:75}.},
 \begin{equation*}
\begin{split}
|\text{IKM}\rangle\sim&\int d\Omega \mathcal{D}_{MK}^I(\Omega)|\text{Nilsson}(\Omega)\rangle;\; E_I=(\hbar^2/2\mathcal I)I(I+1); \mathcal{I}=\mathcal{I}_{rig}.
\end{split}
\end{equation*}
One can also view such bands as the limit $(C\rightarrow 0, D\,(=\mathcal I)$ finite) of low energy ($\omega\rightarrow 0$), large--amplitude collective vibration. 
Similar dynamic and static spontaneous symmetry breaking phenomena take place in connection with particle--particle ($\beta=+2$ transfer quantum number) and hole--hole ($\beta=-2$) correlations, namely in gauge space (see Fig. \ref{fig1D1}; subject discussed also in Sect. \ref{App1E} (dynamic: pairing vibration) and also below (static: pairing rotation); see also Figs. \ref{fig1.1}, \ref{fig1.3} and \ref{fig1.4}). For a consistent discussion of these subjects\footnote{See \cite{Bes:90}.}.




\subsection{independent--pair motion}\label{C1AppDS2}
Let us make use of the constant pairing matrix element approximation $\langle j_1j_2|v|j_3j_4\rangle=G$, that is, 
\begin{equation}
H_P=-G\sum_{\nu, \nu'>0}a^{\dagger}_\nu a^{\dagger}_{\bar\nu}a_{\nu'} a_{\bar\nu'}.
\end{equation}
 The abnormal density is related to the finite value of the pair operator. The associated independent pair states are written in the BCS approximation as
\begin{equation}
\left(U_\nu^2+V_\nu^2\right)=1;\quad |\varphi_{\nu\bar\nu}\rangle=\left(U_j+V_ja^\dagger_{jm}a^\dagger_{\tilde{jm}}\right)|0\rangle,\quad\left.\begin{array}{c}
V_\nu \\ 
U_\nu
\end{array}\right\}=\frac{1}{\sqrt{2}}\left(1\mp\frac{\epsilon_\nu}{E_\nu}\right)^{1/2},
\end{equation}
where $E_\nu=\sqrt{\epsilon_\nu^2+\Delta^2}$ and $\epsilon_\nu=\varepsilon_\nu-\lambda$, $\lambda=\varepsilon_F$.
The BCS ground state,
\begin{equation}
|BCS\rangle=\Pi_{\nu>0}\left(U_j+V_ja^\dagger_{jm}a^\dagger_{\tilde{jm}}\right)|0\rangle,
\end{equation}
describes independent pair motion, namely a situation correctly described in term of strongly overlapping (quasibosonic) pairs of fermions (Fig. \ref{fig1A5}), at variance with of that (erroneous) shown in Fig. \ref{fig1A4}.
Let us introduce the phasing\footnote{cf. e.g. \cite{Schrieffer:73}.}, 

\begin{equation}
U_\nu=|U_\nu|=U_\nu';\quad V_\nu=e^{-2i\phi}V_\nu';\quad(V_\nu'\equiv|V_\nu|)\;(\nu\equiv j,m),
\end{equation}
where $\phi$ is the gauge angle. One can then write the  (BCS) wavefunction as, 
\begin{equation}
\begin{split}
|BCS(\phi)\rangle_{\mathcal K}&=\prod_{\nu>0} \left(U_\nu'+V_\nu'e^{-2i\phi}a_\nu^\dagger a_{\bar{\nu}}^\dagger\right)|0\rangle=\prod_{\nu>0}\left(U_\nu'+V_\nu'a_\nu^{\dagger'} a_{\bar{\nu}}^{\dagger'}\right)|0\rangle\\
&=|BCS(\phi=0)\rangle_{\mathcal{K'}}:\text{lab. system},\mathcal{K}:\text{intr. system}\,\mathcal{K}',
\end{split}
\end{equation}
where $a_{\nu}^{'\dagger}=e^{-i\phi}a_{\nu}^{\dagger}$ is the single--particle creation operator referred to the intrinsic system.
The BCS  order parameter,  two--nucleon spectroscopic amplitudes and number and gap equations are\footnote{See \cite{Potel:17}.},
\begin{equation}
\begin{split}
\braket{BCS|\sum_{\nu>0}a^\dagger_\nu a^\dagger_{\bar\nu}|BCS}=&\alpha_0'e^{-2i\phi}\; ; \;\;\alpha_0'=\sum_{\nu>0}U_\nu'V_\nu' \;\;\;;\Delta=G\alpha_0,\\
\end{split}
\end{equation}


\begin{equation}
B_\nu=\langle BCS|\frac{[a_\nu'^\dagger a_\nu'^\dagger]_0}{\sqrt{2}}|BCS\rangle=(j_\nu+1/2)^{1/2}U'_\nu V'_\nu,
\end{equation}

and
\begin{equation}
\begin{split} N_0=2\sum_{\nu>0}V^2_\nu;\;\;\frac{1}{G}=\sum_{\nu>0}\frac{1}{2E_\nu}.
\end{split}
\end{equation}
Examples of $B_\nu$--coefficients for the reaction $^{124}$Sn$(p,t)^{122}$Sn (gs) are given in Table \ref{tab1D1}. 


The wavefunction and energies of the members of the pairing rotational band, can be written as  
\begin{equation*}
\begin{split}
&|N_0\rangle\sim\int_0^{2\pi}d\phi e^{-iN_0\phi}|BCS(\phi)\rangle_{\mathcal K}\sim\left(\sum_{\nu>0}c_\nu a_\nu^\dagger a_{\bar{\nu}}^\dagger\right)^{N_0/2}|0\rangle,\,c_\nu=V_\nu/U_\nu;\\
&E_N=(\hbar^2/2\mathcal{I})N^2 ;\;\;\mathcal{I}\approx 2\hbar^2/G,
\end{split}
\end{equation*}
respectively\footnote{See e.g. \cite{Brink:05} App. H; \cite{Mottelson:02} and refs. therein.}.

In the case of a quadrupole deformed nucleus, the system acquires not only a privileged orientation in gauge space, but also in 3D--space. Now, as summarized above, in a superfluid system, Cooper pairs and not single--particles are the building blocks of the system (see Figs. \ref{fig1D2} and \ref{fig1A5})\footnote{In connection with Fig. \ref{fig1D2}, the estimate $2R=20/k_F$ was carried out with the help of the Fermi gas model (see e.g. \cite{Bohr:69}). The Fermi momentum is written as $k_F\approx (3\pi^2 A/2V)^{1/3}\approx (\frac{3\pi^2}{2}\rho(0))^{1/3}$. Making use of $\rho(0)\approx 0.17$ fm $^{-3}$ one obtains $k_F\approx 1.36 $ fm$^{-1}$. Let us now rewrite the relation between $k_F$ and the volume $V(=(4\pi/3) R^3=(4\pi/3) r_0^3 A)$. That is $k_F\approx (9\pi/8)^{1/3}/r_0(=1.52/r_0)$. To employ $r_0=1.2$ fm and still keep 1.36 fm $^{-1}$, one has to modify the above relation to $k_F\approx 1.63/r_0$. We now write the diameter of a heavy nucleus of mass $A\approx 200$ ($A^{1/3}\approx 5.85$). i.e. $2R=2r_0A^{1/3}\approx 20/k_F$.
 This is the value used in Fig. \ref{fig1D2}.}. But while the mean square radius of a nucleon at the Fermi energy ($\langle r^2\rangle^{1/2}\approx (3/5)^{1/2}R_0 \,(R_0=1.2 A^{1/3}\text{fm}))$ is about 4.6 fm $(A\approx 120)$, that of a Cooper pair is determined by the correlation length ($\xi\approx\hbar v_F/\pi\Delta\approx 30$ fm) between the two nucleons forming the pair (see Figs. \ref{fig1A3} and \ref{fig1A5}). Consequently, orienting the quadrupole deformed potential in different directions (Euler angles $\Omega$), will have  less  effect on Cooper pairs than on independent particles. Thus the reduction of the moment of inertia from $\mathcal I_r$ to $\approx\mathcal{I}_r/2$, $\mathcal{I}_r$ being the rigid moment of inertia. Within this context one can mention the fact that low--lying nuclear collective vibrations (and rotations) are essentially not observed at intrinsic excitation energies corresponding to temperatures of $\approx$1-2 MeV. In this case, this is because the surface is strongly fluctuating (thermally) and thus not well defined, making it non operative it's anisotropic orientation in space.
 \begin{table}[h!]
 {\begin{tabular}{|c|c|c|c|c|}
 \cline{2-5} 
 \multicolumn{1}{c|}{ }& \multicolumn{2}{|c|}{ $^{112}$Sn($p,t)^{110}$Sn(gs)}&\multicolumn{2}{|c|}{$^{124}$Sn($p,t)^{122}$Sn(gs)} \\
 \hline
 $nlj^{\,a)}$ & BCS$^{b)}$ & $V_{low-k}^{c)}$ & BCS$^{d)}$& NuShell$^{e)}$  \\
 \hline
 $1g_{7/2}$ & 0.96 &-1.1073 & 0.44 & 0.63  \\
 $2d_{5/2}$ & 0.66 & -0.7556& 0.35 & 0.60  \\
 $2d_{3/2}$ & 0.54 &  -0.4825& 0.58 & 0.72  \\
 $3s_{1/2}$ & 0.45 &  -0.3663&  0.36 & 0.52  \\
 $1h_{11/2}$ & 0.69 & -0.6647 & 1.22 & -1.24  \\
 \hline 
 \end{tabular}}
 \caption{Two--nucleon transfer spectroscopic amplitudes associated with the reactions $^{112}$Sn$(p,t)^{110}$Sn(gs) and $^{124}$Sn$(p,t)^{122}$Sn(gs). \textbf{a}) quantum numbers of the two--particle configurations $(nlj)^2_{J=0}$ coupled to angular momentum $J=0$. \textbf{b})--\textbf{d}) $\langle BCS|T_\nu|BCS\rangle=\sqrt{(2j_\nu+1)/2}\,U_\nu(A) V_\nu(A+2)\;(A+2=112$ and $ 124$ respectively), where $T_\nu=[a^\dagger_{\nu}a^\dagger_\nu]^0/\sqrt{2} \,(\nu\equiv nlj)$ (cf. \cite{Potel:11,Potel:13,Potel:13b}) \textbf{c}) two--nucleon transfer spectroscopic amplitudes calculated making use of initial and final state wavefunctions obtained by diagonalizing a $v_{low-k}$, that is a renormalized, low--momentum interaction derived from the CD--Bonn nucleon--nucleon potential (see \cite{Guazzoni:06} and references therein). \textbf{e}) Two--neutron overlap functions obtained making use of the shell--model wavefunctions for the ground state of $^{122}$Sn and $^{124}$Sn calculated with the code NuShell \citep{Brown:07}. The wavefunctions were obtained starting with a $G$--matrix derived from the $CD$--Bonn nucleon--nucleon interaction \cite{Machleidt:96}. These amplitudes were used in the calculation of $^{124}$Sn$(p,t)^{122}$Sn absolute cross sections carried out by I.J. Thompson \citep{Thompson:13}.}\label{tab1D1}
 \end{table}
\begin{figure}
\centerline{\includegraphics*[width=\textwidth,angle=0]{nutshell/figs/Resume.pdf}}
\caption{Classical localization and zero point fluctuations, associated with independent--particle (normal density) and independent--pair motion (abnormal density).}\label{fig1D2}
\end{figure}





\begin{figure}\centerline{\includegraphics*[width=\textwidth,angle=0]{nutshell/figs/fig1D3.pdf}}
\caption{Schematic representation of the steps to be taken to extract from a two--body interaction independent particle motion ($U$, mean field) and independent pair motion $(V_p=-G\alpha_0(P^\dagger+P)$, pair potential, $P^\dagger$ being the pair creation operator), in terms of a generalized quasiparticle transformation, and leading to a sharp step--function occupation distribution and a smooth (sigmoidal) occupation distribution around the Fermi surface respectively.}\label{fig1D3}
\end{figure}




Because in FMBS  quantal fluctuations are very important\footnote{see \cite{Bertsch:05} and references therein.}, deformation in such systems explicit themselves through rotational bands. In particular, superfluid nuclei display well defined pairing rotational bands, an example of such bands being provided by the ground states of the superfluid Sn--isotopes. In this case, the moment of inertia is directly related to the  pairing interaction. Pairing  rotational bands are specifically excited in two nucleon transfer reactions (cf, Figs. \ref{fig1.3} and \ref{fig1.4}). 
A summary of the physics which is at the basis of independent single--particle and single--pair motion is given in Figs. \ref{fig1D2} and \ref{fig1D3}. 







\section[Pair vibration spectroscopic amplitudes]{Two--nucleon spectroscopic amplitudes associated with pairing vibrational modes in closed shell systems: the $^{208}$Pb case.}\label{App1E}
The solution of the pairing Hamiltonian
\begin{equation*}
H=H_{sp}+H_p,
\end{equation*}
where
\begin{equation*}
H_{sp}=\sum_{\nu}\epsilon_\nu a_\nu^\dagger a_\nu,
\end{equation*}
and 
\begin{equation*}
H_p=-GP^\dagger P,
\end{equation*}
with
\begin{equation*}
P^\dagger=\sum_{\nu>0}a_\nu^\dagger a_{\bar \nu}^\dagger,
\end{equation*}
lead, in the case of closed shell systems and within the harmonic approximation (RPA), to pair addition $(a)$ pair removal $(r)$ two--particle, two--hole correlated modes, the associated creation and annihilation operator being
\begin{equation*}
\Gamma_a^\dagger(n)=\sum_k X_n^a(k)\Gamma_k^\dagger+\sum_iY^a_n(i)\Gamma_i,
\end{equation*}
and
\begin{equation*}
\Gamma_r^\dagger(n)=\sum_i X_n^r(i)\Gamma_i^\dagger+\sum_kY^r_n(k)\Gamma_k,
\end{equation*}
with
\begin{equation*}
\sum X^2-Y^2=1,
\end{equation*}
and
\begin{equation*}
\Gamma_k^\dagger=a_k^\dagger a_{\tilde k}^\dagger,\quad (\epsilon_k>\epsilon_F).
\end{equation*}
Similarly,
\begin{equation*}
\Gamma_i^\dagger=a_{\tilde i} a_i ,\quad (\epsilon_i\leq\epsilon_F).
\end{equation*}
The relations
\begin{equation*}
[H,\Gamma_a^\dagger(n)]=\hbar W_n (\beta=+2),
\end{equation*}
and
\begin{equation*}
[H,\Gamma_r^\dagger(n)]=\hbar W_n (\beta=-2),
\end{equation*}
where $\beta$ is the transfer quantum number, while $n$ labels the roots of the corresponding dispersion relations\footnote{cf. \cite{Bes:66}.}, 
\begin{equation*}
\frac{1}{G(\pm2)}=\sum_k \frac{(\Omega_k/2)}{2\epsilon_k\mp W_n(\pm2)}+\sum_i \frac{(\Omega_i/2)}{2\epsilon_i\pm W_n(\pm2)},
\end{equation*}
$n$ labeling the corresponding solutions in increasing order of energy. In the above equation, $\Omega_j=j+1/2$ is the pair degeneracy of the orbital with total angular momentum $j$.

For the case of the (neutron) pair addition and pair substraction modes of $^{208}$Pb the above equations are  graphically solved in Fig \ref{fig1E1} (see also Table \ref{tab1E1}). The minimum of the dispersion relation defines the Fermi energy of the system under study. This is in keeping with the fact that in the case in which $W_1 (\beta=+2)=W_1(\beta=-2)=0$, situation corresponding to the  transition between normal and superfluid phases, the energy value at which the dispersion relation touches for the first time the energy axis, coincides with the BCS $\lambda$ variational parameter. It is of notice that, as a rule, the Fermi energy of closed shell nuclei is empirically defined as half the energy difference between the last occupied and the first empty single particle state\footnote{cf. e.g. \cite{Mahaux:85}.}. Making use of the values (see Fig. \ref{fig1E1} and Table \ref{tab1E1})
  \begin{figure}
  \centerline{\includegraphics*[width=\textwidth,angle=0]{nutshell/figs/dispersion.pdf}}
  \caption{The right hand side of the RPA pairing vibrational dispersion relation for neutrons in the case of the closed shell system $^{208}$Pb (cf. \cite{Bes:66}) in the region between the two neighboring shells ($p_{1/2}$ and $g_{9/2}$). All quantities are in MeV. For each $G$ there is a straight horizontal line, which is divided by the the curve in three sections. The first one from the left corresponds to the pairing correlation energy of the nucleus $^{206}$Pb (two correlated neutron hole states) while the last segment to the right measures the pairing correlation energy of $^{210}$Pb (two correlated neutrons above closed shell) the intermediate segment measures the energy of the two phonon (correlated ($2p-2h$)) pairing vibrational state  of $^{208}$Pb.}\label{fig1E1}
  \end{figure}
\begin{table}
\begin{tabular}{|c|c|c|}
\hline \rule[-2ex]{0pt}{5.5ex}   orbit& $\epsilon_j$  & $\epsilon_{p_{1/2}}-\epsilon_k\equiv|\epsilon_k|-|\epsilon_{p_{1/2}}|$  \\ 
\hline
$0h_{9/2}$&-10.62 &3.47\\
$1f_{7/2}$& -9.50&2.35\\
$0i_{13/2}$& -8.79&1.64\\
$2p_{3/2}$& -8.05&0.90\\
$1f_{5/2}$& -7.72&0.57\\
$2p_{1/2}$& -7.15&0\\
\hline \rule[-2ex]{0pt}{5.5ex}   $\epsilon_F=-5.825$ MeV&   & $\epsilon_k-\epsilon_{g_{9/2}}\equiv|\epsilon_{g_{9/2}}|-|\epsilon_k|$  \\ 
\hline
$1g_{9/2}$&-3.74 &0.\\
$0i_{11/2}$& -2.97&0.77\\
$0j_{15/2}$& -2.33&1.41\\
$2d_{5/2}$& -2.18&1.56\\
$3s_{1/2}$& -1.71&2.03\\
$1g_{7/2}$& -1.27&2.47\\
$2d_{3/2}$& -1.23&2.51\\
\hline
\end{tabular}\caption{Valence single--particle levels of $^{208}$Pb. In the upper part the occupied levels ($\epsilon_i\leq\epsilon_F$) are shown while in the lower part the empty levels ($\epsilon_k\geq\epsilon_F$). Of notice that $\epsilon_{p_{1/2}}-\epsilon_{g_{9/2}}=3.41$ MeV, is the single--particle gap associated associated with $N=126$ shell closure (from Nuclear Data Center).}\label{tab1E1}
 \end{table}
 
 \begin{equation*}
 \left\{
 \begin{array}{c}
  E_{corr}(+2)=BE(208)+BE(210)-2BE(209)=1.248\,\text{MeV},\\ 
  E_{corr}(-2)=BE(208)+BE(206)-2BE(207)=0.640\,\text{MeV},
 \end{array}
 \right.
 \end{equation*}
 one obtains
 $W_1(-2)+W_1(+2)=(BE(208)-BE(206))-(BE(210)-BE(208))$
 $=14.11-9.115=4.995 \,\text{MeV}$. Notice that in the above calculations all energies differences are positive. In particular (see Table \ref{tab1E1})  
\begin{equation*}
\epsilon_i<\epsilon_F\Rightarrow \epsilon_F-\epsilon_i=-|\epsilon_F|+|\epsilon_i|=|\epsilon_i|-|\epsilon_F|>0,
\end{equation*}
and
\begin{equation*}
\epsilon_k>\epsilon_F\Rightarrow \epsilon_k-\epsilon_F=-|\epsilon_k|+|\epsilon_F|=|\epsilon_F|-|\epsilon_k|>0.
\end{equation*}
Thus,
\begin{equation*}
\left\{
\begin{array}{c}
 2(\epsilon_F-\epsilon_{p_{1/2}})=W_1(-2)+E_{corr}(-2)>0,\\ 
 2(\epsilon_{g_{9/2}}-\epsilon_F)=W_1(+2)+E_{corr}(+2)>0.
\end{array}
\right.
\end{equation*}
From Fig. \ref{fig1E1} and Table \ref{tab1E1} one can then write,
\begin{equation*} 
2\times(-5.825-(-7.5))\,\text{MeV}=2.650\,\text{MeV}=W_1(-2)+0.640\,\text{MeV}
\end{equation*}
and 
\begin{equation*} 
2\times(-3.74\, \text{MeV}-(-5.825)\,\text{MeV})=4.17\, \text{MeV}=W_1(+2)+1.248\, \text{MeV}.
\end{equation*}
Consequently,
\begin{equation*} 
W_1(-2)=2.01\,\text{MeV}\quad \text{and} \quad W_1(+2)=2.92\,\text{MeV},
\end{equation*}
leading to,
\begin{equation*} 
W_1(+2)+W_1(-2)=4.93\,\text{MeV}.
\end{equation*}

\subsection{Pair removal mode}
In Fig. \ref{fig1E2} the graphical representation of the forwards going RPA amplitude of the pair removal mode is shown. Its expression is 
\begin{equation*}
X_1^r(i)=\frac{\frac{1}{2}\Omega_i^{1/2}\Lambda(-2)}{2(\epsilon_F-\epsilon_i)-W_1(-2)},
\end{equation*}
where
\begin{equation*}
\begin{split}
2\times(\epsilon_F-\epsilon_i)-W_1(-2)&=2\times(\epsilon_F-\epsilon_i)-2\times(\epsilon_F-\epsilon_{p_{1/2}})+E_{corr}(-2)\\
&=2\times(\epsilon_{p_{1/2}}-\epsilon_i)+E_{corr}(-2)=2\times(|\epsilon_i|-|\epsilon_{p_{1/2}}|)+E_{corr}(-2).
\end{split}
\end{equation*}
Thus,
\begin{equation*}
X_1^r(i)=\frac{\frac{1}{2}\Omega_i^{1/2}\Lambda(-2)}{2(|\epsilon_i|-|\epsilon_{p_{1/2}}|)+E_{corr}(-2)}.
\end{equation*}
Making use of the empirical value of $E_{corr}(-2)$ worked out above one obtains, 
  \begin{figure}
  \centerline{\includegraphics*[width=0.1\textwidth,angle=0]{nutshell/figs/removal_forward.pdf}}
  \caption{NFT representation of the forwards going RPA amplitude of the pair removal mode (double downward going arrowed line) describing a two correlated hole state (single downward going arrowed line for each hole with quantum numbers collectively labeled $i$).}\label{fig1E2}
  \end{figure}
\begin{equation*}
X_1^r(i)=\frac{\frac{1}{2}\Omega_i^{1/2}\Lambda(-2)}{2(|\epsilon_i|-|\epsilon_{p_{1/2}}|)+0.640\,\text{MeV}}.
\end{equation*}
In Fig. \ref{fig1E3} we display the graphical process associated with the backwards going RPA amplitude,
\begin{equation*}
Y_1^r(k)=\frac{\frac{1}{2}\Omega_k^{1/2}\Lambda(-2)}{2(\epsilon_k-\epsilon_F)+W_1(-2)}.
\end{equation*}
  \begin{figure}
  \centerline{\includegraphics*[width=0.2\textwidth,angle=0]{nutshell/figs/removal_backward.pdf}}
  \caption{Same as Fig. \ref{fig1E2} but for the backwards going amplitudes.}\label{fig1E3}
  \end{figure}
  Making use of

\begin{equation*}
\begin{split}
2\times(\epsilon_F&-\epsilon_{p_{1/2}})-E_{corr}(-2)=W_1(-2),
\end{split}
\end{equation*}
one can write
\begin{equation*}
\begin{split}
2\times(\epsilon_F-\epsilon_{p_{1/2}})+2\times(\epsilon_k-\epsilon_{F})-E_{corr}(-2)=2\times(\epsilon_k-\epsilon_{F})+W_1(-2),
\end{split}
\end{equation*}
leading to 
\begin{equation*}
\begin{split}
2\times(|\epsilon_{p_{1/2}}|-|\epsilon_k|)-E_{corr}(-2)=2\times(|\epsilon_{p_{1/2}}|-|\epsilon_{g_{9/2}}|)+2\times(|\epsilon_{g_{9/2}}|-|\epsilon_k|)-E_{corr}(-2).
\end{split}
\end{equation*}
Thus, 
\begin{equation*}
Y_1^r(k)=\frac{\frac{1}{2}\Omega_k^{1/2}\Lambda(-2)}{2(|\epsilon_{g_{9/2}}|-|\epsilon_k|)+2(|\epsilon_{p_{1/2}}|-|\epsilon_{g_{9/2}}|)-E_{corr}(-2)}.
\end{equation*}
With the help of  $2\times(|\epsilon_{p_{1/2}}|-|\epsilon_{g_{9/2}}|)-E_{corr}(-2)=6.82 \text{MeV}-0.640 \text{MeV}=6.18 \text{MeV}$, one obtains,
\begin{equation*}
Y_1^r(k)=\frac{\frac{1}{2}\Omega_k^{1/2}\Lambda(-2)}{2(|\epsilon_{g_{9/2}}|-|\epsilon_k|)+6.18\,\text{MeV}}.
\end{equation*}
The above  expressions of $X_1^r(i)$ and $Y_1^r(k)$ contain the experimental values of the $2$--hole correlation energies (0.640 MeV). Because (see Fig. \ref{fig1E1}) the associated values of $G$ does not lead to the observed correlation energy of the pair addition mode (1.248 MeV), we prefer to choose a single intermediate value of $G$ and use the resulting $E_{corr}(-2)$ (=0.5 MeV) and $E_{corr}(+2)$ (=1.5 MeV), correlation energies, to calculate the corresponding $X,Y$ amplitudes for both the lowest removal and lowest addition pairing modes. Making use of, 
\begin{equation*}
\begin{split}
2\times(|\epsilon_{p_{1/2}}|-|\epsilon_{g_{9/2}}|)=6.82\,\text{MeV}\quad\text{and}\quad 2&\times(|\epsilon_{p_{1/2}}|-|\epsilon_{g_{9/2}}|)-E_{corr}(-2)\\
&=(6.82-0.5)\,\text{MeV}=6.32\,\text{MeV},
\end{split}
\end{equation*}
one can write
\begin{equation*}
\begin{split}
X_1^r(i)&=\frac{\frac{1}{2}\Omega_i^{1/2}\Lambda(-2)}{2(|\epsilon_i|-|\epsilon_{p_{1/2}}|)+0.5\,\text{MeV}},\\ Y_1^r(k)&=\frac{\frac{1}{2}\Omega_k^{1/2}\Lambda(-2)}{2(|\epsilon_{g_{9/2}}|-|\epsilon_k|)+6.32\,\text{MeV}}.
\end{split}
\end{equation*}
Tables \ref{tab1E2} and \ref{tab1E3} contain the amplitudes of the pair removal mode of $^{208}$Pb ($\Gamma^\dagger_r(1)=\sum X^r_{1}(i)\Gamma^\dagger_i+\sum Y^r_{1}(k)\Gamma_k$), that is of the two neutron  correlated hole state describing $|^{206}\text{Pb (gs)}\rangle=\Gamma^\dagger_r(1)|0\rangle$. 


It is of notice that the coupling strength $\Lambda (-2)$ with which the pair removal mode couples to the single--particle (--hole) states is calculated by normalizing the amplitudes: 1) (Tamm Dancoff, TD)$\sum_iA^2(i)=1.5549 \,\text{MeV}^{-2}$ and thus $\Lambda (-2)=0.802$ MeV, ($\sum X(i)^2_{TD}=1$); 2) RPA, $\Lambda_1^2 (-2)\times(\sum_i A^2(i)-\sum_k B^2(k))=\Lambda_1^2(-2)\times1.45073=1$. Thus $\Lambda_1(-2)=0.830$ MeV. The above results shows that there is a few percentage difference between the two values of $\Lambda$ (TD and RPA), as well as for the corresponding $X$ amplitudes. Nonetheless, ground state correlations as expressed by the $Y$ amplitudes, gives rise to a 52\% increase in the $^{206}$Pb$(t,p)^{208}$Pb(gs) absolute cross section, from 0.34 mb to 0.52 mb to be compared with experimental data $\sigma=0.68\pm 0.24$ mb (see Fig. \ref{fig2A4}).


\subsection{Pair addition mode}
\begin{table}
\begin{tabular}{|c|c|c|c|c|c|}
\hline units &  &MeV  &MeV$^{-1}$  & RPA & TD \\ 
\hline  $nlj$&$\Omega_i$  &$|\epsilon_i|-|\epsilon_{p_{1/2}}|$  &$A(i)=\frac{\frac{1}{2}\Omega_i^{1/2}}{2(|\epsilon_i|-|\epsilon_{p_{1/2}}|)+0.5\,\text{MeV}}$  & $X_1^r(i)$ & $X_1^r(i)$ \\ 
\hline  $2p_{1/2}$& 1 &  0& 1 &  0.83&  0.80\\ 
\hline $1f_{5/2}$ & 3 &  0.57&0.528  & 0.44 & 0.42 \\ 
\hline  $2p_{3/2}$& 2 &  0.90&  0.307&  0.25&  0.25\\ 
\hline  $0i_{13/2}$& 7 & 1.64 & 0.350 & 0.29 &  0.28\\ 
\hline  $1f_{7/2}$&  4& 2.35 &  0.192& 0.16 &  0.15\\ 
\hline  $0h_{9/2}$&  5& 3.47 & 0.150 &  0.12&0.12  \\ 
\hline 
\end{tabular}\caption{Forwards going RPA amplitudes of the pair removal mode of $^{208}$Pb (i.e. $|^{206}\text{Pb}\rangle$ state), cf. Table XVI \cite{Broglia:73}.}\label{tab1E2}
\end{table}
\begin{table}
\begin{tabular}{|c|c|c|c|c|}
\hline units &  &MeV  &MeV$^{-1}$  &  RPA  \\ 
\hline  $nlj$&$\Omega_k$  &$|\epsilon_{g_{9/2}}|-|\epsilon_k|$  &$B(k)=\frac{\frac{1}{2}\Omega_i^{1/2}}{2(|\epsilon_{g_{9/2}}|-|\epsilon_k|)+6.23\,\text{MeV}}$  & $Y_1^r(i)$  \\ 
\hline  $1g_{9/2}$& 5 &  0& 0.179 &  -0.15\\ 
\hline $0i_{11/2}$ & 6 &  0.77&0.158  & -0.13  \\ 
\hline  $0j_{15/2}$& 8 &  1.41&  0.156&  -0.13\\ 
\hline  $2d_{5/2}$& 3 & 1.56 & 0.093 & -0.08 \\ 
\hline  $3s_{1/2}$&  1& 2.03 &  0.046& -0.04\\ 
\hline  $1g_{7/2}$&  4& 2.47 & 0.090 &  -0.07  \\ 
\hline  $2d_{3/2}$&  2& 2.51 & 0.063 &  -0.05 \\ 
\hline 
\end{tabular}\caption{Same as Table \ref{tab1E2} but for the backwards amplitude.}\label{tab1E3}
\end{table}
In Fig. \ref{fig1E4} the $X$--amplitude of the pair addition mode is shown (NFT diagram). The associated expression
\begin{equation*}
X_1^a(k)=\frac{\frac{1}{2}\Omega_k^{1/2}\Lambda_1(+2)}{2(\epsilon_k-\epsilon_F)-W_1(+2)},
\end{equation*}
can be written, making use of
  \begin{figure}
  \centerline{\includegraphics*[width=0.1\textwidth,angle=0]{nutshell/figs/addition_forward.pdf}}
  \caption{Same as Fig. \ref{fig1E2} but for the pair addition mode}\label{fig1E4}
  \end{figure}
  
\begin{equation*}
\begin{split}
2\times(\epsilon_k-\epsilon_F)-W_1(+2)&=2\times(\epsilon_k-\epsilon_F)-2\times(\epsilon_{g_{9/2}}-\epsilon_F)+E_{corr}(+2)\\
&=2\times(\epsilon_k-\epsilon_{g_{9/2}})+E_{corr}(+2)=2\times(|\epsilon_{g_{9/2}}|-|\epsilon_k|)+E_{corr}(+2),
\end{split}
\end{equation*}
as
\begin{equation*}
X_1^a(k)=\frac{\frac{1}{2}\Omega_k^{1/2}\Lambda_1(+2)}{2(|\epsilon_{g_{9/2}}|-|\epsilon_k|)+E_{corr}(+2)}.
\end{equation*}
Similarly (cf. Fig. \ref{fig1E5}),
\begin{equation*}
Y_1^a(i)=\frac{\frac{1}{2}\Omega_i^{1/2}\Lambda_1(+2)}{2(\epsilon_F-\epsilon_i)+W_1(+2)},
\end{equation*}
can be written, with the help of the relation
\begin{equation*}
\begin{split}
2\times(\epsilon_F-\epsilon_i)+W_1(+2)&=2\times(\epsilon_F-\epsilon_i)-2\times(\epsilon_{g_{9/2}}-\epsilon_F)-E_{corr}(+2)\\
&=2\times(\epsilon_{p_{1/2}}-\epsilon_i)+2\times(\epsilon_{g_{9/2}}-\epsilon_{p_{1/2}})-E_{corr}(+2)\\
&=2\times(|\epsilon_i|-|\epsilon_{p_{1/2}}|)+2\times(|\epsilon_{p_{1/2}}|-|\epsilon_{g_{9/2}}|)-E_{corr}(+2),
\end{split}
\end{equation*}
as
  \begin{figure}
  \centerline{\includegraphics*[width=0.2\textwidth,angle=0]{nutshell/figs/addition_backward.pdf}}
  \caption{Same as Fig. \ref{fig1E3} but for the pair addition mode}\label{fig1E5}
  \end{figure}

\begin{equation*}
Y_1^a(i)=\frac{\frac{1}{2}\Omega_i^{1/2}\Lambda_1(+2)}{2(|\epsilon_i|-|\epsilon_{p_{1/2}}|)+2\Delta\epsilon_{sp}-E_{corr}(+2)}.
\end{equation*}
Making use of $E_{corr}(+2)=1.5$ MeV (cf. Fig. \ref{fig1E1}) and 
\begin{equation*}
\Delta\epsilon_{sp}=2\times(|\epsilon_{p_{1/2}}|-|\epsilon_{g_{9/2}}|)=6.28\,\text{MeV},
\end{equation*}
one can write $2\Delta\epsilon_{sp}-E_{corr}(+2)=(6.82-1.5)$ MeV=5.32 MeV, leading to
\begin{equation*}
\left\{\begin{array}{c}
X_1^a(k)=\frac{\frac{1}{2}\Omega_k^{1/2}\Lambda(-2)}{2(|\epsilon_{g_{9/2}}|-|\epsilon_k|)+1.5\,\text{MeV}}, \\ 
Y_1^a(i)=-\frac{\frac{1}{2}\Omega_i^{1/2}\Lambda(+2)}{2(|\epsilon_i|-|\epsilon_{p_{1/2}}|)+5.32\,\text{MeV}}.
\end{array}\right.
\end{equation*}
The corresponding numerical values are displayed in Tables \ref{tab1E4} and \ref{tab1E5}, while in Fig. \ref{fig1E6} we display a schematic summary of the graphical solution of the dispersion relations.

  \begin{figure}
  \centerline{\includegraphics*[width=\textwidth,angle=0]{nutshell/figs/fig1E6.pdf}}
  \caption{Schematic representation of the quantal phase transition taking place as a function of the pairing coupling
  constant in a (model) closed shell nucleus. (a)
  dispersion relation associated with the RPA diagonalization of the Hamiltonian
  $H = H_{sp} + H_p$ for the pair addition and pair removal modes. In the insets are
  shown the two-particle transfer processes exciting these modes, which testify to
  the fact that the associated zero point fluctuations (ZPF) which diverge at
  $G = G_{crit}$, blur the distinction between occupied and empty states typical of
  closed shell nuclei. (b) occupation number associated with the single-particle
  levels. For $G < G_{crit}$ there is a dynamical depopulation (population) of levels
  $ i(k)$ below (above) the Fermi energy. For $G > G_{crit}$,
  the deformation of the Fermi
  surface becomes static, although with a non--vanishing dynamic component (cf. Fig. \ref{fig1.2}).}\label{fig1E6}
  \end{figure}
\begin{table}
\begin{tabular}{|c|c|c|c|c|}
\hline units &  &MeV  &MeV$^{-1}$  &    \\ 
\hline  $nlj$&$\Omega_k$  &$|\epsilon_{g_{9/2}}|-|\epsilon_k|$  &$C(k)=\frac{\frac{1}{2}\Omega_k^{1/2}}{2(|\epsilon_{g_{9/2}}|-|\epsilon_k|)+1.5\,\text{MeV}}$  $\vphantom{\prod}^{a)}$& $X_1^a(k)$  \\ 
\hline  $1g_{9/2}$& 5 &  0& 0.745 &  0.82\\ 
\hline $0i_{11/2}$ & 6 &  0.77&0.403  & 0.44  \\ 
\hline  $0j_{15/2}$& 8 &  1.41&  0.327&  0.36\\ 
\hline  $2d_{5/2}$& 3 & 1.56 & 0.187 & 0.21 \\ 
\hline  $3s_{1/2}$&  1& 2.03 &  0.090& 0.10\\ 
\hline  $1g_{7/2}$&  4& 2.47 & 0.155 &  0.17  \\ 
\hline  $2d_{3/2}$&  2& 2.51 & 0.108 &  0.12 \\ 
\hline 
\end{tabular}\caption{Forwards going RPA amplitudes associated with the pair addition mode of $^{208}$Pb (cf.  Table XVI \cite{Broglia:73}). a) $\sum_{k}C^2(k)=0.903$}\label{tab1E4}
\end{table}
\begin{table}
\begin{tabular}{|c|c|c|c|c|}
\hline units &  &MeV  &MeV$^{-1}$  &    \\ 
\hline  $nlj$&$\Omega_i$  &$|\epsilon_i|-|\epsilon_{p_{1/2}}|$  &$D(i)=\frac{\frac{1}{2}\Omega_i^{1/2}}{2(|\epsilon_i|-|\epsilon_{p_{1/2}}|)+5.32\,\text{MeV}}$ $^{a)}$ & $Y_1^a(i)$  \\ 
\hline  $2p_{1/2}$& 1 &  0& -0.094 &  -0.1\\ 
\hline $1f_{5/2}$ & 3 &  0.57& -0.134  & -0.15 \\ 
\hline  $2p_{3/2}$& 2 &  0.90&  -0.099&  -0.11\\ 
\hline  $0i_{13/2}$& 7 & 1.64 & -0.154 & -0.17 \\ 
\hline  $1f_{7/2}$&  4& 2.35 &  -0.100& -0.11 \\ 
\hline  $0h_{9/2}$&  5& 3.47 & -0.091 &  -0.10  \\ 
\hline 
\end{tabular}\caption{Same as Table \ref{tab1E4} but for the backwards going amplitude. a) $\sum_i D^2(i)=0.079$ and $\Lambda^2(+2) (\sum_k C^2(k)-D^2(i))=\Lambda^2(+2) (0.903-0.079)$ MeV$^{-2}=0.824$MeV$^{-2}$; $\Lambda(+2)=(0.824)^{-1/2}$MeV, thus $\Lambda(+2)=1.102$ MeV.}\label{tab1E5}
\end{table}

Let us conclude this Section by noting that while the harmonic (RPA) description of the pair vibrational mode of $^{208}$Pb provides a fair picture of the two neutron transfer spectroscopic amplitudes, in keeping with the collective character of these (coherent) states, conspicuous anharmonicities in the multi--phonon spectrum have been observed and calculated\footnote{Cf. for example \cite{Flynn:72},\cite{Lanford:73},\cite{Bortignon:78}, \cite{Clark:06}.}. Within the framework of Fig. \ref{fig1D1}, we schematically emphasize in Fig. \ref{fig1_E8} the relative importance of dynamic and static pairing distortions, in comparison with the corresponding quantities in the case of quadrupole surface distortions in 3D--space\footnote{For details cf. \cite{Bes:77}, \cite{Broglia:68}, \cite{Bes:88},\cite{Barranco:87a} \cite{Shimizu:89}, \cite{Shimizu:13}, \cite{Vaquero:13} and references therein.}  These results underscore the major role pairing vibrations play in nuclei around closed shells, while those collected in Fig. \ref{fig1.2} their importance in gauge invariance restoration in systems far away from closed shells.
  \begin{figure}
  \centerline{\includegraphics*[width=\textwidth,angle=0]{nutshell/figs/fig1E8.pdf}}
  \caption{Relative importance of dynamic and static pairing distortion ($\alpha_{dyn}$ and $\alpha_0$ respectively) associated with closed shell and open shell  nuclei, calculated in terms of a two level model, as compared with similar quantities for the case of quadrupole surface degrees of freedom ($\beta_2$--values). The parameter $x'$ (product of the effective pairing strength $G'=Z_\omega^2(v_p^{bare}+v_p^{ind})$ and of the effective density of levels at the Fermi energy $N'(0)=Z_\omega^{-1}N(0)=Z^{-1}_\omega(2\Omega/D)=2\Omega'/D=2\Omega/D';\Omega'=Z_\omega^{-1}\Omega,D'=Z_\omega D$), measures the relative importance of the single--particle gap $D'=Z_\omega D$ and of the pair correlation $G'\Omega$ (cf. \cite{Brink:05} App. H, Sect. H.4).}\label{fig1_E8}
  \end{figure}
\section[Halo pair addition mode and pygmy]{Halo pair addition mode and pygmy: a new mechnism to break gauge invariance}\label{App1AF}
Pairing is intimately connected with particle number violation and thus spontaneous breaking of gauge invariance, as testified by the order parameter\\ \mbox{$\langle BCS|P^{\dagger}|BCS\rangle=\alpha_0$}.  In the nuclear case and, at variance with condensed matter, dynamical breaking of gauge symmetry is similarly important to that associated with static distortions (e.g. pairing vibrations around closed shell nuclei, cf. Fig. \ref{fig1.1}; see also Fig. \ref{fig1D2} and Fig. \ref{fig1_E8}) The fact that the average single--particle field acts as an external potential (like e.g. a magnetic field in metallic superconductors) is one of the reasons of the existence of a critical value $G_c$ of the pairing strength $G$ to bind Cooper pairs in nuclei. Spatial quantization in finite systems at large and in nuclei in particular, is intimately connected with the paramount role the surface plays in these systems\footnote{cf. \cite{Broglia:02d} and references therein.}. Another consequence of this role is  the fact that in nuclei an important fraction ($\approx$50\%) of Cooper pair binding is due to the exchange of collective vibrations between the partners of the pair\footnote{ Cf. e.g. \cite{Barranco:99}, \cite{Brink:05}, \cite{Saperstein:12}, \cite{Avdenkov:12}, \cite{Lombardo:12}, and references therein; cf. also \cite{Bohr:75}, p. 432.}, the rest being associated with the bare $NN$--interaction in the $^1S_0$ channel (cf. Fig. \ref{fig1F1}) plus possible $3N$ corrections\footnote{Cf. e.g. \cite{Lesinski:12}, \cite{Pankratov:11}, \cite{Hergert:09}.}. Within this context we note that the results displayed in Fig. \ref{fig1.2} provide one of the clearest quantitative examples of the central and ubiquitous role pairing vibrations play in nuclear pairing correlations.


The study of light exotic nuclei lying along the neutron drip line have revealed a novel aspect of the interplay between shell effects and induced pairing interaction. It has been found  that there are situations in which spatial quantization screens, essentially completely, the bare nucleon--nucleon paring interaction. This happens in the case in which the nuclear valence orbitals ($^{10}$Li) are $s,p$--states at threshold\footnote{Pairing anti--halo effect; \cite{Bennaceur:00} 
, \cite{Hamamoto:03}, \cite{Hamamoto:04}.}. An example of situations of this type is provided by $N=6$ (parity inversion; see Chapter \ref{C6}, Section \ref{C6S2.2x}) isotones (see also Sect. \ref{C6S2} in connection with $^{11}$Be$_7$). In particular, by $^{11}$Li, in which case the strongly renormalized $s_{1/2}$ and $p_{1/2}$ valence orbitals are a virtual and a resonant state lying at $\approx$0.2 and 0.5 MeV in the continuum, respectively. Let us elaborate on this point. 
The binding provided by a contact pairing interaction $V_\delta (|\mathbf{r}-\mathbf{r}'|)$ ($\delta$--force) to a pair of fermions moving in time--reversal states\footnote{cf. e.g. Eq. (2.12) \cite{Brink:05}.} is given by the matrix element,
\begin{equation*}
M_j=\langle j^2(0)|V_\delta|j^2(0)\rangle=-\frac{(2j+1)}{2} V_0 I(j)\approx \frac{(2j+1)}{2}V_0\frac{3}{R^3}.
\end{equation*} 
Of notice that $G=V_0I(j)$. 
The ratio of the above matrix element for the halo nucleus $^{11}$Li and for an hypothetical normal nucleus of mass $A=11$ is
\begin{equation*}
r=\frac{(M_j)_{halo}}{(M_j)_{core}}=\frac{2}{(2j+1)}\left(\frac{R_0}{R}\right)^3.
\end{equation*}
The quantities $R_0=1.2 A^{1/3}$fm$=2.7$fm ($A=11$), and $R=\sqrt{\frac{5}{3}}\langle r^2\rangle^{1/2}_{^{11}\text{Li}}=\sqrt{\frac{5}{3}}(3.55\pm0.1)$ fm =$(4.6\pm 0.13)$ fm are the radius of a stable nucleus of mass $A=11$ (systematics), and  the measured radius of $^{11}$Li, respectively. The quantity $j$ is the angular momentum representative for a nucleus of mass $A=11 (j\sim k_F R_0\approx 3-4)$. One thus obtains $r=0.048$. Consequently, the bare $NN$--nucleon pairing interaction is expected to become strongly screened, the resulting effective $G$--value 
\begin{equation}\label{eq1C2AppF}
G'=G\times r=0.048\times 25 \text{MeV}/A\approx 1 \text{MeV}/A\approx 0.1\,\text{MeV},
\end{equation}
becoming subcritical and thus unable to bind the halo Cooper pair ($2\epsilon_{s_{1/2}}=0.4$ MeV) to the $^9$Li core.


 Further insight into this question can be shed making use of the multipole expansion of a general interaction
\begin{equation*}
v(|\mathbf{r}_1-\mathbf r_2|)=\sum_{\lambda}V_{\lambda}(r_1,r_2)P_\lambda(\cos\theta_{12}).
\end{equation*}
Because the function $P_\lambda$ drops from its maximum at $\theta_{12}=0$ in an angular distance $1/\lambda$, particles 1 and 2 interact through the component $\lambda$ of the force, only if $r_{12}=|\mathbf{r}_1-\mathbf{r}_2|<R/\lambda$, where $R$ is the mean value of the radii $\mathbf{r}_1$ and $\mathbf{r}_2$. Thus, as $\lambda$ increases, the effective force range decreases. For a force of range much greater than the nuclear size, only the $\lambda\approx0$ (long wavelength) term is important. At the other extreme, a $\delta$--function force has coefficients $V_\lambda(r_1,r_2)\left(=\tfrac{(2\lambda+1)}{4\pi r_1^2}\delta(r_1-r_2)\right)$ that increase with $\lambda$. In the case of $^{11}$Li(gs) we are thus forced to accept the need for a long range, low $\lambda$ pairing interaction, as responsible for the binding of the dineutron, halo Cooper pair to the $^9$Li core. This is equivalent to saying, an induced pairing interaction arising from the exchange of vibrations with low $\lambda$--value.
\subsection{Cooper pair binding: a novel embodiment of the Axel--Brink hypothesis.}\label{sect1F1}
In what follows we discuss a possible novel test of the Axel--Brink hypothesis\footnote{The color of an object can be determined in two ways: by illuminating it with white light and see which wavelength it absorbs, or by heating it up and see the same wavelength it emits. In both cases one is talking about dipole radiation. To describe the de--excitation process of hot nuclei requires the knowledge of the photon interactions with excited states. The common assumption, known as the Axel--Brink hypothesis, has been that each excited state of a nucleus carries a giant dipole resonance (GDR) on top of it, and that the properties of such resonances are unaffected by any excitation of the nucleus (\cite{Brink:55}, \cite{Lynn:68} pag. 321, \cite{Axel:62}; cf. also \cite{Bertsch:86} and \cite{Bortignon:98})}.Within the $s,p$ subspace, the most natural low multipolarity, long wavelength vibration is the dipole mode. From systematics, the centroid of these vibrations is $\hbar \omega_{GDR}\approx 100$ MeV$/R$, $R$ being the nuclear radius\footnote{cf. e.g. \cite{Bortignon:98} and \cite{Bertsch:05}.}. Thus, in the case of $^{11}$Li, one expects the centroid of the Giant Dipole Resonance carrying $\approx$100\% of the energy weighted sum rule (EWSR) at $\hbar \omega_{GDR}\approx 100$ MeV$/2.7\approx 37$ MeV. Now, such a high frequency mode can hardly be expected to give rise to anything, but polarization effects. On the other hand, there exists experimental evidence which testifies to the presence of a rather sharp dipole state with centroid at $\lessapprox1$ MeV and carrying $\approx 8$\% of the EWSR\footnote{\cite{Zinser:97}, \cite{Nakamura:06}, \cite{Shimoura:95}, \cite{Ieki:93}, \cite{Sackett:93}, \cite{Kanungo:15,Kobayashi:89}.}. The existence of this ``pygmy resonance'' which can be viewed as a simple consequence of the existence of a low--lying particle--hole state associated with the transition $s_{1/2}\rightarrow p_{1/2}$ testifies, arguably, to the coexistence\footnote{Within this context one can mention similar situations concerning the coexistence of spherical and quadrupole deformed states (cf. e.g. \cite{Wimmer:10}, \cite{Federman:65}, \cite{Federman:66}, \cite{Donau:67} and refs. therein; cf. also \cite{Bohr:63}), typically of nuclei with $N\approx Z$. The fact that the associated inhomogeneous damping on the GDR has modest consequences concerning dipole strength at low energies as compared with radial (isotropic) deformations in $^{11}$Li is understood in terms of the (non--Newtonian) plasticity of the atomic nucleus regarding quadrupole deformations (low--lying collective $2^+$ surface vibrations, fission, exotic decay (cf. \cite{Barranco:88}, \cite{Barranco:89,Bertsch:88b}, \cite{Bertsch:87})), and of the little compressibility displayed by the same system and connected with saturation properties.} 
of two states with rather different radii in the ground state. One, closely connected with the $^{9}$Li core, ($\approx 2.5$ fm), the second with the diffuse halo ($\approx 4.6$ fm), namely displaying a large radial deformation (neutron skin), and thus able to induce a conspicuous inhomogeneous damping to the dipole mode. 


Before proceeding, let us estimate the overlap $\mathcal{O}$ between the two ``ground states''. Making use of a schematic expression for the single--particle radial wavefunctions\footnote{\cite{Bohr:69}.}
\begin{equation}
\mathcal{R}=\sqrt{3/R_0^3}\;\Theta(r-R_0),
\end{equation}
where 
\begin{equation*}
\Theta=1 \quad (r\leq R_0);\quad 0 \quad (r>R_0),
\end{equation*}
leading to,
\begin{equation}
\int_0^{\infty}dr r^2 \mathcal{R}^2(r)=\frac{3}{R_0^3}\int_0^{R_0}dr^3/3=1,
\end{equation}
one can work out the overlap $\mathcal{O}$ between the two halo neutrons an the core nucleons. That is, 
\begin{equation}\label{eq2.6.4}
\begin{split}
\mathcal{O}&=|\langle\mathcal{R}_{halo}|\mathcal{R}_{core}\rangle|^2=\left(\sqrt{\frac{3}{R_0^3}}\sqrt{\frac{3}{R^3}}\int_0^{\infty}dr\,r^2\Theta(r-R)\Theta(r-R_0)\right)^2\\
&=\left(\sqrt{\frac{3}{R_0^3}}\sqrt{\frac{3}{R^3}}\int_0^{R_0}dr^3/3\,\right)^2=(R_0/R)^3=0.16,
\end{split}
\end{equation}
where use has been made of $\Theta(r-R)\Theta(r-R_0)=\Theta(r-R_0), R_0=1.2A^{1/3} \text{fm}=2.5 \text{fm} (A=9)$ and $R=(4.6\pm 0.13)$ fm.
Because of the small value of this overlap, one can posit that a \textit{bona fide} dipole pygmy resonance is a GDR based on an exotic, unusually extended state as compared to systematics $(A\approx (4.6/1.2)^3\approx 60)$, i.e., in the present case, to  a system with an effective $A$ mass number about 5 times that predicted by systematics.


Of notice that, the small values of $r$ and of $\mathcal{O}$ have essentially the same origin. On the other hand, they have apparently, rather different physical consequences. In fact, the first makes the bare pairing interaction strength $G$ subcritical, while the second one screens the repulsive symmetry potential $V_1(\approx +25 $ MeV)\footnote{See e.g. \cite{Bortignon:98} Eq. (3.48) and refs. therein.}, that is, the price one has to pay to separate protons from neutrons. This effect allows for a consistent fraction of the dipole Thomas--Reiche--Kuhn sum rule, that is of the $J^{\pi}=1^-$ energy weighted sum rule (EWSR), to come low in energy ($p_{1/2}-s_{1/2}$ transition) from the value $E_{GDR}\approx(80/A^{1/3})$ MeV and, acting as an intermediate boson between the two halo neutrons, glue them to the $^{9}$Li core. Summing up, the halo anti--pairing effect $G_{screened}=r\times G\ll G<G_{crit}$ triggers ($\mathcal{O}V_1\ll V_1$) the virtual presence of a ``gas'' of dipole (pygmy) bosons which, exchanged between the two halo neutrons (cf. Fig. \ref{pigmy}), overcompensates the reduction of the bare pairing interaction, leading to the binding of the halo Cooper pair to the core (anti--(halo anti--pairing effect)). It can thus be stated that the halo of $^{11}$Li and the pygmy resonance built on top of it constitute a pair of symbiotic states (see also Chapter \ref{C8}, in particular Fig. \ref{fig8_2_4x}).

Let us further elaborate on these issues. Making use of the relation $\langle r^2\rangle^{1/2}\approx (3/5)^{1/2}R$ between mean square radius and radius, one may write
\begin{equation*}
\langle r^2\rangle_{^{11}\text{Li}}\approx \frac{3}{5}R_{eff}^2(^{11}\text{Li}).
\end{equation*}
 with
\begin{equation*}
R_{eff}^2(^{11}\text{Li})=\left(\frac{9}{11}R_0^2(^9\text{Li})+\frac{2}{11}\left(\frac{\xi}{2}\right)^2\right),
\end{equation*}
where
\begin{equation*}
R_0(^9\text{Li})=2.5 \text{fm},
\end{equation*}
is the $^9$Li radius ($R_0=r_0A^{1/3}, r_0=1.2$fm), while $\xi$ is the correlation length of the Cooper pair neutron halo. An estimate of this quantity is provided by the relation
 \begin{equation*}
\xi=\frac{\hbar v_F}{2E_{corr}}\approx 20 \, \text{fm},
 \end{equation*}
in keeping with the fact that in $^{11}$Li, $(v_F/c)\approx 0.1$ and $E_{corr}\approx0.5$ MeV. Consequently, 
\begin{align}\label{eq2.F.5}
R_{eff}\,(^{11}\text{Li})\approx 4.83 \;\text{fm}
\end{align} 
and  $\langle r^2\rangle_{^{11}\text{Li}}^{1/2}\approx 3.74$fm, in overall agreement with the experimental value $\langle r^2\rangle_{^{11}\text{Li}}^{1/2}= 3.55\pm0.1$fm\footnote{\cite{Kobayashi:89}.}. It is of notice that this value implies  the radius $R$($^{11}$Li)$=\sqrt{5/3\langle r^2\rangle_{^{11}\text{Li}}}=4.58\pm 0.13$ fm.


We now proceed to the calculation of the centroid of the dipole pygmy resonance of $^{11}$Li in the RPA making use of the separable interaction
 \begin{equation}\label{eq2.F.6}
H_D=-\kappa_1\vec D\cdot\vec D
 \end{equation}
where $\vec D=\vec r$ and
 \begin{equation}
\kappa_1=\frac{-5V_1}{AR^2}.
 \end{equation}
The resulting dispersion relation is\footnote{cf. (3.30) p.55 of \cite{Bortignon:98}.}
\begin{equation}
W(E)=\sum_{k,i}\frac{2(\epsilon_k-\epsilon_i)|\langle \tilde i|F|k\rangle|^2}{(\epsilon_k-\epsilon_i)^2-E^2}.
\end{equation}
 Making use of this relation and of the fact that\footnote{See Fig. \ref{fig1F3}; see also  p.264 \cite{Brink:05}.} 
 $\epsilon_{\nu_k}-\epsilon_{\nu_i}=\epsilon_{p_{1/2}}-\epsilon_{s_{1/2}}\approx 0.3 $MeV, and that the EWSR associated with the $^{11}$Li pigmy resonance is $\approx 8$\% of the total Thomas--Reiche--Kuhn sum rule
 \begin{equation}
\sum_n |\langle0|F|n\rangle|^2(E_n-E_0)=\frac{\hbar^2}{2M}\int d\mathbf r |\vec\nabla F|^2 \rho(r),
 \end{equation}
 which, for $F=r$ has the value $\hbar^2 A/2M$ one can write\footnote{cf. \cite{Bertsch:05} pag. 53.}, 
\begin{equation*}
2\times 0.08\times \frac{\hbar^2A}{2M}=\frac{1}{\kappa_1}[(0.3\text{MeV})^2-(\hbar \omega_{pygmy})^2],
\end{equation*}
 and thus
\begin{equation*}
 (\hbar\omega_{pygmy})^2=(0.3 \text{MeV})^2-2\times 0.08\times\frac{\hbar^2A}{2M}\kappa_1,
\end{equation*}
 where\footnote{see \cite{Bortignon:98}.}
\begin{equation*}
\kappa_1=-\frac{5V_1}{A(\xi/2)^2}\left(\frac{2}{11}\right)=-\frac{125\text{MeV}}{A 100 \text{fm}^2}\left(\frac{2}{11}\right)\approx \kappa_1^0\times 0.045= -\frac{2.5}{A^2}\text{fm}^{-2}\text{MeV},
\end{equation*}
the ratio in parenthesis reflecting the fact that only 2 out of 11 nucleons, slosh back and forth in an extended configuration with little overlap with the other nucleons, while
\begin{equation}
\kappa_1^0=-\frac{5V_1}{AR^2_{eff}(^{11}\text{Li})}\approx 0.49\; \text{MeV fm}^{-2},
\end{equation}
is the standard self consistent dipole strength\footnote{cf. \cite{Bohr:75}.}.
 One then obtains,
\begin{equation*}
-2\times 0.08\frac{\hbar^2A}{2M}\kappa_1=2\times 0.08\times 20\,\text{MeV fm}^2\times A\times\frac{2.5}{A^2}\text{fm}^{-2}\text{MeV}\approx 0.73 \text{MeV}^2\approx (0.85\text{MeV})^2.
\end{equation*} 
 Consequently
  \begin{figure}
  \centerline{\includegraphics*[width=0.7\textwidth,angle=0]{nutshell/figs/fig1F1.pdf}}\caption{(top) Nuclear density $\rho$ in units of fm$^{-3}$, plotted as a function of the distance $r$ (in units of fm) from
    the centre of the nucleus (after \cite{Bohr:69}). Saturation density correspond to $\approx$0.17 fm$^{-3}$, equivalent to $2.8\times 10^{14}$ g/cm$^3$. Because of the short range of
    the nuclear force, the strong force, the nuclear density changes from 90\% of saturation density to 10\% within 0.65 fm, i.e. within the
    nuclear diffusivity. (bottom) Phase shift  parameter associated with the elastic scattering of two nucleons moving in states of time reversal, so
    called $^1S_0$ phase shift, in keeping with the fact that the system is in a singlet state of spin zero. The solution of the Schr\"odinger equation
    describing the elastic scattering of a nucleon from a scattering centre (in this case another nucleon) is, at large distances from the
    scattering centre a superposition of the incoming wave and of the outgoing, scattering wave. The interaction of the incoming particle
    with the target particle changes only the amplitude of the outgoing wave. This amplitude can be written in terms of a real phase
    shift  or scattering phase $\delta$. Positive values of $\delta$ implies an attractive interaction, negative a repulsive one. For low relative velocities
    (kinetic energies $E_L$), i.e. around the nuclear surface where the density is low, the $^1S_0$ phase shift arizing from the exchange of mesons
    (i.g. pions, represented by an horizontal dotted  line) between nucleons (represented by upward pointing arrowed lines)
    is attractive. This mechanism provides about half of the glue to nucleons moving in time reversal states to form Cooper pairs. These
    pairs behaves like boson and eventually condense in a single quantal state leading to nuclear superfluidity. Cooper pair formation is
    further assisted by the exchange of collective surface vibrations (wavy curve in the scattering process) between the members of the
    pair (after \cite{Broglia:02d}).}\label{fig1F1}
  \end{figure}
\begin{equation*}
\hbar \omega_{pygmy}=\sqrt{(0.3)^2+(0.85)^2}\text{MeV}\approx 0.9\, \text{MeV},
\end{equation*}  
 in overall agreement with the experimental findings\footnote{\cite{Zinser:97,Kanungo:15}.}. It is of notice that the centroid of the pigmy resonance calculated in the RPA with the help of a separable dipole interaction is\footnote{\cite{Barranco:01}; see also Fig. 11.3(a) p.269, \cite{Brink:05}.} $\approx (0.6\,\text{MeV}+ 1.6\, \text{MeV})/2\approx 1.1\, \text{MeV}$.
 \begin{figure}
 \centerline{\includegraphics*[width=0.5\textwidth,angle=0]{nutshell/figs/pigmy.pdf}}
 \caption{Diagrammatic representation of the exchange of a collective $1^-$ pygmy resonance between pairs of nucleons moving in the time--reversal configurations $s_{1/2}^2(0)$ and $p_{1/2}^2(0)$. It is of notice that both these configurations can act as initial states  the figure showing only one of the two possibilities. Consequently, the energy denominator to be used in the simple estimate (\ref{eq2.F.10}) is the average value $DEN=(DEN_1+DEN_2)/2=-\hbar\omega_{pygmy}$ where $DEN_1=\Delta \epsilon-\hbar\omega_{pygmy}$ and $DEN_2=-\Delta\epsilon-\hbar\omega_{pygmy}$, while $\Delta\epsilon=\epsilon_{s_{1/2}}-\epsilon_{p_{1/2}}$.}\label{pigmy}
 \end{figure}
  \begin{figure}
  \centerline{\includegraphics*[width=\textwidth,angle=0]{nutshell/figs/fig1F3.pdf}}
  \caption{(Color online) In (I) and (II) the NFT processes renormalizing the single--particle motion ($^{10}$Li) and leading to the effective interaction, sum of the bare (horizontal dotted lines) and induced (wavy curves) interactions which bind the two--neutron halo to the core of $^{9}$Li  thus leading to the $|^{11}$Li$\rangle$ ground state are displayed. In a) and b) are also displayed the  spatial structure of the pure $|s_{1/2}^2(0)\rangle$ configuration and that of the two--neutron halo $|\tilde 0\rangle$ Cooper pair. The modulus squared wave function $|\Psi_0(\mathbf{r}_1,\mathbf{r}_2)|^2=|\langle \mathbf{r}_1,\mathbf{r}_1|0^+\rangle|^2$ describing the motion of the two halo neutrons around the $^9$Li core 
  is shown as a function of the cartesian coordinates of particle 2, for fixed values of the
  position of particle 1 ($r_1 = 5$ fm) represented  by a solid dot, while the core $^9$Li is shown as a red
  circle. The numbers appearing on the $z$--axis of the three-dimensional plots displayed on the left side of the figure are in units of fm$^{-−2}$. After \cite{Barranco:01}.}\label{fig1F3}
  \end{figure}

 Let us now estimate the binding energy which the exchange of the pigmy resonance between two neutron of the Cooper pair halo of $^{11}$Li can provide.
The associated particle--vibration coupling\footnote{cf. e.g. \cite{Brink:05} Eq. (8.42) p.189.} is $\Lambda= \left(\partial W(E)/\partial E|_{\hbar\omega_{pygmy}}\right)^{-1/2}$. Note the use in what follows of a dimensionless dipole single--particle field $F'=F/R_{eff}(^{11}\text{Li})$). This is in keeping with the fact that one wants to obtain a quantity with energy dimensions ($[\Lambda]=$ MeV), and that $\kappa_1$ has been introduced through the Hamiltonian $H_D$ with the self consistent value normalized in terms of $R^2_{eff}(^{11}$Li). 
 One then obtains
\begin{equation*}
\begin{split}
\Lambda^2&=\left\{2\hbar \omega_{pygmy}\frac{2\times 0.08(\frac{\hbar^2A}{2M})/R^2_{eff}(^{11}\text{Li})}{\left[(\epsilon_{p_{1/2}}-\epsilon_{s_{1/2}})^2-(\hbar\omega_{pygmy})^2\right]^2}\right\}^{-1},\\
&=\left\{2\text{MeV}\frac{0.16(\hbar^2A/2M)(1/4.83^2\,\text{fm}^2}{\left[(0.3)^2-(1\text{MeV})^2\right]^2\text{MeV}^4}\right\}^{-1},\\
&=\left(\frac{3\,\text{MeV}^2}{(0.91)^2\,\text{MeV}^4}\right)^{-1},\\
&=\left(\frac{1}{1.7}\right)^2\,\text{MeV}^2\approx 0.35\,\text{MeV}^2,
\end{split}
\end{equation*}   
 leading to $\Lambda\approx 0.6\,\text{MeV}$. The value of the induced interaction matrix elements is then given by (Fig. \ref{pigmy}),
 \begin{equation}\label{eq2.F.10}
M_{ind}=\frac{2\Lambda^2}{DEN}\approx-\frac{2\Lambda^2}{\hbar\omega_{pygmy}}\approx-0.7\,\text{MeV},
 \end{equation}
 the factor of two resulting from the two time ordering contributions (see Fig. \ref{pigmy}). The resulting correlation energy is thus $E_{corr}=|2\epsilon_{s_{1/2}}-G'+M_{ind}|=|0.4-0.1-0.7|\approx 0.4$ MeV, in overall agreement with the experimental\footnote{\cite{Bachelet:08}, \cite{Smith:08}.} findings (0.380 MeV). Of notice that in this estimate the (subcritical) effect of the screened bare pairing interaction has also been used (see Eq. (\ref{eq1C2AppF}))\footnote{That new physics, namely a novel mechanism to (dynamically) violate gauge invariance, finds a scenario of a barely bound Cooper pair at the drip line (half life 8.75 ms) to express itself seems to confirm a recurrent expectation. That truly new complex phenomena appear at the border between rigid order and randomness (see \cite{DeGennes:94}).}.
 
 
 
 This schematic model has been implemented with microscopic detail\footnote{cf. \cite{Barranco:01}.} within the framework of a field theoretical description of the interweaving of collective vibrations and single--particle motion, and is discussed in more detail within the context of single--particle (Chapter \ref{C6}) and two--particle (Chapter \ref{C8}) transfer processes. Here we provide a summary of the theoretical findings. 
 
 
 
 
 In Fig. \ref{fig1F3} (I), the single--particle neutron resonances in $^{10}$Li are given\footnote{See however \cite{Cavallaro:17}.}. The 
  position of the levels $s_{1/2}$ and $p_{1/2}$ determined making use
 of mean-field theory is shown (hatched area and thin horizontal
 line, respectively). The coupling of a single--neutron (upward
 pointing arrowed line) to a vibration (wavy line) calculated
 making use of NFT Feynman diagrams 
 (schematically depicted also in terms of either solid dots (neutron)
 or open circles (neutron hole) moving in a single--particle
 level around or in the $^9$Li core (hatched area)), leads to conspicuous
 shifts in the energy centroid of the $s_{1/2}$ and $p_{1/2}$ resonances
 (shown by thick horizontal lines) and eventually to
 an inversion in their sequence. In Fig. \ref{fig1F3} (II) the  processes binding the  halo neutron system $^{11}$Li are displayed. 
 
 
 
 
 
  Starting with the clothed mean  field
 picture in which two neutrons (solid dots) move in
 time--reversal states around the core $^{9}$Li (hatched area) in the
 $s_{1/2}$ virtual state leading to an unbound $s^2
 _{1/2}(0)$ state where the
 two neutrons are coupled to  angular momentum zero. The associated spatial structure of the uncorrelated pair is shown in \textbf{a)}. The exchange
 of vibrations between the two neutrons displayed in the upper
 part of the figure leads to a density--dependent interaction
 which, added to the nucleon--nucleon bare interaction (see boxed inset) which, as can be seen from the figure, is subcritical, correlates the
 two-neutron system leading to a bound state $|\tilde 0\rangle$ whose wavefunction is  displayed in \textbf{ b)}, together with the spatial structure of the Cooper pair. It is of notice that a large fraction of the induced interaction arises from the exchange of the pigmy resonance (see Fig. \ref{pigmy}) between the two halo neutrons.  Within this scenario one can posit that the $^{11}$Li dipole pigmy resonance can hardly be viewed but in symbiosis with the $^{9}$Li halo neutron pair addition mode and vice versa. For details see Chapter \ref{C8} as well as\footnote{\cite{Barranco:01}.}.
 
 
 Let us conclude this Section by stating that the detailed consequences  of the diagonalization of self--energy processes and of the bare and induced interactions  tantamount to the diagonalization of the many--body Hamiltonian, provides in the case of $^{10}$Li an example of minimal mean field description  (App. \ref{C6AppA}) and, in the case of $^{11}$Li, an example of the fact that pairs of dressed single--particle states lead to abnormal density (induced pairing interaction), also in the case of closed shell systems, due to the strong ZPF associated with pairing vibrations (see also discussion around Eq. (\ref{eq2_2_2})). In keeping with the fact that $^9$Li is a normal, bound nucleus, while $^{10}$Li is not bound testifies to the fact that the binding of two neutrons to the $^9$Li core leading to $^{11}$Li ground state ($S_{2n}\approx 380$ keV), is a pairing phenomenon. 
 \begin{subappendices}
 \section{Nuclear van der Waals Cooper pair}\label{C2SG2}
 The atomic van der Waals (dispersive; retarded) interaction which, like gravitation,  acts between all atoms and molecules, also non--polar, can be written for two systems placed at a distance $R$ as (see App. \ref{C2AppD}), 
 \begin{align}\label{eq1C2AppG}
 \Delta E=-\frac{6\times e^2 \times a_0^5}{R^6}=-\frac{6\times e^2}{(R/a_0)^6}\frac{1}{a_0},
 \end{align}
 where $a_0$ is the Bohr radius. A possible nuclear parallel can be established making the following correspondences,
 \begin{align*}
e^2\rightarrow \Lambda R_0 (^{11}\text{Li})=0.6\, \text{MeV}\times 2.7 \,\text{fm};\quad a_0\rightarrow d=4\,\text{fm};\quad R\rightarrow R_{eff} (^{11}\text{Li})=4.83\,\text{fm}.
 \end{align*} 
 That is,
  \begin{align*}
  \Delta E&=-\frac{6\times \Lambda\times R_0}{R^6}=-\frac{6\times e^2}{(R_{eff} (^{11}\text{Li})/d)^6}\frac{1}{d}=\frac{6\times 0.6\,\text{MeV}\times 2.7 \,\text{fm}}{(4.83/4)^6}\frac{1}{4\,\text{fm}}\\
&  =-\frac{9.72 
\,\text{MeV}}{12.4}\approx -0.8\,\text{MeV}\rightarrow M_{ind}.
  \end{align*}
  Thus,
 \begin{align*}
E_{corr}=|2E_{s_{1/2}}-G'+\Delta E|=|0.4\,\text{MeV}-0.1\,\text{MeV}-0.8\,\text{MeV}|\approx 0.5
 \end{align*} 
to be compared to
 \begin{align*}
(S_{2n})_{exp}\approx 0.380 \,\text{MeV}.
 \end{align*} 


   \begin{figure}
   \centerline{\includegraphics*[width=17cm,angle=0]{nutshell/figs/VdW.pdf}}\caption{NFT Feynman diagrams describing the binding of the halo Cooper pair through pygmy. That is, producing the symbiotic mode involving the pair addition mode and the GDPR. The single--particle states $s_{1/2}$ and $p_{1/2}$ are labeled in (a) $s$ and $p$ for simplicity. The different particle--vibration coupling vertices (either with the quadrupole ($2^+$) or with the pygmy ($1^-$) modes drawn as solid wavy lines) are denoted by a solid dot, and numbered in increasing sequence so as to show that diagram (b) emerges from (a) through time ordering. The motion of the neutrons are drawn in terms of continuous solid curves. In keeping with the fact that the occupation of the single--particle states is neither 1 nor 0 (cf. Eqs. (\ref{eq8_2_1})--(\ref{eq8_2_3})), these states are treated as quasiparticle states. Thus no arrow is drawn on them. Diagram (\textbf{a}) emphasizes the self--energy renormalization of the state $s_{1/2}$ lying in the continuum and which   through its clothing with the quadrupole mode is brought down becoming a virtual ($\epsilon_{s_{1/2}}=0.2$ MeV) state (see (I) and (II)), while (III) contributes to the induced pairing interaction through pygmy (see also Fig. \ref{pigmy}). The ``eagle'' diagram (\textbf{b}) contains (cf. (IV) and (V)) Pauli corrections which push the bound state $p_{1/2}$ into a resonant state in the continuum ($\epsilon_{p_{1/2}}=0.5$ MeV). In other words, processes (I), (II), (III), (IV) and (V) are at the basis of parity inversion, and of the appearance of the new magic number $N=6$. Processes (VI) and (VII) are associated with the pygmy ZPF, while (VIII) contributes to the induced pairing interaction through pygmy (van der Waals--like process).}\label{fig2.A.1}
   \end{figure}
 
 \section{Renormalized coupling constants $^{11}$Li: resum\'e}\label{C2SF2}
 Let us make use of the experimental (empirical),
 \begin{align*}
 \epsilon_{s_{1/2}}&=0.2\, \text{MeV},\\
  \epsilon_{p_{1/2}}&=0.5\, \text{MeV},\\
  V_1&=25\, \text{MeV},
 \end{align*}
and theoretical
  \begin{align*}
R_0(^{11}\text{Li})&=1.2 (11)^{1/3}\,\text{fm}=2.7\,\text{fm},\\
\xi&=20\text{fm},\\
R_{eff}(^{11}\text{Li})&=4.83\,\text{MeV},\\
G&=\frac{25}{A}\,\text{MeV}=2.3\,\text{fm},\\
\kappa_1^0&=-\frac{5V_1}{A R_{eff}^2(^{11}\text{Li})}\approx -0.49\,\text{MeV fm}^{-2},\\
\kappa_1&=-\frac{5V_1}{A \left(\xi/2\right)^2}\left(\frac{2}{11}\right)= -0.021\,\text{MeV fm}^{-2},
  \end{align*}
 inputs.
 
 
  One can then calculate the ratio
    \begin{align*}
r=\frac{2}{(2j+1)}\frac{R_0}{R_{eff}}^3\approx 0.042,
   \end{align*}
 where use was made of $(2j+1)\approx (2k_FR_0+1)\approx 8.34$. Thus, the screened bare pairing interaction is,
\begin{align*}
(G)_{scr}=rG=0.042\times\frac{25}{A}\, \text{MeV}=\frac{1\,\text{MeV}}{A}\approx 0.1\,\text{MeV}.
\end{align*}
 Similarly
 \begin{align*}
\kappa_1=s\kappa_1^0,
 \end{align*}
 where the screening factor is 
  \begin{align*}
s=\frac{R_{eff}^2}{\left(\xi/2\right)^2}\frac{2}{11}\approx 0.042.
  \end{align*}
Thus, the screened symmetry potential is,
  \begin{align*}
(V_1)_{scr}=sV_1=0.042\times 25\, \text{MeV}=1\,\text{MeV}.
  \end{align*}
The fact that $r$ and $s$ coincide within numerical approximations is in keeping with the fact that both quantities are closely related to the overlap
  \begin{align*}
\mathcal{O}=\left(\frac{R_0}{R_{eff}}\right)^3=\left(\frac{2.7\,\text{fm}}{4.83\,\text{fm}}\right)^3=0.17,
  \end{align*}
quantity which has a double hit effect concerning the mechanism which is at the basis of much of the nuclear structure of exotic nuclei at threshold: 1) it makes subcritical the screened bare $NN$--pairing interaction $(G)_{scr}=rG<G_c$ ($(G)_{scr}=1\,\text{MeV}/A$); 2) it screens the symmetry potential drastically, reducing the price one has to pay to separate protons from delocalized neutrons, permitting a consistent chunk ($\approx 8$\%) of the TRK sum rule to  become essentially degenerate with the  ground state ($(V_1)_{scr}=1 \,\text{MeV}$), thus allowing for the first nuclear example of a van der Waals Cooper pair and a novel mechanism to break dynamically gauge invariance: dipole--dipole fluctuating fields associated with the exchange of the pygmy dipole resonance between the halo neutrons of $^{11}$Li. As a result, a new, (composite) elementary mode of nuclear excitation joins the ranks of the previously known: the halo pair addition mode carrying on top of it, a low--lying collective pygmy resonance. This symbiotic mode can be studies through two--particle transfer reactions, eventually in coincidence with $\gamma$--decay. In particular, making use of the reactions,\vspace{0.2cm}
\centerline{$^{9}$Li$(t,p)^{11}$Li$(f)$,}
\vspace{0.2cm}
\centerline{$|f\rangle$; ground state ($L=0$), pygmy ($L=1$),}
\vspace{0.3cm}
and\\
\vspace{0.3cm}
\centerline{$^{10}$Be$(t,p)^{12}$Be$(f)$,}
\vspace{0.2cm}
\centerline{$|f\rangle$; first excited $0^+$ state $(E_x=2.24$ MeV) $(L=0)$,}
\vspace{0.2cm}
pygmy on top of it ($L=1$, arguably the state at $E_x=2.70$ MeV is part of it).\footnote{\cite{Iwasaki:00}.}
\section[Lindemann criterion and  quantality parameter]{Lindemann criterion and connection with quantality parameter}\label{C2AppC}
The original Lindemann criterion\footnote{\cite{Lindemann:10}.} compares the atomic fluctuation amplitude $\langle\Delta r^2\rangle^{1/2}$ with the lattice constant $a$ of a crystal. If this ratio, which is defined as the disorder parameter $\Delta L$, reaches a certain value, fluctuations cannot increase without damaging or destroying the crystal lattice. The results of experiments and simulations show that the critical value of $\Delta_L$ for simple solids is in the range of 0.10 to 0.15, relatively independent of the type of substance, the nature of the interaction potential, and the crystal structure\footnote{\cite{Bilgram:87,Lowen:94,Stillinger:95}.}. Applications of this criterion to an inhomogeneous finite system like a protein in its native state (aperiodic crystal\footnote{\cite{Schrodinger:44}.}, requires evaluation of the generalized Lindemann parameter\footnote{\cite{Stillinger:90}.}
\begin{align}
\Delta_L=\frac{\sqrt{\sum_i\langle \Delta r_i^2\rangle/N}}{a'},
\end{align}  
where $N$ is the number of atoms and $a'$ the most probable non--bonded near--neighbor distance, $\mathbf r_i$ is the position of atom $i$, $\Delta r_i^2=(\mathbf r_i-\langle \mathbf r_i^2\rangle)$, and $\langle\rangle$ denotes configurational averages at the conditions of measurement or simulations (e.g. biological, in which case $T\approx 310$ K, PH$\approx 7$, etc.\footnote{Fluctuations, classical (thermal) or quantal imply a probabilistic description. While one can only predict the odds for a given outcome of an experiment, probabilities themselves evolve in a deterministic fashion.}). The dynamics as a function of the distance from the geometric center of the protein is characterized by defining an interior ($int$) Lindemann parameter, 
\begin{align}
\Delta^{int}_L(r_{cut})=\frac{\sqrt{\sum_{i,r_i<r_{cut}}\langle \Delta r_i^2\rangle/N}}{a'},
\end{align}  
which is obtained by averaging over the atoms that are within a chosen cutoff distance, $r_{cut}$, from the center of mass of the protein.

Simulations and experimental data for a number of proteins, in particular Barnase, Myoglobin, Crambin and Ribonuclease A indicate 0.14 as the critical value distinguishing between solid--like and liquid--like behaviour, and $r_{cut}\approx 6$ \AA. As can be seen from Table \ref{tab2C1}, the interior of a protein, under physiological conditions, is solid--like  ($\Delta_L<0.14$), while its surface is liquid--like ($\Delta_L>0.14$). The beginning of thermal denaturation in the simulations appears to be related to the melting of its interior (i.e. $\Delta^{int}_L>0.14$), so that the entire protein becomes liquid--like. This is also the situation of the denatured state of a protein under physiological conditions\footnote{see e.g. \cite{Rosner:17}.} 



\begin{table}[h]
 \begin{tabular}{|c|c|c|c|c|}
 \hline
 &\multicolumn{4}{|c|}{$\Delta_L(\Delta_L^{int}(6\;\text{\AA}))(300$ K)}\\
 \cline{2-5}
 &\multicolumn{3}{|c|}{MD simulations}&X--ray data\\
 \hline
 Proteins&Barnase&Myoglobin&Crambin&Ribonuclease A\\
 \hline
 all atoms&0.21(0.12)&0.16(0.11)&0.16(0.09)&0.16(0.12)\\
 backbone atoms only&0.16(0.10)&0.12(0.09)&0.12(0.08)&0.13(0.10)\\
 side--chain atoms only&0.25(0.14)&0.18(0.12)&0.19(0.10)&0.19(0.13)\\
 \hline
 \end{tabular}
 \caption{The heavy--atom $\Delta_L(\Delta_L^{int})$ value, for four proteins at 300 K. After \cite{Zhou:99}.}\label{tab2C1}
 \end{table}

\subsection{Lindemann (``disorder'') parameter for a nucleus}
An estimate of  $\sqrt{\sum_i\langle \Delta r_i^2\rangle/A}$ in the case of nuclei considered as a sphere of nuclear matter of radius $R_0$, is provided by the ``spill out'' of nucleons due to quantal effects. That is\footnote{\cite{Bertsch:05}, see e.g. Ch. 5.} $\sqrt{\quad}\approx 0.69\times a_0$, where $a_0$ is of the order of the range of nuclear forces ($\approx 1$ fm).


The average internucleon distance can be determined from the relation\footnote{\cite{Brink:05}, App. C.}
\begin{align}
a'=\left(\frac{V}{A}\right)^{1/3}=\left(\frac{\frac{4\pi}{3}R^3}{A}\right)^{1/3}=\left(\frac{4\pi}{3}\right)^{1/3}\times 1.2\; \text{fm}\approx 2\;\text{fm}
\end{align} 
Thus,
\begin{align}
\Delta_L=\frac{0.69 a_0}{2\;\text{fm}}\approx0.35.
\end{align} 
While it is difficult to compare among them crystals, aperiodic finite crystals and atomic nuclei, arguably, the above value indicates that a nucleus is liquid--like. More precisely, it is made out of a non--Newtonian fluid, which reacts elastically to sudden so\-li\-ci\-ta\-tions $(\lesssim 10^{-22}$ s),  and plastically to long lasting strain $(\gtrsim 10^{-21}$ s). In any case, one expects from $\Delta_L\approx 0.35$ that the nucleon mean free path is long, larger than nuclear dimensions, as also indicated by the quantality parameter (see Sect. \ref{App1A}). 

\section{The van der Waals interaction}\label{C2AppD}













\end{subappendices}









\newpage
\renewcommand{\bibname}{Bibliography Ch 2}
 \bibliographystyle{abbrvnat}
% \bibliography{C:/Gregory/Broglia/notas_ricardo/nuclear_bib}
 \bibliography{../nuclear_bib}

