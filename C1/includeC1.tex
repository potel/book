 \chapter*{Preface}\label{preface}
The elementary modes of nuclear excitation are vibrations and rotations, single-particle  motion, and pairing vibrations and rotations. The reactions which specifically probe  them  are, inelastic scattering plus Coulomb excitation, and single- and two- particle transfer processes, respectively. 

The interweaving of the elementary modes of excitation leads to the renormalization of  energy, radial wavefunction and particle content of the single-particles. Also to the renormalization of energy, width and collectivity of vibrations and rotations. This implies renormalization of transition densities and formfactors, as well as of deformation (order) parameters, both in 3D- and in gauge space. As a consequence, the emergence of a variety of properties, like generalized rigidity and long range correlations in connection with collective pairing states.  Implying, for example, that pair transfer is dominated by the successive tunneling of entangled nucleons and, consequently, the need to go beyond lowest order distorted wave Born approximation, that is, to second order DWBA.    

Within this context one can posit that nuclear structure (bound) and reactions (continuum) are but two aspects of the same physics.  Even more so concerning the study of light exotic halo nuclei, in which case the distinction between bound and continuum states is, to a large extent blurred. This is also the reason why these two aspects of nuclear physics are treated in the present monograph on equal footing,  within the framework of a unified nuclear field theory of structure and reactions (NFT)$_\text{(s+r)}$. 


This theory provides the (graphical) rules to diagonalize in a compact and economic way the nuclear Hamiltonian for both bound and continuum states. It does so in terms of Feynman diagrams which describe the coupling of elementary modes of excitation, correcting for the overcompletness of the basis  (structure) and for the  non-orthogonality of the scattering states (reaction), as well as for Pauli principle violation. The outcome connects directly with observables: absolute reaction cross sections and decay probabilities. 


(NFT)$_\text{(s+r)}$ focuses on the scattering amplitudes which determine the absolute cross sections for the variety of physical processes, involving also those in which (quasi) bosons and fermions are created or annihilated, connecting such processes to formfactors and transition densities. Processes where one set of particles with given energies, momenta, angular momenta, etc. go in and another group (or the same), comes out. 

Pairing vibrations and rotations, closely connected with nuclear superfluidity are  paradigms of quantal nuclear phenomena. They  play an important  role within the field of nuclear structure. It is natural that two-nucleon transfer plays a similar role concerning the probing of the structure of the nucleus.

 
At the basis of fermionic pairing phenomena\footnote{\cite{Bardeen:57a,Bardeen:57b}.} one finds Cooper pairs\footnote{\cite{Cooper:56}.}, weakly bound, very extended, strongly overlapping (quasi-) bosonic entities, made out of pairs of nucleons dressed by collective vibrations and interacting  through the exchange of these vibrations as well as through the bare $NN$-interaction.
Cooper pairs change, under certain conditions\footnote{Value of the intrinsic excitation energy, rotational frequency and distance of closest approach ($D_0$) in Cooper pair tunneling between superfluid nuclei in nuclear reactions, smaller than the correlation length ($\xi$).}, the statistics of the nuclear stuff around the Fermi surface and, condensing, the properties of nuclei close to their ground state. They also display a rather remarkable mechanism of tunneling between  target and projectile in  direct two-nucleon transfer reactions.


Cooper pair partners being weakly bound ($\ll \epsilon_F$, Fermi energy), are correlated over distances (correlation length $\xi$) much larger than nuclear dimensions ($\gg R$, nuclear radius). On the other hand, Cooper pairs --building blocks of the so-called abnormal (pair) density-- are forced to be confined within regions in which normal, single-particle density is present. That is, within nuclear dimensions. Said it differently, the mean field acts on Cooper pairs, as a strong external field,  distorting their spatial structure.


The correlation length paradigm comes into evidence, for example, when two nuclei are set into weak coupling  in a direct nuclear reaction with distance of closest approach $D_0\lesssim\xi$. In such a  case,  the partner nucleons of a Cooper pair have a finite probability to be confined each within the mean field of a different nucleus, equally well pairing correlated than when both nucleons are in the same nucleus. It is then natural that a Cooper pair can also tunnel between target and projectile, equally well correlated, through simultaneous than through successive transfer processes. Because of the weak binding of the Cooper pair, let alone the fact that tunneling probability falls off exponentially with increasing mass, successive is the dominant transfer process.
 
 
Although one does not expects supercurrents in nuclei, one can study long-range pairing correlations in terms of individual quantal states and of the tunneling of single Cooper pairs. Such (weak coupling) Cooper pair transfer reminds  the tunneling mechanism of electronic Cooper pairs across a barrier (e.g. a dioxide layer of dimensions much smaller than the correlation length) separating two low-temperature, metallic superconductors, and known as a Josephson junction\footnote{\cite{Josephson:62,Anderson:64b}.}.
  
  
In the nuclear time dependent junction transiently established in  direct two-nucleon transfer process, only one or sometimes none of the two weakly interacting nuclei are superfluid.  On the other hand in nuclei, paradigmatic example of finite quantum   many-body systems (FQMBS), zero point fluctuations  (ZPF) in general, and those associated with pair addition and pair subtraction modes known as pairing vibrations in particular are, as  rule, much stronger than in condensed matter. Thus, pairing correlations based on even  a single Cooper pair can lead to distinct  effects in two-nucleon transfer processes. 
  
  
Nucleonic Cooper pair tunneling has played and is playing a central role in the probing of these subtle quantal phenomena, both in the case of  light exotic nuclei as well as of medium and heavy nuclei lying along the stability valley. They  have been instrumental in shedding light on the subject of pairing in nuclei at large, and on nuclear superfluidity in particular. Consequently, and as already said, the subject of two-nucleon transfer reactions occupy  a central place in the present monograph. Both concerning the conceptual and the computational aspects of the description of nuclear pairing, as well as regarding the quantitative confrontation of the theoretical  results  with the experimental findings, in terms of absolute differential cross sections.
 

Concerning exotic nuclei, experimental studies carried out at TRIUMF, Vancouver, (Canada)\footnote{\cite{Tanihata:08}.}, have provided the basis for what can be considered a nuclear embodiment\footnote{\cite{Barranco:01,Potel:10}.} of the Cooper pair model: a pair of fermions (nucleons $N$) moving in time reversal states on top of the Fermi surface and interacting through the short range, bare $NN$- and the long range, induced-pairing interaction\footnote{\cite{Frohlich:52,Bardeen:55}.}. This last one resulting from the exchange of a long wavelength dipole vibration (quasi-boson), leading to an extended, weakly bound system.

 

Regarding medium heavy nuclei lying along the stability valley,  studies of heavy ion reactions between superfluid nuclei carried out at energies around and  below the Coulomb barrier at the National Laboratory of Legnaro (Italy)\footnote{\cite{Montanari:14}.} have provided a measure of the neutron Cooper pair size (mean square radius or correlation length\footnote{\cite{Potel:21}.}). 


In the present monograph interdisciplinarity\footnote{Quoting \cite{deGennes:74}: ``\dots what a theorist can and should systematically introduce is comparison with other fields''. } is employed as a tool to attack concrete nuclear problems. Also, and  making use of the unique laboratory provided by  the atomic nucleus\footnote{\cite{Bohr:19} (Overview).}, to shed light on condensed matter results, in terms of analogies involving individual, quantal  states and tunneling of single Cooper pairs.


Because of the central role the interweaving of the variety of elementary modes of nuclear excitation play in nuclear superfluidity, the study of Cooper pair tunneling in nuclei requires  a consistent description of nuclear structure in terms of dressed quasiparticles and, making use of the resulting renormalized wavefunctions (formfactors),  of single-nucleon transfer processes\footnote{In other words, one recognizes the difficulties of extracting spectroscopic factors from experiment, in terms of single-particle transfer cross sections calculated making use of mean field wavefunctions.}. This is similar to the situation encountered in superconductors, in connection with strongly renormalized systems\footnote{\cite{Eliashberg:60}.},  studied through one-electron tunneling experiments\footnote{\cite{Giaver:73}.}. 

In the present monograph the general physical arguments and technical computational details concerning the calculation of  absolute one-and two- nucleon  transfer differential cross sections within the framework of DWBA, making use of state of the art NFT structure input, are discussed in detail. 


As a result of this approach, it is expected that both theoretical and experimental nuclear practitioners can use the present monograph at profit. To help this use, the basic nuclear structure formalism, in particular that associated with single-particle and collective motion in both normal and superfluid nuclei, is economically introduced through general physical arguments. This is also in keeping with the availability in the current literature, of detailed discussions of the corresponding material. Within this context, the monographs \emph{Nuclear Superfluidity} by Brink and Broglia and \emph{Oscillations in Finite Quantum Systems}  by Bertsch and Broglia, published also by Cambridge University Press, can be considered companion volumes to the present one. Volume which shares with those a similar aim: to provide a broad physical view of central issues in the study of finite quantal many-body nuclear systems accessible to motivated students and practitioners. However, neither the present one, nor the other two are introductory texts. In particular the present one in which an attempt at unifying structure and reactions, is made. On the other hand, unifying discrete (mainly structure) and continuum  (mainly reactions) configuration spaces, implies that one will be dealing with those structure results which can be tested by means of experiment. A fact which makes the subject of the present monograph a chapter of quantum mechanics, and thus opened to a wide range of practitioners\footnote{Within this context let us mention the intimately correlated subjects of Random Phase Approximation (RPA) and Particle Vibration Coupling (PVC) not found in a fourth year curriculum. They are explained and refer to in a number of places throughout the present monograph, starting from a pedestrian level and for both surface (particle-hole) and pairing (particle-particle and hole-hole) vibrations (Sect. \ref{Sect1.2}, Fig. \ref{fig1.0.7} (inset)  and Sect. \ref{Sect1.3}), and then extended to include further details and facets (see Sects. \ref{S1.5}, \ref{Sect2.6} and \ref{appintroD} Fig. (\ref{fig:4.2}), Sect. \ref{Sect2.3.2} (Fig. \ref{fig:4.10}), App. \ref{App1.D} and  Sect. \ref{C1S1} (Fig. \ref{fig2.1.5})). Furthermore, in the case of RPA of pairing vibrations around the closed shell system $^{208}$Pb, one provides in Sect. \ref{App1E}, detailed documentation of the numerical calculation of the associated wavefunctions ($X$- and $Y$-amplitudes) at the level of an exercise in a fourth year course. A similar situation is encountered in connection with the subject of the Distorted Wave Born Approximation (DWBA), again a subject not found in fourth year curricula. It is treated at the pedestrian level in Sects. \ref{C6AppE} and \ref{C7AppB} in connection with one-particle and two-particle transfer reactions respectively. And,  once more in full detail, without eschewing complexities, but again in the style of an example-exercise  in Sects. \ref{C4S1} and \ref{C7S2} (within this context we quote, ``Sometimes one has to say difficult things, but one ought to say them as simply as one know-how'' (G. H. Hardy, A mathematician's apology, Cambridge University Press, Cambridge (1969))).}.

Concerning the notation, we have divided each chapter into sections. Each section may, in turn, be broken down into subsections. Equations and Figures are identified by the number of the chapter and that of the section. Thus (5.1.33) labels the thirtythird equation of section 1 of chapter 5. Similarly, Fig. 5.1.2 labels the second figure of section 1 of chapter 5. Concerning the Appendices, they are labelled by the chapter number and by a Latin letter in alphabetical order, e.g. App. 2.A, App. 2.B, etc. Concerning equations and Figures, a sequential number is added. Thus (2.B.2) labels the second equation of Appendix B of chapter 2, while Fig. 2.B.1 labels the first figure of Appendix B of Chapter 2. References are called  in terms of the author's surname and publication year and are found in alphabetic order in the bibliography at the end of the monograph.

A methodological approach used in the present monograph concerns a certain degree of repetition. Similar, but not the same issues are dealt with more than once using different but equatable terminologies. This approach reflects the fact that useful concepts like reaction channels, or correlation length, let alone elementary modes of excitation, are easy to understand but difficult to define\footnote{\label{f10}``Gentagelsen den er virkeligheden, og Tilv\ae{}relsens Alvor'' (Repetition is the reality and the seriousness of life: S. Kierkegaard Gjentagelsen (1843)).}. This is because their validity is not exhausted by a single perspective. But even more important, because their power in helping at connecting\footnote{``The concepts and propositions get ``meaning'' viz. ``content'', only through their connection with sense-experiences\dots The degree of certainty with which this connection, viz., intuitive combination, can be undertaken, and nothing else, differentiates empty fantasy from scientific ``truth''\dots A correct proposition borrows its ``truth'' from the truth-content of the system to which it belongs'' (A. Einstein, Autobiographical notes, in Albert Einstein, Ed. P. A. Schilpp, Vol I, Harper, New York (1951) p. 13.).} seemingly unrelated results and phenomena is difficult to be fully appreciated the first time around, spontaneous symmetry breaking and associated emergent properties providing an example of this fact.

Throughout, a number of footnotes are found. This is in keeping with the fact that footnotes can play a special role within the framework of an elaborated presentation. In particular, they are useful to emphasize relevant issues in an economic way. Being outside the main text, they give the possibility of stating eventual important results, without the need of elaborating on the proof, but referring to the corresponding sources.
 Within this context, and keeping the natural distances, one can mention that in the paper  in which Born\footnote{\cite{Born:26}. Within this context, it is of notice that the extension of Born probabilistic interpretation to the case of many-particle systems is also found in a footnote (\cite{Pauli:27}, footnote on p. 83 of this reference; see  \cite{Pais:86}).} introduces the probabilistic interpretation of Schr\"odinger's  wavefunction, the fact that this probability is connected with its [modulus] squared and not with the wavefunction itself, is only referred to in a footnote.



  A large fraction of the material contained in this monograph have been the subject of the lectures of the fourth year course ``Nuclear Structure Theory'' which RAB delivered throughout the years at the Department of Physics of the University of Milan, as well as at the Niels Bohr Institute (Copenhagen) and at Stony Brook (State University of New York). Part of it was also presented by the authors in the course Nuclear Reactions held at the PhD School of Physics of the University of Milan.

GP wants to thank the tutoring of  Ben Bayman concerning specific aspects of two-particle transfer reactions. Discussions with Ian Thompson and Filomena Nunes on a variety of reaction subjects are gratefully acknowledged. 
RAB  acknowledges the essential role the collaboration with Francisco Barranco and Enrico Vigezzi has played concerning the variety of nuclear structure aspects of the present monograph. His debt with the late Aage Winther regarding the reaction aspects of it is difficult to express in words. The overall contributions of Daniel B\`{e}s, Ben Bayman and the late Pier Francesco Bortignon are only too explicitly evident throughout the text and constitute a daily source of inspiration.  G. P. and R. A. B. have received important suggestions and comments regarding concrete points and the overall presentation of the material discussed below from Ben Bayman, Pier Francesco Bortignon,  Willem Dickhoff,  Vladimir Zelevinsky and the late David Brink and are here gratefully acknowledged. We are specially beholden to Elena Litvinova and Horst Lenske for much constructive criticism and suggestions.
\begin{flushleft}
Gregory Potel Aguilar\\
Livermore
\end{flushleft}
\vspace{-1.7cm}
\begin{flushright}
Ricardo A. Broglia\\
 Copenhagen
\end{flushright}

%\renewcommand{\bibname}{Bibliography}
%\bibliographystyle{abbrvnat}
%\bibliography{../nuclear_bib.bib}

%%% Local Variables:
%%% mode: latex
%%% TeX-master: "../main_libro_CUP"
%%% End:
