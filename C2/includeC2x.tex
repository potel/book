
\chapter{Introduction}
In a scattering experiment, a beam of particles is directed at a target containing the scattering material, and the energy and angular distributions of the outgoing particles is measured. In the case in which the scattering material (target) corresponds to unstable, short lived nuclei, like e.g. $^{132}$Sn (39.7 sec, see \cite{Jones:10}) (but not e.g. in the case of $^{210}$Pb (22.3y)), one can carry out the experiment by exchanging the roles of target and projectile, a method known as inverse kinematics.

There are two basic lengths governing nuclear reactions, namely the radius of the nucleus ($\approx 10^{-13}$cm) and the distance from the target nucleus to the detector ($\approx 10^2$cm). Because of the difference of these two characteristic magnitudes, one can divide the scattering process in two separate parts, namely,

\begin{enumerate}
\item{the analysis of the outgoing beam properties in terms of optical potentials and of single--particle strengths and spreading widths, (effective) deformation parameters and average value of the pair transfer operator, and}
\item{the relation between these parameters and the motion of nucleons in the nuclei.}
\end{enumerate}

In the first part of the monograph the motion of a particle displaying a large mean free path (quantality parameter of order of unity, see \ref{appD}) in a medium, can be economically described in terms of a (complex) dielectric constant (function). In the nuclear case, this function is known as the optical potential for particles moving in the continuum (scattering states). The average, single--particle potential (Hartree--Fock mean field), corresponds to the real part of this function (see \ref{appF}). The imaginary part describes the coupling between the entrance, elastic channel, and the different reaction channels (inelastic, transfer, compound, etc.), leading to the depopulation of the incoming beam. Particularly strong couplings cannot be treated this way (average imaginary function), and have to be included explicitly, as a rule, within the framework of a coupled channel formalism. For bound nucleons the dielectric function is known as the self--energy function. The real part is connected with the single--particle energy centroid and strength (quasiparticle pole and residue $E_{qp}, S$ (see \ref{appF})). The imaginary part provides a compact measure of the range of energy over which the remaining strength is distributed (single--particle fractionation), as a result of the interweaving of single--particle and collective motion (quasiparticle lifetime $\hbar/\Gamma$). In the case in which the particle--vibration coupling strength becomes too strong ($\Gamma\gtrsim E_{qp} $), a full diagonalization is called for. As example one can refer to the self--energy function arizing from the coupling with collective quadrupole surface vibrations of closed shell nuclei. In this case the self--energy function can be calculated perturbatively. As nucleons are progressively added it may happen that $E_{2^+}\rightarrow 0$ and the nucleus acquires a permanent static deformation. The large breaking of the $j$--strength into the two--fold degenerate (Kramers degeneracy) calls for a change of basis and the used of a ``deformed'' average mean field (Nilsson model).


The real and imaginary parts of the dielectric functions describing the effects, on the nucleon motion, of the virtual quantal processes which do not conserve (off--the energy shell) conserve (on--the energy shell) energy, are not independent of each other, but must fulfill a dispersion relation known as the Kramers--Kr\"{o}nig relation. This has profound physical implications, as well as practical (computational) consequences. In fact, virtual processes, renormalizing the properties of a particle like mass, charge, etc., can lead to divergences, which forces one to introduce (energy--, momentum--, angular momentum--, etc.) cut--offs. Now, energy conserving contributions are free from such divergences. Consequently, calculating the imaginary part of the dielectric function and making use of the (so--called substracted) dispersion relations, can provide non--divergent real components of the dielectric function. This is a distinctive property of, so called, asymptotic free theories. In these theories one knows that something quite spectacular can happen in the infrared end of the spectrum (e.g. spontaneous breaking of rotational invariance associated with nuclear deformation), but that the consequences of such a phenomenon will not depend on contributions to observables from processes above a certain cut--off which can be simply defined introducing by just choosing a model (e.g. the Nilsson model in the case under discussion). In Part I of the monograph, the techniques necessary to deal with problem 1) will be worked out, as for as needed to deal with problem 2) which is the central subject of the present monograph and is treated in Part II. Appendices are given that provide the elements of nuclear structure needed for the calculation of the differential cross sections associated with the variety of reaction processes. In other words, the spectroscopic amplitudes associated with one-- and two-- nucleon transfer processes, and the effective deformation parameters and transition densities associated with anelastic processes. Each chapter introduces the subject in term of the definition of the quantities needed for the calculation of the differential cross sections. Approximations are then introduced (plane--wave, no--recoil, etc.) which allows to work out most of the technical aspects of the reaction machinery almost analytically. This is done to be able to explicit the nuclear structure (details on nucleonic motion in terms of single-- and two--particle as well as (particle--hole)--wavefunctions), needed to calculate the differential cross section associated with the process, setting special emphasis in the nuclear structure information one can extract from the comparison with the experimental data. In the second part of each chapter, and eventually in appendices, the full details of spectroscopic amplitudes, formfactors and of the differential cross sections, without introducing but very generic approximations, are given eventually supplemented by numerical examples. The recent availability of low energy, light ion reaction data on exotic nuclei, but not only, requires the availability of the theoretical tools to extract the corresponding nuclear structure information. In particular concerning (dressed) single--particle and (medium renormalized) pairing degrees of freedom.


Many of these questions are dealt with in a unified fashion, within the framework of the applications which constitute thepart II of this monograph.


A CD with software which allows to apply some of the concepts, ideas and techniques developed in different chapters is also provided.


\section{Reaction channel}

Let us consider the case in which the nucleus $^{18}O$ is the target and the projectile is a proton. The following processes can take place, among others:

\begin{equation}
{\rm entrance\;channel} \left\{
\begin{array}{rl}
p+^{18}O \rightarrow & \;p + ^{18}O \;({\rm gs})\;\; (Q=0) \;\;\; {\rm elastic\; scattering} \\
p+^{18}O \rightarrow & \left. \begin{array}{ll}
     p + ^{18}O^* \;({\rm 6 MeV}) & + Q_1 \\
     d + ^{17}O \;({\rm gs})    & + Q_2 \\
     t + ^{16}O \;({\rm gs})    & + Q_3
		 \end{array} \right\} \;\; {\rm reaction\;channels,}
\end{array}
\right.
\end{equation}

\noindent where

\begin{equation}
Q_1 = M_p + M(^{18}O) - (M_p + ^{18}O \; (6 \;{\rm MeV})) = -6 {\rm MeV}
\end{equation}

\begin{equation}
Q_2 = M_p + M(^{18}O) - (M_d + ^{17}O \; ({\rm gs}))
\end{equation}

\begin{equation}
Q_3 = M_p + M(^{18}O) - (M_t + ^{16}O \; ({\rm gs}))
\end{equation}
and where $^{18}$O$^*$ labels the nucleus $^{18}$O in an excited state. In general if $Q>0$ the reaction can proceed at zero bombarding energy. For $Q<0$ the reaction is not observed below the threshold $E_t$ which is defined, for a general reaction $A(a,b)B$ as

\begin{equation}
E_{CM} = \frac{1}{2} \frac{M_a M_A}{M_a + M_A} V_a^2 = \frac{M_A}{M_a + M_A} E_{lab} = \frac{1}{1+(M_a/M_A)} E_{lab}
\end{equation}

\noindent ($1/2 M_a V_a^2$ is the total energy of the system). If $E=|Q|$ we have

\begin{equation}
E_t^{lab} = \frac{M_a + M_A}{M_A} |Q| \rightarrow E_t^{lab} = \left( 1+ \frac{M_a}{M_A} \right)  |Q|
\end{equation}

\noindent For the particular case

\begin{equation}
\begin{array}{l}
p + ^{18}O \rightarrow p + ^{18}O^* \;\;\; (6 \;{\rm MeV}) \\
E_t^{lab} = \frac{19}{18} \times 6 \;{\rm MeV}
\end{array}
\end{equation}

A complete specification of the type of two--particle breakup and of the internal states of the two particles, is called a channel and is specified by the product $\Psi_{\alpha} = \Psi_a \Psi_A$ of the (bound) internal wavefunctions of the two nuclei. Here $A$ and $a$ denote the nuclei into which the system breaks up and also their state of excitation, their angular momenta and the projection of their angular momenta.

The same word channel is understood sometimes to include all the properties already mentioned, together with a definite value of the orbital angular momentum of the relative motion of the centers of mass of the two separating systems.


\section{The reaction cross section}

The initial situation can be described by a plane wave \footnote{This free particle wavefunction can be normalized in a given volume or requiring the function to obey periodic boundary conditions inside a box (see Appendix)},

\begin{equation}
\Psi_{inc} = \Psi_{\alpha} e^{i k_{\alpha} z}
\label{eqn:psi_inc}
\end{equation}

\noindent which represents a beam of particles of unit density, incident upon the scattering center in the $z$-direction. The price to pay for such a simplification is that momentum and energy are then absolutely defined, and it is no longer possible to follow the scattering process neither in space nor in time (from $\Delta x \Delta p \ge \frac{\hbar}{2}$ we loose the localization in space and from $\Delta E \Delta t \ge \frac{\hbar}{2}$ the one in time).

This way of describing the incident beam is of course an idealization, and in most cases there is no problem of principle, as one can approach the situation described by Eq.(\ref{eqn:psi_inc}) as closely as desired.
{\it One must be able, however, to place the detector outside the beam}. It is important to verify that the corresponding spread of the wave packet does not produce any essential limitation. Let us call $W$ the width of the beam (see Fig. \ref{fig1st_1}). The velocity perpendicular to the beam can be easily calculated as follows,

\begin{equation}
\begin{array}{rl}
\Delta x \Delta p_x \sim \hbar &, p_x = m v_x, \\
\Delta p_x = m \Delta v_x &, \Delta X = W, \\
W(m \Delta v_x) &\sim \hbar, \\
\Delta v_x &\sim \frac{\hbar}{MW},
\end{array}
\label{eqn:10}
\end{equation}
\begin{figure}
\centerline{\includegraphics*[width=10cm,angle=0]{C2/figs_C2/fig_1_1.pdf}}
\caption{Schematic representation of a reaction experiment}\label{fig1st_1}
\end{figure}

A particle leaving the collimator ($\langle v_x \rangle = 0$) has an uncertainty in $v_x$ given by the equation (\ref{eqn:10}). In the most unfavourable case $v_x = \Delta v_x$. After a time $\Delta t = t$ (we choose $t=0$ that time when the particle leaves the collimator, namely for $z=0$), the particle has travelled a distance $L$, and the spreading of the wave packet along $x$ is equal to

\begin{equation}
\begin{array}{rl}
\Delta W = v_x t &\approx \Delta v_x t, \;\;\;\;\; t = \frac{L}{v}, \\
\Delta W = \frac{\Delta v_x}{v} L &\approx \frac{\hbar L}{m W v}.
\end{array}
\end{equation}

\noindent The relative increase of the wave packet is then equal to

\begin{equation}
\frac{\Delta W}{W} \approx \frac{\hbar L}{m W^2 v},
\end{equation}

\noindent Let us put some numbers, for example those corresponding to the accelerator at Ris\o, where some of the pioneer measurements of reaction processes were carried out,

\begin{equation}
\begin{array}{rl}
L &\approx 10^2 {\rm cm}, \\
v &\approx 10^9 {\rm cm/s}\;(\approx 13 \text{MeV}), \\
W &\approx 10^{-1} {\rm cm}, \\
\frac{\Delta W}{W} &\approx 10^{-8},
\end{array}
\end{equation}

\noindent which is very small indeed.

The most dangerous case corresponds to reactions with slow neutrons, which usually are done also in large equipments (time of flight techniques). Let us put

\begin{equation}
\begin{array}{rl}
L &\approx 10^4 {\rm cm}, \\
v &\approx 2 \times 10^5 {\rm cm/s}, \\
W &\approx 10^{-1} {\rm cm}, \\
\frac{\Delta W}{W} &\approx 10^{-2},
\end{array}
\end{equation}

The ratio is still small, but just on the limit of becoming important. Still if this ratio would be of order unit we could use the same concepts, but we should treat with more detail the problem of how the wave packet is constructed and the possible interference at the edge of the beam, with the outgoing wave packet.

{\it The scattered wave (asymptotic region) must be the solution of the free field equation}

\begin{equation}
\begin{array}{rl}
H \Psi_{\rm scatt} &= E \Psi_{\rm scatt}, \\
E &= \frac{\hbar^2 k_{\alpha}^2}{2m},
\end{array}
\end{equation}

\noindent with

\begin{equation}
H = \frac{p^2}{2m} = - \frac{\hbar^2}{2m} \nabla^2 = - \frac{\hbar^2}{2m} \left\{ \frac{1}{r^2} \frac{\partial}{\partial r} \left( r^2 \frac{\partial}{\partial r} \right) + \frac{1}{r^2} \hat{L}^2 \right\},
\end{equation}

\noindent where $\hat{L}$ is the angular momentum operator, $m = \frac{M_B M_b}{M_b + M_B}$ the reduced mass, and $k$ and $r$ the relative momentum and coordinate of the nuclei $b$ and $B$.

\noindent At large distances the angular momentum terms drops out as $\frac{1}{r^2}$ and it is easy to verify that the asymptotic solution is

\begin{equation}
\Psi_{\rm scatt} = \frac{e^{ik_{\alpha} r}}{r} f_{\alpha \alpha}(E, \theta, \phi) \Psi_{\alpha},
\end{equation}
where $\Psi_{\alpha}$ is the intrinsic channel wavefunction.
\noindent Let us now calculate the incoming current and the scattered current of particles.

For a given wave function $\Psi$, (describing the motion of a particle of mass $m$), the associated current is equal to

\begin{equation}
\begin{array}{rl}
\vec{I} &= \frac{\hbar}{2im} \left( \Psi^* \vec{\nabla} \Psi - \left( \vec{\nabla} \Psi^* \right) \Psi \right) \\
&= \frac{\hbar}{m} {\cal I}_m \left( \Psi^* \vec{\nabla} \Psi \right).
\end{array}
\end{equation}

The incident current is equal to (cf. eq. (9))

\begin{equation}
\begin{array}{rl}
\vec{I}_{\rm inc} &= \frac{\hbar}{m} {\cal I}_m e^{-i k_{\alpha} z} \frac{\rm d}{{\rm d}z} \left( e^{i k_{\alpha} z} \right) \hat{z} \\
&= \frac{\hbar k_{\alpha}}{m} \hat{z} = v_{\infty} \hat{z}.
\end{array}
\end{equation}

\noindent where $v_{\infty}$ is the velocity corresponding to the projectile incident energy.

The scattered current is equal to \footnote{In deriving this equation one assumes that $r \rightarrow \infty$ (asymptotic region).}

\begin{equation}
\vec{I}_{\rm scatt} \approx \frac{|f(\theta,\phi)|^2}{r^2} \frac{\hbar k_{\alpha}}{m} \hat{r}.
\end{equation}

The differential cross section is defined as the flux of particles going into the solid angle ${\rm d}\Omega$ at angle $(\theta,\phi)$, divided by the incoming flux of incoming particles.

The flux of outgoing particles is given by the projection of $I_{\rm scatt}$ on the unit $\vec{{\rm d}s} = r^2 {\rm d}\Omega \hat{r}$ of solid angle, namely

\begin{equation}
\vec{I}_{\rm scatt} {\rm d}\vec{s} = |f(\theta,\phi)|^2 \frac{\hbar k_{\alpha}}{m} {\rm d}\Omega.
\end{equation}

\noindent The incident flux is given by

\begin{equation}
\vec{I}_{\rm inc} \hat{z} = \frac{\hbar k_{\alpha}}{m}.
\end{equation}

\noindent The differential cross section is defined as the flux of particles into the solid angle ${\rm d}\Omega$ at angle $(\theta,\phi)$, divided by the incoming flux of incoming particles, namely

\begin{equation}
{\rm d}\sigma = |f_{\alpha \alpha}(E,\theta,\phi)|^2 {\rm d}\Omega
\end{equation}

In other channels there will be no incident wave, and in general, the outgoing waves would have a different value of the wave number, i.e.

\begin{figure}
\centerline{\includegraphics*[width=10cm,angle=0]{C2/figs_C2/1_2.pdf}}
\caption{}\label{fig1st_2}
\end{figure}
\begin{equation}
\Psi_{\rm scatt} = \frac{1}{r} e^{i k_{\beta} r} f_{\alpha \beta} (E',\theta,\phi).
\end{equation}

The symmetries of the problem can produce limitations in the form of $f_{\alpha \beta}(k_{\beta},\theta,\phi)$. In general, using unpolarized particles, and not considering spin, $f_{\alpha \beta}(k_{\beta},\theta,\phi)$ will not depend on the angle $\phi$. This is a consequence of the fact that the incoming beam has zero projection of the angular momentum in the direction of the incident beam. Therefore, in the outgoing channel the angular momentum will mantain its zero projection and therefore the outgoing wave function cannot depend on $\phi$.



\section{Evaluation of the spreading of the wave packet}
The proton mass is
\begin{equation}
m_p = 1.7 \times 10^{-24} {\rm gr},
\end{equation}
where
\begin{equation}
1 {\rm MeV} = 1.6 \times 10^{-6} {\rm erg} = 1.6 \times 10^{-6} {\rm gr} \frac{{\rm cm}^2}{{\rm sec}^2}
\end{equation}
Thus, the relation between velocity and energy for a proton can be written as
\begin{equation}
\begin{array}{rl}
v &= \sqrt{\frac{2E}{m}} = \sqrt{\frac{3.2 E \times 10^{-6}}{1.7 \times 10^{-24} {\rm gr}}{\rm gr} \frac{{\rm cm}^2}{{\rm sec}^2}}, \\
  &= 10^9 \sqrt{1.88 E ({\rm MeV})}, \\
	&\approx 1.4 \times 10^9 \sqrt{E({\rm MeV})} \frac{{\rm cm}}{{\rm sec}}.
\end{array}
\end{equation}

For $20$ MeV protons ($E = 20$ MeV) one obtains

\begin{equation}
v \approx 6 \times 10^{9} \frac{{\rm cm}}{{\rm sec}}
\end{equation}
Thus
\begin{equation}
\frac{\Delta W}{W} \approx \frac{\hbar L}{m W^2 v},
\end{equation}
Typical values of $L$ and $W$ are
\begin{equation}
L \approx 10^2 {\rm cm}, \qquad \qquad W \approx 10^{-1} {\rm cm}.
\end{equation}
Using
\begin{equation}
\hbar = 1.054 \times 10^{-27} {\rm erg \; sec}
\end{equation}
one obtains
\begin{equation}
\begin{array}{rl}
\frac{\Delta W}{W} &= \frac{\left( 1.054 \times 10^{-27} {\rm gr} \frac{{\rm cm^2}}{{\rm sec^2}} \right) \left( {\rm cm} \times 10^2 \right)}{1.7 \times 10^{-24} {\rm gr} \times 10^{-2} {\rm cm^2} \times 6 \times 10^9 \frac{{\rm cm}}{{\rm sec}}} \\
&= \frac{1.054 \times 10^{-25}}{1.7 \times 6} \times 10^{17} = \frac{1.054}{1.7 \times 6} 10^{-8} \\
&\approx 10^{-9}
\end{array}
\end{equation}
