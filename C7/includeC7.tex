
 \chapter{Two-particle transfer}\label{C7}
  \epigraph{A unified picture of pairing and two-nucleon transfer, where the two subjects are blended together (which is what happens in nature) }{B. F.  Bayman}



 



The present Chapter is structured in the following way. In Section \ref{C7S1} we present a summary of two-nucleon transfer reaction theory. It provides, together with Section \ref{C2S1} the elements needed to calculate the absolute two-nucleon transfer differential cross sections in second order DWBA, and thus to compare theory with experiment.
For the practitioner in search of details and clarification we carry out, in section \ref{C7S2}  a  derivation of the equations presented in section \ref{C7S1}. These equations are implemented and made operative  in the software \textsc{cooper}  used in the applications (cf. App. \ref{C8AppD}).

A number of Appendices are provided.   Appendices \ref{C7AppD}--\ref{C7AppN} contain relations used in Sect. \ref{C7S2} as well as in the derivation of some two-nucleon transfer spectroscopic amplitudes. Finally, Appendix \ref{C7AppO} provides a glimpse of original material due to Ben Bayman\footnote{\cite{Bayman:70}, \cite{Bayman:71}, \cite{Bayman:82}.} which was instrumental to render quantitative the studies of two-nucleon transfer, studies which can now be carried out in terms of absolute differential cross sections and not only of relative ones.
\section{Summary of second order DWBA}\label{C7S1}
Let us illustrate the  theory of second order DWBA two-nucleon transfer reactions  with the  $A+t \rightarrow B(\equiv A+2)+p$ reaction, in which $A+2$ and $A$ are even nuclei in their $0^+$ ground state. The extension of the  expressions to the transfer of pairs coupled to arbitrary angular momentum is discussed in subsection \ref{C7S2S10}. 


The wavefunction of the nucleus $A+2$ can be written as 
\begin{equation}\label{eq7_1_1}
\Psi_{A+2}(\xi_A,\mathbf r_{A1},\sigma_1,\mathbf r_{A2},\sigma_2)=\psi_A(\xi_A)\sum_{l_i,j_i}[\phi^{A+2}_{l_i,j_i}(\mathbf r_{A1},\sigma_1,\mathbf r_{A2},\sigma_2)]^0_0,
\end{equation} 
where 
\begin{equation}\label{eq7_1_2}
[\phi^{A+2}_{l_i,j_i}(\mathbf r_{A1},\sigma_1,\mathbf r_{A2},\sigma_2)]^0_0=\sum_{nm}a_{nm}\left[\varphi^{A+2}_{n,l_i,j_i}(\mathbf r_{A1},\sigma_1)\varphi^{A+2}_{m,l_i,j_i}(\mathbf r_{A2},\sigma_2)\right]^0_0,
\end{equation} 
while  $\varphi^{A+2}_{n,l_i,j_i}(\mathbf r)$ are single-particle wavefunctions.
%eigenfunctions of a Saxon--Woods potential
%\begin{equation}\label{Eq17}
%U(r)=-\frac{V_0}
%{1+\exp\left[\frac{r-R_0}{a}\right]},\quad\quad R_0=r_0 A^{1/3},
%\end{equation}
%of depth $V_0$  adjusted to reproduce the experimental single--particles energies, together with a standard spin--orbit potential.
 The radial dependence of the  wavefunction of the two neutrons in the triton is written as $\phi_t(\mathbf r_{p1},\mathbf r_{p2})=\rho(r_{p1})\rho(r_{p2})\rho(r_{12})$, where $r_{p1},r_{p2},r_{12}$ are the distances between neutron 1 and the proton, neutron 2 and the proton and between neutrons 1 and 2 respectively, while $\rho(r)$ is the hard core $(r_{core}=0.45$ fm) potential  wavefunction  depicted in\footnote{\cite{Tang:65}.} Fig \ref{fig7_1_1}.
 

 
 
\begin{figure}
\centerline{\includegraphics*[width=.55\textwidth,angle=0]{C7/figs_C7/tritium.pdf}}
\caption{Radial function $\rho(r)$ (hard core 0.45 fm) entering the tritium wavefunction.}\label{fig7_1_1}
\end{figure}
\begin{figure}
\centerline{\includegraphics*[width=.55\textwidth,angle=0]{C7/figs_C7/deuteron.pdf}}
\caption{Radial wavefunction $\rho_d(r)$ (hard core 0.45 fm) entering the deuteron wavefunction.}\label{fig7_1_2}
\end{figure}
The two-nucleon transfer differential cross section is written as
\begin{equation}\label{eq5.1.4}
\frac{d\sigma}{d\Omega}=\frac{\mu_i\mu_f}{(4\pi\hbar^2)^2}\frac{k_f}{k_i}\left|T^{(1)}(\theta)+T^{(2)}_{succ}(\theta)-T^{(2)}_{NO}(\theta)\right|^2,
\end{equation}

where\footnote{See \cite{Bayman:82} and App. \ref{C7AppO}.},
\begin{subequations}
\begin{multline}\label{eq1_40}
T^{(1)}(\theta)=2\sum_{l_i,j_i}\sum_{\sigma_1 \sigma_2}\int d\mathbf{r}_{tA}d\mathbf{r}_{p1}d\mathbf{r}_{A2}
  [\phi^{A+2}_{l_i,j_i}(\mathbf r_{A1},\sigma_1,\mathbf r_{A2},\sigma_2)]^{0*}_0\chi^{(-)*}_{pB}(\mathbf{r}_{pB})\\
 \times v({r}_{p1}) \phi_t(\mathbf r_{p1},\sigma_1,\mathbf r_{p2},\sigma_2)\chi^{(+)}_{tA}(\mathbf{r}_{tA}), \idx{Two-nucleon transfer!simultaneous}
\end{multline}
\begin{multline}\label{eq1_41}
T^{(2)}_{succ}(\theta)=2\sum_{l_i,j_i}\sum_{l_f,j_f,m_f}\sum_{\substack{\sigma_1 \sigma_2\\\sigma'_1 \sigma'_2}}
\int d\mathbf{r}_{dF}d\mathbf{r}_{p1}d\mathbf{r}_{A2}
[\phi^{A+2}_{l_i,j_i}(\mathbf r_{A1},\sigma_{1},\mathbf r_{A2},\sigma_2)]^{0*}_0\chi^{(-)*}_{pB}(\mathbf{r}_{pB})
 v({r}_{p1})\\
 \times\phi_d(\mathbf r_{p1},\sigma_1)\varphi^{A+1}_{l_f,j_f,m_f}(\mathbf r_{A2},\sigma_2) \int d\mathbf{r}'_{dF}d\mathbf{r}'_{p1}d\mathbf{r}'_{A2}G(\mathbf{r}_{dF},\mathbf{r}'_{dF})\\
 \times \phi_d(\mathbf r'_{p1},\sigma'_1)^*\varphi^{A+1*}_{l_f,j_f,m_f}(\mathbf r'_{A2},\sigma'_2) \frac{2\mu_{dF}}{\hbar^2}v({r}'_{p2})
 \phi_d(\mathbf r'_{p1},\sigma'_1)\phi_d(\mathbf r'_{p2},\sigma'_2) \chi^{(+)}_{tA}(\mathbf{r}'_{tA}), \idx{Two-nucleon transfer!successive}
\end{multline}
\begin{multline}\label{eq1_42}
T^{(2)}_{NO}(\theta)=2\sum_{l_i,j_i}\sum_{l_f,j_f,m_f}\sum_{\substack{\sigma_1 \sigma_2\\\sigma'_1 \sigma'_2}}
\int d\mathbf{r}_{dF}d\mathbf{r}_{p1}d\mathbf{r}_{A2}
[\phi^{A+2}_{l_i,j_i}(\mathbf r_{A1},\sigma_1,\mathbf r_{A2},\sigma_2)]^{0*}_0\chi^{(-)*}_{pB}(\mathbf{r}_{pB})
 v({r}_{p1})\\
 \times \phi_d(\mathbf r_{p1},\sigma_1)\varphi^{A+1}_{l_f,j_f,m_f}(\mathbf r_{A2},\sigma_2)\int d\mathbf{r}'_{p1}d\mathbf{r}'_{A2}d\mathbf{r}'_{dF}\\
 \times\phi_d(\mathbf r'_{p1},\sigma'_1)^*\varphi^{A+1*}_{l_f,j_f,m_f}(\mathbf r'_{A2},\sigma'_2) 
  \phi_d(\mathbf r'_{p1},\sigma'_1)\phi_d(\mathbf r'_{p2},\sigma'_2)\chi^{(+)}_{tA}(\mathbf{r}'_{tA}).\idx{Two-nucleon transfer!non-orthogonality correction}
\end{multline}
\end{subequations}
The quantities $\mu_i, \mu_f (k_i,k_f)$ are the reduced masses (relative linear momenta) in both entrance (initial, $i$) and exit (final, $f$) channels, respectively.
In the above expressions, $\varphi^{A+1}_{l_f,j_f,m_f}(\mathbf r_{A1})$ are the wavefunctions describing the intermediate states of the nucleus $F(\equiv (A+1))$, generated as solutions of a Woods-Saxon potential,  $\phi_d(\mathbf r_{p2})$ being the  the deuteron bound wavefunction (see Fig. \ref{fig7_1_2})\footnote{\cite{Tang:65}.}. Note that some or all of the single-particle states described by the wavefunctions $\varphi^{A+1}_{l_f,j_f,m_f}(\mathbf r_{A1})$ may lie in the continuum ( case in which the nucleus $F$ is loosely bound or unbound). Although there are a number of ways to exactly treat such states, discretization processes may be sufficiently	accurate. They can be implemented by, for example,  embedding the Woods--Saxon potential in a spherical box of sufficiently large  radius. In actual calculations involving e.g. the halo nucleus $^{11}$Li, and where $|F\rangle=|^{10}$Li$\rangle$, one achieved convergence making use of approximately  20 continuum states and a box of 30 fm of radius. Concerning the components of the triton wavefunction describing the relative motion of the dineutron, it was  generated with the $p-n$ interaction\footnote{ \cite{Tang:65}.}
\begin{align}\label{eq7_1_6}
v(r)&=-v_0\exp\left(-k(r-r_c)\right) \quad r>r_c\\
v(r)&=\infty \quad r<r_c,
\end{align}
where $k=2.5$ fm$^{-1}$ and $r_c=0.45$ fm, the depth $v_0$ being adjusted to reproduce the experimental separation energies.
The positive--energy wavefunctions  $\chi^{(+)}_{tA}(\mathbf{r}_{tA})$ and $\chi^{(-)}_{pB}(\mathbf{r}_{pB})$ are the ingoing distorted wave in the initial channel and the outgoing distorted wave in the final channel respectively. They are continuum solutions of the Schr\"{o}dinger equation associated with the corresponding optical potentials.


The transition potential responsible for the transfer of the pair is, in the \emph{post} representation (cf. Fig. \ref{figC7C1}),
\begin{equation}\label{eq1_43}
    V_\beta=v_{pB}-U_{\beta},
\end{equation}
where $v_{pB}$ is the interaction between the proton and nucleus $B$, and $U_{\beta}$ is the optical potential in the final channel. We make the assumption that $v_{pB}$ can be decomposed into a term containing the interaction between $A$ and $p$ and the potential describing the interaction between the proton and each of the transferred nucleons, namely
\begin{equation}\label{eq1_44}
    v_{pB}=v_{pA}+v_{p1}+v_{p2},
\end{equation}
where $v_{p1}$ and $v_{p2}$ is the hard--core potential (\ref{eq7_1_6}). The transition potential is
\begin{equation}\label{eq1_45}
    V_\beta=v_{pA}+v_{p1}+v_{p2}-U_{\beta}.
\end{equation}
Assuming that $\langle \beta |v_{pA}|\alpha \rangle \approx	 \langle \beta |U_{\beta}|\alpha \rangle $ (i.e, assuming that the matrix element of the core--core interaction between the initial and final states is very similar to the matrix element of the real part of the optical potential), one obtains the final expression of the transfer potential in the \emph{post-post} representation, namely,	
\begin{equation}\label{eq1_45x}
    V_\beta\simeq v_{p1}+v_{p2}=v(\mathbf{r}_{p1})+v(\mathbf{r}_{p2}).
\end{equation}
We make the further approximation of using the same interaction potential in all the (i.e. initial, intermediate and final) channels.


The extension to a heavy--ion reaction $A+a(\equiv b+2) \longrightarrow B(\equiv A+2)+b$ imply no essential modifications in the formalism. The deuteron and triton wavefunctions appearing in Eqs. (\ref{eq1_40}), (\ref{eq1_41}) and (\ref{eq1_42}) are to be substituted with the corresponding wavefunctions $\Psi_{b+2}(\xi_b,\mathbf r_{b1},\sigma_1,\mathbf r_{b2},\sigma_2)$, constructed in a similar way as those appearing in (\ref{eq7_1_1} and \ref{eq7_1_2}). The interaction potential used in  Eqs. (\ref{eq1_40}), (\ref{eq1_41}) and (\ref{eq1_42})  will now be the Saxon--Woods used to define the initial (final) state in the post (prior) representation, instead of the proton--neutron interaction (\ref{eq7_1_6}).

The Green's function $G(\mathbf{r}_{dF},\mathbf{r}'_{dF})$ appearing in (\ref{eq1_41}) propagates the intermediate channel $d,F$. It  can be expanded in partial waves as,
\begin{equation}\label{eq7_1_12}
G(\mathbf{r}_{dF},\mathbf{r}_{dF}')=i\sum_{l}\sqrt{2l+1}
\frac{f_{l}(k_{dF},r_<)g_{l}(k_{dF},r_>)}{k_{dF}r_{dF}r_{dF}'}
\left[  Y^{l} (\hat r_{dF}) Y^{l} (\hat r_{dF}')\right]_0^0.
\end{equation}
The $f_{l}(k_{dF},r)$ and $g_l(k_{dF},r)$ are the regular and the irregular solutions of a Schr\"{o}dinger equation for  a suitable optical potential and an energy equal to the kinetic energy of the intermediate state. In most cases of interest, the result is hardly altered if we use the same energy of  relative motion for all the intermediate states. This representative energy is calculated when both intermediate nuclei are in their corresponding ground states. It is of notice that the validity of this approximation can break down in some particular cases. If, for example, some relevant intermediate state become off shell, its contribution is significantly quenched. An subtle situation can arise when this happens to all possible intermediate states, so they can only be virtually populated.
\section{Detailed derivation of second order DWBA}\label{C7S2}
\idx{Two-nucleon transfer!second order DWBA}
\subsection{Simultaneous transfer: distorted waves}\label{Sect6.2.1}
For a $(t,p)$ reaction, the triton is represented by an incoming distorted wave. We make the assumption that the two neutrons are in an $S=L=0$ state, and that the relative motion of the proton with respect to the dineutron is also $l=0$. Consequently, the total spin of the triton   is entirely due to the spin of the proton. We will explicitly treat it, as we will consider a spin--orbit term in the optical potential acting between the triton and the target\footnote{In what follows we will use the notation of \cite{Bayman:71}, see also  App. \ref{C7AppO}.}.



Following (\ref{eqC7L1}), we can write the triton distorted wave as
\begin{equation}\label{eq88}
    \psi^{(+)}_{m_t}(\mathbf{R},\mathbf{k}_i,\sigma_p)=\sum_{l_t}\exp\left(i\sigma_{l_t}^t\right)g_{l_tj_t}Y_0^{l_t}
    (\hat{\mathbf{R}})\frac{\sqrt{4\pi(2l_t+1)}}{k_iR}\chi_{m_t}(\sigma_p),
\end{equation}
where use was made of $Y_0^{l_t}(\hat{\mathbf{k}}_i)=i^{l_t}\sqrt{\tfrac{2l_t+1}{4\pi}}\delta_{m_t,0}$, in keeping with the fact that $\mathbf{k}_i$ is oriented along the $z$--axis. Note the phase difference with eq. (7) of \cite{Bayman:71}, due to the use of time--reversal rather than Condon--Shortley phase convention. Making use of the relation
\begin{equation}\label{eq89}
    Y_0^{l_t}(\hat{\mathbf{R}})\chi_{m_t}(\sigma_p)=\sum_{j_t}\langle l_t \;0\;1/2\;m_t|j_t\;m_t\rangle \left[Y^{l_t}(\hat{\mathbf{R}})\chi(\sigma_p)\right]^{j_t}_{m_t},
\end{equation}
we have
\begin{equation}\label{eq90}
\begin{split}
    \psi^{(+)}_{m_t}(\mathbf{R},\mathbf{k}_i,\sigma_p)=\sum_{l_t,j_t}&\exp\left(i\sigma_{l_t}^t\right)
    \frac{\sqrt{4\pi(2l_t+1)}}{k_iR}g_{l_tj_t}(R)\\
    &\times\langle l_t \;0\;1/2\;m_t|j_t\;m_t\rangle \left[Y^{l_t}(\hat{\mathbf{R}})\chi(\sigma_p)\right]^{j_t}_{m_t}.
\end{split}
\end{equation}
We now turn our attention to the outgoing proton distorted wave, which, following (\ref{eq35}) can be written as
\begin{equation}\label{eq91}
    \psi^{(-)}_{m_p}(\boldsymbol{\zeta},\mathbf{k}_f,\sigma_p)=\sum_{l_pj_p}\frac{4\pi}{k_f\zeta}i^{l_p}
    \exp\left(-i\sigma_{l_p}^p\right)f_{l_pj_p}^*(\zeta)\sum_m Y_m^{l_p}
    (\hat{\boldsymbol{\zeta}})Y_m^{l_p*}
    (\hat{\mathbf{k}}_f)\chi_{m_p}(\sigma_p).
\end{equation}
Making use of the relation
\begin{equation}\label{eq92}
\begin{split}
\sum_m Y_m^{l_p}&
    (\hat{\boldsymbol{\zeta}})Y_m^{l_p*}
    (\hat{\mathbf{k}}_f)\chi_{m_p}(\sigma_p)=\sum_{m,j_p} Y_m^{l_p*}
    (\hat{\mathbf{k}}_f)\langle l_p \;m\;1/2\;m_p|j_p\;m+m_p\rangle \\
    &\times \left[Y^{l_p}
    (\hat{\boldsymbol{\zeta}})\chi_{m_p}(\sigma_p)\right]^{j_p}_{m+m_p}\\
    &=\sum_{m,j_p} Y_{m-m_p}^{l_p*}
    (\hat{\mathbf{k}}_f)\langle l_p \;m-m_p\;1/2\;m_p|j_p\;m\rangle \left[Y^{l_p}
    (\hat{\boldsymbol{\zeta}})\chi_{m_p}(\sigma_p)\right]^{j_p}_{m},
\end{split}
\end{equation}
one obtains
\begin{equation}\label{eq93}
\begin{split}
    \psi^{(-)}_{m_p}(\boldsymbol{\zeta},\mathbf{k}_f,\sigma_p)
    &=\frac{4\pi}{k_f\zeta}\sum_{l_pj_p,m}i^{l_p}
    \exp\left(-i\sigma_{l_p}^p\right)f_{l_pj_p}^*(\zeta)Y_{m-m_p}^{l_p*}
    (\hat{\mathbf{k}}_f)\\
    &\times \langle l_p \;m-m_p\;1/2\;m_p|j_p\;m\rangle \left[Y^{l_p}
    (\hat{\boldsymbol{\zeta}})\chi(\sigma_p)\right]^{j_p}_{m}.
\end{split}
\end{equation}
\subsection{matrix element for the transition amplitude}\label{C7SS722}
\idx{Two-nucleon transfer!second order DWBA}
We now turn our attention to the evaluation of
\begin{equation}\label{eq94}
  \begin{split}
  \langle \Psi_f^{(-)}&(\mathbf{k}_f)|V(r_{1p})|\Psi_i^{(+)}(k_i,\hat {\mathbf{z}})\rangle=\frac{(4\pi)^{3/2}}{k_ik_f}\sum_{l_pl_tj_pj_tm}\bigl((\lambda \tfrac{1}{2})_k(\lambda \tfrac{1}{2})_k|(\lambda \lambda)_0(\tfrac{1}{2}\tfrac{1}{2})_0\bigr)_0\sqrt{2l_t+1}\\
  &\times \langle l_p \;m-m_p\;1/2\;m_p|j_p\;m\rangle\langle l_t \;0\;1/2\;m_t|j_t\;m_t\rangle\,i^{-l_p}\exp\bigl[i(\sigma_{l_p}^p+\sigma_{l_t}^t)\bigr]\\
  &\times 2 Y_{m-m_p}^{l_p}(\hat{\mathbf{k}}_f)\sum_{\sigma_1\sigma_2\sigma_p} \int \frac {d\boldsymbol{\zeta}d\mathbf{r}d\boldsymbol{\eta}}{\zeta R} u_{\lambda k}(r_1)u_{\lambda k}(r_2)\left[Y^{\lambda}(\hat{\mathbf{r}}_1)Y^{\lambda}(\hat{\mathbf{r}}_2)\right]^{0*}_{0}\\
  &\times f_{l_pj_p}(\zeta)g_{l_tj_t}(R)\Bigl[\chi(\sigma_1)\chi(\sigma_2)\Bigr]^{0*}_{0}\left[Y^{l_p}
    (\hat{\boldsymbol{\zeta}})\chi(\sigma_p)\right]^{j_p*}_{m} V(r_{1p}) \\
  &\times \theta_0^0(\mathbf{r},\mathbf{s}) \Bigl[\chi(\sigma_1)\chi(\sigma_2)\Bigr]^{0}_{0}\left[Y^{l_t}(\hat{\mathbf{R}})\chi(\sigma_p)\right]^{j_t}_{m_t},
  \end{split}
\end{equation}
where
\begin{equation}\label{eq95}
    \begin{split}
    \mathbf{r}=&\mathbf{r}_2-\mathbf{r}_1,  \\
    \mathbf{s}=&\frac{1}{2}\left(\mathbf{r}_1+\mathbf{r}_2\right)-\mathbf{r}_p,\\
    \boldsymbol{\eta}=&\frac{1}{2}\left(\mathbf{r}_1+\mathbf{r}_2\right),\\
    \boldsymbol{\zeta}=&\mathbf{r}_p-\frac{\mathbf{r}_1+\mathbf{r}_2}{A+2}.\\
    \end{split}
\end{equation}
The sum over $\sigma_1,\sigma_2$ in (\ref{eq94}) is found to be equal to 1. We will now simplify the term $\left[Y^{l_p}
    (\hat{\boldsymbol{\zeta}})\chi(\sigma_p)\right]^{j_p*}_{m}\left[Y^{l_t}(\hat{\mathbf{R}})\chi(\sigma_p)\right]^{j_t}_{m_t}$,  noting that, (\ref{eq13})
\begin{equation}\label{eq96}
\left[Y^{l_p}(\hat{\boldsymbol{\zeta}})\chi(\sigma_p)\right]^{j_p*}_{m}=(-1)^{1/2-\sigma_p+j_p-m}
\left[Y^{l_p}(\hat{\boldsymbol{\zeta}})\chi(-\sigma_p)\right]^{j_p}_{-m}.
\end{equation}
and that
\begin{equation}\label{eq7_2_10}
\begin{split}
\left[Y^{l_p}(\hat{\boldsymbol{\zeta}})\right.&\left.\chi(-\sigma_p)\right]^{j_p}_{-m}
 \left[Y^{l_t}(\hat{\mathbf{R}})\chi(\sigma_p)\right]^{j_t}_{m_t}=\sum_{JM}\langle j_p \;-m\;j_t\;m_t|J\;M\rangle\\
 &\times\left\{\left[Y^{l_p}
 (\hat{\boldsymbol{\zeta}})\chi(-\sigma_p)\right]^{j_p}
 \left[Y^{l_t}(\hat{\mathbf{R}})\chi(\sigma_p)\right]^{j_t}\right\}_{M}^{J}\\
 \end{split}
\end{equation}

The only term which does not vanish after the integration is performed is the one in which the angular and spin functions are coupled to $L=0,S=0,J=0$. Thus,
\begin{equation}\label{eq7_2_11}
\begin{split}
\langle j_p & \;-m\;j_t\;m_t|0\;0\rangle \left\{\left[Y^{l_p}
 (\hat{\boldsymbol{\zeta}})\chi(-\sigma_p)\right]^{j_p}
 \left[Y^{l_t}(\hat{\mathbf{R}})\chi(\sigma_p)\right]^{j_t}\right\}_{0}^{0}\delta_{l_pl_t}\delta_{j_pj_t}\delta_{mm_t}\\
& =\frac{(-1)^{j_p+m_t}}{\sqrt{2j_p+1}}\left\{\left[Y^{l_p}
 (\hat{\boldsymbol{\zeta}})\chi(-\sigma_p)\right]^{j_p}
 \left[Y^{l_t}(\hat{\mathbf{R}})\chi(\sigma_p)\right]^{j_t}\right\}_{0}^{0}\delta_{l_pl_t}\delta_{j_pj_t}\delta_{mm_t}.
 \end{split}
\end{equation}
\idx{Two-nucleon transfer!second order DWBA}
Coupling separately the spin and angular functions, one obtains
 \begin{equation}\label{eq7_2_12}
\begin{split}
\left\{\left[Y^{l}
 (\hat{\boldsymbol{\zeta}})\chi(-\sigma_p)\right]^{j}\right.&\left.
 \left[Y^{l}(\hat{\mathbf{R}})\chi(\sigma_p)\right]^{j}\right\}_{0}^{0}\\
& =\bigl((l \tfrac{1}{2})_{j}(l \tfrac{1}{2})_{j}|(l l)_0(\tfrac{1}{2}\tfrac{1}{2})_0\bigr)_0
\left[\chi(-\sigma_p)\chi(\sigma_p)\right]^{0}_{0}
 \left[Y^{l}(\hat{\boldsymbol{\zeta}})Y^{l}(\hat{\mathbf{R}})\right]^{0}_{0}.
 \end{split}
\end{equation}
 We substitute (\ref{eq96}), (\ref{eq107}),(\ref{eq108}) in (\ref{eq94}) to obtain
\begin{equation}\label{eq98}
  \begin{split}
  \langle \Psi_f^{(-)}&(\mathbf{k}_f)|V(r_{1p})|\Psi_i^{(+)}(k_i,\hat {\mathbf{z}})\rangle=-\frac{(4\pi)^{3/2}}{k_ik_f}\sum_{lj}\bigl((\lambda \tfrac{1}{2})_k(\lambda \tfrac{1}{2})_k|(\lambda \lambda)_0(\tfrac{1}{2}\tfrac{1}{2})_0\bigr)_0\sqrt{\frac{2l+1}{2j+1}}\\
  &\times \langle l \;m_t-m_p\;1/2\;m_p|j\;m_t\rangle\langle l \;0\;1/2\;m_t|j\;m_t\rangle\,i^{-l}\exp\bigl[i(\sigma_{l}^p+\sigma_{l}^t)\bigr]\\
  &\times 2 Y_{m_t-m_p}^{l}(\hat{\mathbf{k}}_f) \int \frac {d\boldsymbol{\zeta}d\mathbf{r}d\boldsymbol{\eta}}{\zeta R} u_{\lambda k}(r_1)u_{\lambda k}(r_2)\left[Y^{\lambda}(\hat{\mathbf{r}}_1)Y^{\lambda}(\hat{\mathbf{r}}_2)\right]^{0*}_{0}\\
  &\times f_{lj}(\zeta)g_{lj}(R)\left[Y^{l}(\hat{\boldsymbol{\zeta}})Y^{l}(\hat{\mathbf{R}})\right]^{0}_{0} V(r_{1p}) \theta_0^0(\mathbf{r},\mathbf{s})\\
  &\times \bigl((l \tfrac{1}{2})_{j}(l \tfrac{1}{2})_{j}|(l l)_0(\tfrac{1}{2}\tfrac{1}{2})_0\bigr)_0\sum_{\sigma_p}(-1)^{1/2-\sigma_p}\left[\chi(-\sigma_p)\chi(\sigma_p)\right]^{0}_{0}.
  \end{split}
\end{equation}
The last sum over $\sigma_p$ leads to
\begin{equation}\label{eq99}
  \begin{split}
  \sum_{\sigma_p}(-1)^{1/2-\sigma_p}&\left[\chi(-\sigma_p)\chi(\sigma_p)\right]^{0}_{0}=
  \sum_{\sigma_p m}(-1)^{1/2-\sigma_p}\langle 1/2 \;m\;1/2\;-m|0\;0\rangle\\
  &\times \chi_m(-\sigma_p)\chi_{-m}(\sigma_p)\\
  &=\frac{1}{\sqrt 2}\sum_{\sigma_p m}(-1)^{1/2-\sigma_p}(-1)^{1/2-m}
   \delta_{m,-\sigma_p}\delta_{-m,\sigma_p}=-\sqrt 2.\\
  \end{split}
\end{equation}
The $9j$--symbols can be evaluated to find
\idx{Two-nucleon transfer!second order DWBA}
\begin{equation}\label{eq100}
  \begin{split}
\bigl((\lambda \tfrac{1}{2})_k(\lambda \tfrac{1}{2})_k|(\lambda \lambda)_0(\tfrac{1}{2}\tfrac{1}{2})_0\bigr)_0&=\sqrt{\frac{2k+1}{2(2\lambda+1)}}\\
\bigl((l \tfrac{1}{2})_{j}(l \tfrac{1}{2})_{j}|(l l)_0(\tfrac{1}{2}\tfrac{1}{2})_0\bigr)_0
&=\sqrt{\frac{2j+1}{2(2l+1)}},
  \end{split}
\end{equation}
and consequently,
\begin{equation}\label{eq101}
  \begin{split}
  \langle \Psi_f^{(-)}&(\mathbf{k}_f)|V(r_{1p})|\Psi_i^{(+)}(k_i,\hat {\mathbf{z}})\rangle=\frac{(4\pi)^{3/2}}{k_ik_f}\sum_{lj}\sqrt{\frac{2k+1}{2\lambda+1}}\\
  &\times \langle l \;m_t-m_p\;1/2\;m_p|j\;m_t\rangle\langle l \;0\;1/2\;m_t|j\;m_t\rangle\,i^{-l}\exp\bigl[i(\sigma_{l}^p+\sigma_{l}^t)\bigr]\\
  &\times \sqrt 2 Y_{m_t-m_p}^{l}(\hat{\mathbf{k}}_f) \int \frac {d\boldsymbol{\zeta}d\mathbf{r}d\boldsymbol{\eta}}{\zeta R} u_{\lambda k}(r_1)u_{\lambda k}(r_2)\left[Y^{\lambda}(\hat{\mathbf{r}}_1)Y^{\lambda}(\hat{\mathbf{r}}_2)\right]^{0*}_{0}\\
  &\times f_{lj}(\zeta)g_{lj}(R)\left[Y^{l}(\hat{\boldsymbol{\zeta}})Y^{l}(\hat{\mathbf{R}})\right]^{0}_{0} V(r_{1p}) \theta_0^0(\mathbf{r},\mathbf{s}).
  \end{split}
\end{equation}
The  values of the Clebsh--Gordan coefficients are, for $j=l-1/2$,
 \begin{equation}\label{eq7_2_17}
 \begin{split}
\langle l \;m_t-m_p\;1/2\;m_p|l-1/2\;m_t\rangle & \langle l \;0\;1/2\;m_t|l-1/2\;m_t\rangle\\
&=\left\{
\begin{aligned}
\frac{l}{2l+1} \qquad &\text{if}\; m_t=m_p\\
-\frac{\sqrt{l(l+1)}}{2l+1}\qquad &\text{if} \;m_t=-m_p
\end{aligned}
\right.
\end{split}
\end{equation}
and, for $j=l+1/2$:
 \begin{equation}\label{eq7_2_18}
 \begin{split}
\langle l \;m_t-m_p\;1/2\;m_p|l+1/2\;m_t\rangle & \langle l \;0\;1/2\;m_t|l+1/2\;m_t\rangle\\
&=\left\{
\begin{aligned}
\frac{l+1}{2l+1} \qquad &\text{if}\; m_t=m_p\\
\frac{\sqrt{l(l+1)}}{2l+1}\qquad &\text{if} \;m_t=-m_p
\end{aligned}
\right.
\end{split}
\end{equation}
One thus can write,
\begin{equation}\label{eq102}
  \begin{split}
  \langle \Psi_f^{(-)}&(\mathbf{k}_f)|V(r_{1p})|\Psi_i^{(+)}(k_i,\hat {\mathbf{z}})\rangle=\frac{(4\pi)^{3/2}}{k_ik_f}\sum_{l}
  \frac{1}{(2l+1)}\sqrt{\frac{(2k+1)}{(2\lambda+1)}}\exp\bigl[i(\sigma_{l}^p+\sigma_{l}^t)\bigr]i^{-l}\\
  &\times \sqrt 2 Y_{m_t-m_p}^{l}(\hat{\mathbf{k}}_f) \int \frac {d\boldsymbol{\zeta}d\mathbf{r}d\boldsymbol{\eta}}{\zeta R} u_{\lambda k}(r_1)u_{\lambda k}(r_2)\left[Y^{\lambda}(\hat{\mathbf{r}}_1)Y^{\lambda}(\hat{\mathbf{r}}_2)\right]^{0*}_{0}\\
  &\times  V(r_{1p})\theta_0^0(\mathbf{r},\mathbf{s})
  \left[Y^{l}(\hat{\boldsymbol{\zeta}})Y^{l}(\hat{\mathbf{R}})\right]^{0}_{0}\\
  &\times \left[\Bigl(f_{ll+1/2}(\zeta)g_{ll+1/2}(R)(l+1)+f_{ll-1/2}(\zeta)g_{ll-1/2}(R)l\Bigr)\delta_{m_p,m_t}\right.\\
  &\left.+\Bigl(f_{ll+1/2}(\zeta)g_{ll+1/2}(R)\sqrt{l(l+1)}-f_{ll-1/2}(\zeta)g_{ll-1/2}(R)\sqrt{l(l+1)}\Bigr)\delta_{m_p,-m_t}\right].
  \end{split}
\end{equation}
We can further simplify this expression using
\begin{equation}\label{eq103}
  \begin{split}
\left[Y^{\lambda}(\hat{\mathbf{r}}_1)Y^{\lambda}(\hat{\mathbf{r}}_2)\right]^{0*}_{0}&=
\left[Y^{\lambda}(\hat{\mathbf{r}}_1)Y^{\lambda}(\hat{\mathbf{r}}_2)\right]^{0}_{0}=\sum_m \langle \lambda \;m\;\lambda\;-m|0\;0\rangle Y^{\lambda}_m(\hat{\mathbf{r}}_1)Y^{\lambda}_{-m}(\hat{\mathbf{r}}_2)\\
&=\sum_m (-1)^{\lambda-m} \langle \lambda \;m\;\lambda\;-m|0\;0\rangle Y^{\lambda}_m(\hat{\mathbf{r}}_1)Y^{\lambda*}_{m}(\hat{\mathbf{r}}_2)\\
& =\frac{1}{\sqrt{2\lambda+1}}\sum_m Y^{\lambda}_m(\hat{\mathbf{r}}_1)Y^{\lambda*}_{m}(\hat{\mathbf{r}}_2)\\
&=\frac{\sqrt{(2\lambda+1)}}{4\pi}P_\lambda(\cos \theta_{12}).
  \end{split}
\end{equation}
Note that when using Condon-Shortley phases this last expression is to be multiplied by $(-1)^\lambda$, and that
\begin{equation}\label{eq7_2_21}
  \begin{split}
\left[Y^{l}(\hat{\boldsymbol{\zeta}})Y^{l}(\hat{\mathbf{R}})\right]^{0}_{0}&=\sum_m \langle l \;m\;l\;-m|0\;0\rangle Y^{l}_m(\hat{\boldsymbol{\zeta}}) Y^{l}_{-m}(\hat{\mathbf{R}})\\
&=\frac{1}{\sqrt{(2l+1)}}\sum_m (-1)^{l+m}Y^{l}_m(\hat{\boldsymbol{\zeta}}) Y^{l}_{-m}(\hat{\mathbf{R}}).\\
  \end{split}
\end{equation}
Because the integral of the above expression is independent of $m$, one can eliminate the $m$--sum and multiply by $2l+1$ the $m=0$ term, leading to \idx{Two-nucleon transfer!second order DWBA}
\begin{equation}\label{eq115}
  \begin{split}
\left[Y^{l}(\hat{\boldsymbol{\zeta}})Y^{l}(\hat{\mathbf{R}})\right]^{0}_{0}&\Rightarrow(-1)^l\sqrt{(2l+1)}  \, Y^{l}_0(\hat{\boldsymbol{\zeta}})_0Y^{l}(\hat{\mathbf{R}})\\
&=\sqrt{(2l+1)}Y^{l}_0(\hat{\boldsymbol{\zeta}}) Y_0^{l*}(\hat{\mathbf{R}}).
\end{split}
\end{equation}
\idx{Two-nucleon transfer!second order DWBA}
We now change the integration variables from $(\boldsymbol{\zeta},\mathbf{r},\boldsymbol{\eta})$ to $(\mathbf{R},\alpha,\beta,\gamma,r_{12},r_{1p},r_{2p})$, the quantity
\begin{equation}\label{eq113}
\left|\frac{\partial(\mathbf{r},\boldsymbol{\eta},\boldsymbol{\zeta})}
{\partial(\mathbf{R},\alpha,\beta,\gamma,r_{12},r_{1p},r_{2p})}\right|=r_{12}r_{1p}r_{2p}\sin\beta,
\end{equation}
being the Jacobian of the transformation.
%and, similarly,
%\begin{equation}\label{eq104}
%\left[Y^{l}(\hat{\boldsymbol{\zeta}})Y^{l}(\hat{\mathbf{R}})\right]^{0}_{0}=\frac{\sqrt{(2l+1)}}{4\pi}P_l(\cos \theta_{\zeta R}).
%\end{equation}
Finally,
\begin{equation}\label{eq114}
  \begin{split}
  \langle \Psi_f^{(-)}&(\mathbf{k}_f)|V(r_{1p})|\Psi_i^{(+)}(k_i,\hat {\mathbf{z}})\rangle=\frac{\sqrt{8\pi}}{k_ik_f}\sum_{l}
  \sqrt{\frac{2k+1}{2l+1}}\exp\bigl[i(\sigma_{l}^p+\sigma_{l}^t)\bigr]i^{-l}\\
  &\times Y_{m_t-m_p}^{l}(\hat{\mathbf{k}}_f) \int d\mathbf{R} Y_0^{l*}(\hat{\mathbf{R}})\int \frac {d\alpha\, d\beta\, d\gamma \, dr_{12}\,dr_{1p}\,dr_{2p}\,\sin\beta} {\zeta R}Y^{l}_0(\hat{\boldsymbol{\zeta}})\\
  &\times u_{\lambda k}(r_1)u_{\lambda k}(r_2)V(r_{1p})\theta_0^0(\mathbf{r},\mathbf{s})
  P_\lambda(\cos \theta_{12})r_{12}r_{1p}r_{2p}\\
  &\times \left[\Bigl(f_{ll+1/2}(\zeta)g_{ll+1/2}(R)(l+1)+f_{ll-1/2}(\zeta)g_{ll-1/2}(R)l\Bigr)\delta_{m_p,m_t}\right.\\
  &\left.+\Bigl(f_{ll+1/2}(\zeta)g_{ll+1/2}(R)\sqrt{l(l+1)}-f_{ll-1/2}(\zeta)g_{ll-1/2}(R)
  \sqrt{l(l+1)}\Bigr)\delta_{m_p,-m_t}\right].
  \end{split}
\end{equation}
It is noted that the second integral is a function of solely $\mathbf{R}$ transforming under rotations as $Y_0^{l}(\hat{\mathbf{R}})$, in keeping with the fact that the full dependence on the orientation of $\mathbf{R}$ is contained in the spherical harmonic $Y^{l}_0(\hat{\boldsymbol{\zeta}})$. The second integral can thus be cast into the form
\begin{equation}\label{eq116}
  \begin{split}
A(R)Y_0^{l}(\hat{\mathbf{R}})=\int d\alpha\, d\beta\, d\gamma &\, dr_{12}\,dr_{1p}\,dr_{2p}\,\sin\beta \\ &\times F(\alpha,\beta,\gamma,r_{12},r_{1p},r_{2p},R_x,R_y,R_z).
  \end{split}
\end{equation}
To evaluate $A(R)$, we set $\mathbf{R}$ along the $z$--axis
\begin{equation}\label{eq117}
  \begin{split}
A(R)=2\pi i^{-l}\sqrt{\frac{4\pi}{2l+1}}\int  d\beta\, d\gamma &\, dr_{12}\,dr_{1p}\,dr_{2p}\,\sin\beta \\ &\times F(\alpha,\beta,\gamma,r_{12},r_{1p},r_{2p},0,0,R),
  \end{split}
\end{equation}
where a factor $2\pi$  results from the integration over $\alpha$, the integrand not depending on $\alpha$. Substituting (\ref{eq116}) and (\ref{eq117}) in (\ref{eq114}) and, after integration over the angular variables of $\mathbf{R}$, we obtain
\begin{equation}\label{eq118}
  \begin{split}
  \langle \Psi_f^{(-)}&(\mathbf{k}_f)|V(r_{1p})|\Psi_i^{(+)}(k_i,\hat {\mathbf{z}})\rangle=2\frac{(2\pi)^{3/2}}{k_ik_f}\sum_{l}
  \sqrt{\frac{2k+1}{2l+1}}\exp\bigl[i(\sigma_{l}^p+\sigma_{l}^t)\bigr]i^{-l}\\
  &\times Y_{m_t-m_p}^{l}(\hat{\mathbf{k}}_f) \int dR \, d\beta\, d\gamma \, dr_{12}\,dr_{1p}\,dr_{2p}\,R\sin\beta \, r_{12}r_{1p}r_{2p}  \\
  &\times u_{\lambda k}(r_1)u_{\lambda k}(r_2)V(r_{1p})\theta_0^0(\mathbf{r},\mathbf{s})
  P_\lambda(\cos \theta_{12})P_l(\cos \theta_\zeta)\\
  &\times \left[\Bigl(f_{ll+1/2}(\zeta)g_{ll+1/2}(R)(l+1)+f_{ll-1/2}(\zeta)g_{ll-1/2}(R)l\Bigr)\delta_{m_p,m_t}\right.\\
  &\left.+\Bigl(f_{ll+1/2}(\zeta)g_{ll+1/2}(R)\sqrt{l(l+1)}-f_{ll-1/2}(\zeta)g_{ll-1/2}(R)
  \sqrt{l(l+1)}\Bigr)\delta_{m_p,-m_t}\right]/\zeta,
  \end{split}
\end{equation}
where  use was made of the relation
\begin{equation}\label{eq119}
Y^{l}_0(\hat{\boldsymbol{\zeta}})=i^l\sqrt{\frac{2l+1}{4\pi}}P_l(\cos \theta_\zeta).
\end{equation}
\idx{Two-nucleon transfer!second order DWBA}



The final expression of the differential cross section involves a sum over the spin orientations:
\begin{equation}\label{eq106}
\frac{d\sigma}{d\Omega}(\hat{\mathbf{k}}_f)=\frac{k_f}{k_i}\frac{\mu_i\mu_f}{(2\pi\hbar^2)^2}\frac{1}{2}\sum_{m_tm_p}|\langle \Psi_f^{(-)}(\mathbf{k}_f)|V(r_{1p})|\Psi_i^{(+)}(k_i,\hat {\mathbf{z}})\rangle|^2.
\end{equation}
When $m_p=1/2, m_t=1/2$ or $m_p=-1/2, m_t=-1/2$, the terms proportional to $\delta_{m_p,m_t}$  including the factor
\begin{equation}\label{eq107}
|Y_{m_t-m_p}^{l}(\hat{\mathbf{k}}_f)\delta_{m_p,m_t}|=|Y_{0}^{l}(\hat{\mathbf{k}}_f)|=
\left|i^l \sqrt{\frac{2l+1}{4\pi}}P_l^0(\cos\theta)\right|,
\end{equation}
in the case in which $m_p=-1/2, m_t=1/2$
\begin{equation}\label{eq108}
|Y_{m_t-m_p}^{l}(\hat{\mathbf{k}}_f)\delta_{m_p,-m_t}|=|Y_{1}^{l}(\hat{\mathbf{k}}_f)|=
\left|i^l \sqrt{\frac{2l+1}{4\pi}\frac{1}{l(l+1)}}P_l^1(\cos\theta)\right|,
\end{equation}
and
\begin{equation}\label{eq109}
|Y_{m_t-m_p}^{l}(\hat{\mathbf{k}}_f)\delta_{m_p,-m_t}|=|Y_{-1}^{l}(\hat{\mathbf{k}}_f)|=|Y_{1}^{l}(\hat{\mathbf{k}}_f)|
=\left|i^l \sqrt{\frac{2l+1}{4\pi}\frac{1}{l(l+1)}}P_l^1(\cos\theta)\right|,
\end{equation}
 when $m_p=1/2, m_t=-1/2$
Taking the squared modulus of (\ref{eq118}), the sum over $m_t$ and $m_p$ yields a factor 2 multiplying each one of the 2 different terms of the sum ($m_t=m_p$ and $m_t=-m_p$). This is equivalent to multiply each amplitude by $\sqrt{2}$, so the final constant that multiply the amplitudes is\idx{Two-nucleon transfer!second order DWBA}
\begin{equation}\label{eq238}
\frac{8\pi^{3/2}}{k_ik_f}.
\end{equation}
Now, for the triton wavefunction we use	
\begin{equation}\label{eq188}
\theta_0^0(\mathbf{r},\mathbf{s})=\rho(r_{1p})\rho(r_{2p})\rho(r_{12}),
\end{equation}
 $\rho(r)$ being a Tang--Herndon (1965) wave function also used by  \cite{Bayman:71}.
We obtain
\begin{equation}\label{eq110}
\frac{d\sigma}{d\Omega}(\hat{\mathbf{k}}_f)=\frac{1}{2 E_i^{3/2} E_f^{1/2}}\sqrt{\frac{\mu_f}{\mu_i}}\left(|I_{\lambda k}^{(0)}(\theta)|^2+|I_{\lambda k}^{(1)}(\theta)|^2\right),
\end{equation}
where
\begin{equation}\label{eq111}
  \begin{split}
  I_{\lambda k}^{(0)}(\theta)&=\sum_{l}P_l^0(\cos\theta)
  \sqrt{2k+1}\exp\bigl[i(\sigma_{l}^p+\sigma_{l}^t)\bigr]\\
  &\times  \int dR \, d\beta\, d\gamma \, dr_{12}\,dr_{1p}\,dr_{2p}\,R\sin\beta\,\rho(r_{1p})\rho(r_{2p})\rho(r_{12})   \\
  &\times u_{\lambda k}(r_1)u_{\lambda k}(r_2)V(r_{1p})
  P_\lambda(\cos \theta_{12})P_l(\cos \theta_\zeta)r_{12}r_{1p}r_{2p}\\
  &\times \Bigl(f_{ll+1/2}(\zeta)g_{ll+1/2}(R)\,(l+1)+f_{ll-1/2}(\zeta)g_{ll-1/2}(R)\,l\Bigr)/\zeta,
  \end{split}
\end{equation}
and
\begin{equation}\label{eq112}
  \begin{split}
  I_{\lambda k}^{(1)}(\theta)&=\sum_{l}P_l^1(\cos\theta)
  \sqrt{2k+1}\exp\bigl[i(\sigma_{l}^p+\sigma_{l}^t)\bigr]\\
  &\times  \int dR \, d\beta\, d\gamma \, dr_{12}\,dr_{1p}\,dr_{2p}\,R\sin\beta \,\rho(r_{1p})\rho(r_{2p})\rho(r_{12})  \\
  &\times u_{\lambda k}(r_1)u_{\lambda k}(r_2)V(r_{1p})
  P_\lambda(\cos \theta_{12})P_l(\cos \theta_\zeta)r_{12}r_{1p}r_{2p}\\
  &\times \Bigl(f_{ll+1/2}(\zeta)g_{ll+1/2}(R)-f_{ll-1/2}(\zeta)g_{ll-1/2}(R)\Bigr)/\zeta.
  \end{split}
\end{equation}
Note that the absence of the $(-1)^\lambda$ factor with respect to what is found in the work of Bayman\footnote{  \cite{Bayman:71}.}, is due to the use of time--reversed phases instead of Condon--Shortley phasing. This is compensated in the total result by a similar difference in the expression of the spectroscopic amplitudes. This ensures that, in either case, the contribution of all the single particle transitions tend to have the same phase for superfluid nuclei, adding coherently to enhance the transfer cross section.
\subsubsection{Heavy-ion Reactions}
\idx{Two-nucleon transfer!second order DWBA}
 In  dealing with a heavy ion reaction, $\theta_0^0(\mathbf{r},\mathbf{s})$ is the spatial part of the wavefunction
 \begin{equation}\label{eq189}
 \begin{split}
\Psi(\mathbf{r}_{b1},\mathbf{r}_{b2},\sigma_1,\sigma_2)&=\left[\psi ^{j_i} (\mathbf{r}_{b1},\sigma_1) \psi ^{j_i} (\mathbf{r}_{b2},\sigma_2) \right] _0^{0}\\
&=\theta_0^0(\mathbf{r},\mathbf{s})\Bigl[\chi(\sigma_1)\chi(\sigma_2)\Bigr]^{0}_{0},
 \end{split}
\end{equation}
where $\mathbf{r}_{b1},\mathbf{r}_{b2}$ are the positions of the two neutrons with respect to the $b$ core. It can be shown to be
 \begin{equation}\label{eqC7.2.39}
\theta_0^0(\mathbf{r},\mathbf{s})=\frac{u_{l_i j_i}(r_{b1})u_{l_i j_i}(r_{b2})}{4\pi}\sqrt{\frac{2j_i+1}{2}}P_{l_i}(\cos{\theta_i}),
\end{equation}
where $\theta_i$ is the angle between $\mathbf{r}_{b1}$ and $\mathbf{r}_{b2}$. Neglecting the spin--orbit term in the optical potential, as is usually done for heavy ion reactions, one obtains
\begin{equation}\label{eq191}
\frac{d\sigma}{d\Omega}(\hat{\mathbf{k}}_f)=\frac{\mu_f\mu_i}{16\pi^2\hbar^4k_i^3k_f}| T^{(1)}(\theta)|^2,
\end{equation}
where\idx{Two-nucleon transfer!second order DWBA}
\begin{equation}\label{eq192}
  \begin{split}
  T^{(1)}(\theta)&=\sum_{l}(2l+1)P_l(\cos\theta)
  \sqrt{(2j_i+1)(2j_f+1)}\exp\bigl[i(\sigma_{l}^p+\sigma_{l}^t)\bigr]\\
  &\times  \int dR \, d\beta\, d\gamma \, dr_{12}\,dr_{b1}\,dr_{b2}\,R\sin\beta\,u_{l_i j_i}(r_{b1})u_{l_i j_i}(r_{b2})   \\
  &\times u_{l_f j_f}(r_{A1})u_{l_f j_f}(r_{A2})V(r_{b1})
  P_\lambda(\cos \theta_{12})P_l(\cos \theta_\zeta)\\
  &\times r_{12}r_{b1}r_{b2} P_{l_i}(\cos{\theta_i})\frac{f_{l}(\zeta)g_{l}(R)}{\zeta},
  \end{split}
\end{equation}
obtained by using Eq. (\ref{eqC7.2.39}) in Eq. (\ref{eq94}) instead of (\ref{eq188}),  $\mathbf{r}_{A1},\mathbf{r}_{A2}$ being the coordinates of the two transferred  neutrons with respect to the $A$ core.


For control, in what follows we work out the same transition amplitude but starting from the  distorted waves for a reaction taking place between spinless nuclei, namely
\begin{equation}\label{eq206}
    \psi^{(+)}(\mathbf{r}_{Aa},\mathbf{k}_{Aa})=\sum_{l}\exp\left(i\sigma_{l}^i\right)g_{l}Y_0^{l}
    (\hat{\mathbf{r}}_{aA})\frac{\sqrt{4\pi(2l+1)}}{k_{aA}r_{aA}},
\end{equation}
and
\begin{equation}\label{eq207}
\begin{split}
    \psi^{(-)}(\mathbf{r}_{bB},\mathbf{k}_{bB})
    =\frac{4\pi}{k_{bB}r_{bB}}\sum_{\tilde l}i^{\tilde l}
    \exp\left(-i\sigma_{\tilde l}^f\right)f_{\tilde l}^*(r_{bB})\sum_{m}Y_{m}^{\tilde l*}
    (\hat{\mathbf{k}}_{bB})Y_{m}^{\tilde l}(\hat{\mathbf{r}}_{bB}).
\end{split}
\end{equation}
One can then write,
\begin{equation}\label{eq208}
  \begin{split}
  T^{1step}_{2N}=\langle \Psi_f^{(-)}&(\mathbf{k}_{bB})|V(r_{1p})|\Psi_i^{(+)}(k_{aA},\hat {\mathbf{z}})\rangle=\frac{(4\pi)^{3/2}}{k_{aA}k_{bB}}\sum_{l \tilde l m}\bigl((l_f \tfrac{1}{2})_{j_f}(l_f \tfrac{1}{2})_{j_f}|(l_f l_f)_0(\tfrac{1}{2}\tfrac{1}{2})_0\bigr)_0\\
  &\times\bigl((l_i \tfrac{1}{2})_{j_i}(l_i \tfrac{1}{2})_{j_i}|(l_i l_i)_0(\tfrac{1}{2}\tfrac{1}{2})_0\bigr)_0\sqrt{2l+1}
 i^{-l_p}\exp\bigl[i(\sigma_{\tilde l}^f+\sigma_{l}^i)\bigr]\\
  &\times 2 Y_{m}^{\tilde l}(\hat{\mathbf{k}}_{bB})\sum_{\sigma_1\sigma_2} \int \frac {d\mathbf{r}_{bB}d\mathbf{r}d\boldsymbol{\eta}}{r_{bB} r_{aA}} u_{l_f j_f}(r_{A1})u_{l_f j_f}(r_{A2})u_{l_i j_i}(r_{b1})u_{l_i j_i}(r_{b2})\\
  &\times \left[Y^{l_f}(\hat{\mathbf{r}}_{A1})Y^{l_f}(\hat{\mathbf{r}}_{A2})\right]^{0*}_{0}
  \left[Y^{l_i}(\hat{\mathbf{r}}_{b1})Y^{l_i}(\hat{\mathbf{r}}_{b2})\right]^{0}_{0}\\
  &\times f_{\tilde l}(r_{bB})g_{l}(r_{aA})\Bigl[\chi(\sigma_1)\chi(\sigma_2)\Bigr]^{0*}_{0}Y_{m}^{\tilde l*}(\hat{\mathbf{r}}_{bB}) V(r_{1p}) \\
  &\times \Bigl[\chi(\sigma_1)\chi(\sigma_2)\Bigr]^{0}_{0}Y_0^{l}
    (\hat{\mathbf{r}}_{aA}),
  \end{split}
\end{equation}
which, after a number of simplifications becomes \idx{Two-nucleon transfer!second order DWBA}
\begin{equation}\label{eq209}
  \begin{split}
  \langle \Psi_f^{(-)}&(\mathbf{k}_{bB})|V(r_{1p})|\Psi_i^{(+)}(k_{aA},\hat {\mathbf{z}})\rangle=\frac{(4\pi)^{3/2}}{k_{aA}k_{bB}}\sum_{l \tilde l m}
  \sqrt{\frac{(2j_f+1)(2j_i+1)}{(2l_f+1)(2l_i+1)}}\\
  &\times \sqrt{2l+1}
 i^{-\tilde l}\exp\bigl[i(\sigma_{\tilde l}^f+\sigma_{l}^i)\bigr]\\
  &\times  Y_{m}^{\tilde l}(\hat{\mathbf{k}}_{bB}) \int \frac {d\mathbf{r}_{bB}d\mathbf{r}d\boldsymbol{\eta}}{r_{bB} r_{aA}} u_{l_f j_f}(r_{A1})u_{l_f j_f}(r_{A2})u_{l_i j_i}(r_{b1})u_{l_i j_i}(r_{b2})\\
  &\times \left[Y^{l_f}(\hat{\mathbf{r}}_{A1})Y^{l_f}(\hat{\mathbf{r}}_{A2})\right]^{0*}_{0}
  \left[Y^{l_i}(\hat{\mathbf{r}}_{b1})Y^{l_i}(\hat{\mathbf{r}}_{b2})\right]^{0}_{0}\\
  &\times f_{\tilde l}(r_{bB})g_{l}(r_{aA})Y_{m}^{\tilde l*}(\hat{\mathbf{r}}_{bB}) V(r_{1p})Y_0^{l}
    (\hat{\mathbf{r}}_{aA}),
  \end{split}
\end{equation}
where $l=\tilde l$ and $m=0$. Making use of  Legendre polynomials leads to,
\begin{equation}\label{eq7_2_46}
  \begin{split}
  \langle \Psi_f^{(-)}&(\mathbf{k}_{bB})|V(r_{1p})|\Psi_i^{(+)}(k_{aA},\hat {\mathbf{z}})\rangle=\frac{(4\pi)^{-1/2}}{k_{aA}k_{bB}}\sum_{l}
  \sqrt{(2j_f+1)(2j_i+1)}\\
  &\times \sqrt{2l+1}
 i^{-l}\exp\bigl[i(\sigma_{l}^f+\sigma_{l}^i)\bigr]Y_{0}^{l}(\hat{\mathbf{k}}_{bB})\\
  &\times \int \frac {d\mathbf{r}_{bB}d\mathbf{r}d\boldsymbol{\eta}}{r_{bB} r_{aA}} u_{l_f j_f}(r_{A1})u_{l_f j_f}(r_{A2})u_{l_i j_i}(r_{b1})u_{l_i j_i}(r_{b2})\\
  &\times P_{l_f}(\cos \theta_A)
  P_{l_i}(\cos \theta_b)\\
  &\times f_{l}(r_{bB})g_{l}(r_{aA})Y_{0}^{l*}(\hat{\mathbf{r}}_{bB}) V(r_{1p})Y_0^{l}
    (\hat{\mathbf{r}}_{aA}).
  \end{split}
\end{equation}
Changing the integration variables and proceeding as in last section, (implying the   multiplicative factor  $2\pi\sqrt{\frac{4\pi}{2l+1}}$), the above expression becomes
\begin{equation}\label{eq7_2_47}
  \begin{split}
  \langle \Psi_f^{(-)}&(\mathbf{k}_{bB})|V(r_{1p})|\Psi_i^{(+)}(k_{aA},\hat {\mathbf{z}})\rangle=\frac{2\pi}{k_{aA}k_{bB}}\sum_{l}
  \sqrt{(2j_f+1)(2j_i+1)}\\
  &\times
 i^{-l}\exp\bigl[i(\sigma_{l}^f+\sigma_{l}^i)\bigr]Y_{0}^{l}(\hat{\mathbf{k}}_{bB})\\
  &\times \int dr_{aA} \, d\beta\, d\gamma \, dr_{12}\,dr_{b1}\,dr_{b2}\,r_{aA}\sin\beta \, r_{12}r_{b1}r_{b2} \\
  &\times P_{l_f}(\cos \theta_A)
  P_{l_i}(\cos \theta_b)u_{l_f j_f}(r_{A1})u_{l_f j_f}(r_{A2})u_{l_i j_i}(r_{b1})u_{l_i j_i}(r_{b2})\\
  &\times f_{l}(r_{bB})g_{l}(r_{aA})Y_{0}^{l*}(\hat{\mathbf{r}}_{bB}) V(r_{1p})/r_{bB},
  \end{split}
\end{equation}
which eventually can be recasted, through the use of Legendre polynomials, in the expression,\idx{Two-nucleon transfer!second order DWBA}
\begin{equation}\label{eq7_2_48}
  \begin{split}
  T^{1step}_{2N}=\langle \Psi_f^{(-)}&(\mathbf{k}_{bB})|V(r_{1p})|\Psi_i^{(+)}(k_{aA},\hat {\mathbf{z}})\rangle=\frac{1}{2k_{aA}k_{bB}}\sum_{l}
  \sqrt{(2j_f+1)(2j_i+1)}\\
  &\times
 i^{-l}\exp\bigl[i(\sigma_{l}^f+\sigma_{l}^i)\bigr]P_l(\cos\theta)(2l+1)\\
  &\times \int dr_{aA} \, d\beta\, d\gamma \, dr_{12}\,dr_{b1}\,dr_{b2}\,r_{aA}\sin\beta \, r_{12}r_{b1}r_{b2} \\
  &\times P_{l_f}(\cos \theta_A)
  P_{l_i}(\cos \theta_b)u_{l_f j_f}(r_{A1})u_{l_f j_f}(r_{A2})V(r_{1p})\\
  &\times u_{l_i j_i}(r_{b1})u_{l_i j_i}(r_{b2})f_{l}(r_{bB})g_{l}(r_{aA})P_l(\cos\theta_{if}) /r_{bB},
  \end{split}
\end{equation}
expression which gives the same results as (\ref{eq192})
\subsection{Coordinates for the calculation of simultaneous transfer}\label{csc}
\idx{Two-nucleon transfer!second order DWBA}
In what follows we explicit the coordinates used in the calculation of the above equations. Making use of the notation of \cite{Bayman:71}, we find the expression of the variables appearing in the integral as functions of the integration variables $r_{1p},r_{2p},r_{12},R,\beta,\gamma$ (remember that $\mathbf{R}=R\,\hat{\mathbf{z}}$, see last section). $\mathbf{R}$ being the center of mass coordinate. Thus, one can write
\begin{equation}\label{eq65}
    \mathbf{R} =\frac{1}{3}\left(\mathbf{r}_1+ \mathbf{r}_2+ \mathbf{r}_p\right)=\frac{1}{3}\left(\mathbf{R}+ \mathbf{d}_1+ \mathbf{R}+ \mathbf{d}_2+ \mathbf{R}+ \mathbf{d}_p\right),
\end{equation}
so
\begin{equation}\label{eq66}
     \mathbf{d}_1+ \mathbf{d}_2+ \mathbf{d}_p=0.
\end{equation}
Together with
\begin{equation}\label{eq67}
   \mathbf{d}_1+\mathbf{r}_{12}=\mathbf{d}_2 \qquad \mathbf{d}_2+ \mathbf{r}_{2p}=\mathbf{d}_p,
\end{equation}
we find
\begin{equation}\label{eq68}
   \mathbf{d}_1 =\frac{1}{3}\left(2\mathbf{r}_{12}+\mathbf{r}_{2p}\right),
\end{equation}
and
\begin{equation}\label{eq69}
    d_1^2 =\frac{1}{9}\left(4 r_{12}^2+r_{2p}^2+4\mathbf{r}_{12}\mathbf{r}_{2p}\right).
\end{equation}
Making use of
\begin{equation}\label{eq70}
\begin{split}
&\mathbf{r}_{12}+\mathbf{r}_{2p}=\mathbf{r}_{1p}\\
&r_{1p}^2=r_{12}^2+r_{2p}^2+2\mathbf{r}_{12}\mathbf{r}_{2p}\\
&2\mathbf{r}_{12}\mathbf{r}_{2p}=r_{1p}^2-r_{12}^2-r_{2p}^2.
\end{split}
\end{equation}
one obtains
\begin{equation}\label{eq71}
    d_1=\frac{1}{3}\sqrt{2r_{12}^2+2r_{1p}^2-r_{2p}^2}.
\end{equation}
Similarly,
\begin{equation}\label{eq72}
    d_2=\frac{1}{3}\sqrt{2r_{12}^2+2r_{2p}^2-r_{1p}^2}\qquad d_p=\frac{1}{3}\sqrt{2r_{2p}^2+2r_{1p}^2-r_{12}^2}.
\end{equation}
We now express the angle $\alpha$ between $\mathbf{d}_1$ and $\mathbf{r}_{12}$. We have
\begin{equation}\label{eq73}
-\mathbf{d}_1\mathbf{r}_{12}=r_{12}d_1\cos(\alpha),
\end{equation}
and
\begin{equation}\label{eq74}
\begin{split}
  & \mathbf{d}_1+ \mathbf{r}_{12}=\mathbf{d}_{2}\\
    & d_1^2+r_{12}^2+2\mathbf{d}_1\mathbf{r}_{12}=d_2^2.
\end{split}
\end{equation}
Consequently,
\begin{equation}\label{eq75}
\cos(\alpha)=\frac{d_1^2+r_{12}^2-d_2^2}{2r_{12}d_1}.
\end{equation}
The complete determination of $\mathbf{r}_1,\mathbf{r}_2,\mathbf{r}_{12}$ can be made by writing their expression in a simple configuration, in which the triangle lies in the $xz$--plane with $\mathbf{d}_1$ pointing along the positive $z$--direction, and $\mathbf{R}=0$. Then, a first rotation $\mathcal{R}_z(\gamma)$ of an angle $\gamma$ around the $z$--axis, a second rotation $\mathcal{R}_y(\beta)$ of an angle $\beta$ around the $y$--axis, and a translation along $\mathbf{R}$ will bring the vectors to the most general configuration. In other words,
\begin{equation}\label{eq76}
\begin{split}
\mathbf{r}_1=&\mathbf{R}+\mathcal{R}_y(\beta)\mathcal{R}_z(\gamma)\mathbf{r}'_1,\\
\mathbf{r}_{12}=&\mathcal{R}_y(\beta)\mathcal{R}_z(\gamma)\mathbf{r}'_{12},\\
\mathbf{r}_2=&\mathbf{r}_1+\mathbf{r}_{12},
\end{split}
\end{equation}
with
\begin{equation}\label{eq77}
\mathbf{r}'_1=
\begin{bmatrix}
  0\\
 0\\
 d_1\\
\end{bmatrix},
\end{equation}
\begin{equation}\label{eq78}
\mathbf{r}'_{12}=r_{12}
\begin{bmatrix}
  \sin(\alpha)\\
 0\\
 -\cos(\alpha)\\
\end{bmatrix},
\end{equation}
and the rotation matrixes are \idx{Two-nucleon transfer!second order DWBA}
\begin{equation}\label{eq80}
\mathcal{R}_y(\beta)=
\begin{bmatrix}
  \cos(\beta)&0&\sin(\beta)\\
 0&1&0\\
 -\sin(\beta)&0&\cos(\beta)\\
\end{bmatrix},
\end{equation}
and
\begin{equation}\label{eq81}
\mathcal{R}_z(\gamma)=
\begin{bmatrix}
  \cos(\gamma)&-\sin(\gamma)&0\\
 \sin(\gamma)&\cos(\gamma)&0\\
 0&0&1\\
\end{bmatrix}.
\end{equation}
then
\begin{equation}\label{eq82}
\mathbf{r}_1=
\begin{bmatrix}
  d_1\sin(\beta)\\
 0\\
 R+d_1\cos(\beta)\\
\end{bmatrix},
\end{equation}
\begin{equation}\label{eq83}
\mathbf{r}_{12}=
\begin{bmatrix}
  r_{12}\cos(\beta)\cos(\gamma)\sin(\alpha)-r_{12}\sin(\beta)\cos(\alpha)\\
 r_{12}\sin(\gamma)\sin(\alpha)\\
 -r_{12}\sin(\beta)\cos(\gamma)\sin(\alpha)-r_{12}\cos(\alpha)\cos(\beta)\\
\end{bmatrix},
\end{equation}
\begin{equation}\label{eq84}
\mathbf{r}_{2}=
\begin{bmatrix}
  d_1\sin(\beta)+r_{12}\cos(\beta)\cos(\gamma)\sin(\alpha)-r_{12}\sin(\beta)\cos(\alpha)\\
 r_{12}\sin(\gamma)\sin(\alpha)\\
 R+d_1\cos(\beta)-r_{12}\sin(\beta)\cos(\gamma)\sin(\alpha)-r_{12}\cos(\alpha)\cos(\beta)\\
\end{bmatrix}.
\end{equation}
We also need $\cos(\theta_{12})$, $\zeta$ and $\cos(\theta_{\zeta})$, $\theta_{12}$ being the angle between $\mathbf{r}_{1}$ and $\mathbf{r}_{2}$, $\boldsymbol{\zeta}=\mathbf{r}_p-\tfrac{\mathbf{r}_1+\mathbf{r}_2}{A+2}$ the position of the proton with respect to the final nucleus, and $\theta_{\zeta}$ the angle between $\boldsymbol{\zeta}$ and the $z$--axis:
\begin{equation}\label{eq85}
    \cos(\theta_{12})=\frac{\mathbf{r}_{1}\mathbf{r}_{2}}{r_1r_2},
\end{equation}
and
\begin{equation}\label{eq86}
    \boldsymbol{\zeta}=3\mathbf{R}-\frac{A+3}{A+2}(\mathbf{r}_1+\mathbf{r}_2),
\end{equation}
where we have used (\ref{eq65}).\idx{Two-nucleon transfer!second order DWBA}


For heavy ions, we find instead
\begin{equation}\label{eq198}
    \mathbf{R} =\frac{1}{m_a}\left(\mathbf{r}_{A1}+\mathbf{r}_{A2}+m_b\mathbf{r}_{Ab}\right),
\end{equation}
\begin{equation}\label{eq193}
    \mathbf{d}_1=\frac{1}{m_a}\left(m_b\mathbf{r}_{b2}-(m_b+1)\mathbf{r}_{12}\right),
\end{equation}
\begin{equation}\label{eq194}
    d_1=\frac{1}{m_a}\sqrt{(m_b+1)r_{12}^2+m_b(m_b+1)r_{b1}^2-m_br_{b2}^2},
\end{equation}
\begin{equation}\label{eq195}
    d_2=\frac{1}{m_a}\sqrt{(m_b+1)r_{12}^2+m_b(m_b+1)r_{b2}^2-m_br_{b1}^2},
\end{equation}
and
\begin{equation}\label{eq199}
    \boldsymbol{\zeta}=\frac{m_a}{m_b}\mathbf{R}-\frac{m_B+m_b}{m_bm_B}(\mathbf{r}_{A1}+\mathbf{r}_{A2}).
\end{equation}
The rest of the formulae are identical to the $(t,p)$ ones. We list them for convenience,
\begin{equation}\label{eq196}
\mathbf{r}_{A1}=
\begin{bmatrix}
  d_1\sin(\beta)\\
 0\\
 R+d_1\cos(\beta)\\
\end{bmatrix},
\end{equation}
\begin{equation}\label{eq197}
\mathbf{r}_{A2}=
\begin{bmatrix}
  d_1\sin(\beta)+r_{12}\cos(\beta)\cos(\gamma)\sin(\alpha)-r_{12}\sin(\beta)\cos(\alpha)\\
 r_{12}\sin(\gamma)\sin(\alpha)\\
 R+d_1\cos(\beta)-r_{12}\sin(\beta)\cos(\gamma)\sin(\alpha)-r_{12}\cos(\alpha)\cos(\beta)\\
\end{bmatrix}.
\end{equation}
We we also find
\begin{equation}\label{eq200}
    \mathbf{r}_{b1}=\frac{1}{m_b}(\mathbf{r}_{A2}+(m_b+1)\mathbf{r}_{A1}-m_a\mathbf{R}),
\end{equation}
and
\begin{equation}\label{eq201}
    \mathbf{r}_{b2}=\frac{1}{m_b}(\mathbf{r}_{A1}+(m_b+1)\mathbf{r}_{A2}-m_a\mathbf{R}).
\end{equation}
One can readily obtain
\begin{equation}\label{eq202}
    \cos\theta_{12}=\frac{r_{A1}^2+r_{A2}^2-r_{12}^2}{2r_{A1}r_{A2}},
\end{equation}
and
\begin{equation}\label{eq203}
    \cos\theta_{i}=\frac{r_{b1}^2+r_{b2}^2-r_{12}^2}{2r_{b1}r_{b2}}.
\end{equation}



\subsection{Alternative derivation (heavy ions)}
\idx{Two-nucleon transfer!second order DWBA}
In what follows we work out an alternative derivation of $T^{1step}_{2N}$, more closely related to heavy ion reactions\footnote{\cite{Bayman:82}.}, that is
\begin{equation}\label{eq141}
 \begin{split}
T^{(1)}(\theta)=&2\frac{(4\pi)^{3/2}}{k_{Aa}k_{Bb}}\sum_{l_pj_pml_tj_p}i^{-l_p}
\exp\bigl[i(\sigma_{l_p}^p+\sigma_{l_t}^t)\bigr] \sqrt{2l_t+1}\\
&\times \langle l_p \;m-m_p\;1/2\;m_p|j_p\;m\rangle\langle l_t \;0\;1/2\;m_t|j_t\;m_t\rangle Y_{m-m_p}^{l_p}(\hat{\mathbf{k}}_{Bb})\\
 &\times \sum_{\sigma_1 \sigma_2 \sigma_p}\int d\mathbf{r}_{Cc}d\mathbf{r}_{b1}d\mathbf{r}_{A2}\left[ \psi ^{j_f} (\mathbf{r}_{A1},\sigma_1) \psi ^{j_f} (\mathbf{r}_{A2},\sigma_2) \right] _0^{0*}\\
 &\times  v(r_{b1})
\left[ \psi ^{j_i} (\mathbf{r}_{b1},\sigma_1) \psi ^{j_i} (\mathbf{r}_{b2},\sigma_2) \right] _0^{0} \frac{g_{l_tj_t}(r_{Aa})f_{l_pj_p}(r_{Bb})}{r_{Aa}r_{Bb}}\\
 &\times \left[Y^{l_t}(\hat{\mathbf{r}}_{Aa})\chi(\sigma_p)\right]^{j_t}_{m_t}\left[Y^{l_p}
    (\hat{\mathbf{r}}_{Bb})\chi(\sigma_p)\right]^{j_p*}_{m}.\\
 \end{split}
\end{equation}
As shown above one can write,
\begin{equation}\label{eq142}
 \begin{split}
\sum_{\sigma_p}& \langle l_p \;m-m_p\;1/2\;m_p|j_p\;m\rangle\langle l_t \;0\;1/2\;m_t|j_t\;m_t\rangle \left[Y^{l_t}(\hat{\mathbf{r}}_{Aa})\chi(\sigma_p)\right]^{j_t}_{m_t}\left[Y^{l_p}
    (\hat{\mathbf{r}}_{Bb})\chi(\sigma_p)\right]^{j_p*}_{m}\\
    &=-\frac{\delta_{l_p,l_t}\delta_{j_p,j_t}\delta_{m,m_t}}{\sqrt{2l+1}}
    \left[Y^{l}(\hat{\mathbf{r}}_{Aa})Y^{l}(\hat{\mathbf{r}}_{Bb})\right]^{0}_{0}
    \left\{
\begin{aligned}
\frac{l}{2l+1} \qquad &\text{if}\; m_t=m_p\\
-\frac{\sqrt{l(l+1)}}{2l+1}\qquad &\text{if} \;m_t=-m_p
\end{aligned}
\right.
 \end{split}
\end{equation}
when $j=l-1/2$ and
\begin{equation}\label{eq143}
 \begin{split}
\sum_{\sigma_p}& \langle l_p \;m-m_p\;1/2\;m_p|j_p\;m\rangle\langle l_t \;0\;1/2\;m_t|j_t\;m_t\rangle \left[Y^{l_t}(\hat{\mathbf{r}}_{Aa})\chi(\sigma_p)\right]^{j_t}_{m_t}\left[Y^{l_p}
    (\hat{\mathbf{r}}_{Bb})\chi(\sigma_p)\right]^{j_p*}_{m}\\
    &=-\frac{\delta_{l_p,l_t}\delta_{j_p,j_t}\delta_{m,m_t}}{\sqrt{2l+1}}
    \left[Y^{l}(\hat{\mathbf{r}}_{Aa})Y^{l}(\hat{\mathbf{r}}_{Bb})\right]^{0}_{0}
    \left\{
\begin{aligned}
\frac{l+1}{2l+1} \qquad &\text{if}\; m_t=m_p\\
\frac{\sqrt{l(l+1)}}{2l+1}\qquad &\text{if} \;m_t=-m_p
\end{aligned}
\right.
 \end{split}
\end{equation}
if $j=l+1/2$. One then gets
\begin{equation}\label{eq144}
 \begin{split}
T^{(1)}(\mu=0;\theta)=&2\frac{(4\pi)^{3/2}}{k_{Aa}k_{Bb}}\sum_{l}i^{-l}
\frac{\exp\bigl[i(\sigma_{l}^p+\sigma_{l}^t)\bigr]}{2l+1}Y_{m_t-m_p}^{l}(\hat{\mathbf{k}}_{Bb})\\
 &\times \sum_{\sigma_1 \sigma_2}\int \frac{d\mathbf{r}_{Cc}d\mathbf{r}_{b1}d\mathbf{r}_{A2}}{r_{Aa}r_{Bb}}\left[ \psi ^{j_f} (\mathbf{r}_{A1},\sigma_1) \psi ^{j_f} (\mathbf{r}_{A2},\sigma_2) \right] _0^{0*}\\
 &\times  v(r_{b1})
\left[ \psi ^{j_i} (\mathbf{r}_{b1},\sigma_1) \psi ^{j_i} (\mathbf{r}_{b2},\sigma_2) \right] _0^{0}
 \left[Y^{l}(\hat{\mathbf{r}}_{Aa})Y^{l}(\hat{\mathbf{r}}_{Bb})\right]^{0}_{0}\\
 &\times \left[\Bigl(f_{ll+1/2}(r_{Bb})g_{ll+1/2}(r_{Aa})(l+1)+f_{ll-1/2}(r_{Bb})g_{ll-1/2}(r_{Aa})l\Bigr)\delta_{m_p,m_t}\right.\\
  &\left.+\Bigl(f_{ll+1/2}(r_{Bb})g_{ll+1/2}(r_{Aa})\sqrt{l(l+1)}-f_{ll-1/2}(r_{Bb})g_{ll-1/2}(r_{Aa})
  \sqrt{l(l+1)}\Bigr)\delta_{m_p,-m_t}\right].\\
 \end{split}
\end{equation}
Making use of the relations,\idx{Two-nucleon transfer!second order DWBA}
\begin{equation}\label{eq145}
 \begin{split}
\left[ \psi ^{j_f}\right.& \left.(\mathbf{r}_{A1},\sigma_1) \psi ^{j_f} (\mathbf{r}_{A2},\sigma_2) \right] _0^{0*}\\
&=\bigl ( (l_f \tfrac{1}{2})_{j_f} (l_f \tfrac{1}{2})_{j_f} |(l_f l_f)_0 (\tfrac{1}{2}\tfrac{1}{2})_0 \bigr )_0 u_{l_f}(r_{A1})u_{l_f}(r_{A2})\\
&\times \left[ Y ^{l_f}(\hat{\mathbf{r}}_{A1})  Y ^{l_f}(\hat{\mathbf{r}}_{A2}) \right] _0^{0*}\left[\chi(\sigma_1)\chi(\sigma_2)\right]^{0*}_{0}\\
&=\sqrt{\frac{2j_f+1}{2(2l_f+1)}}u_{l_f}(r_{A1})u_{l_f}(r_{A2})\\
&\times \left[ Y ^{l_f}(\hat{\mathbf{r}}_{A1})  Y ^{l_f}(\hat{\mathbf{r}}_{A2}) \right] _0^{0*}\left[\chi(\sigma_1)\chi(\sigma_2)\right]^{0*}_{0}\\
&=\sqrt{\frac{2j_f+1}{2}}\frac{u_{l_f}(r_{A1})u_{l_f}(r_{A2})}{4\pi}
P_{l_f}(\cos\omega_A)\left[\chi(\sigma_1)\chi(\sigma_2)\right]^{0*}_{0},
\end{split}
\end{equation}
and
\begin{equation}\label{eq146}
 \begin{split}
\left[ \psi ^{j_i}\right.& \left.(\mathbf{r}_{b1},\sigma_1) \psi ^{j_i} (\mathbf{r}_{b2},\sigma_2) \right] _0^{0}\\
&=\bigl ( (l_i \tfrac{1}{2})_{j_i} (l_i \tfrac{1}{2})_{j_i} |(l_i l_i)_0 (\tfrac{1}{2}\tfrac{1}{2})_0 \bigr )_0
u_{l_i}(r_{b1})u_{l_i}(r_{b2})\\
&\times \left[ Y ^{l_i}(\hat{\mathbf{r}}_{b1})  Y ^{l_i}(\hat{\mathbf{r}}_{b2}) \right] _0^{0}\left[\chi(\sigma_1)\chi(\sigma_2)\right]^{0}_{0}\\
&=\sqrt{\frac{2j_i+1}{2(2l_i+1)}}u_{l_i}(r_{b1})u_{l_i}(r_{b2})\\
&\times\left[ Y ^{l_i}(\hat{\mathbf{r}}_{b1})  Y ^{l_i}(\hat{\mathbf{r}}_{b2}) \right] _0^{0}\left[\chi(\sigma_1)\chi(\sigma_2)\right]^{0}_{0}\\
&=\sqrt{\frac{2j_i+1}{2}}\frac{u_{l_i}(r_{b1})u_{l_i}(r_{b2})}{4\pi}
P_{l_i}(\cos\omega_b)\left[\chi(\sigma_1)\chi(\sigma_2)\right]^{0}_{0},
\end{split}
\end{equation}
\idx{Two-nucleon transfer!second order DWBA}
where $\omega_A$ is the angle between $\mathbf{r}_{A1}$ and $\mathbf{r}_{A2}$, and $\omega_b$ is the angle between $\mathbf{r}_{b1}$ and $\mathbf{r}_{b2}$. Consequently
\begin{equation}\label{eq147}
 \begin{split}
T^{(1)}(\theta)=&(4\pi)^{-3/2}\frac{\sqrt{(2j_i+1)(2j_f+1)}}{k_{Aa}k_{Bb}}\sum_{l}i^{-l}
\frac{\exp\bigl[i(\sigma_{l}^p+\sigma_{l}^t)\bigr]}{\sqrt{2l+1}}Y_{m_t-m_p}^{l}(\hat{\mathbf{k}}_{Bb})\\
 &\times \int \frac{d\mathbf{r}_{Cc}d\mathbf{r}_{b1}d\mathbf{r}_{A2}}{r_{Aa}r_{Bb}}P_{l_f}(\cos\omega_A)P_{l_i}(\cos\omega_b)
 P_{l}(\cos\omega_{if})\\
 &\times  v(r_{b1})u_{l_i}(r_{b1})u_{l_i}(r_{b2})u_{l_f}(r_{A1})u_{l_f}(r_{A2})
\\
 &\times \left[\Bigl(f_{ll+1/2}(r_{Bb})g_{ll+1/2}(r_{Aa})(l+1)+f_{ll-1/2}(r_{Bb})g_{ll-1/2}(r_{Aa})l\Bigr)\delta_{m_p,m_t}\right.\\
  &\left.+\Bigl(f_{ll+1/2}(r_{Bb})g_{ll+1/2}(r_{Aa})\sqrt{l(l+1)}-f_{ll-1/2}(r_{Bb})g_{ll-1/2}(r_{Aa})
  \sqrt{l(l+1)}\Bigr)\delta_{m_p,-m_t}\right],\\
 \end{split}
\end{equation}
\idx{Two-nucleon transfer!second order DWBA}
where $\omega_{if}$ is the angle between $\mathbf{r}_{Aa}$ and $\mathbf{r}_{Bb}$. For heavy ions, we can consider that the the optical potential does not have a spin--orbit term, and the distorted waves are independent of $j$. We thus have
\begin{equation}\label{eq167}
 \begin{split}
T^{(1)}(\theta)=&(4\pi)^{-3/2}\frac{\sqrt{(2j_i+1)(2j_f+1)}}{k_{Aa}k_{Bb}}\sum_{l}i^{-l}
\exp\bigl[i(\sigma_{l}^p+\sigma_{l}^t)\bigr]Y_{0}^{l}(\hat{\mathbf{k}}_{Bb})\sqrt{2l+1}\\
 &\times \int \frac{d\mathbf{r}_{Cc}d\mathbf{r}_{b1}d\mathbf{r}_{A2}}{r_{Aa}r_{Bb}}P_{l_f}(\cos\omega_A)P_{l_i}(\cos\omega_b)
 P_{l}(\cos\omega_{if})\\
 &\times  v(r_{b1})u_{l_i}(r_{b1})u_{l_i}(r_{b2})u_{l_f}(r_{A1})u_{l_f}(r_{A2})f_{l}(r_{Bb})g_{l}(r_{Aa}).
 \end{split}
\end{equation}
Changing variables one obtains,
\begin{equation}\label{eq168}
 \begin{split}
T^{(1)}(\theta)=&(4\pi)^{-1}\frac{\sqrt{(2j_i+1)(2j_f+1)}}{k_{Aa}k_{Bb}}\sum_{l}
\exp\bigl[i(\sigma_{l}^p+\sigma_{l}^t)\bigr]P_{l}(\cos\theta)(2l+1)\\
 &\times \int dr_{1A}\,dr_{2A}\,dr_{Aa}\,d(\cos\beta)\,d(\cos\omega_A)\,d\gamma \,r_{1A}^2r_{2A}^2r_{Aa}^2 \\ &\times P_{l_f}(\cos\omega_A)P_{l_i}(\cos\omega_b)
 P_{l}(\cos\omega_{if})v(r_{b1})\\
 &\times  u_{l_i}(r_{b1})u_{l_i}(r_{b2})u_{l_f}(r_{A1})u_{l_f}(r_{A2})f_{l}(r_{Bb})g_{l}(r_{Aa}).
 \end{split}
\end{equation}
\idx{Two-nucleon transfer!second order DWBA}
\subsection[Coordinates for simultaneous transfer]{Coordinates used to derive the transition amplitude  (Eq. (\ref{eq168}))}
We determine the relation between the integration variables in (\ref{eq147}) and the coordinates needed to evaluate the quantities in the integrand. Noting that
\begin{equation}\label{eq177}
\mathbf{r}_{Aa}=\frac{\mathbf{r}_{A1}+\mathbf{r}_{A2}+m_b\mathbf{r}_{Ab}}{m_b+2},
\end{equation}
one has
\begin{equation}\label{eq7_2_90}
 \begin{split}
\mathbf{r}_{b1}=\mathbf{r}_{bA}+\mathbf{r}_{A1}=\frac{(m_b+1)\mathbf{r}_{A1}+\mathbf{r}_{A2}-(m_b+2)\mathbf{r}_{Aa}}{m_b},
 \end{split}
\end{equation}
\begin{equation}\label{eq178}
 \begin{split}
\mathbf{r}_{b2}=\mathbf{r}_{bA}+\mathbf{r}_{A2}=\frac{(m_b+1)\mathbf{r}_{A2}+\mathbf{r}_{A1}-(m_b+2)\mathbf{r}_{Aa}}{m_b},
 \end{split}
\end{equation}
and
\begin{equation}\label{eq7_2_93}
\begin{split}
\mathbf{r}_{Cc}=\mathbf{r}_{CA}+\mathbf{r}_{A1}+&\mathbf{r}_{1c}=
-\frac{1}{m_A+1}\mathbf{r}_{A2}+\mathbf{r}_{A1}-\frac{m_b}{m_b+1}\mathbf{r}_{b1}\\
&=\frac{m_b+2}{m_b+1}\mathbf{r}_{Aa}-\frac{m_b+2+m_A}{(m_b+1)(m_A+1)}\mathbf{r}_{A2}
\end{split}
\end{equation}
Since,
\begin{equation}\label{eq179}
\mathbf{r}_{AB}=\frac{\mathbf{r}_{A1}+\mathbf{r}_{A2}}{m_A+2},
\end{equation}
one obtains \idx{Two-nucleon transfer!second order DWBA}
\begin{equation}\label{eq180}
\mathbf{r}_{Bb}=\mathbf{r}_{BA}+\mathbf{r}_{Ab}=\frac{m_b+2}{m_b}\mathbf{r}_{Aa}-\frac{m_A+m_b+2}{(m_A+2)m_b}
(\mathbf{r}_{A1}+\mathbf{r}_{A2}).
\end{equation}
Using the same rotations as those used in  Section \ref{csc} one gets,
\begin{equation}\label{eq169}
\mathbf{r}_{A1}=r_{A1}
\begin{bmatrix}
  \sin\alpha\\
 0\\
 \cos\alpha\\
\end{bmatrix},
\end{equation}
and
\begin{equation}\label{eq170}
\mathbf{r}_{A2}=r_{A2}
\begin{bmatrix}
  -\cos\alpha\cos\gamma\sin\omega_A+\sin\alpha\cos\omega_A\\
 -\sin\gamma\sin\omega_A\\
\sin\alpha\cos\gamma\sin\omega_A+\cos\alpha\cos\omega_A\\
\end{bmatrix},
\end{equation}
with
\begin{equation}\label{eq171}
\cos\alpha=\frac{r_{A1}^2-d_1^2+r_{Aa}^2}{2r_{A1}r_{Aa}},
\end{equation}
and
\begin{equation}\label{eq172}
d_1=\sqrt{r_{A1}^2-r_{Aa}^2\sin^2\beta}-r_{Aa}\cos\beta.
\end{equation}
Note that though $\beta,r_{1A},r_{Aa}$ are independent integration variables, they have to fulfill the condition
\begin{equation}\label{eq173}
r_{Aa}\sin\beta\leq r_{A1}, \quad \text{for}\;0\leq\beta\leq\pi.
\end{equation}
The expression of the remaining quantities appearing in the integral are now straightforward,
\begin{equation}\label{eq174}
\begin{split}
r_{b1}&=m_b^{-1}\left|(m_b+1)\mathbf{r}_{A1}+\mathbf{r}_{A2}-(m_b+2)\mathbf{r}_{Aa}\right|\\
&=m_b^{-1}\Bigl((m_b+2)^2r_{Aa}^2+(m_b+1)^2r_{A1}^2+r_{A2}^2\\
&-2(m_b+2)(m_b+1)\mathbf{r}_{Aa}\,\mathbf{r}_{A1}-
2(m_b+2)\mathbf{r}_{Aa}\,\mathbf{r}_{A2}+2(m_b+1)\mathbf{r}_{A1}\mathbf{r}_{A2}\Bigr)^{1/2},
\end{split}
\end{equation}
\idx{Two-nucleon transfer!second order DWBA}
\hspace{0.5cm}
\begin{equation}\label{eq175}
\begin{split}
r_{b2}&=m_b^{-1}\left|(m_b+1)\mathbf{r}_{A2}+\mathbf{r}_{A1}-(m_b+2)\mathbf{r}_{Aa}\right|\\
&=m_b^{-1}\Bigl((m_b+2)^2r_{Aa}^2+(m_b+1)^2r_{A2}^2+r_{A1}^2\\
&-2(m_b+2)(m_b+1)\mathbf{r}_{Aa}\,\mathbf{r}_{A2}-
2(m_b+2)\mathbf{r}_{Aa}\,\mathbf{r}_{A1}+2(m_b+1)\mathbf{r}_{A2}\mathbf{r}_{A1}\Bigr)^{1/2},
\end{split}
\end{equation}
\hspace{0.5cm}
\begin{equation}\label{eq176}
\begin{split}
r_{Bb}&=\left|\frac{m_b+2}{m_b}\mathbf{r}_{Aa}-\frac{m_A+m_b+2}{(m_A+2)m_b}
(\mathbf{r}_{A1}+\mathbf{r}_{A2})\right|\\
&=\Biggl[\left(\frac{m_b+2}{m_b}\right)^2r_{Aa}^2+
\left(\frac{m_A+m_b+2}{(m_A+2)m_b}\right)^2(r_{A1}^2+r_{A2}^2+2\mathbf{r}_{A1}\mathbf{r}_{A2})\\
&-2\frac{(m_b+2)(m_A+m_b+2)}{(m_A+2)m_b^2}\mathbf{r}_{Aa}(\mathbf{r}_{A1}+\mathbf{r}_{A2})\Biggr]^{1/2},
\end{split}
\end{equation}
\hspace{0.5cm}
\begin{equation}\label{eq186}
\begin{split}
r_{Cc}&=\left|\frac{m_b+2}{m_b+1}\mathbf{r}_{Aa}-\frac{m_b+2+m_A}{(m_b+1)(m_A+1)}\mathbf{r}_{A2}\right|\\
&=\Biggl[\left(\frac{m_a}{(m_a-1)}\right)^2r_{Aa}^2+
\left(\frac{m_A+m_a}{(m_A+1)(m_a-1)}\right)^2r_{A2}^2\\
&-2\frac{m_Am_a+m_a^2}{(m_A+1)(m_a-1)^2}\mathbf{r}_{Aa}\mathbf{r}_{A2}\Biggr]^{1/2},
\end{split}
\end{equation}
\hspace{0.5cm}
\begin{equation}\label{eq181}
\begin{split}
\cos\omega_b=\frac{\mathbf{r}_{b1}\mathbf{r}_{b2}}{r_{b1}r_{b2}},
\end{split}
\end{equation}
\hspace{0.5cm}
\begin{equation}\label{eq182}
\begin{split}
\cos\omega_{if}=\frac{\mathbf{r}_{Aa}\mathbf{r}_{Bb}}{r_{Aa}r_{Bb}},
\end{split}
\end{equation}
with
\begin{equation}\label{eq183}
\begin{split}
\mathbf{r}_{Aa}\mathbf{r}_{A1}=r_{Aa}r_{A1}\cos\alpha,
\end{split}
\end{equation}
\begin{equation}\label{eq184}
\begin{split}
\mathbf{r}_{Aa}\mathbf{r}_{A2}=r_{Aa}r_{A2}(\sin\alpha\cos\gamma\sin\omega_A+\cos\alpha\cos\omega_A),
\end{split}
\end{equation}
\begin{equation}\label{eq185}
\begin{split}
\mathbf{r}_{A1}\mathbf{r}_{A2}=r_{A1}r_{A2}\cos\omega_A.
\end{split}
\end{equation}
\subsection{Successive transfer}
The successive two--neutron transfer amplitudes can be written as (\cite{Bayman:82}):
\begin{equation}\label{eq1}
 \begin{split}
T_{succ}^{(2)}(\theta)=&\frac{4\mu_{Cc}}{\hbar^2}\sum_{\substack{\sigma_1 \sigma_2 \\ \sigma_1' \sigma_2' \\ KM}}\int d^3r_{Cc}d^3r_{b1}d^3r_{A2}d^3r_{Cc}'
d^3r_{b1}'d^3r_{A2}' \,\chi^{(-)*}(\mathbf{k}_{Bb},\mathbf{r}_{Bb})\\
 & \times \left[ \psi ^{j_f} (\mathbf{r}_{A1},\sigma_1) \psi ^{j_f} (\mathbf{r}_{A2},\sigma_2) \right] _0^{0*} v(r_{b1})
\left[ \psi ^{j_f} (\mathbf{r}_{A2},\sigma_2) \psi ^{j_i} (\mathbf{r}_{b1},\sigma_1) \right] _M^{K}\\
& \times G(\mathbf{r}_{Cc},\mathbf{r}_{Cc}')
\left[ \psi ^{j_f} (\mathbf{r}_{A2}',\sigma_2') \psi ^{j_i} (\mathbf{r}_{b1}',\sigma_1') \right] _M^{K*} v(r_{c2}')\\
& \times \left[ \psi ^{j_i} (\mathbf{r}_{b1}',\sigma_1') \psi ^{j_i} (\mathbf{r}_{b2}',\sigma_2') \right] _0^{0}
\chi^{(+)}( \mathbf{r}_{Aa}').
 \end{split}
\end{equation}
It is of notice that the time--reversal phase convention is used throughout.
Expanding the Green function and the distorted waves in a basis of angular momentum eigenstate one can write,\idx{Two-nucleon transfer!second order DWBA}
\begin{equation}\label{eq2}
\chi^{(-)*}(\mathbf{k}_{Bb},\mathbf{r}_{Bb})= \sum_{\tilde l}\frac{ 4\pi }{k_{Bb} r_{Bb}} i^{-\tilde l}
e^{i\sigma_f^{\tilde l}} F_{\tilde l} \sum_m Y_m^{\tilde l} (\hat r_{Bb}) Y_m^{\tilde l*} (\hat k_{Bb}),
\end{equation}
the sum over $m$ being
\begin{equation}\label{eq3}
\sum_m (-1)^{\tilde l-m} Y_m^{\tilde l} (\hat r_{Bb}) Y_{-m}^{\tilde l}(\hat k_{Bb})=\sqrt{2\tilde l+1}
\left[  Y^{\tilde l} (\hat r_{Bb}) Y^{\tilde l} (\hat k_{Bb})\right]_0^0,
\end{equation}
where we have used (\ref{eq21}) and (\ref{eq22}), so
\begin{equation}\label{eq5}
\chi^{(-)*}(\mathbf{k}_{Bb},\mathbf{r}_{Bb})=  \sum_{\tilde l}\sqrt{2\tilde l+1}\frac{ 4\pi }{k_{Bb} r_{Bb}} i^{-\tilde l}
e^{i\sigma_f^{\tilde l}} F_{\tilde l}  (r_{Bb})
\left[  Y^{\tilde l} (\hat r_{Bb}) Y^{\tilde l} (\hat k_{Bb})\right]_0^0.
\end{equation}
Similarly,
\begin{equation}\label{eq6}
\chi^{(+)}( \mathbf{r}_{Aa}')= \sum_{ l}i^l \sqrt{2l+1}\frac{ 4\pi }{k_{Aa} r_{Aa}'}
e^{i\sigma_i^{ l}} F_{ l}  (r_{Aa}')
\left[  Y^{l} (\hat r_{Aa}') Y^{l} (\hat k_{Aa})\right]_0^0
\end{equation}
where we have taken into account the choice $\hat k_{Aa} \equiv \hat z$.
The Green function can be written as \idx{Two-nucleon transfer!second order DWBA}
\begin{equation}\label{eq7}
G(\mathbf{r}_{Cc},\mathbf{r}_{Cc}')=i\sum_{l_c}\sqrt{2l_c+1}
\frac{f_{l_c}(k_{Cc},r_<)P_{l_c}(k_{Cc},r_>)}{k_{Cc}r_{Cc}r_{Cc}'}
\left[  Y^{l_c} (\hat r_{Cc}) Y^{l_c} (\hat r_{Cc}')\right]_0^0.
\end{equation}
Finally
\begin{equation}\label{eq8}
 \begin{split}
T_{succ}^{(2)}(\theta)=&\frac{4\mu_{Cc}(4\pi)^2 i}{\hbar^2 k_{Aa}k_{Bb}k_{Cc}}\sum_{l,l_c,\tilde l}
e^{i(\sigma_i^l +\sigma_f^{\tilde l})}  i^{l- \tilde l} \sqrt{(2 l+1)(2 l_c+1)(2 \tilde l+1)}\\
& \times \sum_{\substack{\sigma_1 \sigma_2 \\ \sigma_1' \sigma_2'}} \int d^3r_{Cc}d^3r_{b1}d^3r_{A2}d^3r_{Cc}'
d^3r_{b1}'d^3r_{A2}' v(r_{b1}) v(r_{c2}')
 \left[  Y^{\tilde l} (\hat r_{Bb}) Y^{\tilde l} (\hat k_{Bb})\right]_0^0 \\
& \times \left[  Y^{ l} (\hat r_{Aa}') Y^{ l} (\hat k_{Aa}')\right]_0^0
 \left[  Y^{l_c} (\hat r_{Cc}) Y^{l_c} (\hat r_{Cc}')\right]_0^0 \frac{F_{\tilde l}(r_{Bb})}{r_{Bb}}
\frac{F_{ l}(r_{Aa}')}{r_{Aa}'} \\
& \times \frac{f_{l_c}(k_{Cc},r_<)P_{l_c}(k_{Cc},r_>)}{r_{Cc}r_{Cc}'}
\left[ \psi ^{j_f} (\mathbf{r}_{A1},\sigma_1) \psi ^{j_f} (\mathbf{r}_{A2},\sigma_2) \right] _0^{0*} \\
& \times \left[ \psi ^{j_i} (\mathbf{r}_{b1}',\sigma_1') \psi ^{j_i} (\mathbf{r}_{b2}',\sigma_2') \right] _0^{0}
\sum_{KM}  \left[ \psi ^{j_f} (\mathbf{r}_{A2},\sigma_2) \psi ^{j_i} (\mathbf{r}_{b1},\sigma_1) \right] _M^{K} \\
& \times \left[ \psi ^{j_f} (\mathbf{r}_{A2}',\sigma_2') \psi ^{j_i} (\mathbf{r}_{b1}',\sigma_1') \right] _M^{K*}.
 \end{split}
\end{equation}
Let us now perform the integration over $\mathbf{r}_{A2}$, \idx{Two-nucleon transfer!second order DWBA}
\begin{equation}\label{eq18}
 \begin{split}
\sum_{\sigma_1, \sigma_2} &\int d\mathbf{r}_{A2} \left[ \psi ^{j_f} (\mathbf{r}_{A1},\sigma_1) \psi ^{j_f} (\mathbf{r}_{A2},\sigma_2) \right] _0^{0*} \left[ \psi ^{j_f} (\mathbf{r}_{A2},\sigma_2) \psi ^{j_i} (\mathbf{r}_{b1},\sigma_1) \right] _M^{K}\\
&=\sum_{\sigma_1, \sigma_2} (-1)^{1/2-\sigma_1+1/2-\sigma_2}\int d\mathbf{r}_{A2} \left[ \psi ^{j_f} (\mathbf{r}_{A1},-\sigma_1) \psi ^{j_f} (\mathbf{r}_{A2},-\sigma_2) \right] _0^{0} \left[ \psi ^{j_f} (\mathbf{r}_{A2},\sigma_2) \psi ^{j_i} (\mathbf{r}_{b1},\sigma_1) \right] _M^{K}\\
&=-\sum_{\sigma_1, \sigma_2} (-1)^{1/2-\sigma_1+1/2-\sigma_2}\int d\mathbf{r}_{A2} \left[ \psi ^{j_f} (\mathbf{r}_{A2},-\sigma_2) \psi ^{j_f} (\mathbf{r}_{A1},-\sigma_1) \right] _0^{0} \left[ \psi ^{j_f} (\mathbf{r}_{A2},\sigma_2) \psi ^{j_i} (\mathbf{r}_{b1},\sigma_1) \right] _M^{K}\\
&=-\bigl ( (j_f j_f)_0 (j_f j_i)_K |(j_f j_f)_0 (j_f j_i)_K \bigr )_K \sum_{\sigma_1, \sigma_2} (-1)^{1/2-\sigma_1+1/2-\sigma_2}\\
&\times \int d\mathbf{r}_{A2} \left[ \psi ^{j_f} (\mathbf{r}_{A2},-\sigma_2) \psi ^{j_f} (\mathbf{r}_{A2},\sigma_2) \right] _0^{0} \left[ \psi ^{j_f} (\mathbf{r}_{A1},-\sigma_1) \psi ^{j_i} (\mathbf{r}_{b1},\sigma_1) \right] _M^{K}\\
&=\frac{1}{2j_f+1}\sqrt{2j_f+1} \bigl ( (l_f \tfrac{1}{2})_{j_f} (l_i \tfrac{1}{2})_{j_i} |(l_f l_i)_K (\tfrac{1}{2} \tfrac{1}{2})_0 \bigr )_K \\
&\times u_{l_f}(r_{A1})u_{l_i}(r_{b1}) \left[ Y ^{l_f} (\hat r_{A1}) Y ^{l_i} (\hat r_{b1}) \right] _M^{K}
\sum_{\sigma_1}(-1)^{1/2-\sigma_1}\left[ \chi^{1/2} (-\sigma_1) \chi^{1/2} (\sigma_1) \right] _0^{0}\\
&=-\sqrt{\frac{2}{2j_f+1}} \bigl ( (l_f \tfrac{1}{2})_{j_f} (l_i \tfrac{1}{2})_{j_i} |(l_f l_i)_K (\tfrac{1}{2} \tfrac{1}{2})_0 \bigr )_K \left[ Y ^{l_f} (\hat r_{A1}) Y ^{l_i} (\hat r_{b1}) \right] _M^{K} u_{l_f}(r_{A1})u_{l_i}(r_{b1}),
 \end{split}
\end{equation}
where we have evaluated the $9j$--symbol \idx{Two-nucleon transfer!second order DWBA}
\begin{equation}\label{eq187}
    \bigl ( (j_f j_f)_0 (j_f j_i)_K |(j_f j_f)_0 (j_f j_i)_K \bigr )_K=\frac{1}{2j_f+1},
\end{equation}
as well as (\ref{eq16}).
We proceed in a similar way to evaluate the integral over $\mathbf{r}_{b1}'$,
\begin{equation}\label{eq19}
 \begin{split}
\sum_{\sigma_1', \sigma_2'} &\int d\mathbf{r}_{b1}' \left[ \psi ^{j_i} (\mathbf{r}_{b1}',\sigma_1') \psi ^{j_i} (\mathbf{r}_{b2}',\sigma_2') \right] _0^{0} \left[ \psi ^{j_f} (\mathbf{r}_{A2}',\sigma_2') \psi ^{j_i} (\mathbf{r}_{b1}',\sigma_1') \right] _M^{K*}\\
&=-(-1)^{K-M}\sum_{\sigma_1', \sigma_2'} \int d\mathbf{r}_{b1}'\left[ \psi ^{j_f} (\mathbf{r}_{A2}',-\sigma_2') \psi ^{j_i} (\mathbf{r}_{b1}',-\sigma_1') \right] _{-M}^{K}\\
&\times \left[  \psi ^{j_i} (\mathbf{r}_{b2}',\sigma_2') \psi ^{j_i} (\mathbf{r}_{b1}',\sigma_1')\right] _0^{0}
(-1)^{1/2-\sigma_1'+1/2-\sigma_2'}\\
&= -(-1)^{K-M} \bigl ( (j_f j_i)_K (j_i j_i)_0 |(j_f j_i)_K (j_i j_i)_0 \bigr )_K (-\sqrt{2 j_i+1}) \\
&\times \bigl ( (l_f \tfrac{1}{2})_{j_f} (l_i \tfrac{1}{2})_{j_i} |(l_f l_i)_K (\tfrac{1}{2} \tfrac{1}{2})_0 \bigr )_K (-\sqrt{2})
 u_{l_f}(r_{A2}')u_{l_i}(r_{b2}') \left[ Y ^{l_f} (\hat r_{A2}') Y ^{l_i} (\hat r_{b2}') \right] _{-M}^{K}\\
&= -\sqrt{\frac{2}{2j_i+1}}\bigl ( (l_f \tfrac{1}{2})_{j_f} (l_i \tfrac{1}{2})_{j_i} |(l_f l_i)_K (\tfrac{1}{2} \tfrac{1}{2})_0 \bigr )_K
\left[ Y ^{l_f} (\hat r_{A2}') Y ^{l_i} (\hat r_{b2}') \right] _{M}^{K*} u_{l_f}(r_{A2}')u_{l_i}(r_{b2}').\\
 \end{split}
\end{equation}
Setting the different elements together one obtains
\begin{equation}\label{eq23}
 \begin{split}
T_{succ}^{(2)}(\theta)=&\frac{4\mu_{Cc}(4\pi)^2 i}{\hbar^2 k_{Aa}k_{Bb}k_{Cc}}\frac{2}{\sqrt{(2j_i+1)(2j_f+1)}}\sum_{K,M}
\bigl ( (l_f \tfrac{1}{2})_{j_f} (l_i \tfrac{1}{2})_{j_i} |(l_f l_i)_K (\tfrac{1}{2} \tfrac{1}{2})_0 \bigr )_K ^2\\
& \times \sum_{l_c,l,\tilde l} e^{i(\sigma _i^l+\sigma _f^{\tilde l})} \sqrt{(2l_c+1)(2l+1)(2\tilde l+1)} \, i^{l-\tilde l}\\
& \times \int d^3 r_{Cc}d^3 r_{b1}d^3 r_{Cc}'d^3 r_{A2}' v(r_{b1})v(r_{c2}') u_{l_f}(r_{A1})u_{l_i}(r_{b1}) u_{l_f}(r_{A2}')u_{l_i}(r_{b2}')\\
& \times \left[ Y ^{l_f} (\hat r_{A2}') Y ^{l_i} (\hat r_{b2}') \right] _{M}^{K*}
\left[ Y ^{l_f} (\hat r_{A1}) Y ^{l_i} (\hat r_{b1}) \right] _{M}^{K}
\frac{F_l(r_{Aa}')F_{\tilde l}(r_{Bb}')f_{l_c}(k_{Cc},r_<)P_{l_c}(k_{Cc},r_>)}{r_{Aa}'r_{Bb}r_{Cc}r_{Cc}'}\\
& \times \left[ Y ^{\tilde l} (\hat r_{Bb}) Y ^{\tilde l} (\hat k_{Bb}) \right] _{0}^{0}
\left[ Y ^{ l} (\hat r_{Aa}') Y ^{l} (\hat k_{Aa}) \right] _{0}^{0} \left[ Y ^{ l_c} (\hat r_{Cc}) Y ^{l_c} (\hat r_{Cc}') \right] _{0}^{0}.
 \end{split}
\end{equation}
We now proceed to write this expression in a more compact way. \idx{Two-nucleon transfer!second order DWBA}
For this purpose one writes
\begin{equation}\label{eq24}
 \begin{split}
\left[ Y ^{\tilde l} (\hat r_{Bb}) Y ^{\tilde l} (\hat k_{Bb})  \right]_{0}^{0}&
\left[ Y ^{ l} (\hat r_{Aa}') Y ^{l} (\hat k_{Aa}) \right] _{0}^{0}=\\
& \bigl ( (l \, l)_0 (\tilde l \, \tilde l)_0 |(l \, \tilde l)_0 (l \, \tilde l)_0 \bigr )_0
\left[ Y ^{\tilde l} (\hat r_{Bb}) Y ^{ l} (\hat r_{Aa}') \right] _{0}^{0}
\left[ Y ^{ \tilde l} (\hat k_{Bb}) Y ^{l} (\hat k_{Aa}) \right] _{0}^{0}\\
& =\frac{\delta_{\tilde l \, l}}{2l+1}
\left[ Y ^{l} (\hat r_{Bb}) Y ^{ l} (\hat r_{Aa}') \right] _{0}^{0}
\left[ Y ^{l} (\hat k_{Bb}) Y ^{l} (\hat k_{Aa}) \right] _{0}^{0}.\\
 \end{split}
\end{equation}
Taking into account the relations \idx{Two-nucleon transfer!second order DWBA}
\begin{equation}\label{eq25}
\left[ Y ^{l} (\hat k_{Bb}) Y ^{l} (\hat k_{Aa}) \right] _{0}^{0}=\frac{(-1)^l}{\sqrt{4 \pi}} Y_0^l(\hat k_{Bb}) i^l,
\end{equation}
and
\begin{equation}\label{eq26}
 \begin{split}
\left[ Y ^{l} (\hat r_{Bb}) Y ^{ l} (\hat r_{Aa}') \right] _{0}^{0}&
\left[ Y ^{l_c} (\hat r_{Cc}) Y ^{l_c} (\hat r_{Cc}') \right] _{0}^{0}=\\
& \bigl ( (l \, l)_0 (l_c \,  l_c)_0 |(l \,l_c)_K (l \, l_c)_K \bigr )_0 \left\lbrace
\left[Y^{l} (\hat r_{Bb}) Y ^{l_c} (\hat r_{Cc}) \right]^{K}
\left[Y^{l} (\hat r_{Aa}') Y ^{l_c} (\hat r_{Cc}') \right]^{K}\right\rbrace _0^0\\
& = \sqrt{\frac{2K+1}{(2l+1)(2l_c+1)}}\\
& \times \sum_{M'} \frac{(-1)^{K+M'}}{\sqrt{2K+1}}
\left[Y^{l} (\hat r_{Bb}) Y ^{l_c} (\hat r_{Cc}) \right]^{K}_{-M'} \left[Y^{l} (\hat r_{Aa}') Y ^{l_c} (\hat r_{Cc}') \right]^{K}_{M'}\\
& = \sqrt{\frac{1}{(2l+1)(2l_c+1)}}\\
& \times \sum_{M'}
\left[Y^{l} (\hat r_{Bb}) Y ^{l_c} (\hat r_{Cc}) \right]^{K*}_{M'} \left[Y^{l} (\hat r_{Aa}') Y ^{l_c} (\hat r_{Cc}') \right]^{K}_{M'}.\\
 \end{split}
\end{equation}
It is of notice that the integrals
\begin{equation}\label{eq29}
 \begin{split}
\int d \hat r_{Cc}d \hat r_{b1} \left[Y^{l} (\hat r_{Bb}) Y ^{l_c} (\hat r_{Cc}) \right]^{K*}_{M}
\left[ Y ^{l_f} (\hat r_{A1}) Y ^{l_i} (\hat r_{b1}) \right] _{M}^{K},
 \end{split}
\end{equation}
and
\begin{equation}\label{eq30}
 \begin{split}
\int d \hat r_{Cc}'d \hat r_{A2}' \left[Y^{l} (\hat r_{Aa}') Y ^{l_c} (\hat r_{Cc}') \right]^{K}_{M}
\left[ Y ^{l_f} (\hat r_{A2}') Y ^{l_i} (\hat r_{b2}') \right] _{M}^{K*},
 \end{split}
\end{equation}
over the angular variables do not depend on $M$. Let us see why this is so with the help of (\ref{eq29}), \idx{Two-nucleon transfer!second order DWBA}
\begin{equation}\label{eq31}
 \begin{split}
\left[Y^{l}(\hat r_{Bb}) Y ^{l_c} (\hat r_{Cc}) \right]^{K*}_{M}&
\left[ Y ^{l_f}  (\hat r_{A1}) Y ^{l_i} (\hat r_{b1}) \right] _{M}^{K}=
 (-1)^{K-M}\left[Y^{l} (\hat r_{Bb}) Y ^{l_c} (\hat r_{Cc}) \right]^{K}_{-M} \\
& \times \left[ Y ^{l_f} (\hat r_{A1}) Y ^{l_i} (\hat r_{b1}) \right] _{M}^{K}=(-1)^{K-M} \sum_J \langle K\;K\;M\;-M|J\;0\rangle \\
& \times \left\lbrace \left[Y^{l} (\hat r_{Bb}) Y ^{l_c} (\hat r_{Cc}) \right]^{K}
\left[ Y ^{l_f} (\hat r_{A1}) Y ^{l_i} (\hat r_{b1}) \right]^{K} \right\rbrace ^J_0.
\end{split}
\end{equation}
After integration, only the term \idx{Two-nucleon transfer!second order DWBA}
\begin{equation}\label{eq32}
 \begin{split}
(-1)^{K-M} & \langle K\;K\;M\;-M|0\;0\rangle \left\lbrace \left[Y^{l} (\hat r_{Bb}) Y ^{l_c} (\hat r_{Cc}) \right]^{K}
\left[ Y ^{l_f} (\hat r_{A1}) Y ^{l_i} (\hat r_{b1}) \right]^{K} \right\rbrace ^0_0=.\\
&\frac{1}{\sqrt{2K+1}} \left\lbrace \left[Y^{l} (\hat r_{Bb}) Y ^{l_c} (\hat r_{Cc}) \right]^{K}
\left[ Y ^{l_f} (\hat r_{A1}) Y ^{l_i} (\hat r_{b1}) \right]^{K} \right\rbrace ^0_0
\end{split}
\end{equation}
corresponding to $J=0$ survives, which is indeed independent of $M$.
We can thus omit the sum over $M$ in (\ref{eq23}) and multiply by $(2K+1)$, obtaining
\begin{equation}\label{eq27}
 \begin{split}
T_{succ}^{(2)}(\theta)=&\frac{64\mu_{Cc}(\pi)^{3/2} i}{\hbar^2 k_{Aa}k_{Bb}k_{Cc}}\frac{i^{-l}}{\sqrt{(2j_i+1)(2j_f+1)}}\\
&\times \sum_{K}(2K+1)
\bigl ( (l_f \tfrac{1}{2})_{j_f} (l_i \tfrac{1}{2})_{j_i} |(l_f l_i)_K (\tfrac{1}{2} \tfrac{1}{2})_0 \bigr )_K ^2\\
&\times \sum_{l_c,l}\frac{e^{i(\sigma _i^l+\sigma _f^{l})}}{\sqrt{(2l+1)}} Y_0^l (\hat k_{Bb})S_{K,l,l_c},
 \end{split}
\end{equation}
where \idx{Two-nucleon transfer!second order DWBA}
\begin{equation}\label{eq121}
 \begin{split}
S_{K,l,l_c}=&\int d^3 r_{Cc}d^3 r_{b1} v(r_{b1}) u_{l_f}(r_{A1})u_{l_i}(r_{b1})\frac{s_{K,l,l_c}(r_{Cc})}{r_{Cc}}\frac{F_l(r_{Bb})}{r_{Bb}} \\
& \times \left[ Y ^{l_f} (\hat r_{A1}) Y ^{l_i} (\hat r_{b1}) \right] _{M}^{K}
\left[Y ^{l_c} (\hat r_{Cc}) Y^{l} (\hat r_{Bb})  \right]^{K*}_{M},
 \end{split}
\end{equation}
and
\begin{equation}\label{eq122}
 \begin{split}
s_{K,l,l_c}(r_{Cc})=&\int_{r_{Cc}fixed} d^3 r_{Cc}'d^3 r_{A2}' v(r_{c2}') u_{l_f}(r_{A2}')u_{l_i}(r_{b2}') \frac{F_l(r_{Aa}')}{r_{Aa}'}\frac{f_{l_c}(k_{Cc},r_<)P_{l_c}(k_{Cc},r_>)}{r_{Cc}'}\\
& \times \left[ Y ^{l_f} (\hat r_{A2}') Y ^{l_i} (\hat r_{b2}') \right] _{M}^{K*}
\left[Y ^{l_c} (\hat r_{Cc}') Y^{l} (\hat r_{Aa}')  \right]^{K}_{M}.
 \end{split}
\end{equation}
It can be shown that the integrand in (\ref{eq121}) is independent of $M$. Consequently, one can sum over $M$ and divide by $(2K+1)$, to get
\begin{equation}\label{eq123}
 \begin{split}
 \frac{1}{2K+1}v(r_{b1})& u_{l_f}(r_{A1})u_{l_i}(r_{b1})\frac{s_{K,l,l_c}(r_{Cc})}{r_{Cc}}\frac{F_l(r_{Bb})}{r_{Bb}}\\
 &\times \sum_M \left[ Y ^{l_f} (\hat r_{A1}) Y ^{l_i} (\hat r_{b1}) \right] _{M}^{K}
\left[Y ^{l_c} (\hat r_{Cc}) Y^{l} (\hat r_{Bb})  \right]^{K*}_{M} .
 \end{split}
\end{equation}

This integrand is rotationally invariant (it is proportional to a $T_M^L$ spherical tensor with $L=0$, $M=0$), so one can  evaluate it in the ``standard'' configuration  in which $\mathbf{r}_{Cc}$ is directed along the $z$--axis and multiply by $8\pi^2$ (see \cite{Bayman:82}), obtaining the final expression for $S_{K,l,l_c}$:
\begin{equation}\label{eq124}
 \begin{split}
S_{K,l,l_c}=&\frac{4\pi^{3/2}\sqrt{2l_c+1}}{2K+1}i^{-l_c}\\
&\times\int r_{Cc}^2 \, d r_{Cc}\,r_{b1}^2\, d r_{b1} \,\sin\theta\, d\theta \, v(r_{b1}) u_{l_f}(r_{A1})u_{l_i}(r_{b1})\\
& \times \frac{s_{K,l,l_c}(r_{Cc})}{r_{Cc}}\frac{F_l(r_{Bb})}{r_{Bb}}\\
&\times\sum_M \langle l_c \;0\;l\;M|K\;M\rangle \left[ Y ^{l_f} (\hat r_{A1}) Y ^{l_i} (\theta+\pi,0) \right] _{M}^{K}
 Y^{l*}_M (\hat r_{Bb}).
 \end{split}
\end{equation}
Similarly, one has \idx{Two-nucleon transfer!second order DWBA}
\begin{equation}\label{eq125}
 \begin{split}
s_{K,l,l_c}(r_{Cc})=&\frac{4\pi^{3/2}\sqrt{2l_c+1}}{2K+1}i^{l_c}\\
&\times\int r_{Cc}^{'2} \, d r'_{Cc}\,r_{A2}^{'2}\, d r'_{A2} \,\sin\theta'\, d\theta' \, v(r_{c2}') u_{l_f}(r_{A2}')u_{l_i}(r_{b2}')  \\
& \times \frac{F_l(r_{Aa}')}{r_{Aa}'}\frac{f_{l_c}(k_{Cc},r_<)P_{l_c}(k_{Cc},r_>)}{r_{Cc}'}\\
&\times\sum_M \langle l_c \;0\;l\;M|K\;M\rangle \left[ Y ^{l_f} (\hat r_{A2}') Y ^{l_i} (\hat r_{b2}') \right] _{M}^{K*}
 Y^{l}_M (\hat r_{Aa}').
 \end{split}
\end{equation}
Introducing  the further approximations $\mathbf{r}_{A1}\approx\mathbf{r}_{C1}$ and $\mathbf{r}_{b2}\approx\mathbf{r}_{c2}$, one obtains the final expression
\begin{equation}\label{eq126}
 \begin{split}
T_{succ}^{(2)}(\theta)=&\frac{1024\mu_{Cc}\pi^{9/2} i}{\hbar^2 k_{Aa}k_{Bb}k_{Cc}}\frac{1}{\sqrt{(2j_i+1)(2j_f+1)}}\\
&\times \sum_{K}\frac{1}{2K+1}
\bigl ( (l_f \tfrac{1}{2})_{j_f} (l_i \tfrac{1}{2})_{j_i} |(l_f l_i)_K (\tfrac{1}{2} \tfrac{1}{2})_0 \bigr )_K ^2\\
&\times \sum_{l_c,l}e^{i(\sigma _i^l+\sigma _f^{l})}\frac{(2l_c+1)}{\sqrt{2l+1}} Y_0^l(\hat k_{Bb})S_{K,l,l_c},
 \end{split}
\end{equation}
with
\begin{equation}\label{eq127}
 \begin{split}
S_{K,l,l_c}=&\int r_{Cc}^2 \, d r_{Cc}\,r_{b1}^2\, d r_{b1} \,\sin\theta\, d\theta \, v(r_{b1}) u_{l_f}(r_{C1})u_{l_i}(r_{b1})\\
& \times \frac{s_{K,l,l_c}(r_{Cc})}{r_{Cc}}\frac{F_l(r_{Bb})}{r_{Bb}}\\
&\times\sum_M \langle l_c \;0\;l\;M|K\;M\rangle \left[ Y ^{l_f} (\hat r_{C1}) Y ^{l_i} (\theta+\pi,0) \right] _{M}^{K}
 Y^{l*}_M (\hat r_{Bb}),
 \end{split}
\end{equation}
and
\begin{equation}\label{eq128}
 \begin{split}
s_{K,l,l_c}(r_{Cc})=&\int r_{Cc}^{'2} \, d r'_{Cc}\,r_{A2}^{'2}\, d r'_{A2} \,\sin\theta'\, d\theta' \, v(r_{c2}') u_{l_f}(r_{A2}')u_{l_i}(r_{c2}')  \\
& \times \frac{F_l(r_{Aa}')}{r_{Aa}'}\frac{f_{l_c}(k_{Cc},r_<)P_{l_c}(k_{Cc},r_>)}{r_{Cc}'}\\
&\times\sum_M \langle l_c \;0\;l\;M|K\;M\rangle \left[ Y ^{l_f} (\hat r_{A2}') Y ^{l_i} (\hat r_{c2}') \right] _{M}^{K*}
 Y^{l}_M (\hat r_{Aa}').
 \end{split}
\end{equation}
\subsection{Coordinates for the successive transfer}
\idx{Two-nucleon transfer!second order DWBA}
In the standard configuration in which the integrals (\ref{eq127}) and (\ref{eq128}) are to be evaluated, we have
\begin{equation}\label{eq148}
\mathbf{r}_{Cc}=r_{Cc}\, \hat {\mathbf{z}} \,, \qquad \mathbf{r}_{b1}=r_{b1}(-\cos\theta \, \hat {\mathbf{z}}-\sin\theta \, \hat {\mathbf{x}}).
\end{equation}
Now,
\begin{equation}\label{eq149}
\begin{split}
\mathbf{r}_{C1}&=\mathbf{r}_{Cc}+\mathbf{r}_{c1}=\mathbf{r}_{Cc}+\frac{m_b}{m_b+1}\mathbf{r}_{b1}\\
&=\left(r_{Cc}-\frac{m_b}{m_b+1}r_{b1}\cos\theta \right)\hat {\mathbf{z}}-\frac{m_b}{m_b+1}r_{b1}\sin\theta \hat {\mathbf{x}},
\end{split}
\end{equation}
and \idx{Two-nucleon transfer!second order DWBA}
\begin{equation}\label{eq150}
\mathbf{r}_{Bb}=\mathbf{r}_{BC}+\mathbf{r}_{Cb}=-\frac{1}{m_B}\mathbf{r}_{C1}+\mathbf{r}_{Cb}.
\end{equation}
Substituting the relation
\begin{equation}\label{eq151}
\mathbf{r}_{Cb}=\mathbf{r}_{Cc}+\mathbf{r}_{cb}=\mathbf{r}_{Cc}-\frac{1}{m_b+1}\mathbf{r}_{b1},
\end{equation}
in (\ref{eq150}) one gets
\begin{equation}\label{eq152}
\mathbf{r}_{Bb}=\left(\frac{m_B-1}{m_B}r_{Cc}+\frac{m_b+m_B}{m_B(m_b+1)}r_{b1}\cos\theta\right)\hat {\mathbf{z}}+\frac{m_b+m_B}{m_B(m_b+1)}r_{b1}\sin\theta\hat {\mathbf{x}}.
\end{equation}
The primed variables are arranged in a similar fashion,
\begin{equation}\label{eq153}
\mathbf{r}'_{Cc}=r'_{Cc}\, \hat {\mathbf{z}} \,, \qquad \mathbf{r}'_{A2}=r'_{A2}(-\cos\theta' \, \hat {\mathbf{z}}-\sin\theta' \, \hat {\mathbf{x}}).
\end{equation}
Thus,
\begin{equation}\label{eq154}
\mathbf{r}'_{c2}=\left(-r'_{Cc}-\frac{m_A}{m_A+1}r'_{A2}\cos\theta' \right)\hat {\mathbf{z}}-\frac{m_A}{m_A+1}r'_{A2}\sin\theta' \hat {\mathbf{x}},
\end{equation}
and
\begin{equation}\label{eq7_2_144}
\mathbf{r}'_{Aa}=\left(\frac{m_a-1}{m_a}r'_{Cc}-\frac{m_A+m_a}{m_a(m_A+1)}r'_{A2}\cos\theta'\right)\hat {\mathbf{z}}-\frac{m_A+m_a}{m_a(m_A+1)}r'_{A2}\sin\theta'\hat {\mathbf{x}}.
\end{equation}
\subsection{Simplifying the vector coupling}
We will now turn our attention to the vector--coupled quantities in (\ref{eq127}) and (\ref{eq128}),
\begin{equation}\label{eq155}
\sum_M \langle l_c \;0\;l\;M|K\;M\rangle \left[ Y ^{l_f} (\hat r_{C1}) Y ^{l_i} (\theta+\pi,0) \right] _{M}^{K}
 Y^{l*}_M (\hat r_{Bb}),
\end{equation}
and
\begin{equation}\label{eq156}
\sum_M \langle l_c \;0\;l\;M|K\;M\rangle \left[ Y ^{l_f} (\hat r_{A2}') Y ^{l_i} (\hat r_{c2}') \right] _{M}^{K*}
 Y^{l}_M (\hat r_{Aa}').
\end{equation}
We can express them both as
\begin{equation}\label{eq157}
\sum_M  f(M),
\end{equation}
where e.g. in the case of (\ref{eq155}), one has
\begin{equation}\label{eq158}
f(M)=\langle l_c \;0\;l\;M|K\;M\rangle \left[ Y ^{l_f} (\hat r_{C1}) Y ^{l_i} (\theta+\pi,0) \right] _{M}^{K}
 Y^{l*}_M (\hat r_{Bb}).
\end{equation}
\idx{Two-nucleon transfer!second order DWBA}
Note that all the vectors that come into play in the above expressions are in the $(x,z)$--plane. Consequently, the azimuthal angle $\phi$ is always equal to zero. Under these circumstances and for time--reversed phases, ($Y^{L*}_M(\theta,0)=(-1)^LY^{L}_M(\theta,0)$) one has
\begin{equation}\label{eq164}
f(-M)=(-1)^{l_c+l_f+l_i+l}f(M).
\end{equation}
Consequently, \idx{Two-nucleon transfer!second order DWBA}
\begin{equation}\label{eq166}
\begin{split}
\sum_M \langle l_c \;0\;l&\;M|K\;M\rangle f(M)=\langle l_c \;0\;l\;0|K\;0\rangle f(0)\\
&+\sum_{M>0}\langle l_c \;0\;l\;M|K\;M\rangle f(M)\Bigl(1+(-1)^{l_c+l+l_i+l_f}\Bigr).
\end{split}
\end{equation}
Consequently, in the case in which $l_c+l+l_i+l_f$ is odd, we have only  to evaluate the $M=0$ contribution. This consideration is useful to restrict the number of numerical operations needed to calculate the transition amplitude.
\subsection{Non-orthogonality term}\label{S6.2.9}
We write the non-orthogonality contribution to the transition amplitude (see \cite{Bayman:82}): \idx{Two-nucleon transfer!second order DWBA}
\begin{equation}\label{129}
 \begin{split}
T_{NO}^{(2)}(\theta)=&2\sum_{\substack{\sigma_1 \sigma_2 \\ \sigma_1' \sigma_2' \\ KM}}\int d^3r_{Cc}d^3r_{b1}d^3r_{A2}
d^3r_{b1}'d^3r_{A2}' \,\chi^{(-)*}(\mathbf{k}_{Bb},\mathbf{r}_{Bb})\\
 & \times \left[ \psi ^{j_f} (\mathbf{r}_{A1},\sigma_1) \psi ^{j_f} (\mathbf{r}_{A2},\sigma_2) \right] _0^{0*} v(r_{b1})
\left[ \psi ^{j_f} (\mathbf{r}_{A2},\sigma_2) \psi ^{j_i} (\mathbf{r}_{b1},\sigma_1) \right] _M^{K}\\
& \times
\left[ \psi ^{j_f} (\mathbf{r}_{A2}',\sigma_2') \psi ^{j_i} (\mathbf{r}_{b1}',\sigma_1') \right] _M^{K*} \left[ \psi ^{j_i} (\mathbf{r}_{b1}',\sigma_1') \psi ^{j_i} (\mathbf{r}_{b2}',\sigma_2') \right] _0^{0}
\chi^{(+)}( \mathbf{r}_{Aa}').
 \end{split}
\end{equation}
This expression is equivalent to (\ref{eq1}) if we make the replacement
\begin{equation}\label{eq136}
    \frac{2\mu_{Cc}}{\hbar^2}G(\mathbf{r}_{Cc},\mathbf{r}'_{Cc})v(r_{A2}')\rightarrow \delta(\mathbf{r}_{Cc}-\mathbf{r}'_{Cc}).
\end{equation}
Looking at the partial--wave expansions of $G(\mathbf{r}_{Cc},\mathbf{r}'_{Cc})$ and $\delta(\mathbf{r}_{Cc}-\mathbf{r}'_{Cc})$ (see App. \ref{C7AppD}), we find that we can use the above expressions for the successive transfer with the replacement
\begin{equation}\label{eq137}
    i\frac{2\mu_{Cc}}{\hbar^2}\frac{f_{l_c}(k_{Cc},r_<)P_{l_c}(k_{Cc},r_>)}{k_{Cc}}\rightarrow \delta(r_{Cc}-r'_{Cc}).
\end{equation}
We thus have
\begin{equation}\label{eq138}
 \begin{split}
T_{2NT}^{NO}=&\frac{512\pi^{9/2}}{ k_{Aa}k_{Bb}}\frac{1}{\sqrt{(2j_i+1)(2j_f+1)}}\\
&\times \sum_{K}
\bigl ( (l_f \tfrac{1}{2})_{j_f} (l_i \tfrac{1}{2})_{j_i} |(l_f l_i)_K (\tfrac{1}{2} \tfrac{1}{2})_0 \bigr )_K ^2\\
&\times \sum_{l_c,l}e^{i(\sigma _i^l+\sigma _f^{l})}\frac{(2l_c+1)}{\sqrt{2l+1}} Y_0^l(\hat k_{Bb})S_{K,l,l_c},
 \end{split}
\end{equation}
with
\begin{equation}\label{eq139}
 \begin{split}
S_{K,l,l_c}=&\int r_{Cc}^2 \, d r_{Cc}\,r_{b1}^2\, d r_{b1} \,\sin\theta\, d\theta \, v(r_{b1}) u_{l_f}(r_{C1})u_{l_i}(r_{b1})\\
& \times \frac{s_{K,l,l_c}(r_{Cc})}{r_{Cc}}\frac{F_l(r_{Bb})}{r_{Bb}}\\
&\times\sum_M \langle l_c \;0\;l\;M|K\;M\rangle \left[ Y ^{l_f} (\hat r_{C1}) Y ^{l_i} (\theta+\pi,0) \right] _{M}^{K}
 Y^{l*}_M (\hat r_{Bb}),
 \end{split}
\end{equation}
and
\begin{equation}\label{eq140}
 \begin{split}
s_{K,l,l_c}(r_{Cc})=&r_{Cc}\int  d r'_{A2}\, r_{A2}^{'2}\,\sin\theta'\, d\theta' \, u_{l_f}(r_{A2}')u_{l_i}(r_{c2}') \frac{F_l(r_{Aa}')}{r_{Aa}'} \\
&\times\sum_M \langle l_c \;0\;l\;M|K\;M\rangle \left[ Y ^{l_f} (\hat r_{A2}') Y ^{l_i} (\hat r_{c2}') \right] _{M}^{K*}
 Y^{l}_M (\hat r_{Aa}').
 \end{split}
\end{equation}
\subsection{General orbital momentum transfer}\label{C7S2S10}
We will now examine the case in which the two transferred nucleons carry an angular momentum $\Lambda$ different from 0. Let us assume that two nucleons coupled to angular momentum $\Lambda$ in the initial nucleus $a$ are transferred into a final state of zero angular momentum in nucleus $B$. The transition amplitude is given by the \idx{Two-nucleon transfer!second order DWBA}
 integral
 \begin{equation}\label{eq227}
 \begin{split}
2\sum_{\sigma_1 \sigma_2} & \int d\mathbf{r}_{cC} d\mathbf{r}_{A2}d\mathbf{r}_{b1} \chi^{(-)*}(\mathbf{r}_{bB}) \left[ \psi^{j_f} (\mathbf{r}_{A1}, \sigma_1) \psi^{j_f} (\mathbf{r}_{A2}, \sigma_2) \right] _{0}^{0*}\\
& \times v(r_{b1})\Psi^{(+)}(\mathbf{r}_{aA},\mathbf{r}_{b1},\mathbf{r}_{b2},\sigma_1,\sigma_2).
\end{split}
\end{equation}
If we neglect core excitations, the above expression is exact as long as \\\mbox{$\Psi^{(+)}(\mathbf{r}_{aA},\mathbf{r}_{b1},\mathbf{r}_{b2},\sigma_1,\sigma_2)$} is the exact wavefunction. We can instead obtain an approximation for the transfer amplitude using
  \begin{equation}\label{eq228}
 \begin{split}
\Psi^{(+)}(\mathbf{r}_{aA},\mathbf{r}_{b1},&\mathbf{r}_{b2},\sigma_1,\sigma_2)\approx \chi^{(+)}(\mathbf{r}_{aA}) \left[ \psi^{j_{i1}} (\mathbf{r}_{b1}, \sigma_1) \psi^{j_{i2}} (\mathbf{r}_{b2}, \sigma_2) \right]_{\mu}^{\Lambda}\\
 &+\sum_{K,M}\mathcal U_{K,M}(\mathbf{r}_{cC})\left[ \psi^{j_f} (\mathbf{r}_{A2}, \sigma_2) \psi^{j_{i1}} (\mathbf{r}_{b1}, \sigma_1) \right] _{M}^{K}
\end{split}
\end{equation}
as an approximation for the incoming state. The first term of (\ref{eq228}) gives rise to the simultaneous amplitude, while from second one leads to both the successive and the non-orthogonality contributions.
To extract the amplitude $\mathcal U_{K,M}(\mathbf{r}_{cC})$, we define $f_{KM}(\mathbf{r}_{cC})$ as the scalar product
\begin{equation}\label{eq229}
f_{KM}(\mathbf{r}_{cC})=\left \langle \left[ \psi^{j_f} (\mathbf{r}_{A2}, \sigma_2) \psi^{j_{i1}} (\mathbf{r}_{b1}, \sigma_1) \right] _{M}^{K} \Big | \Psi^{(+)}(\mathbf{r}_{aA},\mathbf{r}_{b1},\mathbf{r}_{b2},\sigma_1,\sigma_2) \right \rangle
\end{equation}
for fixed $\mathbf{r}_{cC}$, which can be seen to obey the equation
  \begin{equation}\label{eq230}
 \begin{split}
\left(\frac{\hbar^2}{2\mu_{cC}}k_{cC}^2\right. & \left.+\frac{\hbar^2}{2\mu_{cC}}\nabla^2_{r_{cC}}-U(r_{cC})\right)f_{KM}(\mathbf{r}_{cC})\\
&=\left \langle \left[ \psi^{j_f} (\mathbf{r}_{A2}, \sigma_2) \psi^{j_{i1}} (\mathbf{r}_{b1}, \sigma_1) \right] _{M}^{K} \Big | v(r_{c2}) \Big| \Psi^{(+)}(\mathbf{r}_{aA},\mathbf{r}_{b1},\mathbf{r}_{b2},\sigma_1,\sigma_2) \right \rangle.
\end{split}
\end{equation}
The solution can be written in terms of the Green function $G(\mathbf{r}_{cC},\mathbf{r}'_{cC})$ defined by\idx{Two-nucleon transfer!second order DWBA}
  \begin{equation}\label{eq231}
\left(\frac{\hbar^2}{2\mu_{cC}}k_{cC}^2\right.  \left.+\frac{\hbar^2}{2\mu_{cC}}\nabla^2_{r_{cC}}-U(r_{cC})\right)
G(\mathbf{r}_{cC},\mathbf{r}'_{cC})=\frac{\hbar^2}{2\mu_{cC}}\delta(\mathbf{r}_{cC}-\mathbf{r}'_{cC}).
\end{equation}
Thus,
\begin{equation}\label{eq232}
 \begin{split}
f_{KM}(\mathbf{r}_{cC})&=\frac{2\mu_{cC}}{\hbar^2}\int d\mathbf{r}'_{cC} G(\mathbf{r}_{cC},\mathbf{r}'_{cC})\left \langle \left[ \psi^{j_f} (\mathbf{r}'_{A2}, \sigma'_2) \psi^{j_{i1}} (\mathbf{r}'_{b1}, \sigma'_1) \right] _{M}^{K} \Big | v(r_{C2}) \Big| \Psi^{(+)}(\mathbf{r}'_{aA},\mathbf{r}'_{b1},\mathbf{r}'_{b2},\sigma'_1,\sigma'_2) \right \rangle\\
&\approx \frac{2\mu_{cC}}{\hbar^2}\sum_{\sigma'_1 \sigma'_2} \int d\mathbf{r}'_{cC}d\mathbf{r}'_{A2}d\mathbf{r}'_{b1}G(\mathbf{r}_{cC},\mathbf{r}'_{cC}) \left[ \psi^{j_f} (\mathbf{r}'_{A2}, \sigma'_2) \psi^{j_{i1}} (\mathbf{r}'_{b1}, \sigma'_1) \right] _{M}^{K*}\\
&\times v(r'_{c2}) \chi^{(+)}(\mathbf{r}'_{aA})\left[ \psi^{j_{i1}} (\mathbf{r}'_{b1}, \sigma'_1) \psi^{j_{i2}} (\mathbf{r}'_{b2}, \sigma'_2) \right]_{\mu}^{\Lambda}=\mathcal U_{K,M}(\mathbf{r}_{cC})\\
&+\left \langle \left[ \psi^{j_f} (\mathbf{r}'_{A2}, \sigma_2) \psi^{j_{i1}} (\mathbf{r}'_{b1}, \sigma_1) \right] _{M}^{K} \Big | \chi^{(+)}(\mathbf{r}'_{aA})\left[ \psi^{j_{i1}} (\mathbf{r}'_{b1}, \sigma'_1) \psi^{j_{i2}} (\mathbf{r}'_{b2}, \sigma'_2) \right]_{\mu}^{\Lambda} \right \rangle.
\end{split}
\end{equation}
Therefore
\begin{equation}\label{eq233}
 \begin{split}
\mathcal U_{K,M}(\mathbf{r}_{cC})&=\frac{2\mu_{cC}}{\hbar^2}\sum_{\sigma'_1 \sigma'_2} \int d\mathbf{r}'_{cC}d\mathbf{r}'_{A2}d\mathbf{r}'_{b1}G(\mathbf{r}_{cC},\mathbf{r}'_{cC}) \left[ \psi^{j_f} (\mathbf{r}'_{A2}, \sigma'_2) \psi^{j_{i1}} (\mathbf{r}'_{b1}, \sigma'_1) \right] _{M}^{K*}\\
&\times v(r'_{c2}) \chi^{(+)}(\mathbf{r}'_{aA})\left[ \psi^{j_{i1}} (\mathbf{r}'_{b1}, \sigma'_1) \psi^{j_{i2}} (\mathbf{r}'_{b2}, \sigma'_2) \right]_{\mu}^{\Lambda}\\
&-\left \langle \left[ \psi^{j_f} (\mathbf{r}'_{A2}, \sigma_2) \psi^{j_{i1}} (\mathbf{r}'_{b1}, \sigma_1) \right] _{M}^{K} \Big | \chi^{(+)}(\mathbf{r}'_{aA})\left[ \psi^{j_{i1}} (\mathbf{r}'_{b1}, \sigma'_1) \psi^{j_{i2}} (\mathbf{r}'_{b2}, \sigma'_2) \right]_{\mu}^{\Lambda} \right \rangle.
\end{split}
\end{equation}
When we substitute $\mathcal U_{K,M}(\mathbf{r}_{cC})$ into (\ref{eq228}) and (\ref{eq227}), the first term gives rise to the successive amplitude for the two--particle transfer, while the second term is responsible for the non--orthogonal contribution. \idx{Two-nucleon transfer!second order DWBA}
\subsubsection{Successive transfer contribution}
We need to evaluate the integral
\begin{equation}\label{eq210}
 \begin{split}
T_{succ}^{(2)}(\theta;\mu)=&\frac{4\mu_{cC}}{\hbar^2}\sum_{\sigma_1 \sigma_2}\sum_{K M} \int d\mathbf{r}_{cC} d\mathbf{r}_{A2}d\mathbf{r}_{b1} d\mathbf{r}'_{cC} d\mathbf{r}'_{A2}d\mathbf{r}'_{b1}  \left[ \psi^{j_f} (\mathbf{r}_{A1}, \sigma_1) \psi^{j_f} (\mathbf{r}_{A2}, \sigma_2) \right] _{0}^{0*} \\
&\times \chi^{(-)*}(\mathbf{r}_{bB})G(\mathbf{r}_{cC},\mathbf{r}'_{cC})
\left[ \psi^{j_f} (\mathbf{r}'_{A2}, \sigma'_2) \psi^{j_{i1}} (\mathbf{r}'_{b1}, \sigma'_1) \right] _{M}^{K*}\chi^{(+)}(\mathbf{r}'_{aA})v(r_{c2}')v(r_{b1})\\
&\times \left[ \psi^{j_{i1}} (\mathbf{r}'_{b1}, \sigma'_1) \psi^{j_{i2}} (\mathbf{r}'_{b2}, \sigma'_2) \right]_{\mu}^{\Lambda}
 \left[ \psi^{j_{f}} (\mathbf{r}_{A2}, \sigma_2) \psi^{j_{i1}} (\mathbf{r}_{b1}, \sigma_1) \right]_{M}^{K},
 \end{split}
\end{equation}
where we must substitute the Green function and the distorted waves by their partial wave expansions (see App. \ref{C7AppE}). \idx{Two-nucleon transfer!second order DWBA}
The integral over $\mathbf{r}'_{b1}$ is:
\begin{multline}\label{eq211}
\sum_{\sigma'_1} \int  d\mathbf{r}'_{b1}
\left[ \psi^{j_f} (\mathbf{r}'_{A2}, \sigma'_2) \psi^{j_{i1}} (\mathbf{r}'_{b1}, \sigma'_1) \right] _{M}^{K*}
\left[ \psi^{j_{i1}} (\mathbf{r}'_{b1}, \sigma'_1) \psi^{j_{i2}} (\mathbf{r}'_{b2}, \sigma'_2) \right]_{\mu}^{\Lambda}\\
=\sum_{\sigma'_1} \int  d\mathbf{r}'_{b1}(-1)^{-M+j_f+j_{i1}-\sigma_1-\sigma_2}\left[\psi^{j_{i1}} (\mathbf{r}'_{b1}, -\sigma'_1)\psi^{j_f} (\mathbf{r}'_{A2}, -\sigma'_2) \right] _{-M}^{K}\left[ \psi^{j_{i1}} (\mathbf{r}'_{b1}, \sigma'_1) \psi^{j_{i2}} (\mathbf{r}'_{b2}, \sigma'_2) \right]_{\mu}^{\Lambda}\\
=\sum_{\sigma'_1} \int d\mathbf{r}'_{b1}(-1)^{-M+j_f+j_{i1}-\sigma_1-\sigma_2}\sum_{P} \langle K \;\Lambda\;-M\;\mu|P\;\mu-M\rangle \bigl ( (j_{i1} j_f)_K (j_{i1} j_{i2})_\Lambda |(j_{i1} j_{i1})_0 (j_f j_{i2})_P \bigr )_P\\
\times  \left[ \psi^{j_{i1}} (\mathbf{r}'_{b1}, -\sigma'_1) \psi^{j_{i1}} (\mathbf{r}'_{b1}, \sigma'_1)  \right] ^{0}_0 \left[ \psi^{j_{f}} (\mathbf{r}'_{A2}, -\sigma'_2) \psi^{j_{i2}} (\mathbf{r}'_{b2}, \sigma'_2)  \right] ^{P}_{\mu-M}\\
=(-1)^{-M+j_f+j_{i1}}\sqrt{2j_{i1}+1}\, u_{l_f}(r_{A2})u_{l_{i2}}(r'_{b2})\sum_{P} \langle K \;\Lambda\;-M\;\mu|P\;\mu-M\rangle\\
\times \bigl ((j_{i1} j_f)_K (j_{i1} j_{i2})_\Lambda |(j_{i1} j_{i1})_0 (j_f j_{i2})_P \bigr )_P \bigl ( (l_f \tfrac{1}{2})_{j_f} (l_{i2} \tfrac{1}{2})_{j_{i2}} |(l_f l_{i2})_P\; (\tfrac{1}{2} \tfrac{1}{2})_0 \bigr )_P\\
\times \left[ Y^{l_f} (\hat{\mathbf{r}}'_{A2}) Y^{l_{i2}} (\hat{\mathbf{r}}'_{b2})  \right] ^{P}_{\mu-M}u_{l_{f}}(r_{A2})u_{l_{i2}}(r_{b2}).
\end{multline}
Integrating over $\mathbf{r}_{A2}$ (see (\ref{eq18})) leads to,
\begin{equation}\label{eq212}
\begin{split}
\sum_{\sigma_2} &\int  d\mathbf{r}_{A2}
\left[ \psi^{j_f} (\mathbf{r}_{A1}, \sigma_1) \psi^{j_f} (\mathbf{r}_{A2}, \sigma_2) \right] _{0}^{0*}
\left[ \psi^{j_f} (\mathbf{r}_{A2}, \sigma_2) \psi^{j_{i1}} (\mathbf{r}_{b1}, \sigma_1) \right]_{M}^{K}\\
&=-\sqrt{\frac{2}{2j_f+1}} \bigl ( (l_f \tfrac{1}{2})_{j_f} (l_{i1} \tfrac{1}{2})_{j_{i1}} |(l_f l_{i1})_K\; (\tfrac{1}{2} \tfrac{1}{2})_0 \bigr )_K \left[ Y^{l_f} (\hat{\mathbf{r}}_{A1}) Y^{l_{i1}} (\hat{\mathbf{r}}_{b1})  \right] ^{K}_{M}u_{l_{f}}(r_{A1})u_{l_{i1}}(r_{b1}).
\end{split}
\end{equation}
Let us examine the term
\begin{equation}\label{eq213}
\begin{split}
\sum_M (-1)^M\;\langle K \;\Lambda\;-M\;\mu|P\;\mu-M\rangle  \left[ Y^{l_f} (\hat{\mathbf{r}}_{A1}) Y^{l_{i1}} (\hat{\mathbf{r}}_{b1})  \right] ^{K}_{M} \left[ Y^{l_f} (\hat{\mathbf{r}}'_{A2}) Y^{l_{i2}} (\hat{\mathbf{r}}'_{b2})  \right] ^{P}_{\mu-M}.
\end{split}
\end{equation}
Making use of the relation
\begin{equation}\label{eq214}
\langle l_1 \;l_2\;m_1\;m_2|L\;M_L\rangle=(-1)^{l_2-m_2}\sqrt{\frac{2L+1}{2l_1+1}}\langle L \;l_2\;-M_L\;m_2|l_1\;-m_1\rangle,
\end{equation}
the expression (\ref{eq214}) is equivalent to,
\begin{equation}\label{eq215}
(-1)^K \sqrt{\frac{2P+1}{2\Lambda+1}}\left\lbrace \left[Y^{l_f} (\hat{\mathbf{r}}'_{A2}) Y^{l_{i2}} (\hat{\mathbf{r}}'_{b2}) \right]^{P}
\left[ Y ^{l_f} (\hat{\mathbf{r}}_{A1}) Y ^{l_{i1}} (\hat{\mathbf{r}}_{b1}) \right]^{K} \right\rbrace ^\Lambda_\mu.
\end{equation}
We now recouple the term
\begin{equation}\label{eq216}
\left[ Y ^{l_a} (\hat{\mathbf{r}}'_{aA}) Y ^{l_a} (\hat{\mathbf{k}}_{aA}) \right]^{0}_0 \left[ Y ^{l_b} (\hat{\mathbf{r}}_{bB}) Y ^{l_b} (\hat{\mathbf{k}}_{bB}) \right]^{0}_0,
\end{equation}
 arising from the partial wave expansion of the incoming and outgoing distorted waves to have, \idx{Two-nucleon transfer!second order DWBA}
\begin{equation}\label{eq217}
\bigl ((l_a l_a)_0 (l_b l_b)_0 |(l_a l_b)_\Lambda (l_a l_b)_\Lambda \bigr )_0 \left\lbrace \left[Y^{la} (\hat{\mathbf{r}}'_{aA}) Y^{l_{b}} (\hat{\mathbf{r}}_{bB}) \right]^{\Lambda}
\left[ Y ^{l_a} (\hat{\mathbf{k}}_{aA}) Y ^{l_{b}} (\hat{\mathbf{k}}_{bB}) \right]^{\Lambda} \right\rbrace ^0_0.
\end{equation}
The only term which does not vanish upon integration is
\begin{equation}\label{eq218}
\frac{(-1)^{\Lambda-\mu}}{\sqrt{(2l_a+1)(2l_b+1)}} \left[Y^{la} (\hat{\mathbf{r}}'_{aA}) Y^{l_{b}} (\hat{\mathbf{r}}_{bB}) \right]^{\Lambda}_{-\mu}\left[ Y ^{l_a} (\hat{\mathbf{k}}_{aA}) Y ^{l_{b}} (\hat{\mathbf{k}}_{bB}) \right]^{\Lambda}_\mu.
\end{equation}
Again, the only term surviving
\begin{equation}\label{eq219}
\left\lbrace \left[Y^{l_f} (\hat{\mathbf{r}}'_{A2}) Y^{l_{i2}} (\hat{\mathbf{r}}'_{b2}) \right]^{P}
\left[ Y ^{l_f} (\hat{\mathbf{r}}_{A1}) Y ^{l_{i1}} (\hat{\mathbf{r}}_{b1}) \right]^{K} \right\rbrace ^\Lambda_\mu\left[Y^{la} (\hat{\mathbf{r}}'_{aA}) Y^{l_{b}} (\hat{\mathbf{r}}_{bB}) \right]^{\Lambda}_{-\mu}
\end{equation}
is
\begin{equation}\label{eq220}
\begin{split}
\frac{(-1)^{\Lambda+\mu}}{\sqrt{2  \Lambda+1}}&
 \left [\left\lbrace \left[Y^{l_f} (\hat{\mathbf{r}}'_{A2})
  Y^{l_{i2}} (\hat{\mathbf{r}}'_{b2}) \right]^{P} \right. \vphantom{\left.\left[ Y ^{l_{i1}}\right]^{K} \right\rbrace ^\Lambda} \right.\\
& \left. \left. \left[ Y ^{l_f} (\hat{\mathbf{r}}_{A1}) Y ^{l_{i1}} (\hat{\mathbf{r}}_{b1}) \right]^{K} \right\rbrace ^\Lambda\left[Y^{la} (\hat{\mathbf{r}}'_{aA}) Y^{l_{b}} (\hat{\mathbf{r}}_{bB}) \right]^{\Lambda}\right ]_0^0.
\end{split}
\end{equation}
We now couple this last term with the term $\left[ Y ^{l_c} (\hat{\mathbf{r}'}_{cC}) Y ^{l_{c}} (\hat{\mathbf{r}}_{cC}) \right]^{0}_0$, arising from the partial wave expansion of the Green function. That is, \idx{Two-nucleon transfer!second order DWBA}
\begin{equation}\label{eq221}
\begin{split}
\left [\left\lbrace  \vphantom{\left.\left[ Y ^{l_{i1}}\right]^{K}\right\rbrace ^\Lambda} \right. \right.&\left.\left.
\left[Y^{l_f} (\hat{\mathbf{r}}'_{A2})
  Y^{l_{i2}} (\hat{\mathbf{r}}'_{b2}) \right]^{P}
  \left[ Y ^{l_f} (\hat{\mathbf{r}}_{A1}) Y ^{l_{i1}} (\hat{\mathbf{r}}_{b1}) \right]^{K} \right\rbrace ^\Lambda\left[Y^{la} (\hat{\mathbf{r}}'_{aA}) Y^{l_{b}} (\hat{\mathbf{r}}_{bB}) \right]^{\Lambda}\right ]_0^0\left[ Y ^{l_c} (\hat{\mathbf{r}'}_{cC}) Y ^{l_{c}} (\hat{\mathbf{r}}_{cC}) \right]^{0}\\
  &=\bigl ((l_a l_b)_\Lambda (l_c l_c)_0 |(l_a l_c)_P (l_b l_c)_K \bigr )_\Lambda \left [\left\lbrace  \left[Y^{l_f} (\hat{\mathbf{r}}'_{A2})
  Y^{l_{i2}} (\hat{\mathbf{r}}'_{b2}) \right]^{P}  \vphantom{\left.\left[ Y ^{l_{i1}}\right]^{K} \right\rbrace ^\Lambda}
  \left[ Y ^{l_f} (\hat{\mathbf{r}}_{A1}) Y ^{l_{i1}} (\hat{\mathbf{r}}_{b1}) \right]^{K} \right\rbrace ^\Lambda \right.\\
  &\left.\left\lbrace  \left[Y^{l_a} (\hat{\mathbf{r}}'_{aA})
  Y^{l_{c}} (\hat{\mathbf{r}}'_{cC}) \right]^{P}  \vphantom{\left.\left[ Y ^{l_{i1}}\right]^{K} \right\rbrace ^\Lambda}
  \left[ Y ^{l_b} (\hat{\mathbf{r}}_{bB}) Y ^{l_{c}} (\hat{\mathbf{r}}_{cC}) \right]^{K} \right\rbrace ^\Lambda \right]_0^0=\bigl ((l_a l_b)_\Lambda (l_c l_c)_0 |(l_a l_c)_P (l_b l_c)_K \bigr )_\Lambda\\
  &\times \bigl ((PK)_\Lambda (PK)_\Lambda |(PP)_0 (KK)_0 \bigr )_0\left\lbrace  \left[Y^{l_f} (\hat{\mathbf{r}}'_{A2})
  Y^{l_{i2}} (\hat{\mathbf{r}}'_{b2}) \right]^{P}
  \left[ Y ^{l_a} (\hat{\mathbf{r}}'_{aA}) Y ^{l_{c}} (\hat{\mathbf{r}}'_{cC}) \right]^{P} \right\rbrace ^0_0\\
  &\times\left\lbrace  \left[Y^{l_f} (\hat{\mathbf{r}}_{A1})
  Y^{l_{i1}} (\hat{\mathbf{r}}_{b1}) \right]^{K}
  \left[ Y ^{l_b} (\hat{\mathbf{r}}_{bB}) Y ^{l_{c}} (\hat{\mathbf{r}}_{cC}) \right]^{K} \right\rbrace ^0_0=\bigl ((l_a l_b)_\Lambda (l_c l_c)_0 |(l_a l_c)_P (l_b l_c)_K \bigr )_\Lambda\\
  &\times \sqrt{\frac{2\Lambda+1}{(2K+1)(2P+1)}}\left\lbrace  \left[Y^{l_f} (\hat{\mathbf{r}}'_{A2})
  Y^{l_{i2}} (\hat{\mathbf{r}}'_{b2}) \right]^{P}
  \left[ Y ^{l_a} (\hat{\mathbf{r}}'_{aA}) Y ^{l_{c}} (\hat{\mathbf{r}}'_{cC}) \right]^{P} \right\rbrace ^0_0\\
  &\times\left\lbrace  \left[Y^{l_f} (\hat{\mathbf{r}}_{A1})
  Y^{l_{i1}} (\hat{\mathbf{r}}_{b1}) \right]^{K}
  \left[ Y ^{l_b} (\hat{\mathbf{r}}_{bB}) Y ^{l_{c}} (\hat{\mathbf{r}}_{cC}) \right]^{K} \right\rbrace ^0_0.
 \end{split}
\end{equation}
Collecting all the contributions (including the constants and phases arising from the partial wave expansion of the distorted waves and the Green function), we  get
\begin{equation}\label{eq222}
 \begin{split}
T_{succ}^{(2)}(\theta;\mu)=&(-1)^{j_f+j_{i1}}\frac{2048\pi^{5}\mu_{Cc}}{ \hbar^2 k_{Aa}k_{Bb}k_{Cc}}\sqrt{\frac{(2j_{i1}+1)}{(2\Lambda+1)(2j_f+1)}}\sum_{K,P}
\bigl ( (l_f \tfrac{1}{2})_{j_f} (l_{i2} \tfrac{1}{2})_{j_{i2}} |(l_f l_{i2})_P (\tfrac{1}{2} \tfrac{1}{2})_0 \bigr )_P\\
&\times
\bigl ( (l_f \tfrac{1}{2})_{j_f} (l_{i1} \tfrac{1}{2})_{j_{i1}} |(l_f l_{i1})_K (\tfrac{1}{2} \tfrac{1}{2})_0 \bigr )_K\;
\bigl ( (j_{i1} j_f)_K (j_{i1} j_{i2})_\Lambda |(j_{i1}  j_{i1})_0 (j_f j_{i2})_P \bigr )_P\\
&\times \frac{(-1)^K}{(2K+1)\sqrt{2P+1}} \sum_{l_c,l_a,l_b}\bigl ((l_a l_b)_\Lambda (l_c l_c)_0 |(l_a l_c)_P (l_b l_c)_K \bigr )_\Lambda e^{i(\sigma _i^{l_a}+\sigma _f^{l_b})}i^{l_a-l_b}\\
&\times (2l_c+1)^{3/2} \left[ Y ^{l_a} (\hat{\mathbf{k}}_{aA}) Y ^{l_{b}} (\hat{\mathbf{k}}_{bB}) \right]^{\Lambda}_\mu S_{K,P,l_a,l_b,l_c},
 \end{split}
\end{equation}
with (note that we have reduced the dimensionality of the integrals in the same fashion as for the $L=$0--angular momentum transfer calculation, see (\ref{eq124})) \idx{Two-nucleon transfer!second order DWBA}
\begin{equation}\label{eq223}
 \begin{split}
S_{K,P,l_a,l_b,l_c}=&\int r_{Cc}^2 \, d r_{Cc}\,r_{b1}^2\, d r_{b1} \,\sin\theta\, d\theta \, v(r_{b1}) u_{l_f}(r_{C1})u_{l_i}(r_{b1})\\
& \times \frac{s_{P,l_a,l_c}(r_{Cc})}{r_{Cc}}\frac{F_{l_b}(r_{Bb})}{r_{Bb}}\\
&\times\sum_M \langle l_c \;0\;l_b\;M|K\;M\rangle \left[ Y ^{l_f} (\hat r_{C1}) Y ^{l_{i1}} (\theta+\pi,0) \right] _{M}^{K}
 Y^{l_b}_{-M} (\hat r_{Bb}),
 \end{split}
\end{equation}
and
\begin{equation}\label{eq224}
 \begin{split}
s_{P,l_a,l_c}(r_{Cc})=&\int r_{Cc}^{'2} \, d r'_{Cc}\,r_{A2}^{'2}\, d r'_{A2} \,\sin\theta'\, d\theta' \, v(r_{c2}') u_{l_f}(r_{A2}')u_{l_i}(r_{c2}')  \\
& \times \frac{F_{l_a}(r_{Aa}')}{r_{Aa}'}\frac{f_{l_c}(k_{Cc},r_<)P_{l_c}(k_{Cc},r_>)}{r_{Cc}'}\\
&\times\sum_M \langle l_c \;0\;l_a\;M|P\;M\rangle \left[ Y ^{l_f} (\hat r_{A2}') Y ^{l_{i2}} (\hat r_{c2}') \right] _{M}^{P}
 Y^{l_a}_{-M} (\hat r_{Aa}').
 \end{split}
\end{equation}
We have evaluated the transition matrix element for a particular projection $\mu$ of the initial angular momentum of the two transferred nucleons. If they are coupled to a core of angular momentum $J_f$ to total angular momentum $J_i,M_i$, the fraction of the initial wavefunction with projection $\mu$ is $\langle \Lambda \;\mu\;J_f\;M_i-\mu|J_i\;M_i\rangle$, and the cross section will be \idx{Two-nucleon transfer!second order DWBA}
\begin{equation}\label{eq225_3}
\frac{d\sigma}{d\Omega}(\hat{\mathbf{k}}_{bB})=\frac{k_{bB}}{k_{aA}}\frac{\mu_{aA}\mu_{bB}}{(2\pi\hbar^2)^2}\left|\sum_\mu
\langle \Lambda \;\mu\;J_f\;M_i-\mu|J_i\;M_i\rangle T_{succ}^{(2)}(\theta;\mu)\right|^2.
\end{equation}
For a non polarized incident beam,
\begin{equation}\label{eq225_2}
\frac{d\sigma}{d\Omega}(\hat{\mathbf{k}}_{bB})=\frac{k_{bB}}{k_{aA}}\frac{\mu_{aA}\mu_{bB}}{(2\pi\hbar^2)^2}
\frac{1}{2J_i+1}\sum_{M_i}\left|\sum_{\mu}\langle \Lambda \;\mu\;J_f\;M_i-\mu|J_i\;M_i\rangle T_{succ}^{(2)}(\theta;\mu)\right|^2.
\end{equation}
This would be the differential cross section for a transition to a definite final state $M_f$. If we do not measure $M_f$ we have to sum for all  $M_f$,
\begin{equation}\label{eq225_4}
\frac{d\sigma}{d\Omega}(\hat{\mathbf{k}}_{bB})=\frac{k_{bB}}{k_{aA}}\frac{\mu_{aA}\mu_{bB}}{(2\pi\hbar^2)^2}
\frac{1}{2J_i+1}\sum_{\mu}|T_{succ}^{(2)}(\theta;\mu)|^2 \sum_{M_i,M_f}\left|\langle \Lambda \;\mu\;J_f\;M_f|J_i\;M_i\rangle\right|^2.
\end{equation}
The sum over $M_i,M_f$ of the Clebsh--Gordan coefficients gives $(2J_i+1)/(2\Lambda+1)$ (see Eq. (\ref{eq241})). One then gets,

\begin{equation}\label{eq225_7}
\begin{split}
\frac{d\sigma}{d\Omega}(\hat{\mathbf{k}}_{bB})=\frac{k_{bB}}{k_{aA}}\frac{\mu_{aA}\mu_{bB}}{(2\pi\hbar^2)^2}
\frac{1}{(2\Lambda+1)}\sum_{\mu}|T_{succ}^{(2)}(\theta;\mu)|^2,
\end{split}
\end{equation}
where one can write
\begin{equation}\label{eq226}
 \begin{split}
T_{succ}^{(2)}(\theta;\mu) &=\sum_{l_a,l_b} C_{l_a,l_b} \left[ Y ^{l_a} (\hat{\mathbf{k}}_{aA}) Y ^{l_{b}} (\hat{\mathbf{k}}_{bB}) \right]^{\Lambda}_\mu\\
&=\sum_{l_a,l_b} C_{l_a,l_b}i^{l_a} \sqrt{\frac{2l_a+1}{4\pi}}\langle l_a \;l_b\;0\;\mu|\Lambda\;\mu\rangle Y ^{l_{b}}_\mu (\hat{\mathbf{k}}_{bB}).
 \end{split}
\end{equation}
Note that (\ref{eq225_7}) takes into account only the spins of the heavy nucleus. In a $(t,p)$ or $(p,t)$ reaction, we have to sum over the spins of the proton and of the triton and divide by 2. If a spin-orbit term is present in the optical potential, the sum yields the combination of terms shown in Section (\ref{C7SS722}), \idx{Two-nucleon transfer!second order DWBA}
\begin{equation}\label{eq225_8}
\begin{split}
\frac{d\sigma}{d\Omega}(\hat{\mathbf{k}}_{bB})=\frac{k_{bB}}{k_{aA}}\frac{\mu_{aA}\mu_{bB}}{(2\pi\hbar^2)^2}
\frac{1}{2(2\Lambda+1)}\sum_{\mu}|A_{\mu}|^2+|B_{\mu}|^2.
\end{split}
\end{equation}
\section[ZPF and Pauli principle]{ZPF, exclusion principle and medium polarization effects: self-energy, vertex corrections, induced interaction}\label{C7AppA}
\idx{Surface vibrations!zero-point fluctuations}
Of all quantal phenomena, zero point fluctuations (ZPF), closely connected with virtual states, are likely the  most representative of the essential difference existing between quantum and classical mechanics. In fact, ZPF are intimately connected with the complementary principle\footnote{\cite{Bohr:28}.}, and thus with  indeterminacy\footnote{\cite{Heisenberg:27}.} and non-commutative\footnote{\cite{Born:25a}, \cite{Born:25b}.} relations, and with the probabilistic interpretation\footnote{ \cite{Born:26}.} of the (modulus squared) of the wavefunctions, solution of Schr\"odinger's or Dirac's equations\footnote{\cite{Schrodinger:25}, \cite{Dirac:26}.}.
 \begin{figure}[h!]
 	\begin{center}
\includegraphics*[width=1.2\textwidth]{C7/figs_C7/FigA.pdf}
\end{center}
\caption{Nuclear field theory (NFT) diagrams describing renormalization processes associated with ZPF. For details see caption to Fig. \ref{figB}.}\label{figA}
\end{figure}
 \begin{figure}[h!]
 	\begin{center}
\includegraphics*[width=0.75\textwidth]{C7/figs_C7/FigB.pdf}
\end{center}
\caption{Pauli effects associated (p-h) ZPF dressing a pairing vibrational (pair addition) mode (see inset I) in terms of self-energy (graphs (a)--(c); correlation (CO) and polarization (PO) diagrams, inset II) and vertex correction (graphs (d)--(f); induced particle-particle (pairing) interaction) processes (inset (III)), associated with phonon exchange between nucleons (inset (IV)).}\label{figB}
\end{figure}
Pauli principle\footnote{\cite{Pauli:25}.} brings about essential modifications to the virtual fluctuations of the many-body system, modifications which are instrumental in the dressing and interweaving of the elementary modes of excitation\footnote{Within the present context, see also \cite{Schrieffer:64}.}.




In Fig.  \ref{figA}, NFT diagrams are given which correspond to the lowest order medium polarization effects renormalizing the properties of a particle--hole collective mode (wavy line), correlated particle-hole excitation  ((up-going)--(down-going) arrowed lines)  calculated within the random phase approximation (RPA,QRPA), and leading to the particle-vibration coupling vertex (formfactor and strength, i.e. transition density (solid dot), see inset (I), bottom). The action of an external field on the zero point fluctuations (ZPF) of the vacuum (inset (II)), forces a virtual process to become real, leading to a collective vibration by annihilating a (virtual, spontaneous) particle-hole excitation (backwards RPA $Y$-amplitude) or, in the time ordered process, by creating a particle-hole excitation which eventually, through the particle-vibration coupling vertex, materializes into the collective (coherent) state (forwardsgoing $X$-amplitudes). Now, oyster-like diagrams associated with the vacuum ZPF can occur at any time (see inset (III) of Fig. \ref{figA}). It is of notice that   while virtual states can violate energy conservation, they have to respect conservation rules, as well as Pauli principle. For example, a virtual state  cannot violate  angular momentum neither allow for the  presence of two fermions in the same quantal state. The process shown in the inset III ($\alpha$) leads, through Pauli principle correcting diagrams (exchange of fermionic arrowed lines) to self-energy (inset III ($\beta$), ($\delta$)) and vertex corrections (induced $p-h$ interaction; inset III ($\gamma$), ($\varepsilon$), see also graphs (c), (f), (h) and (i) of Fig. \ref{figA}) processes. Similar processes are found in the interplay between ($p-h$-like) ZPF 	and pair addition modes as shown in Fig. \ref{figB}. Note the parallel between diagrams of Figs. \ref{figA} (g)--(i) and of Figs. \ref{figB} (d)--(f).


 The collective vibrational modes can be viewed as  coherent states\footnote{See e.g. \cite{Glauber:69,Glauber:07}.} exhausting a consistent fraction of the EWSR.
\section{Coherence and effective formfactors}\label{C7AppB}

In what follows we shall work out  a simplified derivation of the simultaneous two--nucleon transfer amplitude, within the framework of first order DWBA specially suited to discuss correlation aspects of pair transfer in general, and of the associated effective formfactors in particular\footnote{\cite{Glendenning:65,Bayman:67}.}. 


We will concentrate on $(t,p)$ reactions, namely reactions of the type $A(\alpha,\beta)B$ where $\alpha=\beta+2$ and $B=A+2$.
The intrinsic wave functions are in this case
\begin{equation}\label{5lec1}
\begin{split}
\psi_\alpha=& \psi_{M_i}^{J_i}(\xi_A) \sum_{s s'_f} \left[ \chi^s(\sigma_\alpha) \chi^{s'_f}(\sigma_\beta) \right] _{M_{s_i}}^{s_i}
\phi_t^{L=0}\left(r_{12},r_{1p},r_{2p}\right)\\
&= \psi_{M_i}^{J_i}(\xi_A) \sum_{M_s M'_{s_f}} (s M'_{s} s'_f M'_{s_f}| s_i M_{s_i}) \chi^s_{M'_s}(\sigma_\alpha) \chi^{s_f'}_{M'_{s_f}}(\sigma_\beta)\\
& \times \phi_t^{L=0}\left(r_{12},r_{1p},r_{2p}\right)
\end{split}
\end{equation}
 \begin{figure}[h!]
 	\begin{center}
\includegraphics*[width=\textwidth]{C7/figs_C7/coord}
\end{center}
\caption{Coordinate system used in the calculation of the two--nucleon transfer amplitude.}\label{fig_coord}
\end{figure}
while

\begin{equation}\label{5lec2}
\begin{split}
\psi_\beta=& \psi_{M_f}^{J_f}(\xi_{A+2}) \chi^{s_f}_{M_{s_f}}(\sigma_\beta)\\
&=\sum_{\substack{n_1 l_1 j_1\\n_2 l_2 j_2}} B(n_1 l_1 j_1,n_2 l_2 j_2;JJ'_iJ_f)
\left[ \phi^J(j_1 j_2) \phi^{J'_i}(\xi_A)\right]^{J_f}_{M_f}\\
&\times \chi^{s_f}_{M_{s_f}}(\sigma_\beta).
\end{split}
\end{equation}

Making use of the above equation one can define the spectroscopic amplitude $B$ as

\begin{equation}\label{5lec3}
\begin{split}
B&(n_1 l_1 j_1,n_2 l_2 j_2;JJ'_iJ_f)\\
&=\left\langle  \psi^{J_f}(\xi_{A+2})\left |\left[ \phi^J(j_1 j_2) \phi^{J_i}(\xi_A)\right]^{J_f}\right. \right\rangle,
\end{split}
\end{equation}

where
\begin{equation}\label{5lec4}
\phi^J(j_1 j_2)=\frac{\left[ \phi_{j_1}(\mathbf r_1) \phi_{j_2}(\mathbf r_2)\right]^{J}-
\left[ \phi_{j_1}(\mathbf r_2) \phi_{j_2}(\mathbf r_1)\right]^{J}}{\sqrt{1+\delta(j_1,j_2)}},
\end{equation}
is an antisymetrized, normalized wave function of the two transferred particles. The function $\chi^{s}_{M_{s}}(\sigma_\beta)$ appearing  both in eq. (\ref{5lec1}) and (\ref{5lec2}) is the spin wave function of the proton while 
\begin{equation}\label{5lec5}
\chi^{s}(\sigma_\alpha)=\left[ \chi^{s_1}(\sigma_{n_1}) \chi^{s_2}(\sigma_{n_2})\right]^{s},
\end{equation}
is the spin function of the two-neutron system.


A simple description of the intrinsic  degrees of freedom of the triton is obtained by using a wavefunction symmetric in the coordinates of all particles, i.e.

\begin{equation}\label{5lec6}
\begin{split}
\phi_t^{L=0}\left(r_{12},r_{1p},r_{2p}\right)&=N_t\,e^{[(r_1-r_2)^2+(r_1-r_p)^2+(r_2-r_p)^2]}\\
&=\phi_{000}(\mathbf r)\phi_{000}(\pmb \rho),
\end{split}
\end{equation}
where
\begin{equation}
\phi_{000}(\mathbf r)=R_{nl}(\nu^{1/2 }r) Y_{lm}(\mathbf{\hat r}).
\end{equation}

The coordinate $\pmb \rho$ is the radius vector which measures the distance between the center of mass of the dineutron and the proton, while the vector $\mathbf r$ is the dineutron relative coordinate (cf. Fig. \ref{fig_coord}).
To obtain the DWBA cross section we have to calculate the integral
\begin{equation}\label{5lec8}
T(\theta)=\int d\xi_A \,d\mathbf r_1 \,d\mathbf r_2 \,d\mathbf r_p \chi^{(-)}_p(\mathbf R_2) \psi^*_\beta(\xi_{A+2},\sigma_\beta) V_\beta \psi_\alpha(\xi_{A},\sigma_\alpha,\sigma_\beta)\psi_t^{(+)}(\mathbf R_1),
\end{equation}
where the final state effective interaction $V_\beta(\rho)$ is assumed to depend only on the distance $\rho$ between the center of mass of the di-neutron and of the proton. 


To carry out the integral (\ref{5lec8}) we transform the wave function (\ref{5lec4}) into center of mass and relative coordinates. If we assume that both $\phi_{j_1}(\mathbf r_1)$ and $\phi_{j_2}(\mathbf r_2)$ are harmonic oscillator wave functions (used as a basis to expand the Saxon-Woods single-particle wavefunctions), this transformation can be carried with the aid of the Moshinsky brackets. If $| n_1 l_1,n_2 l_2; \lambda \mu \rangle$ is a complete system of wave functions in the harmonic oscillator basis, depending on $\mathbf r_1$ and $\mathbf r_2$ and $| n l,N L; \lambda \mu \rangle$ is the corresponding one depending on $\mathbf r$ and  $\mathbf R$, we can write

\begin{equation}\label{5lec9}
\begin{split}
| n_1 l_1,n_2 l_2; \lambda \mu \rangle&= \sum_{n l N L} \left(| n l,N L; \lambda \mu \rangle \langle n l,N L; \lambda \mu |\right)
| n_1 l_1,n_2 l_2; \lambda \mu \rangle \\
&=\sum_{n l N L} | n l,N L; \lambda \mu \rangle \langle n l,N L; \lambda \mu | n_1 l_1,n_2 l_2; \lambda  \rangle.
\end{split}
\end{equation}
The labels $n,l$ are the principal and angular momentum quantum numbers of the relative motion, while $N,L$ are the corresponding ones corresponding to the center of mass motion of the two-neutron system. Because of energy and parity conservation we have
\begin{equation}\label{5lec10}
\begin{split}
2n_1+l_1+2n_2+l_2&=2n+l+2N+L,\\
(-1)^{l_1+l_2}=(-1)^{l+L}.
\end{split}
\end{equation}
The coefficients $\langle n l,N L, L | n_1 l_1,n_2 l_2, L  \rangle$ were first discussed by Moshinsky\footnote{\cite{Moshinsky:59}.}.
	

With the help of eq. (\ref{5lec9}) we can write the wave function $\psi_{M_f}^{J_f}(\xi_{A+2})$ as
\begin{equation}\label{5lec11}
\begin{split}
\psi_{M_f}^{J_f}(\xi_{A+2})&= \sum_{\substack{n_1 l_1 j_1\\n_2 l_2 j_2\\ J J_i}} B(n_1 l_1 j_1,n_2 l_2 j_2;JJ'_i J_f)
\left[ \phi^J(j_1 j_2) \phi^{J'_i}(\xi_A)\right]^{J_f}_{M_f}\\
&= \sum_{\substack{n_1 l_1 j_1\\n_2 l_2 j_2}} \sum_{J J_i }B(n_1 l_1 j_1,n_2 l_2 j_2;JJ'_i J_f)\\
& \times \sum_{M_J M'_{J_i}} \langle J M_J J'_i M_{J_i}|J_f M_{J_f}\rangle \psi_{M'_{J_i}}^{J'_i}(\xi_{A})\\
& \times \sum_{L S'} \langle S' L J |j_1 j_2 J \rangle \sum_{M_L M'_S} \langle L M_L S' M'_S |J M_J  \rangle \chi^{S'}_{M'_S}(\sigma_\alpha)\\
& \times \sum_{n l N \Lambda} \langle n l,N \Lambda, L |n_1 l_1,n_2 l_2, L \rangle \\
& \times \sum_{m_l  M_\Lambda}
\langle l m_l \Lambda M_\Lambda |L M_L \rangle \phi_{n l m_l}(\mathbf r) \phi_{N \Lambda M_\Lambda}(\mathbf R),
\end{split}
\end{equation}
Integration over $\vec r$ gives
\begin{equation}\label{5lec12}
\langle \phi_{n l m_l}(\mathbf r) | \phi_{000}(\mathbf r) \rangle = \delta(l,0) \delta(m_l,0) \Omega_n,
\end{equation}
where
\begin{equation}\label{5lec13}
\Omega_n=\int R_{n l} (\nu_1^{1/2} r)R_{00} (\nu_2^{1/2} r) r^2\, dr.
\end{equation}
\textit{Note that there is no selection rule in the principal quantum number $n$, as the potential in which the two neutrons move in the triton has a frequency $\nu_2$ which is different from the one that the two neutrons are subjected to, when moving in the system $A$ (non-orthogonality effect}).


Integration over $\xi_A$ and multiplication of the spin functions gives
\begin{equation}\label{5lec14}
\begin{split}
\bigl( \psi_{M_{J_i}}^{J_i},V'_\beta(\rho) \psi_{M'_{J_i}}^{J'_i}\bigr)&= \delta(J_i,J'_i)
\delta(M_{J_i},M_{J'_i})V(\rho),\\
\bigl( \chi_{M_S}^{S}(\sigma_\alpha),\chi_{M_{S'}}^{S'}(\sigma_\alpha)\bigr)&= \delta(S,S')
\delta(M_S,M_{S'}),\\
\bigl( \chi_{M_{S_f}}^{S_f}(\sigma_\beta),\chi_{M_{S'_f}}^{S'_f}(\sigma_\beta)\bigr)&= \delta(S_f,S'_f)
\delta(M_{S_f},M_{S'_f}).\\
\end{split}
\end{equation}
The integral (\ref{5lec8}) can then be written as
\begin{equation}\label{5lec15}
\begin{split}
T(\theta)&= \sum_{\substack{n_1 l_1 j_1\\n_2 l_2 j_2}}\sum_{J M_J}\sum_{nN}\sum_{S}B(n_1 l_1 j_1,n_2 l_2 j_2;JJ_i J_f)\\
&\times \langle J M_J J_i M_{J_i}|J_f M_{J_f} \rangle \langle SLJ|j_1 j_2 J \rangle \\
&\times \langle L M_L S M_{S}|J M_{J} \rangle \langle n0,NL,L|n_1 l_1,n_2 l_2,L \rangle \\
&\times \langle S M_S S_f M_{S_f}|S_i M_{S_i}\rangle \Omega_n \\
&\times \int d\mathbf R\,d\mathbf r_p\, \chi^{(+)*}_t(\mathbf R_1) \phi^*_{NLM_L}(\mathbf R) V(\rho) \phi_{000}(\pmb \rho) \chi^{(+)}_t(\mathbf R_1),
\end{split}
\end{equation}
where we have approximated $V'_\beta$ by an effective interaction depending on $\rho=|\pmb \rho|$.
We now define the effective two-nucleon transfer  form factor as
\begin{equation}\label{eqC7B15}
\begin{split}
u^{J_i J_f}_{LSJ}(R)&=\sum_{\substack{n_1 l_1 j_1\\n_2 l_2 j_2}}\sum_{nN} B(n_1 l_1 j_1,n_2 l_2 j_2;JJ_i J_f) \langle S L J|j_1 j_2 J\rangle\\
&\langle n0,NL,L|n_1 l_1,n_2 l_2;L \rangle \Omega_n R_{NL}(R).
\end{split}
\end{equation}
One can then rewrite Eq. (\ref{5lec15}) as
\begin{equation}\label{5lec17}
\begin{split}
T(\theta)&= \sum_J\sum_L\sum_S \bigl( J M_J J_i M_{J_i}|J_f M_{J_f} \bigr) \bigl( S M_S S_f M_{S_f}|S_i M_{S_i}\bigr)
 \bigl( L M_L S M_{S}|J M_{J} \bigr) \\
&\times \int d\mathbf R\,d\mathbf r_p\, \chi^{*(-)}_p(\mathbf R_2)
 u^{J_i J_f}_{LSJ}(R) Y_{L M_L}^*V(\rho) \phi_{000}(\pmb \rho) \chi^{(+)}_t(\mathbf R_1).
\end{split}
\end{equation}
Because the di-neutron has $S=0$, we have that
\begin{equation}\label{5lec18}
 \bigl( L M_L 0 0|J M_{J} \bigr)=\delta(J,L)\delta(M_L,M_J),
\end{equation}
and the summations over $S$ and $L$ disappear from Eq. (\ref{5lec17}). Let us now  make also here, as done in Sect. \ref{C6AppE}, Eq. (\ref{eqC6AppE15}) for one-particle transfer reactions, the zero range approximation, that is,


\begin{equation}\label{5lec19}
V(\rho) \phi_{000}(\pmb \rho)=D_0 \delta(\pmb \rho),
\end{equation}
where $D_0$ is an empirical parameter ($D_0^2=(31.6\pm9.3)\times10^4$ MeV$^2$fm$^2$) determined to reproduce, in average, the observed absolute cross sections\footnote{\cite{Broglia:73}.} with only the simultaneous transfer.
This means that the proton interacts with the center of mass of the di--neutron, only when they are at the same point in space.
Within this approximation (cf. Fig. \ref{fig_coord})
\begin{equation}\label{5lec20}
\begin{split}
\mathbf R=&\mathbf R_1=\mathbf r,\\
\mathbf R_2=&\frac{A}{A+2}\mathbf R.
\end{split}
\end{equation}
Then Eq. (\ref{5lec15}) can be written as
\begin{equation}\label{5lec21}
\begin{split}
T&= D_0 \sum_L \bigl( L M_L J_i M_{J_i}|J_f M_{J_f} \bigr) \\
&\times \int d\mathbf R\, \chi^{*(-)}_p\bigl(\frac{A}{A+2}\mathbf R\bigr)
 u^{J_i J_f}_{L}(R) Y_{L M_L}^*(\mathbf{\hat R}) \chi^{(+)}_t(\mathbf R)
\end{split}
\end{equation}
From Eq. (\ref{5lec21}) it is seen that the change in parity implied by the reaction is given by $\Delta\pi=(-1)^L$. Consequently, the selection rules for $(t,p)$ and  $(p,t)$ reactions in zero--range approximation are,
\begin{equation}\label{5lec22}
\begin{split}
\Delta S&=0\\
\Delta J=&\Delta L=L \\
\Delta\pi&=(-1)^L
\end{split}
\end{equation}
i.e. only normal parity states are excited.
The integral appearing in Eq. (\ref{5lec21}) has the same structure as the DWBA integral appearing in Eq. (\ref{eqC6E16}) which was derived for the case of one-nucleon transfer reactions.


The difference between the two processes manifests itself through the different structure of the two form factors. While $u_l(r)$  is a single--particle bound state wave function (cf. Eq. (\ref{eqC6E1})), $u_L^{J_i J_f}$ is a coherent summation \textit{over the center of mass states of motion of the two transferred neutrons} (see Eq. (\ref{eqC7B15})). In other words, an effective quantity (function). It is of notice that this difference between single-particle transfer formfactors and simultaneous two-nucleon transfer effective formfactors, becomes less pronounced  when one considers dressed particles resulting from the coupling to collective vibrations  and leading to renormalized energies, single-particle content, and renormalized radial wavefunction. 
In view of the relation (\ref{eq3.2.19}), closely connected with Fig. \ref{fig3.2.1} (b) found at the basis of the fact that successive transfer of entangled nucleons over distances of the order of the correlation length is the dominant contribution to pair transfer,  the above  results\footnote{See \cite{Potel:13}, in particular App. C and Fig. 10 of this reference.} are certainly dated. 


The reason to bring the subject here is to set in evidence subtle aspects of the correlations existing between partner nucleons of a Cooper pair to be also found in the formfactors associated with successive transfer although, arguably, in a less evident fashion.


 Examples of two-nucleon transfer form factors are given in\footnote{\cite{Broglia:67}.} Figs \ref{figC7B2}, \ref{figC7B3} and \ref{figC7B4}.
 \begin{figure}[h!]
 	\begin{center}
\includegraphics*[width=0.75\textwidth]{C7/figs_C7/figC7B2}
\end{center}\caption{The upper part of the figure shows the modified formfactor for the $^{206}$Pb(t,p)$^{208}$Pb transition to the ground state ($0^+_1$) and the pairing vibrational state ($0^+_2$) at 4.87 MeV. Both curves are matched with appropriate 	Hankel functions. In the lower part the form factors of the real ($f(r)$) and the imaginary ($g(r)$) part of the optical potential used to calculate the differential cross sections (see Fig. \ref{fig2A4}), are given.}\label{figC7B2}
\end{figure}
 \begin{figure}[h!]
 	\begin{center}
\includegraphics*[width=0.75\textwidth]{C7/figs_C7/figC7B3}
\end{center}\caption{Modified formfactor for the transition to the ground state ($^{206}$Pb(t,p)$^{208}$Pb(gs); see Fig. \ref{fig2A4}) calculated in different spectroscopic models (pure shell--model configuration -------, shell model  plus pairing residual interaction -- -- --, including ground state correlations --o--o--).}\label{figC7B3}
\end{figure}
 \begin{figure}[h!]
 	\begin{center}
\includegraphics*[width=0.75\textwidth]{C7/figs_C7/figC7B4}
\end{center}\caption{Asymptotic behavior of the modified formfactor for the $^{206}$(t,p)$^{208}$Pb(gs) ground state transition for oscillator plus Hankel wave functions (continuous solid curve), oscillator wave functions alone (dash point dashed curve), and Saxon-Woods wave functions with a variety of asymptotic matchings.}\label{figC7B4}
\end{figure}
\section[Successive, simultaneous (semiclassical)]{Relative importance of successive and simultaneous transfer and non-orthogonality corrections (semiclassical)}\label{C7AppC}

 \begin{figure}[h!]
 	\begin{center}
\includegraphics*[width=0.75\textwidth]{C7/figs_C7/Reaction1}
\end{center}
\caption{Graphical representation of simultaneous (I) and non--orthogonality (II) transfer processes. For details see text and  caption to Fig. \ref{figC7C2}. In the present figure, as well as in Fig. \ref{figC7C2}, and at variance with, e.g. Fig. \ref{fig1.9.2}, the particle-pair vibration coupling vertex is represented by a crossed circle. Concerning the jagged line representing recoil see App. \ref{App1C3}.}\label{figC7C1}
\end{figure}
 \begin{figure}[h!]
 	\begin{center}
\includegraphics*[width=\textwidth]{C7/figs_C7/Reaction2}
\end{center}
	\caption{Graphical representation of  the successive transfer of two nucleons. For details see text. Because of energy conservation, the different choices concerning the interactions inducing transfer, lead to identical results.}
\label{figC7C2}
\end{figure}
 \begin{figure}
 	\begin{center}
\includegraphics*[width=0.75\textwidth]{C7/figs_C7/figC7C3}
\end{center}
\caption{One-- and two--neutron separation energies $S(n)$ and $S(2n)$ associated with the channels $\alpha\equiv a(=b+2)+A \rightarrow \gamma \equiv f(=b+1)+F(=A+1)\rightarrow \beta \equiv b+B(=A+2)$.}\label{figC7C3}
\end{figure}
In what follows we discuss the relative importance of successive and simultaneous two-neutron transfer and of non-orthogonality 
corrections associated with the reaction 
\begin{equation}
\alpha \equiv  a(=b+2) + A \to b + B(=A+2) \equiv \beta,
\label{A1}
\end{equation}
in the limits of independent particles and of strongly correlated Cooper pairs, making use  of the semiclassical approximation\footnote{For details cf. \cite{Broglia:04a}.}, in which case the two-particle transfer differential cross section can be written as

\begin{equation}
\frac{d \sigma_{\alpha \to \beta} }{d \Omega} = P_{\alpha \to \beta} (t = +\infty) 
\sqrt{ \left( \frac{d \sigma_{\alpha}}{d \Omega} \right)_{el} }
\sqrt{ \left( \frac{d \sigma_{\beta}}{d \Omega} \right)_{el}}, 
\label{A2}
\end{equation}
where $P$ is the absolute value squared of a quantum mechanical transition amplitude. It gives the probability that the system at $t = + \infty$ is found in the final channel. The quantities $(d \sigma/d\Omega)_{el}$ are the classical elastic cross sections  in the center of mass system, calculated in terms of the deflection function, namely the functional relating the impact parameter and the scattering angle. 

The transfer amplitude can be written as  


\begin{equation}
a(t = + \infty) = a^{(1)}(\infty) - a^{(NO)}(\infty) + \tilde a^{(2)} ( \infty),
\label{A3}
\end{equation}
where $\tilde a^{(2)}(\infty)$ at $t= + \infty$ 
labels  the successive transfer amplitude expressed in the post-prior representation (see below).
The simultaneous transfer amplitude is given by (see Fig. \ref{figC7C1} (I))

\begin{equation}\label{eq7.C.4}
\begin{split}
a^{(1)} (\infty)& = \frac{1}{i \hbar} \int^{\infty}_{-\infty} dt (\psi^b \psi^B, (V_{bB} - <V_{bB}>) \psi^a \psi^A ) \times 
{\rm exp} [\frac{i}{\hbar} (E^{bB} - E^{aA}) t]  \\
&\approx \frac{2}{i \hbar} \int^{\infty}_{- \infty}  dt \left( \phi^{B(A)} (S^B_{(2n)}; \mathbf r_{1A}, \mathbf r_{2A}), U(r_{1b}) 
e^{i (\sigma_1 + \sigma_2)}
\phi^{a(b)} (S^a_{(2n)}; \mathbf r_{1b}, \mathbf r_{2b}) \right)\\
&\times {\rm exp} [\frac{i}{\hbar} (E^{bB} - E^{aA}) t + \gamma(t)] 
\end{split}
\end{equation}
where\footnote{See \cite{Broglia:75}.}, Eq. (W12, p. 361)
\begin{equation}
\sigma_1 + \sigma_2 = \frac{1}{\hbar} \frac{m_n}{m_A} ( m_{aA} \mathbf v_{aA} (t) + m_{bB} \mathbf v_{bB}(t)) \cdot (\mathbf r_{1\alpha}
+\mathbf r_{2 \alpha}),
\end{equation}
the ``$\alpha$-point'' being defined by the relation\footnote{See \cite{Broglia:72d}, Eq. (2.26); see also \cite{Gotz:75} Eq. (5.2 c).}
\begin{equation}
\mathbf r_{\alpha A}=\frac{m_B}{m_a+m_B}\mathbf r_{bA}, 
\end{equation}
and, as a result,
\begin{equation}
\mathbf r_{i A}=\mathbf r_{i \alpha}+\frac{m_B}{m_a+m_B}\mathbf r_{bA},\quad(i=1,2).
\end{equation}
The ${\rm exp} ( i (\sigma_1 + \sigma_2))$ takes care of recoil 
effects (Galilean transformation associated with the mismatch between entrance and exit channels). 
The phase $\gamma (t)$ is related  with the effective $Q-$value of the reaction. In the above expression, $\phi$ indicates an antisymmetrized, correlated two-particle (Cooper pair)  wavefunction, $S(2n)$ being the two-neutron separation energy (see Fig. \ref{figC7C3}), $U(r_{1b})$ being the single particle potential generated by nucleus $b$. The contribution arising from non-orthogonality effects can be written as (see Fig. \ref{figC7C1} (II))
\begin{equation}\label{eq7.C.6}
\begin{split}
a^{(NO)} (\infty) &= \frac{1}{i \hbar} \sum_{f,F}\int^{\infty}_{-\infty} dt (\psi^b \psi^B, (V_{bB} - <V_{bB}>) \psi^f \psi^F )
(\psi^f\psi^F, \psi^a \psi^A) 
{\rm exp} [\frac{i}{\hbar} (E^{bB} - E^{aA}) t]    \\
&\approx \frac{2}{i \hbar}\sum_{f,F} \int^{\infty}_{- \infty} \phi^{B(F)} (S^B_{(n)}, \mathbf r_{1A}), U(r_{1b}) 
e^{i \sigma_1}
(\phi^{f(b)}(S^f(n), \mathbf r_{1b})\\
&  \times\phi^{F(A)} (S^F(n),\mathbf r_{2A}) e^{i \sigma_2} \phi^{a(f)}(S^a(n),\mathbf r_{2b})) {\rm exp} [\frac{i}{\hbar} (E^{bB} - E^{aA}) t + \gamma(t)] ,
\end{split}
\end{equation}
the reaction channel $f= (b+1) + F(=A+1)$ having been introduced, the quantity $S(n)$ being the one-neutron separation 
energy (see Fig. \ref{figC7C3}). The summation over $f(\equiv a'_1,a'_2)$ and $F (\equiv a_1,a_2)$ involves a restricted number of states, namely the valence shells in nuclei $B$ and $a$. It is of notice that the NFT$_{\text{(s+r)}}$ diagrams appearing in Fig. \ref{figC7C1} (II) are the only ones we have encountered in which to a particle-recoil coupling vertex (dashed open rectangle) and thus the starting or/and ending of a recoil mode (jagged line) does not correspond the action of the nuclear interaction or mean field (horizontal short arrow) inducing a transfer process. 

The successive transfer amplitude  $\tilde a^{(2)}_{\infty}$ written making use of the post-prior representation is equal to 
(see Fig. \ref{figC7C2})

\begin{align}\label{eqC7C7}
\nonumber \tilde a^{(2)} (\infty) & = \frac{1}{i \hbar} \sum_{f,F}\int^{\infty}_{-\infty} dt (\psi^b \psi^B, (V_{bB} - <V_{bB}>) e^{i \sigma_1} \psi^f \psi^F )\\
 \nonumber &\times {\rm exp} [\frac{i}{\hbar} (E^{bB} - E^{fF}) t + \gamma_1(t)]   \\
\nonumber &\times \frac{1}{i \hbar} \int^{t}_{-\infty} dt' (\psi^f \psi^F, (V_{fF} - <V_{fF}>) e^{i \sigma_2} \psi^a \psi^A > \\
&\times {\rm exp} [\frac{i}{\hbar} (E^{fF} - E^{aA}) t' + \gamma_2(t')].
\end{align}

To gain insight into the  relative importance of the three terms contributing to Eq. (\ref{A3}) we discuss two situations, namely,
the independent-particle model and the strong-correlation limits. 


Before doing so, let us comment on the graphical description of the transfer amplitudes (\ref{eq7.C.4}) (\ref{eq7.C.6}) and (\ref{eqC7C7}) displayed in Figs. \ref{figC7C1}
and \ref{figC7C2}. The time arrow is assumed to point upwards:
(\textbf{I}) Simultaneous transfer, in which one particle is transferred by the nucleon-nucleon interaction acting either in the entrance $\alpha \equiv a+A$ channel (prior) or in the final $\beta \equiv b + B$ channel (post), while the other particle follows suit making use of the particle-particle correlation (grey area) which binds the Cooper pair (see upper inset labelled (a)), represented by a solid arrow on a double line, to the projectile (curved arrowed lines) or to the target (opened arrowed lines). The above argument provides the explanation why when e.g. $v_{1b}$ acts on one nucleon, the other nucleon also reacts instantaneously. In fact a Cooper pair displays generalized rigidity (emergent property in gauge space).
A crossed open circle represents the particle-pair vibration coupling. The associated strength, together with an energy denominator, determines the amplitude $X_{a'_1 a'_2}$  with which the pair mode (Cooper pair) is in the (time reversed)\footnote{Generalized to include also different number of nodes.} two particle configuration $a'_1 a'_2$. In the transfer process, the  relative motion orbit changes, the readjustement of the corresponding trajectory mismatch being operated by a Galilean transformation induced by the operator ($\textrm{exp}\{ \mathbf k \cdot (\mathbf{r}_{1A}(t)+\mathbf{r}_{2A}(t))\}$). This phenomenon, known as recoil process, is represented by a jagged line which  provides  information on the two transferred nucleons (single time appearing as argument of both single-particle coordinates $r_1$ and $r_2$; see Fig. \ref{figC7C1} inset labeled (b)). In other words, information on the coupling of structure and reaction modes.
(\textbf{II}) Non-orthogonality contribution. While one of the nucleons of the Cooper pairs is transferred under the action of $v$, the other goes, uncorrelatedly over, profiting of the non-orthogonality of the associated single-particle wavefunctions (see inset (c)). 
(\textbf{III}) Successive transfer. In this case, there are two time dependences associated with the acting of the nucleon-nucleon interaction twice (see Fig. \ref{figC7C2} inset (d)). 


\subsection{Independent particle limit}\label{C7S7C1}

In the independent particle limit, the two transferred particles do not interact among themselves but for antisymmetrization. 
Thus, the separation energies fulfill the relations (see Fig. \ref{figC7C3})
\begin{equation}\label{eqC7C8}
S^B(2n) = 2 S^B(n) = 2S^F(n),
\end{equation}
and 
\begin{equation}\label{eqC7C9}
S^a(2n) = 2 S^a(n) = 2 S^f(n).
\end{equation}
In this case 
\begin{equation}\label{eqC7C10}
\phi^{B(A)} (S^B(2n), \mathbf r_{1A},\mathbf r_{2A}) = \sum_{a_1 a_2} \phi_{a_1}^{B(F)} (S^B(n),\mathbf r_{1A}) 
\phi_{a_{2}}^{F(A)} (S^F(n),\mathbf r_{2a}),
\end{equation}
and 
\begin{equation}\label{eqC7C11}
\phi^{a(b)} (S^a(2n), \mathbf r_{1b},\mathbf r_{2b}) = 
\sum_{a^{'}_{1} a^{'}_{2}} \phi_{a^{'}_1}^{a(f)} (S^a(n),\mathbf r_{2b}) 
\phi_{a^{'}_{2}}^{f(b)} (S^f(n),\mathbf r_{1b}),
\end{equation}
where $(a_1, a_2) \equiv F$ and $(a'_1, a'_2) \equiv f$ span, as mentioned above, shells in nuclei $B$ and $a$ respectively. 

Inserting Eqs. (\ref{eqC7C8}--\ref{eqC7C11}) in Eq. (\ref{eq7.C.4}) one can show that 
\begin{equation}\label{eqC7C12}
a^{(1)} (\infty) = a^{(NO)}(\infty).
\end{equation}
It can be further demonstrated  that within the present approximation, $Im \; \tilde a^{(2)} =0,$ and that 
\begin{eqnarray}\label{eqC7C13}
\nonumber \tilde a^{(2)} (\infty) = \frac{1}{i \hbar} \sum_{f,F}\int^{\infty}_{-\infty} dt (\psi^b \psi^B, (V_{bB} - <V_{bB}>) e^{i \sigma_1} \psi^f \psi^F >\\
\nonumber  \times 
{\rm exp} [\frac{i}{\hbar} (E^{bB} - E^{fF}) t + \gamma_1(t)] \nonumber  \\
\nonumber \times \frac{1}{i \hbar} \int^{\infty}_{-\infty} dt' (\psi^f \psi^F, (V_{fF} - <V_{fF}>) e^{i \sigma_2} \psi^a \psi^A ) \\
\times 
{\rm exp} [\frac{i}{\hbar} (E^{fF} - E^{aA}) t' + \gamma_2(t)].
\label{A12}
\end{eqnarray}
The total absolute differential cross section (\ref{A2}), where $P = |a(\infty)|^2 = |\tilde a^{(2)}|^2$, is then equal to the product of two one-particle transfer cross sections (see Fig. \ref{fig6.1.1}),  associated with the (virtual) reaction channels
\begin{equation}
\alpha \equiv a+A \to f +F \equiv \gamma,
\end{equation}
and 
\begin{equation}
\gamma \equiv f +F \to b+B \equiv \beta.
\end{equation}

In fact, Eq.(\ref{A12}) involves no time ordering and consequently the two processes above are completely independent of each other. 
This result was expected because being $v_{12}= 0$, the transfer of one nucleon cannot influence, aside form selecting the
initial state for the second step, the behaviour of the other nucleon.

\subsection{Strong correlation (cluster) limit}

The second limit to be considered is the one in which the correlation betwen the two nucleons is so strong that (see Fig. \ref{figC7C3})
\begin{equation}
S^a(2n) \approx S^a(n) \gg S^f(n),
\label{A15}
\end{equation}
and 
\begin{equation}
S^B(2n) \approx S^B(n) \gg S^F(n).
\label{A16}
\end{equation}
That is, the magnitude of the one-nucleon separation energy is strongly modified by the pair breaking.

There is a different, although equivalent way to express (\ref{A3}) which is  more convenient to discuss the strong coupling limit.
In fact, making use of the post-prior representation one can write
\begin{eqnarray}
a^{(2)}(t) &= \tilde a^{(2)}(t) - a^{(NO)}(t) = 
%\nonumber \\ 
\frac{1}{i \hbar} \sum_{f,F}\int^{\infty}_{-\infty} dt (\psi^b \psi^B, (V_{bB} - <V_{bB}>) e^{i \sigma_1} \psi^f \psi^F ) \nonumber \\ 
&\times {\rm exp} [\frac{i}{\hbar} (E^{bB} - E^{fF}) t + \gamma_1(t)]  \nonumber  \\
&\nonumber\times\frac{1}{i \hbar} \int^{t}_{-\infty} dt' (\psi^f \psi^F, (V_{aA} - <V_{aA}>) \psi^a \psi^A )\\
& \times 
{\rm exp} [\frac{i}{\hbar} (E^{fF} - E^{aA}) t' + \gamma_2(t')].
\end{eqnarray}
The relations  (\ref{A15}), (\ref{A16}) imply 

\begin{equation}
E^{fF} - E^{aA} = S^a(n) - S^F(n) >> \frac{\hbar}{\tau},
\end{equation}
where $\tau$ is the collision time. Consequently the real part of $a^{(2)}(\infty)$ vanishes exponentially  with the $Q-$value of the intermediate transition, while the imaginary part  vanishes inversely proportional to this energy.
One can thus write,
\begin{equation}
Re \;  a^{(2)} (\infty) \approx 0,
\end{equation} 
and 
\begin{equation}
\begin{split}
a^{(2)}(\infty) \approx &\frac{1}{i \hbar} \frac{\tau}{<E^{fF}> - E^{bB}} 
\sum_{fF} (\psi^b \psi^B, 
(V_{bB}- <V_{bB}>) 
\psi^f\psi^F)_{t=0}\\
& \times 
(\psi^f\psi^F,(V_{aA} - <V_{aA}) \psi^a \psi^A)_{t=0},
\end{split}
\end{equation} 
where one has utilized the fact that $E^{bB} \approx E^{aA}$. For $v_{12} \to \infty$, $(<E^{fF}> - E^{bB}) \to \infty$)
and, consequently, 

\begin{equation}
\lim_{v_{12} \to \infty} a^{(2)} (\infty) = 0.
\end{equation} 

Thus the total two-nucleon transfer amplitude is equal, in the strong coupling limit, to the amplitude $a^{(1)} (\infty)$.


Summing up, only when successive transfer and non-orthogonal corrections are included in the description of the two-nucleon 
transfer process, does one obtain a consistent description of the process, which correctly converges to the weak and 
strong correlation limiting values. 


\subsubsection{Parallel with Josephson and Giaever tunneling.}
As already stated (Sect. \ref{trans_nutAppA}), actual nuclei are, as a rule, closer to the independent particle limit than to the cluster one. Thus, it is not surprising that successive constitutes the major contribution to two-nucleon transfer processes. Nonetheless, the slight deviation from independent particle motion arising from pairing correlations, entangles the fermions (nucleons) partners \idx{Cooper pair!entanglement of fermion partners}
 of a Cooper pair, in a subtle fashion. This can be seen by expressing the pair transfer operator  in terms of quasiparticles. Making use of the relation
\begin{align}\label{eq6.5.23}
a^\dagger_\nu=U_\nu\alpha_\nu^\dagger+V_\nu\alpha_{\tilde\nu},
\end{align} 
one obtains 
\begin{align}\label{eq6.5.24}
P^\dagger_\nu=a^\dagger_\nu a^\dagger_{\tilde\nu}=U^2_\nu\alpha_\nu^\dagger\alpha_{\tilde\nu}^\dagger-U_\nu V_\nu(\alpha_\nu^\dagger\alpha_\nu+\alpha_{\tilde\nu}^\dagger\alpha_{\tilde\nu})-V^2_\nu\alpha_{\tilde\nu}\alpha_\nu+U_\nu V_\nu.
\end{align} 
Because one-quasiparticle transfer amplitude is $\sim U_\nu$ and the associated cross section is $\sim U^2_\nu$ (i.e. $\sim T^2$, $T$ being the associated transition amplitude), it is natural that the two-quasiparticle transfer amplitude is $\sim U^2_\nu$ (within this context see Fig. \ref{fig4.6.1}, $S$-$Q$ current) and that the associated cross section is $\sim U^4_\nu$ (i.e. $\sim T^4$). As it is apparent from (\ref{eq6.5.24}), there exists a different possibility to transfer a pair of nucleons, leaving this time the quasiparticle distribution unchanged\footnote{This was the main message of Josephson (\cite{Josephson:62}), and was the possibility which was missed by Giaever (\cite{Giaver:73}), who interpreted supercurrents with zero voltage drop through the oxide barrier separating two superconductors as a metallic short. As he (Giaever) states: ``to make an experimental discovery is not enough to observe something, one must also realize the significance of the observation''. It was also missed by Cohen, Falicov and Phillips (\cite{Cohen:62}), who developed the many-body tunneling Hamiltonian (eventually extended by Josephson) and who found four contributions to the current which correspond to normal (carriers of charge $e$) current of type S-Q. That is, implying quasiparticle excitation.}. Namely, through the last term of (\ref{eq6.5.24}). Thus, a transfer amplitude $\sim U_\nu V_\nu$ (Fig. \ref{fig4.6.1}, $S$-$S$ current) and associated two-nucleon transfer cross section $\sim(U_\nu V_\nu)^2$ $(\sim T^2)$. A result which one finds at the basis of (\ref{eq3.2.19}).













 
 \begin{subappendices}
\section{Spherical harmonics and angular momenta}\label{C7AppD}
With Condon--Shortley phases
\begin{equation}\label{eq20}
Y_m^l(\hat z)=\delta_{m,0} \sqrt{\frac{2l+1}{4 \pi}}, \quad Y_m^{l*}(\hat r)=(-1)^m Y_{-m}^l(\hat r).
\end{equation}
Time--reversed phases consist in multiplying Condon--Shortley phases with a factor $i^l$, so
\begin{equation}\label{eq21}
Y_m^l(\hat z)=\delta_{m,0} i^l \sqrt{\frac{2l+1}{4 \pi}}, \quad Y_m^{l*}(\hat r)=(-1)^{l-m} Y_{-m}^l(\hat r).
\end{equation}
With this phase convention, the relation with the associated Legendre polynomials includes an extra $i^l$ factor with respect to the Condon--Shortley phase,
\begin{equation}\label{eq120}
Y_m^l(\theta,\phi)=i^l \sqrt{\frac{2l+1}{4\pi}\frac{(l-m)!}{(l+m)!}}P_l^m(\cos \theta)e^{im\phi}.
\end{equation}
\subsection{Addition theorem}
The addition theorem for the spherical harmonics states that
\begin{equation}\label{eq129}
P_l(\cos \theta_{12})=\frac{4\pi}{2l+1}\sum_m Y_m^l(\mathbf{r}_1)Y_m^{l*}(\mathbf{r}_2),
\end{equation}
where $\theta_{12}$ is the angle between the vectors $\mathbf{r}_1$ and $\mathbf{r}_2$. This result is independent of the phase convention. With \emph{time--reversed phases},
\begin{equation}\label{eq130}
P_l(\cos \theta_{12})=\frac{4\pi}{\sqrt{2l+1}}\left[ Y^{l}(\hat {\mathbf{r}}_1)Y^{l}(\hat {\mathbf{r}}_2) \right]^{0}_{0}.
\end{equation}
With \emph{Condon--Shortley phases},
\begin{equation}\label{eq131}
P_l(\cos \theta_{12})=(-1)^l\frac{4\pi}{\sqrt{2l+1}}\left[ Y^{l}(\hat {\mathbf{r}}_1)Y^{l}(\hat {\mathbf{r}}_2) \right]^{0}_{0}.
\end{equation}
\subsection{Expansion of the delta function}
The Dirac delta function can be expanded in multipoles, yielding
\begin{equation}\label{eq132}
\begin{split}
\delta(\mathbf{r}_2-\mathbf{r}_1)=\sum_l \delta(r_1&-r_2)\frac{2l+1}{4\pi r_{1}^2}P_l(\cos \theta_{12})\\
&=\sum_l \delta(r_1-r_2)\frac{1}{r_{1}^2}\sum_m Y_m^l(\mathbf{r}_1)Y_m^{l*}(\mathbf{r}_2).
\end{split}
\end{equation}
This result is independent of the phase convention. With \emph{time--reversed phases},
\begin{equation}\label{eq133}
\delta(\mathbf{r}_2-\mathbf{r}_1)=\sum_l \delta(r_1-r_2)\frac{\sqrt{2l+1}}{r_{1}^2}\left[ Y^{l}(\hat {\mathbf{r}}_1)Y^{l}(\hat {\mathbf{r}}_2) \right]^{0}_{0}.
\end{equation}
\subsection{Coupling and complex conjugation}
If $\Psi^{I_1*}_{M_1}=(-1)^{I_1-M_1}\Psi^{I_1}_{-M_1}$ and $\Phi^{I_2*}_{M_2}=(-1)^{I_2-M_2}\Phi^{I_2}_{-M_2}$, as it happens to be the case for spherical harmonics with time-reversed phases, then

\begin{equation}\label{eq9}
\begin{split}
\left[ \Psi^{I_1}\Phi^{I_2}\right]^{I*}_M&=\sum_{\substack{M_1M_2\\(M_1+M_2=M)}}
\langle I_1\; I_2 \; M_1\;M_2 | I M\rangle \Psi^{I_1*}_{M_1} \Phi^{I_2*}_{M_2} \\
&= \sum_{\substack{M_1M_2\\(M_1+M_2=M)}} (-1)^{I-M_1-M_2}
\langle I_1\; I_2 \; -M_1\;-M_2 | I -M\rangle \Psi^{I_1}_{-M_1}\Phi^{I_2}_{-M_2}\\
&= (-1)^{I-M} \sum_{\substack{M_1M_2\\(M_1+M_2=M)}}
\langle I_1\; I_2 \; -M_1\;-M_2 | I -M\rangle \Psi^{I_1}_{-M_1}\Phi^{I_2}_{-M_2}\\
&=(-1)^{I-M} \left[ \Psi^{I_1}\Phi^{I_2}\right]^{I}_{-M},
\end{split}
\end{equation}
where we have used (\ref{eq28}).

Let us care now about the spinor functions $\chi^{1/2}_m(\sigma)$, which have the form
\begin{equation}\label{eq10}
\chi^{1/2}(\sigma=1/2)=\left[ \begin{aligned} 1\\0 \end{aligned}\right]  \quad
\chi^{1/2}(\sigma=-1/2)=\left[ \begin{aligned} 0\\1 \end{aligned}\right],
\end{equation}
or
\begin{equation}\label{eq11}
\chi^{1/2}_m(\sigma)=\delta_{m,\sigma}.
\end{equation}
Thus, $\, \chi^{1/2*}_m(\sigma)=\chi^{1/2}_m(\sigma)=\delta_{m,\sigma}$, but we can also write
\begin{equation}\label{eq12}
\chi^{1/2*}_m(\sigma)=(-1)^{1/2-m+1/2-\sigma}\chi^{1/2}_{-m}(-\sigma).
\end{equation}
This trick enable us to write
\begin{equation}\label{eq13}
\left[ Y^{l}(\hat r)\chi^{1/2}(\sigma)\right]^{J*}_{M}=(-1)^{1/2-\sigma+J-M}
\left[ Y^{l}(\hat r)\chi^{1/2}(-\sigma)\right]^{J}_{-M},
\end{equation}
which can be derived in a similar way as (\ref{eq9}).

\subsection{Angular momenta coupling}
Relation between Clebsh--Gordan and $3j$ coefficients:
\begin{equation}\label{eq204}
\langle j_1\; j_2 \; m_1\;m_2 | J M\rangle=(-1)^{j_1-j_2+M}\sqrt{2J+1}\begin{pmatrix}
  j_1&j_2&J\\
 m_1&m_2&-M\\
\end{pmatrix}.
\end{equation}
Relation between Wigner and $9j$ coefficients:
\begin{equation}\label{eq205}
\begin{split}
\bigl ( (j_1j_2)_{j_{12}}& (j_3j_4)_{j_{34}} |(j_1j_3)_{j_{13}} (j_2j_4)_{j_{24}}) \bigr )_j=\\
&\sqrt{(2j_{12}+1)(2j_{13}+1)(2j_{24}+1)(2j_{34}+1)}\begin{Bmatrix}
  j_1&j_2&j_{12}\\
 j_3&j_4&j_{34}\\
 j_{13}&j_{24}&j\\
\end{Bmatrix}.
\end{split}
\end{equation}
\subsection{Integrals}
Let us now prove
\begin{equation}\label{eq14}
\int d\Omega \left[ Y^{l}(\hat r)Y^{l}(\hat r) \right]^{I}_{M}=\delta_{M,0}\delta_{I,0}\sqrt{2l+1}.
\end{equation}

\begin{equation}\label{eq15}
\begin{split}
\int d\Omega \left[ Y^{l}(\hat r)Y^{l}(\hat r) \right]^{I}_{M}&= \sum_{\substack{m_1,m_2\\(m_1+m_2=M)}}
\langle l\; l \; m_1\;m_2 | I M\rangle \int d\Omega  Y_{m_1}^{l}(\hat r)Y_{m_2}^{l}(\hat r)\\
&=\sum_{\substack{m_1,m_2\\(m_1+m_2=M)}} (-1)^{l+m_1} \langle l\; l \; -m_1\;m_2 | I M\rangle
\int d\Omega  Y_{m_1}^{l*}(\hat r)Y_{m_2}^{l}(\hat r)\\
&= \delta_{M,0}\sum_m (-1)^{l+m} \langle l\; l \; -m\;m | I 0\rangle \\
&= \delta_{M,0}\sqrt{2l+1}\sum_m  \langle l\; l \; -m\;m | I 0\rangle \langle l\; l \; -m\;m | 0 0\rangle \\
&= \delta_{M,0}\delta_{I,0}\sqrt{2l+1},
\end{split}
\end{equation}
where we have used
\begin{equation}\label{eq22}
\langle l \;l \; -m \;m | 0\; 0 \rangle=\frac{(-1)^{l+m}}{\sqrt{2l+1}}
\end{equation}

Let us now prove
\begin{equation}\label{eq16}
\sum_\sigma \int d\Omega (-1)^{1/2-\sigma}
\left[ \Psi^{j}(\hat r,-\sigma)\Psi^{j}(\hat r,\sigma) \right]^{I}_{M}=-\delta_{M,0}\delta_{I,0}\sqrt{2j+1}.
\end{equation}
\begin{equation}\label{eq17}
\begin{split}
\sum_\sigma \int &d\Omega (-1)^{1/2-\sigma}
\left[ \Psi^{j}(\hat r,-\sigma)\Psi^{j}(\hat r,\sigma) \right]^{I}_{M}\\
&=\sum_{\substack{m_1,m_2\\(m_1+m_2=M)}} \langle j\; j \; m_1\;m_2 | I M\rangle \sum_\sigma
 \int d\Omega \Psi_{m_1}^{j}(\hat r,-\sigma)\Psi_{m_2}^{j}(\hat r,\sigma)\\
&=\sum_{\substack{m_1,m_2\\(m_1+m_2=M)}} \langle j\; j \; m_1\;m_2 | I M\rangle \sum_\sigma (-1)^{j+m_1}
 \int d\Omega \Psi_{-m_1}^{j*}(\hat r,\sigma)\Psi_{m_2}^{j}(\hat r,\sigma)\\
&=\sum_{\substack{m_1,m_2\\(m_1+m_2=M)}} \langle j\; j \; m_1\;m_2 | I M\rangle  (-1)^{j+m_1}
 \delta_{-m_1,m_2}\\
&= \delta_{M,0} \sum_m (-1)^{j+m} \langle j\; j \; m\;-m | I 0\rangle  \\
&= -\delta_{M,0}\sqrt{2j+1} \sum_m (-1)^{j+m} \langle j\; j \; m\;-m | I 0\rangle
\langle j\; j \; m\;-m | 0 0\rangle   \\
&=-\delta_{M,0}\delta_{I,0}\sqrt{2j+1}.
\end{split}
\end{equation}

\subsection{Symmetry properties}
Note also another useful property
\begin{equation}
\left[ \Psi^{I_1}\Psi^{I_2}\right]^{I}_M=(-1)^{I_1+I_2-I}\left[ \Psi^{I_2}\Psi^{I_1}\right]^{I}_M,
\end{equation}
by virtue of the symmetry property of the Clebsh-Gordan coefficients
\begin{equation}\label{eq59}
\langle I_1\; I_2 \; m_1\;m_2 | I M\rangle= (-1)^{I_1+I_2-I} \langle I_2\; I_1 \; m_2\;m_1 | I M\rangle.
\end{equation}
Here's another symmetry property of the Clebsh-Gordan coefficients
\begin{equation}\label{eq28}
\langle I_1\; I_2 \; m_1\;m_2 | I M\rangle= (-1)^{I_1+I_2-I} \langle I_1\; I_2 \; -m_2\;-m_1 | I -M\rangle.
\end{equation}
Another one, which can be derived from the simpler properties of $3j$--symbols
\begin{equation}\label{eq239}
\langle I_1\; I_2 \; m_1\;m_2 | I M\rangle= (-1)^{I_1-m_1}\sqrt{\frac{2I+1}{2I_2+1}} \langle I_1\; I \; m_1\;-M | I_2 m_2\rangle.
\end{equation}
Let us use this last property to calculate sums of the type
\begin{equation}\label{eq240}
\sum_{m_1,m_3}\left|\langle I_1\; I_2 \; m_1\;m_2 | I_3 m_3\rangle \right|^2.
\end{equation}
Using (\ref{eq239}), we have
\begin{multline}\label{eq241}
\sum_{m_1,m_3}\left|\langle I_1\; I_2 \; m_1\;m_2 | I_3 m_3\rangle \right|^2=\\
\frac{2I_3+1}{2I_2+1}\sum_{m_1,m_3}\left|\langle I_1\; I_3 \; m_1\;-m_3 | I_2 m_2\rangle \right|^2=
\frac{2I_3+1}{2I_2+1},
\end{multline}
since
\begin{equation}\label{eq242}
\sum_{m_1,m_3}\left|\langle I_1\; I_3 \; m_1\;-m_3 | I_2 m_2\rangle \right|^2=\sum_{m_1,m_3}\left|\langle I_1\; I_3 \; m_1\;m_3 | I_2 m_2\rangle \right|^2=1.
\end{equation}
\section{distorted waves}\label{C7AppE}
Let us have a closer look at the partial wave expansion of the distorted waves
 \begin{equation}\label{eqC7L1}
\chi^{(+)}(\mathbf{k},\mathbf{r})= \sum_{l}\frac{ 4\pi }{k r} i^{l}
e^{i\sigma^{l}} F_{l} \sum_m Y_m^{l} (\hat r) Y_m^{l*} (\hat k).
\end{equation}
Of notice the very important fact that \emph{this definition is independent of the phase convention}, since the $l$--dependent phase is multiplied by its complex conjugate.
\begin{equation}\label{eq34}
\chi^{(-)}(\mathbf{k},\mathbf{r})=\chi^{(+)*}(-\mathbf{k},\mathbf{r})= \sum_{l}\frac{ 4\pi }{k r} i^{-l}
e^{-i\sigma^{l}} F^*_{l} \sum_m Y_m^{l*} (\hat r) Y_m^{l} (-\hat k).
\end{equation}
As $Y_m^{l} (-\hat k)=(-1)^l Y_m^{l} (\hat k)$, we have
\begin{equation}\label{eq35}
\chi^{(-)}(\mathbf{k},\mathbf{r})= \sum_{l}\frac{ 4\pi }{k r} i^{l}
e^{-i\sigma^{l}} F^*_{l} \sum_m Y_m^{l*} (\hat r) Y_m^{l} (\hat k),
\end{equation}
which is also independent of the phase convention.
With time--reversed phase convention
 \begin{equation}\label{eq36}
\chi^{(+)}(\mathbf{k},\mathbf{r})= \sum_{l}\frac{ 4\pi }{k r} i^{l}\sqrt{2l+1}
e^{i\sigma^{l}} F_{l} \left[ Y^{l} (\hat r) Y^{l} (\hat k)\right]^0_0,
\end{equation}
while with Condon--Shortley phase convention we get an extra $(-1)^l$ factor:
 \begin{equation}\label{eq87}
\chi^{(+)}(\mathbf{k},\mathbf{r})= \sum_{l}\frac{ 4\pi }{k r} i^{-l}\sqrt{2l+1}
e^{i\sigma^{l}} F_{l} \left[ Y^{l} (\hat r) Y^{l} (\hat k)\right]^0_0.
\end{equation}
The partial--wave expansion of the Green function $G(\mathbf{r},\mathbf{r}')$ is
 \begin{equation}\label{eq134}
G(\mathbf{r},\mathbf{r}')= i\sum_{l}\frac{f_l(k,r_<)P_l(k,r_>)}{krr'} \sum_m Y_m^{l} (\hat r) Y_m^{l*} (\hat r'),
\end{equation}
where $f_l(k,r_<)$ and $P_l(k,r_>)$ are the regular and the irregular solutions of the homogeneous problem respectively.
With \emph{time--reversed} phase convention
 \begin{equation}\label{eq135}
G(\mathbf{r},\mathbf{r}')= i\sum_{l}\sqrt{2l+1}\frac{f_l(k,r_<)P_l(k,r_>)}{krr'} \left[ Y^{l} (\hat r) Y^{l} (\hat r')\right]^0_0.
\end{equation}
\section{hole states and time reversal}\label{C7AppM}
Let us consider the state $|(jm)^{-1}\rangle$ obtained by removing a $\psi_{jm}$ single--particle state from a $J=0$ closed shell $|0\rangle$. The antisymmetrized product state
 \begin{equation}\label{eq37}
\sum_m \mathcal{A}\{\psi_{jm}|(jm)^{-1}\rangle\} \propto |0\rangle
\end{equation}
is clearly proportional to $|0\rangle$. This gives us the transformation rules of  $|(jm)^{-1}\rangle$ under rotations, which must be such that, when multiplied by a $j,m$ spherical tensor and summed over $m$, yields a $j=0$ tensor. It can be seen that these properties imply that $|(jm)^{-1}\rangle$ transforms like $(-1)^{j-m}T_{j-m}$, $T_{j-m}$ being a spherical tensor. It also follows that the hole state $|(j\bar m)^{-1}\rangle$ transforms like a $j,m$ spherical tensor if $\psi_{j\bar m}$ is defined as the $\mathcal{R}$--conjugate to $\psi_{j m}$ by the relation
 \begin{equation}\label{eq38}
\psi_{j\bar m} \equiv  (-1)^{j+m} \psi_{j-m}.
\end{equation}
In other words, with the latter definition a \emph{hole state} transforms under rotations with the right phase.
We will now show that $\mathcal{R}$--conjugation is equivalent to a rotation of spin and spatial coordinates through an angle $-\pi$ about the $y$--axis:
 \begin{equation}\label{eq39}
e^{i\pi J_y} \psi_{jm}=(-1)^{j+m}\psi_{j-m}\equiv \psi_{j\bar m}.
\end{equation}
Let us begin by calculating $e^{i\pi L_y} Y_l^m$. The rotation matrix about the $y$--axis is
 \begin{equation}\label{eq40}
 R_y(\theta)=
\begin{pmatrix}
  \cos(\theta) & 0 & \sin(\theta) \\
  0& 1 & 0 \\
  -\sin(\theta) & 0 & \cos(\theta) \\
\end{pmatrix},
\end{equation}
so for $R_y(-\pi)$ we get
 \begin{equation}\label{eq41}
 R_y(-\pi)=
\begin{pmatrix}
  -1 & 0 & 0 \\
  0& 1 & 0 \\
  0 & 0 & -1 \\
\end{pmatrix}.
\end{equation}
When applied to the generic direction $(\sin(\theta)\cos(\phi),\sin(\theta)\sin(\phi),\cos(\theta))$, we obtain $(-\sin(\theta)\cos(\phi),\sin(\theta)\sin(\phi),-\cos(\theta))$, which corresponds to making the substitutions
 \begin{equation}\label{eq42}
\theta\rightarrow \pi-\theta, \quad \phi\rightarrow \pi-\phi.
\end{equation}
When we substitute these angular transformations in the spherical harmonic $Y_l^m(\theta,\phi)$,  we obtain the rotated $Y_l^m(\theta,\phi)$:
 \begin{equation}\label{eq43}
e^{i\pi L_y} Y_{l}^m=(-1)^{l+m}Y_{l}^{-m}.
\end{equation}
Let us now turn our attention to the spin coordinates rotation $e^{i\pi s_y}\chi_m$. The rotation matrix in spin space is
\begin{equation}\label{eq44}
\begin{pmatrix}
  \cos(\theta/2) & -\sin(\theta/2) \\
 \sin(\theta/2) & \cos(\theta/2) \\
\end{pmatrix},
\end{equation}
which, for $\theta=-\pi$ is
\begin{equation}\label{eq45}
\begin{pmatrix}
  0 & 1 \\
 -1 & 0 \\
\end{pmatrix}.
\end{equation}
Applying it to the spinors, we find the rule
\begin{equation}\label{eq46}
e^{i\pi s_y}\chi_m=(-1)^{1/2+m}\chi_{-m},
\end{equation}
so
\begin{equation}\label{eq47}
\begin{split}
e^{i\pi J_y} \psi_{jm}&=\sum_{m_lm_s}\langle l \; m_l \;1/2\; m_s|j\; m\rangle\; e^{i\pi L_y}Y_{l}^{m_l}\;e^{i\pi s_y}\chi_{m_s}\\
&=\sum_{m_lm_s}(-1)^{1/2+m_s+l+m_l}\langle l \; m_l \;1/2\; m_s|j\; m\rangle\; Y_{l}^{-m_l}\;\chi_{-m_s}\\
&=\sum_{m_lm_s}(-1)^{1+m-j+2l}\langle l \; -m_l \;1/2\; -m_s|j\; -m\rangle\; Y_{l}^{-m_l}\;\chi_{-m_s}\\
&=(-1)^{m+j}\psi_{j-m}\equiv \psi_{j\bar m},
\end{split}
\end{equation}
where we have used  $(-1)^{1+m-j+2l}=-(-1)^{m-j}=(-1)^{m+j}$, as $j,m$ are always half--integers and $l$ is always an integer.



We now turn our attention to the time reversal operation, which amounts to the transformations
\begin{equation}\label{eq48}
\mathbf{r} \rightarrow \mathbf{r}, \quad \mathbf{p} \rightarrow -\mathbf{p}.
\end{equation}
This is enough to define the operator of time reversal of a spinless particle (see Messiah). In the position representation, in which $\mathbf{r}$ is real and $\mathbf{p}$ pure imaginary, this (unitary antilinear) operator is the complex conjugation operator.


As angular momentum $\mathbf{l}=\mathbf{r}\times\mathbf{p}$ changes sign under time reversal, so does spin:
\begin{equation}\label{eq49}
\mathbf{s} \rightarrow -\mathbf{s},
\end{equation}
which, along with (\ref{eq48}), completes the set of rules that define the time reversal operation on a particle with spin. In the representation of eigenstates of $\mathbf{s}^2$ and $s_z$, complex conjugation alone changes only the sign of $s_y$, so an additional rotation of $-\pi$ around the $y$--axis is necessary to change the sign of $s_x,s_z$ and implement the transformation ($\ref{eq49}$). If we call $K$ the time--reversal operator, we have
\begin{equation}\label{eq50}
K \psi_{jm}=e^{i\pi s_y} \psi_{jm}^*.
\end{equation}
This is completely general and independent of the phase convention. It only depends on the fact that we have used the $\mathbf{r}$ representation for the spatial wave function and the representation of the eigenstates of $\mathbf{s}^2$ and $s_z$ for the spin part. \emph{If we use time--reversal phases for the spherical harmonics} (see(\ref{eq21})),
\begin{equation}\label{eq51}
Y_m^{l*}=(-1)^{l+m}Y_{-m}^{l}=e^{i\pi L_y} Y_{m}^{l}.
\end{equation}
So we can write
\begin{equation}\label{eq52}
K \psi_{jm}=e^{i\pi J_y}\psi_{jm}=\psi_{j\bar m}.
\end{equation}
Note again that this last expression is valid only if we use time--reversal phases for the spherical harmonics. Only in this case time--reversal coincides with $\mathcal{R}$--conjugation and hole states.


In BCS theory, the quasiparticles are defined in terms of linear combinations of particles and holes. With time--reversal phases, holes are equivalent to time--reversed states, and we get the definitions
\begin{equation}\label{eq53}
    \begin{split}
        \alpha^\dagger_\nu=u_\nu a_\nu^\dagger - v_\nu a_{\bar \nu} & \quad
        a^\dagger_\nu=u_\nu \alpha_\nu^\dagger + v_\nu \alpha_{\bar \nu} \\
         \alpha^\dagger_{\bar \nu}=u_\nu a_{\bar \nu}^\dagger + v_\nu a_\nu & \quad
         a^\dagger_{\bar \nu}=u_\nu \alpha_{\bar \nu}^\dagger - v_\nu \alpha_\nu \\
         \alpha_\nu=u_\nu a_\nu - v_\nu a_{\bar \nu}^\dagger & \quad
         a_\nu=u_\nu \alpha_\nu + v_\nu \alpha_{\bar \nu}^\dagger\\
         \alpha_{\bar \nu}=u_\nu a_{\bar \nu} + v_\nu a^\dagger_\nu & \quad
         a_{\bar \nu}=u_\nu \alpha_{\bar \nu} - v_\nu \alpha^\dagger_\nu
     \end{split}
\end{equation}
\section{Spectroscopic amplitudes in the BCS approximation}\label{C7AppN}
The creation operator of a pair of fermions coupled to $J,M$ can be expressed in second quantization as
\begin{equation}\label{eq54}
T^\dagger(j_1,j_2,J M)=N\sum_m \langle j_1 \; m \;j_2 \;M-m|J \; M\rangle \;a^\dagger_{j_1m}a^\dagger_{j_2M-m},
\end{equation}
where $N$ is a normalization constant. To determine it, we write the wave function resulting from the action of (\ref{eq54}) on the vacuum
\begin{equation}\label{eq55}
\begin{split}
  \Psi=T^\dagger(j_1,j_2,J M) |0\rangle &= \frac{N}{\sqrt 2}\sum_m \langle j_1 \; m \;j_2 \;M-m|J \; M\rangle\\
    & \times\left(\phi_{j_1 m}(\mathbf{r_1})\phi_{j_2 M-m}(\mathbf{r_2})-\phi_{j_2 M-m}(\mathbf{r_1})\phi_{j_1,m}(\mathbf{r_2})\right).
\end{split}
\end{equation}
 The norm is
 \begin{equation}\label{eq56}
\begin{split}
  |\Psi|^2&= \frac{N^2}{2}\sum_{mm'} \langle j_1 \; m \;j_2 \;M-m|J \; M\rangle \langle j_1 \; m' \;j_2 \;M-m'|J \; M\rangle\\
    & \times \left(\phi_{j_1 m}(\mathbf{r_1})\phi_{j_2 M-m}(\mathbf{r_2})-\phi_{j_2 M-m}(\mathbf{r_1})\phi_{j_1,m}(\mathbf{r_2})\right)\\
    &\times \left(\phi_{j_1 m'}(\mathbf{r_1})\phi_{j_2 M-m'}(\mathbf{r_2})-\phi_{j_2 M-m'}(\mathbf{r_1})\phi_{j_1,m'}(\mathbf{r_2})\right).
\end{split}
\end{equation}
Integrating we get
 \begin{equation}\label{eq57}
\begin{split}
  1&= \frac{N^2}{2}\sum_{mm'} \langle j_1 \; m \;j_2 \;M-m|J \; M\rangle \langle j_1 \; m' \;j_2 \;M-m'|J \; M\rangle\\
    & \times \left(2\delta_{m,m'}-2\delta_{j_1,j_2}\delta_{m,M-m'}\right)\\
    &=N^2\left(\sum_m \langle j_1 \; m \;j_2 \;M-m|J \; M\rangle^2\right.\\
    &\left.- \delta_{j_1,j_2}\sum_m \langle j_1 \; m \;j_2 \;M-m|J \; M\rangle \langle j_1 \; M-m \;j_2 \;m|J \; M\rangle\right)\\
    &=N^2\left(1-\delta_{j_1,j_2}(-1)^{2j-J}\right),
\end{split}
\end{equation}
where we have used the closure condition for Clebsh--Gordan coefficients and (\ref{eq59}), and $\delta_{j_1,j_2}$ must be interpreted as a $\delta$ function regarding all the quantum numbers but the magnetic one.
We see that two fermions with identical quantum numbers (but the magnetic one) \emph{cannot couple to $J$ odd}. If $J$ is even, the normalization constant is
\begin{equation}\label{eq58}
N=\frac{1}{\sqrt{1+\delta_{j_1,j_2}}}.
\end{equation}
To sum up,
\begin{equation}\label{eq60}
T^\dagger(j_1,j_2,J M)=\frac{1}{\sqrt{1+\delta_{j_1,j_2}}}\sum_m \langle j_1 \; m \;j_2 \;M-m|J \; M\rangle \;a^\dagger_{j_1m}a^\dagger_{j_2M-m}.
\end{equation}


The spectroscopic amplitude \idx{BCS!pair transfer amplitudes} for finding in a $A+2,J_f,M_f$ nucleus a couple of nucleons with quantum numbers $j_1,j_2$ coupled to $J$ on top of a $A,J_i$ nucleus is
\begin{equation}\label{eq61}
B(j_1,j_2(J))=\sum_{M,M_i} \langle J_i \; M_i \; J M|J_f \; M_f\rangle \langle \Psi_{J_f M_f}|T^\dagger(j_1,j_2,J M)|\Psi_{J_i M_i}\rangle.
\end{equation}
This is completely general. It depends on the structure model only through the way the $A+2$ and $A$ nuclei are treated. We now want to turn our attention to the expression of $B(J,j_1,j_2)$ in the BCS approximation when both the $A+2$ and the $A$ are $0^+$, zero--quasiparticle ground states. In order to do this, we write (\ref{eq60}) in terms of quasiparticle operators using (\ref{eq53})\footnote{In what follows, we use the phase convention $\alpha_{j \bar m}=(-1)^{j-m}\alpha_{j-m}$ instead of $\alpha_{j \bar m}=(-1)^{j+m}\alpha_{j-m}$, consistent with (\ref{eq38}). Had we stick to the definition (\ref{eq38}), the amplitude $B(0,j,j)$ calculated below would have a minus sign, which would not have any physical consequence.}:
\begin{equation}\label{eq62}
    \begin{split}
       T^\dagger(j_1,j_2,J M)= & \frac{1}{\sqrt{1+\delta_{j_1,j_2}}}\sum_{m_1,m_2} \langle j_1 \; m_1 \;j_2 \;m_2|J \; M\rangle \left( U_{j_1}U_{j_2}\alpha_{j_1m_1}^\dagger\alpha_{j_2m_2}^\dagger\right.\\
         &  +(-1)^{j_1+j_2-M}V_{j_1}V_{j_2}\alpha_{j_1-m_1}\alpha_{j_2-m_2}\\
         &+(-1)^{j_2-m_2}U_{j_1}V_{j_2}\alpha_{j_1m_1}^\dagger
         \alpha_{j_2-m_2}\\
         &-(-1)^{j_1-m_1} V_{j_1}U_{j_2}\alpha_{j_2m_2}^\dagger\alpha_{j_1-m_1}\\
         &\left.+(-1)^{j_1-m_1} V_{j_1}U_{j_2}\delta_{j_1j_2}\delta_{-m_1m_2}\right).
     \end{split}
\end{equation}
If both nuclei are in the $\ket{BCS}$ ground state (zero-quasiparticles state), the only term that survives is the last one in the above expression, and (\ref{eq61}) becomes
\begin{equation}\label{eq63}
    \begin{split}
     B_j=B(j^2(0))=& \frac{1}{\sqrt 2}\sum_{m} \langle j \; m \; j\;-m |0 \; 0\rangle (-1)^{j-m} V_jU_j\\
         &=\frac{1}{\sqrt 2}\sum_{m} \frac{(-1)^{j-m}}{\sqrt{(2j+1)}} (-1)^{j-m} V_jU_j\\
         &=\frac{1}{\sqrt 2}\sum_{m} \frac{1}{\sqrt{(2j+1)}}V_jU_j.
     \end{split}
\end{equation}
After carrying out the summation one finds, \idx{BCS!pair transfer amplitudes}
\begin{equation}\label{eq64}
 B_j=B(j^2(0))=\sqrt{j+1/2}\;V_jU_j.
\end{equation}
Note that in this final expression $V_j$ refers to the $A+2$ nucleus, while $U_j$ is related to the $A$ nucleus. In practice, it does not make a big difference to calculate both for the same nucleus.

\section[Bayman's two-nucleon transfer amplitudes]{Derivation of two-nucleon transfer transition amplitudes including recoil, non-orthogonality and successive transfer.}\label{C7AppO}
In the present Appendix we reproduce with the permission of the author the first (manuscript) page (Fig. \ref{fig5.h.1}) of what, arguably, was the first complete derivation\footnote{\cite{Bayman:70} (unpublished).} of the different contributions needed to calculate absolute two--nucleon transfer cross sections in a systematic way\footnote{ \cite{Bayman:71} and \cite{Bayman:82}). Within this context we refer to \cite{Broglia:73} and \cite{Potel:13} in particular Fig. 10 of this reference.}. 
\begin{figure}
\centerline{\includegraphics*[width=17cm,angle=0]{C7/figs_C7/fig5_H_1.pdf}}
\caption{First manuscript page of Ben Bayman's derivation of the two--nucleon transfer reaction amplitude, in second order DWBA approximation.}\label{fig5.h.1}
\end{figure}
\end{subappendices}
%\renewcommand{\bibname}{Bibliography Ch 6}
%\bibliographystyle{abbrvnat}
%\bibliography{../nuclear_bib.bib}