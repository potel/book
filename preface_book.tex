	\chapter{Introduction}\label{introduction}
	 \epigraph{Toute th\'eorie physique est fond\'ee sur l'analogie qu'on \'etabli entre des choses malconnues et des choses simples\footnote{Le\c{c}ons de philosophie, Plon, Paris (1959).}}{Simone Weil}
\section{Views of the nucleus}
In the atom, the nucleus provides the Coulomb field in which negatively charged electrons $(-e)$ move independently of each other in single--particle orbitals. The filling of these orbitals explains Mendeleev's periodic table. Thus the valence of the chemical elements as well as the particular stability of the noble gases associated with the closing of shells (2(He), 10(Ne), 18(Ar), 36(Kr), 54(Xe), 86(Ra)). The dimension of the atom is measured in angstroms (\AA=$10^{-8}$cm), and typical energies in eV, the electron mass being $m_e\approx 0.5$ MeV (MeV=$10^6$eV).


The atomic nucleus is made out of positively charged protons ($+e$) and of (uncharged) neutrons, nucleons, of mass $\approx 10^3$ MeV ($m_p=938.3$ MeV, $m_n=939.6$ MeV). Nuclear dimensions are of the order of few fermis (fm$=10^{-13}$ cm). While the stability of the atom is provided by a source external to the electrons, namely the atomic nucleus, this system is  self--bound as a result of the strong interaction of range $a_0\approx 0.9$ fm and strength $v_0\approx -100$ MeV acting among nucleons. 
\subsection{The liquid drop and the shell models}\label{S1.1.1}
While most of the atom is empty space, the density of the atomic nucleus is conspicuous ($\rho=0.17$ nucleon/fm$^3$). The ``closed packed'' nature of this system implies a short mean free path as compared to nuclear dimensions. This can be estimated from classical kinetic theory $\lambda\approx(\rho\sigma)^{-1}\approx1$ fm, where $\sigma\approx 2\pi a_0^2$ is the nucleon--nucleon cross section. It seems then natural to liken the atomic nucleus to a liquid drop (Bohr and Kalckar).
This picture of the nucleus provided the framework to describe the basic features of the fission process\footnote{\cite{Meitner:39,Bohr:39}.}. 



The leptodermic properties of the atomic nucleus are closely connected with the semi--empirical mass formula\footnote{\cite{Weizsacker:35}.}.
\begin{align}
m(N,Z)=(Nm_n+Zm_p)-\frac{1}{c^2}B(N,Z),
\end{align}
the binding energy being
\begin{align}\label{eq1.0.2}
B(N,Z)=\left(b_{vol}A-b_{surf}A^{2/3}-\frac{1}{2} b_{sym}\frac{(N-Z)^2}{A}-\frac{3}{5}\frac{Z^2e^2}{R_c}\right).
\end{align}
The first term is the volume energy representing the binding energy in the limit of large $A$, for $N=Z$ and in the absence of the Coulomb interaction ($b_{vol}\approx15.6$ MeV) . The second term represents the surface energy, where
\begin{align}\label{eq1.0.3}
b_{surf}=4\pi r_0^2\gamma.
\end{align}
The nuclear radius is written as $R=r_0A^{1/3}$, with $r_0=1.2$ fm, the surface tension energy being $\gamma\approx 0.95$ MeV/fm$^2$. 
The third term in (\ref{eq1.0.2}) is the symmetry term which reflects the tendency towards stability for $N=Z$, with $b_{sym}\approx50$ MeV. The symmetry energy can be divided into a kinetic and a potential energy part. A simple estimate of the kinetic energy part can be obtained by making use of the Fermi gas model which gives $(b_{sym})_{kin}\approx(2/3)\epsilon_F\approx25$ MeV ($\epsilon_F\approx 36$ MeV). Consequently,
\begin{align}\label{eq1.0.4bis}
V_1=(b_{sym})_{pot}=b_{sym}-(b_{sym})_{kin}\approx 25\text{ MeV}.
\end{align}
The last term of (\ref{eq1.0.2}) is the Coulomb energy corresponding to a uniformly charged sphere of radius $R_c=1.25\,A^{1/3}$ fm.


When, in a heavy-ion reaction,  two nuclei come within the range of the nuclear forces, the Coulomb  trajectory of relative motion will be changed by the attraction which will act between the nuclear surfaces. This surface interaction is a fundamental quantity in all heavy ion reactions. Assuming two spherical nuclei at a relative distance $r_{aA}=R_a+R_A$, where $R_a$ and $R_A$ are the corresponding half--density radii, the force acting between the two surfaces is
\begin{align}\label{eq1.0.4}
\left(\frac{\partial U_{aA}^N}{\partial r}\right)_{r_{aA}}=4\pi \gamma\frac{R_aR_A}{R_a+R_A}
\end{align}
This result allows for the calculation of the ion-ion (proximity) potential which, supplemented with a position dependent absorption, can be used to accurately describe heavy ion reactions\footnote{\cite{Broglia:04a} and refs. therein.}.


In such reactions, not only elastic processes are observed, but also anelastic reactions in which one, or both  surfaces of the interacting nuclei are set into vibration (Fig. \ref{fig1.0.2}). The restoring force parameter associated with oscillations of multipolarity $\lambda$ is 
\begin{align}\label{eq1.0.4b}
C_\lambda=(\lambda-1)(\lambda+2)R_0^2\gamma-\frac{3}{2\pi}\frac{\lambda-1}{2\lambda+1}\frac{Z^2e^2}{R_c},
\end{align}
where the second term corresponds to the contribution of the Coulomb energy to $C_\lambda$. Assuming the flow associated with surface vibration to be irrotational, the associated inertia for small amplitude oscillations is, 
\begin{align}\label{eq1.0.5}
D_{\lambda}=\frac{3}{4\pi}\frac{1}{\lambda}AMR^2,
\end{align}
the energy of the corresponding mode being
\begin{align}\label{eq1.0.6}
\hbar\omega_\lambda=\hbar\sqrt{\frac{C_\lambda}{D_\lambda}}.
\end{align}
The label $\lambda$ stands for the angular momentum of the vibrational mode. Furthermore, the vibrations can be characterized by the parity quantum number $\pi=(-1)^\lambda$ and the third component of $\lambda$, denoted $\mu$  (see Eq. (\ref{eq1.0.12})). Aside from $\lambda,\mu$, surface vibrations can also be characterized by an integer $n(=1,2,\dots)$, an ordering number indicating increasing energy. For simplicity, a single common label $\alpha$ will  also be used.


Experimental information associated with low--energy quadrupole vibrations, namely $\hbar\omega_{2}$ and the electromagnetic transition probabilities $B(E2)$, allows to determine $C_2$ and $D_2$. The resulting $C_2$ values exhibit variations by more than a factor of 10 both above and below the liquid--drop estimate. The observed values of $D_2$ are large as compared with the mass parameter for irrotational flow, a fact connected with the role played by pairing in nuclei\footnote{See footnote \ref{foot2}; see also \cite{Bohr:75} p. 75.}.

A picture apparently antithetic to that of the liquid drop, the shell model, emerged from the study of experimental data, plotting them against either the number of protons (atomic number), or the number of neutrons in the nuclei, rather than against the mass number.
One of the main nuclear features which led to the development of the shell model was the study of the stability and abundance of nuclear species and the discovery of what are usually called magic numbers \footnote{\cite{Elsasser:33,Mayer:48,Haxel:49}.}. What makes a number magic is that a configuration of a magic number of neutrons, or of protons, is unusually stable whatever the associated number of other nucleons is\footnote{\cite{Mayer:49,Mayer:49b}.}.


The strong binding of a magic number of nucleons and weak binding for one more, reminds the results  concerning the atomic stability o\bibliographystyle{abbrvnat}f rare gases. In the nuclear case,  the spin-orbit coupling play an important role, as can be seen from the level scheme shown in Fig. \ref{fig1.0.3}, obtained by assuming that nucleons move independently of each other in an average potential  of  spherical symmetry.


A closed shell, or a filled level, has angular momentum zero. Thus, nuclei with one nucleon outside (missing from) closed shell, should have the spin and parity of the orbital associated with the odd nucleon (--hole), a prediction confirmed by the data (available at that time) throughout the mass table. Such a picture implies that the nucleon mean free path is large compared to nuclear dimensions.


The systematic studies of the binding energies leading to the shell model found also that the relation (\ref{eq1.0.2}), has to be supplemented to take into account the fact that nuclei with both odd number of protons and of neutrons are energetically unfavored compared with even--even ones by a quantity of the order of $\delta\approx33\text{ MeV}/A^{3/4}$ called the pairing energy\footnote{\label{foot2} \cite{Mayer:55} p.9. Connecting with further developments associated with the BCS theory of superconductivity (\cite{Bardeen:57a,Bardeen:57b}) and its extension to the atomic nucleus (\cite{Bohr:58}), the quantity $\delta$ is identified with the pairing gap $\Delta$ parametrized according to $\Delta=12 $MeV$/\sqrt{A}$ (\cite{Bohr:69}). It is of notice that for typical superfluid nuclei like $^{120}$Sn, the expression of $\delta$ leads to a numerical value which can be parametrized as  $\delta\approx33\text{ MeV}/(A^{1/4}\times A^{1/2})\approx10$ MeV$/\sqrt{A}$.} and at the basis of the odd-even staggering effect.

\subsection{Nuclear excitations}\label{S1.1.2}
In addition to the quantum numbers $\lambda$,  $\mu$ and $\pi$, one can characterize nuclear excitations by additional quantum numbers such as isospin $\tau$ and spin $\sigma$. Furthermore one can assign a particle (baryon or transfer) quantum number $\beta$. For a nucleon moving above the Fermi surface one has $\beta=+1,$ while for a hole in the Fermi sea $\beta=-1$. For (quasi-) bosonic excitations $\beta=0$ for a mode associated with e.g. surface oscillation, which can also be viewed as a correlated particle-hole excitation (within this context see Fig. \ref{fig1.0.7}) In particular, the low-lying quadrupole vibrations of even-even nuclei have quantum numbers $\lambda^\pi=2^+$, $\tau=0$ (protons and neutrons oscillate in phase) and $\sigma=0$ (no spin-flips in the excitation).

For modes which involve the addition or substraction of two correlated nucleons to the nucleus, $\beta=+2$ (Fig. \ref{fig0.3.1}) and $\beta=-2$ respectively. The excitation which connects the ground state of an even nucleus, to the ground state of the next even nucleus, i.e. monopole pairing vibration (e.g. $\lambda^\pi=0^+, \beta=+2$) is of this type.
The structure of excitations with $|\beta|>2$, expresses the various possible clustering effects of corresponding order.

The low--lying excited state of closed shell nuclei can be interpreted as a rule, as a harmonic quadrupole or octupole collective vibration (Fig. \ref{fig1.0.4}) described by the Hamiltonian\footnote{Classically $\Pi_{\lambda\mu}=D_\lambda\dot\alpha_{\lambda\mu}$.}
\begin{align}\label{eq1.0.7}
H_{coll}=\sum_{\lambda\mu}\left(\frac{1}{2D_{\lambda}}|\Pi_{\lambda\mu}|^2+\frac{C_\lambda}{2}|\alpha_{\lambda\mu}|^2\right)
\end{align}
Following \cite{Dirac:26} one can describe the oscillatory motion introducing boson creation (annihilation) operator $\Gamma_{\lambda\mu}^\dagger$ ($\Gamma_{\lambda\mu}$) obeying
\begin{align}\label{eq1.0.8}
\left[\Gamma_{\alpha},\Gamma_{\alpha'}^\dagger\right]=\delta(\alpha,\alpha'),
\end{align}
leading to 
\begin{align}\label{eq1.0.9}
\hat\alpha_{\lambda\mu}=\sqrt{\frac{\hbar\omega_\lambda}{2C_\lambda}}\left(\Gamma_{\lambda\mu}^\dagger+(-1)^\mu\Gamma_{\lambda-\mu}\right),
\end{align}
and a similar expression for the conjugate momentum variable $\hat\Pi_{\lambda\mu}$, resulting in 
\begin{align}\label{eq1.0.9b}
\hat H_{coll}=\sum_{\lambda\mu}\hbar\omega_\lambda\left((-1)^\mu\Gamma_{\lambda\mu}^\dagger\Gamma_{\lambda-\mu}+1/2\right).
\end{align}
The frequency is $\omega_\lambda=(C_\lambda/D_\lambda)^{1/2}$, while $(\hbar\omega_\lambda/2C_\lambda)^{1/2}$ is the amplitude of the zero--point fluctuation of the vacuum state $\ket{0}_B,\ket{n_{\lambda\mu}=1}=\Gamma_{\lambda\mu}^\dagger \ket{0}_B$ being the one--phonon state. To simplify the notation, in many cases one writes $\ket{n_\alpha=1}$.
%\begin{figure}
%	\centerline {
%		\includegraphics*[width=20cm, angle=0.6]{introduccion/figs/figpreface1}
%	}
%	\caption{The values of the atomic ionization potentials. The  closed shells, corresponding to electron number 2(He), 10(Ne), 18(Ar), 36(Kr), 54(Xe), and 86(Ra), are indicated. After \cite{Bohr:69}. In the inset, masses of nuclei with even $A$ are shown (after \cite{Mayer:55}).}
%	\label{fig1.0.1}
%\end{figure}
\begin{figure}
	\centerline {
		\includegraphics*[width=16cm]{introduccion/figs/figpreface2x}
	}
	\caption{Emergent properties (collective nuclear modes) \textbf{(a)} Nucleon-Nucleon ($NN$) interaction in a scattering experiment; \textbf{(b)} assembly of a swarm of nucleons condensing into drops of nuclear matter, examples shown in (c) and (e); \textbf{(c)} anelastic heavy ion reaction $a+A\to a+A^*$ setting the nucleus $A$ into an octupole surface oscillations \textbf{(d)}; in inset \textbf{(I)} the time--dependent nuclear plus Coulomb fields associated with the reaction (c) is represented by a cross followed by a dashed line, while the wavy line labeled $\lambda$ describes the propagation of the surface vibration shown in (d), time running upwards; \textbf{(e)} another possible outcome of nucleon condensation:the (weakly) quadrupole deformed nucleus $^{223}$Ra which can rotate as a whole with moment of inertia smaller than the rigid moment of inertia, but much larger than the irrotational one; \textbf{(f)} the surface zero point fluctuations  (quadrupole ($\lambda=2$), octupole ($\lambda=3$), etc.) can get, with a small but finite probability ($P\approx10^{-10}$), spontaneously in phase and produce a neck-in (saddle conformation) leading eventually to the (exotic) decay mode  $^{223}$Ra$\to^{209}$Pb+$^{14}$C, as experimentally observed \textbf{(g)} (\cite{Rose:84}, see \cite{Brink:05}, Ch. 7 and refs. therein).}
	\label{fig1.0.2}
\end{figure}
\begin{figure}
	\centerline {
		\includegraphics*[width=12cm]{introduccion/figs/figpreface3}
	}
	\caption{Sequence of levels of the harmonic oscillator potential labeled with the principal oscillator quantum number ($N(\hbar\omega)=0,1 (\hbar\omega), 2(\hbar\omega),\dots$ the parity being $\pi=(-1)^N$). The next column shows the splitting of major shell degeneracies obtained using a more realistic potential (Woods-Saxon), the quantum number being the number of radial nodes of the associated single-particle $s,p,d,$ etc., states. The levels shown at the center result when a spin--orbit term is also considered, the quantum numbers $nlj$ characterizing the states of degeneracy $(2j+1)$ ($j=l\pm1/2$). To the left we schematically (in particular in the case of Li which displays non Meyer and Jensen sequence) indicate the Fermi energy associated with a light (exotic), medium, and heavy nucleus, namely $^{11}_3$Li, $^{120}_{50}$Sn and $^{208}_{82}$Pb. In the inset, a schematic graphical representation of the reaction $^{208}$Pb$(d,p)^{209}$Pb(gs) is shown. A cross followed by a horizontal dashed line represents, in the present case,  the $(d,p)$ field, while a  single arrowed line describes the odd nucleon moving in the $g_{9/2}$ orbital above the $N=126$ shell closure (and belonging to the $N=6$ major shell) drawn as a bold line labeled $0^+$.}
	\label{fig1.0.3}
\end{figure}
\begin{figure}
	\centerline {
		\includegraphics*[width=12cm]{introduccion/figs/fig1_1_4}
	}
	\caption{Schematic representation of harmonic quadrupole and octupole liquid drop collective surface vibrational modes.}
	\label{fig1.0.4}
\end{figure}

The ground and low--lying states of nuclei with one nucleon outside closed shell can be described by the Hamiltonian
\begin{align}\label{eq1.0.10}
H_{sp}=\sum_{\nu}\epsilon_\nu a_\nu^\dagger a_\nu,
\end{align}
where $a_\nu^\dagger (a_\nu)$ is the single--particle creation (annihilation) operator,
\begin{align}\label{eq1.0.11}
\ket{\nu}=a_\nu^\dagger\ket{0}_F,
\end{align}
being the single-particle state of quantum numbers $\nu(\equiv nljm$, namely number of nodes, total angular momentum, and its projection) and energy $\epsilon_\nu$, while $\ket{0}_F$ is the Fermion vacuum. 
It is of notice that
\begin{align}\label{eq0.1.14}
\left[H_{coll},\Gamma^\dagger_{\lambda'\mu'}\right]=\hbar\omega_{\lambda'}\Gamma^\dagger_{\lambda'\mu'}
\end{align}
and 
\begin{align}\label{eq0.1.15}
\left[H_{sp},a^\dagger_{\nu'}\right]=\epsilon_{\nu'}a^\dagger_{\nu'}.
\end{align}
	This is an obvious outcome resulting from the bosonic
\begin{align}\label{eq0.1.16}
\left[\Gamma_{\alpha},\Gamma^\dagger_{\alpha'}\right]=\delta(\alpha,\alpha')
\end{align}
	and fermionic
\begin{align}
\left\{a_\nu,a^\dagger_{\nu'}\right\}=\delta(\nu,\nu')
\end{align}
commutation (anti-commutation) relations.


 The existence of drops of nuclear matter displaying both collective surface vibrations independent-particle motion in a self-confining mean field are emergent properties not contained in the particles forming the system, neither in the $NN$-force, but on the fact that these particles behave according to the rules of quantum mechanics, move in a confined volume and that there are many of them.


Expressed it differently, generalized rigidity closely connected to the inertial parameter $D_\lambda$ implies that acting on a nucleus with an external $\beta=0$, time-dependent (nuclear/Coulomb) field, the systems reacts as a whole (collective vibrations; also rotations see Sect. \ref{S1.4}), while acting with fields which change particle number by one ($\beta=\pm1$; e.g. ($d,p$) and $(p,d)$ reactions) the system reacts in terms of independent particle motion, feeling the pushings and pullings of the other nucleons only when trying to leave the nucleus. Such a behaviour can hardly be inferred from the study of the $NN$-forces in free space, being truly emergent properties of the finite, quantum many-body nuclear system.


Collective surface vibrations and independent particle motion are examples of what are called elementary modes of excitation in finite many-body physics, and collective variables in soft-matter physics.

\section{The particle-vibration coupling}\label{Sect1.2}
The oscillation of the nucleus under the influence of the surface tension implies that the potential $U(r,R)$ in which nucleons move independently of each other change with time. For low--energy collective vibrations this change is slow as compared with single--particle motion. Within this scenario the nuclear radius can be written as  
\begin{align}\label{eq1.0.12}
R=R_0\left(1+\sum_{\lambda\mu}\alpha_{\lambda\mu}Y_{\lambda\mu}^*(\hat r)\right)
\end{align}
Assuming small amplitude motion,
\begin{align}\label{eq1.0.13}
U(r,R)=U(r,R_0)+\delta U(r),
\end{align}
where
\begin{align}\label{eq1.0.14}
\delta U=\kappa\hat \alpha \hat F=\Lambda_\alpha\left(\Gamma_{\lambda\mu}^\dagger+(-1)^\mu\Gamma_{\lambda-\mu}\right)\hat F=H_c,
\end{align}
with
\begin{align}\label{eq1.2.4x}
\Lambda_\alpha=\kappa\sqrt{\frac{\hbar\omega_\lambda}{2C_\lambda}},
\end{align}
is the particle-vibration coupling (PVC) (Fig. \ref{fig1.0.5}), product of the dynamic deformation
\begin{align}\label{eq1.2.5x}
\beta_\lambda=\sqrt{2\lambda+1}\sqrt{\frac{\hbar\omega_\lambda}{2C_\lambda}},
\end{align}
and of the strength $\kappa$, while 
\begin{align}\label{eq1.0.15}
\hat F=\sum_{\nu_1\nu_2}\bra{\nu_1}F\ket{\nu_2}a_{\nu_1}^\dagger a_{\nu_2},
\end{align}
is a single-particle field with  dimensionless form factor,
\begin{align}\label{eq1.0.16}
F=-\frac{R_0}{\kappa}\frac{\partial U}{\partial r}Y^*_{\lambda\mu}(\hat r).
\end{align}
An estimate of $\kappa$ is provided below (Eq. (\ref{eq1.2.11})).


 Diagonalizing $\delta U$ making use of the graphical (Feynman) rules of nuclear field theory (NFT) to be discussed in following Chapter, one obtains structure results which can be used in the calculation of absolute transition probabilities and differential reaction cross sections, quantities which can be compared with the experimental findings.

  \begin{figure}
  	\centerline {
  		\includegraphics*[width=10cm]{introduccion/figs/figpreface5}
  	}
  	\caption{Graphical representation of  process       by which a fermion, bouncing inelastically off the surface, sets it into vibration. Particles are represented by an arrowed line pointing upwards which is also the direction of time, while the vibration is represented by a wavy line. In the cartoon to the right, the black dot represents a nucleon moving in a spherical mean field of which it excites, through the PVC vertex, an octupole vibration after bouncing inelastically off the surface.}
  	\label{fig1.0.5}
  \end{figure}
In fact, within the framework of NFT, single--particles are to be calculated as the Hartree-Fock solution of the $NN$--interaction $v(|\mathbf r-\mathbf r'|)$ (Fig. \ref{fig1.0.6}) -- e.g. a regularized $NN$-bare interaction in terms of renormalization group methods or alternative techniques ($v_{low-k}$), taking eventually also 3$N$ terms into account\footnote{\label{f9}\cite{Duguet:13,Duguet:04,Duguet:08,Lesinski:09,Hebeler:09,Baroni:10,Duguet:10,Lesinski:11}.}-- leading, in particular to
\begin{align}\label{eq1.0.18}
U(r)=\int d\mathbf r' \rho(r')v\left(|\mathbf r-\mathbf r'|\right)
\end{align}
that is, the Hartree field\footnote{To this potential one has to add the Fock potential resulting from the fact that nucleons are fermions. This exchange potential (Fig. \ref{fig1.0.6} (b) (2 and 4)) is essential in the determination of single-particle energies and wavefunctions.} expressing the selfconsistency between density $\rho$ and potential $U$ (Fig. \ref{fig1.0.6} (b) (1) and (3)), while vibrations are to be calculated in the Random Phase Approximation (RPA) making use of the same interaction\footnote{\cite{Bohm:51,Bohm:53}. The sum of the so called bubble (ring) diagrams (see Fig. \ref{fig1.0.7}) are taken into account to infinite order in RPA. This is the reason why bubble contributions in the diagonalization of Eq. (\ref{eq1.0.19b}) are not allowed in NFT, being already contained in the basis states (see next chapter, Sect. \ref{appintroA}).} (Fig. \ref{fig1.0.7}), extending the selfconsistency to fluctuations $\delta\rho$ of the density and $\delta U$ of the mean field, that is (see (\ref{eq1.0.13})),
\begin{align}\label{eq1.0.19}
\delta U(r)=\int d\mathbf r' \delta \rho(r')v\left(|\mathbf r-\mathbf r'|\right).
\end{align}
Making use  of the solution to this relation  one obtains the transition density $\delta\rho$. The matrix elements $\braket{n_\lambda=1, \nu_i|\delta\rho|\nu_k}$ provide the  particle-vibration coupling matrix elements to work out the variety of coupling processes between single-particle and collective motion (Fig. \ref{fig1.0.5})\footnote{It is of notice that the so-called scattering vertex shown in Fig. \ref{fig1.0.5} is not operative in RPA. Being a harmonic approximation, either two fermion lines (particle-hole) enter a vertex and a (quasi) boson line comes out (forwards going process), or two-fermion lines and a boson line come out from the vertex (backwards going process). See inset Fig. \ref{fig1.0.7}, graph (b) and (c) respectively.}. That is, the matrix element of the PVC Hamiltonian $H_c$. However, the role of the $NN$-interaction $v$ is not exhausted neither by $H_{HF}$ nor by $H_{RPA}+H_c$. Diagonalizing 
\begin{align}\label{eq1.0.19b}
H=H_{HF}+H_{RPA}+H_c+v,
\end{align}
by applying,  in the basis of single--particle and collective modes, that is solutions of $H_{HF}$ and of $H_{RPA}$ respectively, the graphical  NFT rules  (see next chapter) one obtains an exact solution of the total Hamiltonian, to order $1/\Omega$ of the Feynman diagrams calculated. The quantity $\Omega$ is the effective degeneracy in which the nucleonic excitations are allowed to correlate through $v$ to give rise to the collective modes, $1/\Omega$ being the small parameter of the NFT diagrams\footnote{According to renormalized NFT $v$ is the $NN$-interaction which eventually combined with a $k$-mass, is used to calculate the bare single-particle states (HF-approximation), collective vibrations (RPA) and particle-vibration coupling vertices so that once the corresponding renormalization (dressing) diagrams (including also four-point vertices, i.e. $v$) to the order of $1/\Omega$ required (eventually infinite order if needed), reproduce the experimental findings. It is of notice that if one is interested in the collective vibrations only to dress the single-particle degrees of freedom, one can take them from experiment (empirical renormalization). In other words, determine the $\Lambda$-values by making use of the experimental dynamical deformation parameter $\beta_\lambda=\sqrt{\frac{\hbar\omega_\lambda}{2C_\lambda}}\frac{1}{\sqrt{2\lambda+1}}$ and energies $\hbar\omega_\lambda$ (\cite{Broglia:16}), in conjunction to the expression of the RPA amplitudes $X$ and $Y$ and dispersion relation collected in caption to Fig. \ref{fig1.0.7}.} (see Sect. \ref{Sect1.7.2}, see also \ref{S6.6.2}). Concerning the rules of NFT (Sect. \ref{appintroA}), they codify the way in which $H_c$ (three-point vertices) and $v$ (four-point vertices) are to be treated to all orders of perturbation theory. Also which processes (diagrams) are not allowed because they will imply overcounting of correlations already included in the basis states\footnote{A simple, although not directly related example is provided by Eq. (2A-31) of \cite{Bohr:69} i.e. $G=\tfrac{1}{4}\sum_{\nu_1\nu_2\nu_3\nu_4}\braket{\nu_3\nu_4|G|\nu_1\nu_2}_ aa^\dagger(\nu_4)a^\dagger(\nu_3)a(\nu_1)a(\nu_2)=\tfrac{1}{2}\sum_{\nu_1\nu_2\nu_3\nu_4}\braket{\nu_3\nu_4|G|\nu_1\nu_2} a^\dagger(\nu_4)a^\dagger(\nu_3)a(\nu_1)a(\nu_2)$ where $\braket{\;}_a$ is the antisymmetric matrix element.}. 

As will be shown in the following chapters, NFT allows in an economic fashion, to sum to infinite order weakly convergent processes or particularly important ones, without at the same time be forced to do the same in connection with other processes which either are rapidly convergent, or which lead to small contributions that can be neglected\footnote{The reason for this flexibility is to be partially found in the fact that bare elementary modes of excitation used as basis states in NFT, contain an important fraction of the many-body nuclear correlations, making the diagonalization of the nuclear Hamiltonian a low-dimensional problem (see also Sect. \ref{S1.1}).}.

Before proceeding let us make a simple estimate of the coupling strength $\kappa$, making use of a schematic separable interaction\footnote{See e.g. \cite{Bohr:75} Eq. (6-37).}
\begin{align}\label{eq1.2.9}
v=\kappa\hat F\hat F^\dagger
\end{align}
and of an expansion of the nuclear density $\rho(r,R)$ similar to (\ref{eq1.0.13}), that is,
\begin{align}\label{eq1.2.10}
\delta\rho(\mathbf r)=-R_0\frac{\partial\rho(r)}{\partial r} Y^*_{\lambda\mu}(\hat r).
\end{align}
With the help of the dynamical selfconsistent relation (\ref{eq1.0.19}) on obtains\footnote{\cite{Bohr:75}.},
\begin{align}\label{eq1.2.11}
\kappa=\int r^2 dr R_0\frac{\partial\rho(r)}{\partial r}R_0\frac{\partial U(r)}{\partial r}.
\end{align}
For attractive fields, both $U$ and $\kappa$ are negative.


Because of quantal zero point fluctuations, a nucleon propagating in the nuclear medium moves through a cloud of bosonic  virtual excitations to which it couples becoming dressed and acquiring  effective mass, charge, etc. (Fig. \ref{fig1.0.8}; see also App. \ref{C6AppA} and \ref{C6AppI}). 


Furthermore, the sharp transition expected to take place, in the independent-particle motion, between occupied ($V^2_i=1$, Fig. \ref{fig1.0.6} (7)) and empty states ($V^2_k=0$) becomes blurred due to the processes (b) and (c) (Fig. \ref{fig1.0.8}) which dress the nucleon. This is illustrated in Fig. \ref{fig1.2.5}, where a schematic representation of the occupation probability of neutron orbits in the closed shell system $^{208}$Pb is displayed. Similar results have been obtained in other nuclei, for example the neutron open shell nucleus $^{120}$Sn (Fig. \ref{fig1.0.3}). Within this context let us consider the example of the $1i_{11/2}$ neutron state which, in the independent particle model of $^{208}$Pb, would be unoccupied. Taking the particle vibration coupling mechanism into account, which we limit for simplicity to the consideration of $\beta=0,$ $p-h$ collective surface vibration, all those ZPF of the $^{208}$Pb ``vacuum'' (see Fig. \ref{fig1.0.8} (a)) which involves the $1i_{13/2}$ orbital in the structure of the collective mode would contribute to $n_{1i_{13/2}}$.

Vibrational modes can also become renormalized through the coupling to dressed nucleons which, in intermediate virtual states, can exchange the vibrations which produce their clothing, with the second fermion (hole state). Such a process leads to a renormalization of the PVC vertex\footnote{\label{footnote7} \cite{Bertsch:83,Barranco:04} and refs. therein. It is to be noted that in the case in which the renormalized vibrational modes, i.e. the initial and final wavy lines in Fig. \ref{fig1.0.9} have angular momentum and parity $\lambda^\pi=0^+$, and one uses a model in which there is symmetry between the particle and the hole subspaces, the four diagrams sum to zero, because of particle (gauge) conservation. A fact closely connected with Wick's theorem of QED and generalized Ward's identity (\cite{Ward:50}).} (Fig. \ref{fig1.0.9}), as well as of the bare $NN$-interaction, in particular $^1S_0$ component (bare pairing interaction)\footnote{See e.g. \cite{Brink:05} Ch. 10 and references therein.}. 

\begin{figure}
	\centerline {
		\includegraphics*[width=12cm]{introduccion/figs/figpreface6}
	}
	\caption{\textbf{(a)} Scattering of two nucleons through the  $NN$--interaction; \textbf{(b)} (1) and (3): Contributions to the (direct) Hartree potential;(2) and (4): contributions to the (exchange) Fock potential. In (5) and (6) the ground state correlations associated with the Hartree- and the Fock-terms are displayed. (7) States $\ket{i}$ $(\epsilon_i\leq\epsilon_F)$ are occupied with probability $V^2_i=1$. States $\ket{k}$ $(\epsilon_k>\epsilon_F)$ are empty $V^2_k=1-U^2_k$. (8) Hartree, mean field contribution to the  nuclear density, the density operator being represented by a cross followed by a dashed horizontal line.}
	\label{fig1.0.6}
\end{figure}
\begin{figure}
	\centerline {
		\includegraphics*[width=12cm]{introduccion/figs/figpreface7}
	}
	\caption{(\textbf{A}) typical Feynman diagram diagonalizing the $NN$--interaction $v(|\mathbf r-\mathbf r'|)$ (horizontal dashed line) in a particle--hole basis provided by the Hartree--Fock solution of $v$, in the harmonic approximation (RPA). Bubbles going forward in time (inset (b); leading to the amplitudes $X^\alpha_{ki}=\frac{\Lambda_\alpha\braket{\tilde i|F|k}}{(\epsilon_k-\epsilon_i)-\hbar\omega_\alpha}$) are associated with configuration mixing of particle-hole states, the state $\ket{\tilde i}=\tau\ket{i}$ where $\tau$ is the time reversal operator. It is of notice that $\epsilon_k>\epsilon_F$ and $\epsilon_i\lesssim\epsilon_F$, and $\epsilon_j=\epsilon_{\widetilde j}$ (Kramers degeneracy). Bubbles going backwards in time (inset (c), leading to the amplitudes $Y^\alpha_{ki}=-\frac{\Lambda_\alpha\braket{\tilde i|F|k}}{(\epsilon_k-\epsilon_i)+\hbar\omega_\alpha}$) are associated with zero point  fluctuations (ZPF) of the ground state (term $(1/2)\hbar\omega$ for each degree of freedom in Eq. (\ref{eq1.0.9b})). The self consistent solutions of A, eigenstates of the dispersion relation $\sum_{ki}\frac{2(\epsilon_k-\epsilon_i)|\braket{\tilde i|F|k}|^2}{(\epsilon_k-\epsilon_i)^2-(\hbar\omega_\alpha)^2}=1/\kappa_\alpha$ are represented by a wavy line (inset), that is a collective mode which can be viewed as a correlated particle (arrowed line going upward)- hole (arrowed line going downward) excitation. The quantity $\Lambda_\alpha=\kappa_\alpha\sqrt{\frac{\hbar\omega_\lambda}{2C_\lambda}}$ is the particle-vibration coupling vertex (normalization constant of the RPA amplitudes $\sum_{ki}\left(X^\alpha_{ki}\right)^2-\left(Y^\alpha_{ki}\right)^2=1$. (\cite{Bohm:51,Bohm:53}, see also \cite{Brink:05} Sect. 8.3).}
	\label{fig1.0.7}
\end{figure}

\begin{figure}
	\centerline {
		\includegraphics*[width=12cm]{introduccion/figs/figpreface8}
	}
	\caption{\textbf{(a)} a nucleon (single arrowed line pointing upward) moving in presence of the zero point fluctuation of the nuclear ground state associated with a collective surface vibration; \textbf{(b)} Pauli principle leads to a dressing process of the nucleon (so called correlation (CO) diagram); \textbf{(c)} time ordering gives rise to the second possible lowest order dressing  process (so called polarization (PO) diagram) (time assumed to run upwards). The above are phenomena closely related to the Lamb shift of Quantum Electrodynamics (see e.g. \cite{Weinberg:96} Sect. 14.3).}
	\label{fig1.0.8}
\end{figure}
\begin{figure}
	\centerline {
		\includegraphics*[width=12cm]{introduccion/figs/fig1_2_5}
	}
	\caption{Schematic representation of the occupation probability of $^{208}$Pb neutron orbits: $1f_{7/2},2p_{1/2},1g_{7/2},1h_{11/2},1h_{9/2},2f_{7/2},2f_{5/2},1i_{13/2},3p_{3/2}$ and $3p_{1/2}$ hole states and $2g_{9/2},1i_{11/2},1j_{15/2},3d_{5/2},2g_{7/2},42_{1/2},2d_{3/2},2h_{11/2}$ and $2h_{9/2}$ particle states. The arrows show the location of $\epsilon^-_F=\epsilon_{3p_{1/2}}$ and $\epsilon^+_F=\epsilon_{2g_{9/2}}$.}
	\label{fig1.2.5}
\end{figure}
\begin{figure}
	\centerline {
		\includegraphics*[width=15cm]{introduccion/figs/figpreface9}
	}
	\caption{(Upper part) Examples of renormalization processes dressing a surface collective vibrational state. (Lower part) Intervening with an external electromagnetic field ($E\lambda$: cross followed by dashed horizontal line; bold wavy lines, renormalized vibration of multipolarity $\lambda$) the $B(E\lambda)$ transition strength can be measured. In insets (I) and (I'), the hatched circle in the diagram to the right stands for the renormalized PVC strength resulting from the processes described by the corresponding diagrams to the left (vertex corrections). In  (II') the bold face arrowed curve  (left diagram) represents  the motion of a nucleon of effective mass $m^*$ in a potential $(m/m^*)U(r)$, generated by the self-energy process shown to the right (see also (II) right diagram), $U(r)$  being the potential describing the motion of nucleons (drawn as a thin  arrowed line) of bare mass $m_k$  (inset (II), right diagram. (\cite{Brink:05} App. B).}
	\label{fig1.0.9}
\end{figure}

The analytic procedures equivalent  to the diagrammatic ones to obtain the HF (Fig. \ref{fig1.0.6}) and RPA (Fig. \ref{fig1.0.7}) solutions associated with the bare $NN$-interaction $v$ is provided by the relations (\ref{eq0.1.15}) and (\ref{eq0.1.14}) respectively, replacing the corresponding Hamiltonians by $(T+v)$, where $T$ is the kinetic energy operator. The phonon operator associated with surface vibrations is defined as,
\begin{align}\label{eq1.0.26}
\Gamma_\alpha^\dagger=\sum_{ki}X_{ki}^\alpha\Gamma_{ki}^\dagger+Y_{ki}^\alpha\Gamma_{ki},
\end{align}
 the normalization condition being, 
\begin{align}\label{eq1.0.27}
\left[\Gamma_\alpha,\Gamma_\alpha^\dagger\right]=\sum_{ki}\left(X_{ki}^{\alpha2}-Y_{ki}^{\alpha2}\right)=1.
\end{align}
The operator $\Gamma^\dagger_{ki}=a^\dagger_{k}a_{i} (\epsilon_k>\epsilon_F,\epsilon_i\leq\epsilon_F)$ creates a particle-hole excitation acting on the HF vacuum state $\ket{0}_F$. It is assumed that
\begin{align}\label{eq1.0.28}
\left[\Gamma_{ki},\Gamma_{k'i'}^\dagger\right]=\delta(k,k')\delta(i,i').
\end{align}
\textit{Within this context, RPA is a harmonic, quasi-boson approximation}.


From being antithetic views of the nuclear structure, a proper analysis of the experimental data testifies to the fact that the collective and the independent particle pictures of the nuclear structure require and support each other\footnote{\cite{Bohr:75}.}. To obtain a quantitative description of nucleon  motion and nuclear phonons (vibrations), one needs a proper description of the $k$- and $\omega$-dependent ``dielectric'' function of the nuclear medium, in a similar way in which a proper description of the reaction processes used as probes of the nuclear structure requires the use of the optical potential (continuum ``dielectric'' function). 


The NFT solutions of (\ref{eq1.0.19b}) provide all the elements to calculate the structure properties of nuclei, and also the optical potential needed to describe nucleon-nucleus as well as nucleus-nucleus elastic scattering and reaction processes\footnote{However, in the present monograph this subject is not treated.}.
Furthermore, the NFT solutions of (\ref{eq1.0.19b}) show that both single-particle (fermionic) and collective (bosonic) elementary modes of excitation emerge from the same properties of the $NN$-interaction. For example, a bunching of levels of the same or opposite parity just below and above the Fermi energy, imply low-lying quadrupole (e.g. $^A_{50}$Sn$_{N}$-isotopes), dipole ($^{11}_3$Li$_8$) or octupole ($^{208}_{82}$Pb$_{126}$) collective vibrational modes. And, as a bonus, one has the building blocks to construct the nuclear spectrum, and bring quantitative simplicity into the experimental findings. Within this scenario, the NFT solutions of (\ref{eq1.0.19b}) also indicate the minimum set of experimental probes needed to have a ``complete'' picture of the nuclear structure associated with a given energy region. This is a consequence of the central role played by the quantal many-body renormalization process which interweave the variety of elementary modes of excitation.
Renormalization processes which acts on par on the radial dependence of the wavefunctions (formfactors) and on the single-particle content of the orbitals involved in the reaction process under discussion (see e.g. Sect. \ref{C6S2}). In other words, structure ad reactions are to be treated on equal footing\footnote{Within this context, and referring to one-particle transfer reactions for concreteness, the prescription of using the ratio of the absolute experimental cross section of and the theoretical one --calculated in the Distorted Wave Born Approximation (DWBA) making use of Saxon-Woods single-particle wavefunctions as formfactors-- to extract the single-particle content of the orbital under study (see e.g. \cite{Schiffer:12}), may not be completely appropriate.}. 


The development of experimental techniques and associated hardware has allowed for the identification of a rich variety of elementary modes of excitation aside from collective surface vibrations  independent particle motion: quadrupole and octupole rotational bands, giant resonance of varied multipolarity and isospin, as well as pairing vibrations and rotation, together with giant pairing vibrations of transfer quantum number $\beta\pm 2$. Modes which can be specifically excited in inelastic and Coulomb excitation processes, and one- and two-particle transfer reactions (Ch. \ref{C6} and \ref{C7}).
\section{Pairing vibrations}\label{Sect1.3}
Let us introduce this new type of elementary mode of excitation by making a parallel with quadrupole surface vibrations within the framework of RPA, namely
\begin{align}\label{eq1.0.29}
\left[(H_{sp}+H_i),\Gamma_{\alpha}^\dagger\right]=\hbar\omega_\alpha\Gamma_{\alpha}^\dagger,
\end{align}
where for simplicity we use, instead of $v$, a quadrupole-quadrupole separable interaction $(i=QQ)$ defined as\footnote{It is of notice that in this case, and at variance with (\ref{eq1.2.9})--(\ref{eq1.2.11}), the fact that the force is attractive is explicitly expressed through the minus sign, assuming $\kappa>0$. This is done for didactical purposes, to connect with the standard notation for the pairing interaction given in Eq. (\ref{eq1.0.32}) below.}
\begin{align}\label{eq1.0.30}
H_{QQ}=-\kappa Q^\dagger Q,
\end{align}
with 
\begin{align}\label{eq1.0.31}
 Q^\dagger=\sum_{ki}\braket{k|r^2 Y_{2\mu}|i}a^\dagger_k a_i,
\end{align}
while $H_{sp}$ and $\Gamma^\dagger_\alpha$ are defined in (\ref{eq1.0.10}) and (\ref{eq1.0.26}) supplemented by (\ref{eq1.0.27}).

\begin{figure}
	\centerline {
		\includegraphics*[width=12cm]{introduccion/figs/fig_preface_3_1}
	}
	\caption{Graphical representation of the RPA dispersion relation describing the pair addition  vibrational mode, represented by a double arrowed line.  A cross followed by a dashed horizontal line stands for (see Eq. (\ref{eq1.0.34})): (\textbf{a}) the collective operator $\Gamma_\alpha^\dagger$, (\textbf{b}) the operator $\Gamma_k^\dagger$ creating a pair of nucleons moving in time reversal particle states ($k,\tilde k$) above the Fermi energy $(\epsilon_k>\epsilon_F)$; (\textbf{c}) The operator $\Gamma^\dagger_i$ filling a pair of time reversal hole states associated with ground state correlations ($\epsilon_i\leq\epsilon_F$).}
	\label{fig0.3.1}
\end{figure}
\begin{figure}
	\centerline {
		\includegraphics*[width=15cm, angle=0.]{introduccion/figs/fig1_3_2}
	}
	\caption{The many-phonon pairing vibrational spectrum around the ``vacuum'' $\ket{gs(^{208}\text{Pb})}$. The energies predicted by the pairing vibrational model (harmonic approximation) are displayed as dashed horizontal lines. The harmonic quantum numbers ($n_r,n_a$) are indicated for each level. A schematic representation of the many-particle many-hole structure of the state is also given. The transitions predicted by the model (harmonic approximation) are indicated in units of $r$ and $a$ (see equation (\ref{eq1.3.9}) and subsequent text). The corresponding experimental numbers are also given together with their errors. The dashed line between the states (0,0) and (2,1) indicates that the $^{208}$Pb$(p,t)^{206}$Pb reaction to the three-phonon state in $^{206}$Pb was carried out, and an upper limit of 0.03$r$ for the corresponding cross-section  determined (see \cite{Flynn:72} \cite{Broglia:73}, also \cite{Lanford:73}).}
	\label{fig0.3.2}
\end{figure}

In connection with the pairing energy mentioned above (odd-even staggering),  it is a consequence of correlation of pairs of like nucleons moving in  time reversed states $(\nu,\tilde \nu)$. A similar phenomenon to that found in metals at low temperatures and giving rise to superconductivity\footnote{\cite{Bohr:58}.}. The pairing interaction $(i=P)$ can be written, within the approximation (\ref{eq1.0.30}) used in the case of the quadrupole-quadrupole force, as 
\begin{align}\label{eq1.0.32}
H_P=- GP^\dagger P,
\end{align}
where 
\begin{align}\label{eq1.0.33}
 P^\dagger=\sum_{\nu>0}a^\dagger_\nu a^\dagger_{\bar\nu},
\end{align}
and $G$ is a pairing coupling constant (attractive $G>0$).
Consequently, in this case the concept of independent particle field $\hat Q$ (see also (\ref{eq1.0.15}) and (\ref{eq1.0.16})) associated with particle-hole ($ph$) excitations and carrying transfer quantum number $\beta=0$ has to be generalized to include fields describing independent pair motion of different multipolarity and (normal) parity, in particular $0^+$, in which case $\alpha\equiv(\beta=+2,J^\pi=0^+)$
\begin{align}\label{eq1.0.34}
\Gamma^\dagger_\alpha=\sum_kX^\alpha_{kk}\Gamma_k^\dagger+\sum_iY^\alpha_{ii}\Gamma_i
\end{align}
with 
\begin{align}\label{eq1.0.35}
\Gamma^\dagger_k=a^\dagger_ka^\dagger_{\bar k}\;(\epsilon_k>\epsilon_F),\quad \Gamma^\dagger_i=a^\dagger_ia^\dagger_{\bar i}\;(\epsilon_i\leq\epsilon_F),
\end{align}
and
\begin{align}\label{eq1.0.36}
\sum_kX_{kk}^{\alpha 2}-\sum_i Y_{ii}^{\alpha 2}=1,
\end{align}
for the pair addition ($(pp),\beta=+2$) mode, and a similar expression for the pair removal ($(hh),\beta=-2$) mode. In Fig. \ref{fig0.3.1} the NFT graphical representation of the RPA equations for the pair addition mode is given. The state $\Gamma^\dagger_\alpha(\beta=+2)\ket{\tilde 0}$, where $\ket{\tilde 0}$ is the correlated ground state of a closed shell nucleus, can be viewed as the nuclear embodiment of a Cooper pair found at the basis of the microscopic theory of superconductivity.
While surface vibrations are associated with the normal $(\beta=0)$ nuclear density, pairing vibrations are connected with the so called abnormal ($\beta=\pm2$) nuclear density (density of Cooper pairs), both static and dynamic.

Similar to the quadrupole and octupole vibrational bands built out of $n_\alpha$ phonons of quantum numbers $\alpha\equiv(\beta=0,\lambda^\pi=2^+,3^-)$ schematically shown in Fig. \ref{fig1.0.4} and experimentally observed in inelastic and Coulomb excitation and associated $\gamma$-decay processes, pairing vibrational bands build of $n_\alpha$ phonons of quantum numbers $\alpha\equiv(\beta=\pm2,\lambda^\pi=0^+,2^+)$ have been identified around closed shells in terms of two-nucleon transfer reactions throughout the mass table\footnote{See e.g. \cite{Flynn:72} and references therein.}. In Fig. \ref{fig0.3.2}, the monopole pairing vibrational band built around the vacuum state $\ket{gs(^{208}\text{Pb})}$ in terms of the pair addition and substraction modes
\begin{align}\label{eq1.3.9}
\ket{a}=\ket{n_a=1}=\ket{gs(^{210}\text{Pb})},\quad\text{and}\quad \ket{r}=\ket{n_r=1}=\ket{gs(^{206}\text{Pb})},
\end{align}
is displayed. The indices $n_a(=1,2,\dots)$ and $n_r(=1,2,\dots)$ indicate the number of pair addition and/or of pair removal modes of which the states are made. The label $a$ and $r$ indicate in Fig. \ref{fig0.3.2}, the absolute transition probability, i.e. two-nucleon (Cooper pair) transfer cross section which connects the ground (``vacuum'') state $\ket{gs(^{208}\text{Pb})}$ with $\ket{a}$ and $\ket{r}$ respectively. 
\section{Spontaneously broken symmetry}\label{S1.4}
Because empty space is homogeneous and isotropic, and to the degree of accuracy relevant in many-body problems, also invariant under time-reversal, the nuclear Hamiltonian is translational, rotational and time-reversal invariant. According to quantum mechanics, the corresponding wavefunctions transform in an irreducible way under the corresponding groups of transformation.
When the solution  of the Hamiltonian does not have some of these symmetries, for example defines a privileged direction in space violating rotational invariance, one is confronted with the phenomenon of spontaneously broken symmetry. Strictly speaking, this can take place only for idealized systems that are infinitely large. But when one sees similar phenomena in atomic nuclei, although not so clear or regular, one recognizes that this system is after all a finite quantum many-body system (FQMBS)\footnote{\cite{Anderson:72,Anderson:84b}.}.
\subsection{Quadrupole deformations in 3D-space}
A nuclear embodiment of the spontaneous symmetry breaking phenomenon is provided by a quadrupole deformed  mean field. A situation one is confronted with, when the value of the lowest quadrupole frequency $\omega_2$ of the RPA solution (\ref{eq0.1.14}) tends to zero ($C_2\to0$, $D_2$ finite). A phenomenon resulting from the interplay of the interaction $v$ ($H_{QQ}$ in (\ref{eq1.0.30})), and of the nucleons outside closed shell, leading to tidal-like polarization of the spherical core.
\begin{figure}
	\centerline {
		\includegraphics*[width=8cm, angle=0.]{introduccion/figs/fig0_4_4}
	}
	\caption{Schematic representation of deformation in  3D- (gauge-) space, where the laboratory ($\mathcal K$) and the intrinsic ($\mathcal K'$, body fixed) frames of reference making a relative relative angle (($\omega$, Euler angles); ($\phi$, gauge angle)) are  indicated (see Eqs. (\ref{eq0.4.6}) and (\ref{eq0.1.90})).}
	\label{fig0.4.4}
\end{figure}
\begin{figure}
	\centerline {
		\includegraphics*[width=15cm, angle=0.]{introduccion/figs/fig0_4_2_v4}
	}
	\caption{Nilsson single-particle levels in a quadrupole deformed potential in the regions $8<Z<20$ and $8<N<20$.}
	\label{fig0.4.1}
\end{figure}
\begin{figure}
	\centerline {
		\includegraphics*[width=12cm, angle=0.]{introduccion/figs/fig0_4_2_v3}
	}
	\caption{Ground state rotational band of $^{158}$Er.}
	\label{fig0.4.2}
\end{figure}
\begin{figure}
	\centerline {
		\includegraphics*[width=15cm, angle=0.]{introduccion/figs/fig0_4_4_v2}
	}
	\caption{(a) Independent (dashed line) and BCS occupation numbers; (b) ground state and excited states in the extreme single-particle model and in the pairing-correlated, superfluid model in the case of a system with an odd number of particles. In the first case, the energy of the ground state of the odd system differs from that of the even with one particle fewer by the energy difference $\epsilon_\nu-\epsilon_{\nu'}$ while in the second case by the energy $E_\nu=\sqrt{(\epsilon_\nu-\lambda)^2+\Delta^2}\approx\Delta$, associated with the fact the odd particle has no partner. Excited states can be obtained in the independent particle case, where it is assumed that levels are two-fold degenerate (Kramers degeneracy) by promoting the odd particle to states above the level $\epsilon_\nu$, or one particle to  states above  $\nu'$ (arrows). To the left only a selected number of these excitations are shown. In the superfluid case excited states can be obtained by breaking of pairs in any orbit. The associated quasiparticle energy is drawn also here by an arrow of which the thin part indicates the contribution of the pairing gap and the thick part indicates the kinetic energy contribution, i.e. the contribution arising from the single-particle motion. Note the very different density of levels emerging from these two pictures, which are shown at the far right of the figure.}
	\label{fig0.4.3}
\end{figure}

Coordinate and linear momentum (($x,p_x$) single-particle motion) as well as Euler angles and angular momentum ($\omega,\mathbf I$) associated with rotation in three-dimensional (3D)-space,  are conjugate variables. Similarly, the gauge angle and the number of particles (($\phi,N$) rotation in gauge space), fulfill $[\phi,N]=i$. The operators $e^{-ip_xx}$, $e^{-i\pmb\omega\cdot\mathbf I}$ and $e^{-iN\phi}$ induce Galilean transformation in 1D-space and rotations in 3D- and in gauge space respectively. 

Making again use, for didactical purposes, of $H_{QQ}$ instead of $v$, and calling $\ket{N}$, for Nilsson\footnote{\cite{Nilsson:55}.}, the eventual mean field solution of the Hamiltonian $T+H_{QQ}$, one expects
  \begin{align}\label{eq0.4.1}
\braket{N|\hat Q|N}=Q_0,
  \end{align}
where, for simplicity, we assumed axial symmetry $(\lambda=2,\mu=0)$. That is, the emergence of a static quadrupole deformation.
Rewriting $H_{QQ}$ in terms of ($(\hat Q^\dagger-Q_0)+Q_0)$ and its Hermitian conjugate, one obtains 
  \begin{align}\label{eq0.4.2}
H=H_{sp}+H_{QQ}=H_{MF}+H_{fluct},
\end{align}
where 
  \begin{align}\label{eq0.4.3}
H_{MF}=H_{sp}-\kappa(\hat Q^\dagger+\hat Q),
\end{align}
is the mean field, and
\begin{align}\label{eq0.4.4}
H_{fluct}=-\kappa(\hat Q^\dagger-Q_0)(\hat Q-Q_0)
\end{align}
the residual interaction inducing fluctuations around $Q_0$.


 Let us  concentrate on $H_{MF}$. The original realization of it is known as the Nilsson Hamiltonian. It describes the motion of nucleons in a single-particle potential of radius $R_0(1+\beta_2Y_{20}(\hat r))$, with $\beta_2$ proportional to the intrinsic quadrupole moment\footnote{\cite{Mottelson:59}, where use has been made of $\beta_2\approx0.95\delta$ \cite{Bohr:75} p. 47 and Eq. (4-191).} $Q_0$ ($\beta_2\approx Q_0/(ZR_0^2)$). The reflection invariance and axial symmetry of the Nilsson Hamiltonian implies that parity $\pi$ and projection $\Omega$ of the total angular momentum along the symmetry axis are constants of motion for the one-particle Nilsson states. These states, are two-fold degenerate, since two orbits that differ only in the sign of $\Omega$ represent the same motion, apart from the clockwise and anticlockwise sense of revolution around the symmetry axis. One can thus write the Nilsson creation operators in terms of a linear combination of creation operators carrying good total angular momentum $j$, 
\begin{align}\label{eq0.4.5}
\gamma^\dagger_{a\Omega}=\sum_jA_j^aa^\dagger_{aj\Omega},
\end{align}
where the label $a$ stands for all the quantum numbers aside from $\Omega$, which specify the orbital (see below). 

Expressed in the intrinsic, body-fixed, system of coordinates $\mathcal K'$ (Fig. \ref{fig0.4.4}) where the 3 ($z'$) axis lies along the symmetry axis and the 1 and 2 ($x',y'$) axis lie in a plane perpendicular to it, namely
\begin{align}\label{eq0.4.6}
\gamma'^{\dagger}_{a\Omega}=\sum_jA_j^a\sum_{\Omega'}\mathcal D_{\Omega\Omega'}^2(\omega)a^\dagger_{aj\Omega'},
\end{align}
one can write the Nilsson state as 
\begin{align}\label{eq0.4.7}
\ket{N(\omega)}_{\mathcal K'}=\prod_{a\Omega>0}\gamma'^{\dagger}_{a\Omega}\gamma'^{\dagger}_{a\widetilde\Omega}\ket{0}_F,
\end{align}
where $\omega$ represent the Euler angles, $\ket{0}_F$ is the particle vacuum, and $\ket{a\widetilde\Omega}=\gamma'^{\dagger}_{a\widetilde\Omega}\ket{0}_F$ is the state time-reversed to $\ket{a\Omega}$. For well deformed nuclei, a conventional description of the one-particle motion is based on the similarity of the nuclear potential to that of an anisotropic oscillator potential, 
\begin{align}\label{eq0.4.8}
V=\frac{1}{2}M\left(\omega_3^2x_3^2+\omega^2_{\perp}(x_1^2+x_x^2)\right)=\frac{1}{2}M\omega_0r^2\left(1-\frac{4}{3}\delta P_2(\cos\theta)\right),
\end{align}
with $\omega_3\omega_{\perp}^2=\omega_0^3$. That is a volume which is independent of the deformation $\delta\approx0.95\beta_2$. The corresponding single-particle states have energy
\begin{align}\label{eq0.4.9}
\epsilon(n_3n_{\perp})=(n_3+\tfrac{1}{2})\hbar\omega_3+(n_{\perp}+1)\hbar\omega_{\perp},
\end{align}
where $n_3$ and $n_{\perp}=n_1+n_2$ are the number of quanta along and perpendicular to the symmetry axis, $N=n_3+n_{\perp}$, being the total oscillator quantum number. The degenerate states with the same value of $n_{\perp}$ can be specified by the component $\Lambda$ of the orbital angular momentum along the symmetry axis,
\begin{align}\label{eq0.4.10}
\Lambda=\pm n_{\perp}, \pm(n_{\perp}-2),\dots,\pm 1\text{ or } 0.
\end{align}
\begin{figure}
	\centerline {
		\includegraphics*[width=12cm, angle=0.]{introduccion/figs/fig0_4_5_v3}
	}
	\caption{Pairing rotational band associated with the ground states of the Sn-isotopes. The lines represent the energies calculated according to the expression $BE = B( ^{50+N}\text{Sn}_N ) - 8.124N + 46.33$ (\cite{Brink:05}), subtracting the contribution of the single nucleon addition to the nuclear binding		energy obtained by a linear fitting of the binding energies of the whole Sn chain. The estimate of $\hbar^2/2\mathcal I$ was obtained using the single $j$-shell model (see, e.g., \cite{Brink:05}, Appendix H).The numbers indicated with thin lines  are the absolute value of the experimental $gs\to gs$ cross section (in units of $\mu $b).}
	\label{fig0.4.5}
\end{figure}
The complete expression of the Nilsson potential includes, aside from the central term discussed above, a spin-orbit ($\Sigma=\pm1/2,\;\Omega=\Lambda+\Sigma$) and a term proportional to the orbital angular momentum quantity squared, so as to make the shape of the oscillator to resemble more that of a Saxon-Woods potential. One can  label the Nilsson levels in terms of the asymptotic quantum numbers $\Omega[Nn_3\Lambda]$. The resulting states provide an overall account of the experimental findings. An example of relevance for light nuclei ($N$ and $Z<20$) is given in\footnote{\cite{Mottelson:59}.} Fig. \ref{fig0.4.1}.

The Nilsson intrinsic state (\ref{eq0.4.7}) does not have a definite angular momentum but is rather a superposition of such states,
\begin{align}\label{eq0.4.11}
\ket{N(\omega)}_{\mathcal K'}=\sum c_I\ket{I}.
\end{align}
Because there is no restoring force associated with different orientations of $\ket{N(\omega}_{\mathcal K'}$, fluctuations in the Euler angle diverge in the right way to restore rotational invariance, leading to a rotational band whose members are 
\begin{align}\label{eq0.1.47}
\ket{IKM}\sim\int d\omega \,\mathcal D^I_{MK}(\omega)\,\ket{N(\omega)}_{\mathcal K'},
\end{align}
with energy
\begin{align}\label{eq0.1.48}
E_I=\frac{\hbar^2}{2\mathcal I}I(I+1).
\end{align}
The quantum numbers $I,M,K$ are the total angular momentum $I$, and its third component $M$ and $K$ along the laboratory ($z$) and intrinsic ($z'$) frame references respectively.
Rotational bands have been observed up to rather high angular momenta in terms of individual transitions. In particular, up to $I=60\hbar$ in the case of\footnote{\cite{Nolan:88} and refs. therein.} $^{152}$Dy. An example of rotational bands  is given in Fig. \ref{fig0.4.2}. 
\subsection{Deformation in gauge space}
Let us now turn to the pairing Hamiltonian. In the case in which \mbox{$\hbar\omega(\beta=-2)=\hbar\omega(\beta=2)=0$}, the system deforms, this time in gauge space. Calling $\ket{BCS}$ the  mean field solution of the pairing Hamiltonian\footnote{\cite{Bardeen:57a,Bardeen:57b}.}, leads to the finite expectation value
\begin{align}\label{eq0.4.14}
\alpha_0=\braket{BCS|P^\dagger|BCS},
 \end{align}
   of the pair creation operator $P^\dagger$, quantity which can be viewed as the order parameter of the new deformed  phase of the system in gauge space. The total Hamiltonian can be written as
\begin{align}\label{eq0.1.50}
H=H_{MF}+H_{fluct},
\end{align}
where 
\begin{align}\label{eq0.1.51}
H_{MF}=H_{sp}-\Delta(P^\dagger+P)+\frac{\Delta^2}{G}
\end{align}
and
\begin{align}\label{eq0.1.52}
H_{fluct}=-G(P^\dagger-\alpha_0)(P-\alpha_0).
\end{align}
The quantity 
\begin{align}\label{eq0.1.53}
\Delta=G\alpha_0,
\end{align}
is the so called pairing gap (Fig. \ref{fig0.4.3}), which measures the binding energy of Cooper pairs, the quantity $\alpha_0$ being the number of Cooper pairs.

 The mean field pairing Hamiltonian 
\begin{align}\label{eq0.1.54}
H_{MF}=\sum_{\nu>0}(\epsilon_\nu-\lambda)\,(a^\dagger_\nu a_\nu+a^\dagger_{\tilde\nu} a_{\tilde\nu})-\Delta\sum_{\nu>0}(\epsilon_\nu-\lambda)\,(a^\dagger_\nu a^\dagger_{\tilde\nu}+a_{\tilde\nu} a_{\nu})+\frac{\Delta^2}{G}
\end{align}
is a bilinear expression in the creation and annihilation operator, $\nu$ labeling the quantum numbers of the single-particle orbitals where nucleons are allowed to correlate (e.g. ($nlj,m$)), while $\tilde \nu$ denotes the time reversal state which in this case is degenerate with $\nu$ and has quantum numbers ($nlj,-m$). The condition $\nu>0$ in Eq. (\ref{eq0.1.54}) implies that one sums over $m>0$. It is of notice that 
\begin{align}\label{eq0.1.55}
\hat N=\sum_{\nu>0}(a^\dagger_\nu a_\nu+a^\dagger_{\tilde\nu} a_{\tilde\nu}),
\end{align}
is the number operator, and $\lambda \hat N$ in Eq. (\ref{eq0.1.54}) acts as the Coriolis force in the body-fixed frame of reference in gauge space. 

One can diagonalize $H_{MF}$ by a rotation in the $(a^\dagger,a)$-space. This can be accomplished through the Boguliubov-Valatin transformation
\begin{align}\label{eq0.1.56}
\alpha^\dagger_\nu=U_\nu a^\dagger_\nu-V_\nu a_{\tilde\nu}.
\end{align}
The BCS solution does not change the energies $\epsilon_\nu$ (measured in (\ref{eq0.1.54}) from the Fermi energy $\lambda$) of the single-particle levels or associated wavefunctions $\varphi_{\nu}(\mathbf r)$, but the occupation probabilities for levels around the Fermi energy within an energy range $2\Delta$ $(2\Delta/\lambda\approx$ 2 MeV/36 MeV$\approx0.06$)\footnote{Within this context, see discussion starting just before Eq. (\ref{eq1.4.57}) and ending just before Eq. (\ref{eq0.1.91}).}. The quasiparticle operator $\alpha^\dagger_\nu$ creates a particle in the single-particle state $\nu$ with probability $U^2_\nu$, while it creates a hole (annihilates a particle) with probability $V^2_\nu$. To be able to create a particle, the state $\nu$ should be empty, while to annihilate a particle it has to be filled, so $U^2_\nu$ and $V^2_\nu$ are the probabilities  that the state $\nu$ is empty and is occupied respectively. Within this context, the one quasiparticle states   
\begin{align}\label{eq0.1.57}
\ket{\nu}=\alpha^\dagger_\nu\ket{BCS},
\end{align}
are orthonormal. In particular
\begin{align}\label{eq0.1.58}
\braket{\nu|\nu}=1=\braket{BCS|\alpha_{\nu}\alpha^\dagger_\nu|BCS}=\braket{BCS|\left\{\alpha_{\nu},\alpha^\dagger_\nu\right\}|BCS}=U^2_\nu+V^2_\nu,
\end{align}
where the relations
\begin{align}\label{eq0.1.59}
\left\{a_{\nu},a^\dagger_{\nu'}\right\}=\delta(\nu,\nu'),
\end{align}
and
\begin{align}\label{eq0.1.60}
\left\{a_{\nu},a_{\nu'}\right\}=\left\{a^\dagger_{\nu},a^\dagger_{\nu'}\right\}=0,
\end{align}
have been used. Note that the $\ket{BCS}$ state is the quasiparticle vacuum
\begin{align}\label{eq0.1.61}
\alpha_\nu\ket{BCS}=0,
\end{align}
in a similar way in which $\ket{0}_F$ is the particle vacuum. Inverting the quasiparticle transformation (\ref{eq0.1.56}) and its complex conjugate, i.e. expressing $a_\nu^\dagger$ and $a_\nu$ (and time reversals (tr)) in terms of $\alpha^\dagger_\nu$ and $\alpha_\nu$ (and tr) one can rewrite (\ref{eq0.1.54}) in terms of quasiparticles.

Minimizing the energy $E_0=\braket{BCS|H|BCS}$ in terms of\footnote{For details see e.g. \cite{Ragnarsson:05}. See also \cite{Nathan:65}.} $V_\nu$
\begin{align}\label{eq0.1.62}
\frac{\partial E_0}{\partial V_\nu}=0
\end{align}
 making use of the expression for the average number of particles
\begin{align}\label{eq0.1.63}
N_0=\braket{BCS|\hat N|BCS}=2\sum_{\nu>0}V_\nu^2,
\end{align}
 of the number of Cooper pairs
\begin{align}\label{eq0.1.64}
\alpha_0=\braket{BCS|P^\dagger|BCS}=\sum_{\nu>0}U_\nu V_\nu
\end{align}
and of the pairing gap
\begin{align}\label{eq0.1.65}
\Delta=G\sum_{\nu>0}U_\nu V_{\nu},
\end{align}
one obtains,
\begin{align}\label{eq0.1.66}
H_{MF}=H_{11}+U,
\end{align}
where 
\begin{align}\label{eq0.1.67}
H_{11}=\sum_\nu E_\nu \alpha^\dagger_\nu\alpha_\nu,
\end{align}
and
\begin{align}\label{eq0.1.68}
U=2\sum_{\nu>0}(\epsilon_\nu-\lambda)V^2_\nu-\frac{\Delta^2}{G}.
\end{align}
The quantity
\begin{align}\label{eq0.1.69}
E_\nu=\sqrt{(\epsilon_\nu-\lambda)^2+\Delta^2}
\end{align}
is the quasiparticle energy, while the occupation (emptiness) probability amplitudes are
\begin{align}\label{eq0.1.70}
V_\nu=\frac{1}{\sqrt{2}}\left(1-\frac{\epsilon_\nu-\lambda}{E_\nu}\right)^{1/2},
\end{align}
\begin{align}\label{eq0.1.71}
U_\nu=\frac{1}{\sqrt{2}}\left(1+\frac{\epsilon_\nu-\lambda}{E_\nu}\right)^{1/2},
\end{align}
 respectively. From the relations (\ref{eq0.1.63}),  (\ref{eq0.1.65}) and (\ref{eq0.1.69}) one obtains
\begin{align}\label{eq0.1.72}
N_0=2\sum_{\nu>0}V^2_\nu,
\end{align}
 and 
\begin{align}\label{eq0.1.73}
\frac{1}{G}=\sum_{\nu>0}\frac{1}{2E_\nu}.
\end{align}
These equations allow one to calculate the parameters $\lambda$ and $\Delta$ from the knowledge of $G$ and $\epsilon_\nu$, parameters which completely determine $E_\nu,V_\nu$ and $U_\nu$ and thus the BCS mean field solution (Fig. \ref{fig0.4.3}). The validity of the BCS description of superfluid open shell nuclei have been confirmed throughout the mass table. We provide below recent examples.

 The relation (\ref{eq0.1.61}) implies, as a consequence of Pauli principle, that 
\begin{align}\label{eq0.1.74}
\nonumber \ket{BCS}&=\frac{1}{\text{Norm}}\prod_{\nu>0}\alpha_\nu\alpha_{\tilde\nu}\ket{0}_F=\prod_{\nu>0}\left(U_\nu+V_\nu P^\dagger_\nu\right)\ket{0}_F\\
\nonumber&=\left(\prod_{\nu>0}U_{\nu}\right)\sum_{N\text{ even}} \frac{\left(c_\nu P^\dagger\right)^{N/2}}{(N/2)!}\ket{0}_F\\
&=\left(\prod_{\nu>0}U_\nu\right)\sum_{n=0,1,2,\dots}\frac{\left(c_\nu P^\dagger_\nu\right)^n}{n!}\ket{0}_F,
\end{align}
 where
\begin{align}\label{eq0.1.75}
P^\dagger_\nu=a_\nu^\dagger a^\dagger_{\tilde \nu}\quad \left(P^\dagger=\sum_{\nu>0}P^\dagger_\nu\right), \quad c_\nu=V_\nu/U_\nu,
\end{align}
while $n$ denotes the number of pairs of particles ($2n=N$).
 In the above discussion of BCS we have treated in a rather cavalier fashion the fact that the amplitudes $U_\nu$ and $V_\nu$ are in fact complex quantities. A possible choice of phasing is\footnote{\cite{Schrieffer:73}. The same results as those which will be derived are obtained with the alternative choice $U_\nu=U'_\nu e^{i\phi},V_\nu=V'_\nu e^{-i\phi}$ (see e.g. \cite{Potel:13b}).} 
\begin{align}\label{eq0.1.76}
U_\nu=U'_\nu;\;\;V_\nu=V'_\nu\, e^{-2i\phi},
\end{align}
$U'_\nu$ and $V'_\nu$ being real quantities, while $\phi$ is the gauge angle, conjugate variable to the number of particles operator. Then\footnote{See e.g. \cite{Brink:05} App. H and refs. therein.}
\begin{align}\label{eq0.1.77}
\hat\phi=i\partial/\partial \mathcal N,\quad \mathcal N,
\end{align}
and
\begin{align}\label{eq0.1.78}
\left[\hat \phi,\mathcal N\right]=i,
\end{align}
where $\mathcal N\equiv\hat N$ (Eq. (\ref{eq0.1.55})), the gauge transformations being induced by the operator 
\begin{align}\label{eq0.1.79}
\mathcal G(\phi)=e^{-i\mathcal N\phi}.
\end{align}
Let us replace the amplitudes (\ref{eq0.1.76}) in  (\ref{eq0.1.74}),
\begin{align}\label{eq0.1.80}
\ket{BCS}=\left(\prod_{\nu>0}U'_\nu\right)\sum_{\text{N even}}e^{-iN\phi}\ket{\Phi_{N}}=\left(\prod_{\nu>0}U'_\nu\right)\sum_{\text{N even}}e^{-i\mathcal N\phi}\ket{\Phi_{N}},
\end{align}
where
\begin{align}\label{eq0.1.81}
\ket{\Phi_{N}}=\frac{\left(\sum_{\nu>0}c'_\nu P^\dagger_\nu\right)^{N/2}}{\left(N/2\right)!}\ket{0}_F,
\end{align}
with $c'_\nu=V'_\nu/U'_\nu$. It is of notice that
\begin{align}\label{eq0.1.82}
\sum_{\nu>0}c'_\nu P^\dagger_\nu\ket{0}_F
\end{align}
is the single Cooper pair state.  The  $\ket{BCS}$ state does not have a definite number of particles, but only in average being a wavepacket in $N$. 
In fact, (\ref{eq0.1.80}) defines a privileged direction in gauge space, being an eigenstate of $\hat\phi$ 
\begin{align}\label{eq0.1.83}
\hat\phi\ket{BCS}=i\frac{\partial}{\partial \mathcal N}\left(\prod_{\nu>0}U'_\nu\right)\sum_{\text{N even}}e^{-i\mathcal N\phi}\ket{\Phi_{N}}=\phi\,\ket{BCS}.
\end{align}
Expressing it differently (\ref{eq0.1.80}) can be viewed as an axially symmetric deformed system in gauge space, whose symmetry axis coincides with the $z'$ component of the body-fixed frame of reference $\mathcal K'$, which makes an angle $\phi$ with the laboratory $z$-axis (Fig. \ref{fig0.4.4}). 

With the help of Eq. (\ref{eq0.1.74}) (first line) one can write
\begin{align}\label{eq0.1.84}
\ket{BCS(\phi=0)}_{\mathcal K'}=\prod_{\nu>0}\left(U'_\nu+V'_\nu P_\nu'^\dagger\right)\ket{0}_F,
\end{align}
where use was made of the relations
\begin{align}\label{eq0.1.85}
\mathcal G(\phi)a^\dagger_\nu \mathcal G^{-1}(\phi)=e^{-i\phi}a^\dagger_\nu=a'^\dagger_\nu,
\end{align}
and 
\begin{align}\label{eq0.1.86}
\mathcal G(\phi)P^\dagger_\nu \mathcal G^{-1}(\phi)=e^{-2i\phi}P^\dagger_\nu=P'^\dagger_\nu.
\end{align}
It is to be noted that $\mathcal G$ induces a counter clockwise rotation\footnote{
	\begin{align*}
&[\mathcal N,\phi]=-i,\quad\mathcal G(\chi)=e^{-i\mathcal N\chi}\quad \chi:\text{c-number},\quad[e^{-i\mathcal N\chi},\phi]=e^{-i\mathcal N\chi}\phi-\phi e^{-i\mathcal N\chi},\\
&[e^{-i\mathcal N\chi},\phi]e^{i\mathcal N\chi}=e^{-i\mathcal N\chi}\phi e^{i\mathcal N\chi}-\phi,\quad e^{-i\mathcal N\chi}=1-i\mathcal N\chi\quad \chi^2\ll\chi\\
&[1-i\mathcal N\chi,\phi]=-i[\mathcal\chi,\phi]=-i\left(\mathcal N[\chi,\phi]+[\mathcal N,\phi]\chi\right)=-i(0-i\chi)=-\chi\\
&[e^{-i\mathcal N\chi},\phi]=-\chi e^{-i\mathcal N\chi}\\
&[e^{-i\mathcal N\chi},\phi]e^{i\mathcal N\chi}=-\chi e^{-i\mathcal N\chi}e^{i\mathcal N\chi}=-\chi,\quad -\chi=\mathcal G(\chi)\phi\mathcal G^{-1}(\chi)-\phi\\
&\mathcal G(\chi)\phi\mathcal G^{-1}(\chi)=\phi-\chi
	\end{align*}
},
\begin{align}\label{eq0.1.87}
\mathcal G(\chi)\,\hat\phi\, \mathcal G^{-1}(\chi)=\hat\phi-\chi.
\end{align}
As a consequence, to rotate $\ket{BCS(\phi=0)}_{\mathcal K'}$ back into the laboratory system, use has to be made of the clockwise rotation of angle $\phi$ induced by $\mathcal G^{-1}(\phi)$,
\begin{align}\label{eq0.1.88}
\nonumber \mathcal G^{-1}(\phi)\,\ket{BCS(\phi=0)}_{\mathcal K'}&=\prod_{\nu>0}\left(U'_\nu+V'_\nu\mathcal G^{-1}(\phi)P'^\dagger_\nu\right)\ket{0}_F\\
&=\prod_{\nu>0}\left(U'_\nu+e^{2i\phi}V_\nu P^\dagger_\nu\right)\ket{0}_F=\ket{BCS(\phi)}_{\mathcal K}
\end{align}
where use was made of the invariance of $\ket{0}_F$ with respect to gauge transformations, and of the fact that in (\ref{eq0.1.84}), $P'^\dagger=\mathcal G(\phi)P^\dagger_\nu\mathcal G^{-1}(\phi)$. Thus 
\begin{align}\label{eq0.1.89}
\mathcal G^{-1}(\phi)P'^\dagger_\nu\mathcal G(\phi)=\mathcal G^{-1}(\phi)\left(\mathcal G(\phi)P^\dagger_\nu\mathcal G^{-1}(\phi)\right)\mathcal G(\phi)=P^\dagger_\nu,
\end{align}
leading to
\begin{align}\label{eq0.1.90}
\nonumber\ket{BCS(\phi=0)}_{\mathcal K'}&=\prod_{\nu>0}\left(U'_\nu+V'_\nu P'^\dagger_\nu\right)\ket{0}_F=\prod_{\nu>0}\left(U_\nu+V_\nu P^\dagger_\nu\right)\ket{0}_F\\
&=\prod_{\nu>0}\left(U'_\nu+e^{-2i\phi}V'_\nu P^\dagger_\nu\right)\ket{0}_F.
\end{align}
Spontaneous broken symmetry in nuclei is, as a rule, associated with the presence of rotational bands, as already found in the case of quadrupole deformed nuclei. \textit{Consequently, one expects in nuclei with $\alpha_0\neq0$ rotational bands in which particle number plays the role of angular momentum. That is pairing rotational bands (see Eq. (\ref{eq0.1.101})).}


Before proceeding with the discussion of the term $H_{fluct}$ let us return to the relation (\ref{eq0.1.61}). That is the fact that one can view the state $\ket{BCS}$ as the quasiparticle vacuum. Thus implying  that it can be written as
\begin{align}\label{eq1.4.57}
\ket{BCS}\sim\prod_{\nu}\alpha_\nu\ket{0}_F,
\end{align}
as well as
\begin{align}\label{eq1.4.58}
\ket{BCS}\sim\prod_{\nu>0}\alpha_\nu\alpha_{\tilde\nu}\ket{0}_F,
\end{align}
The state $\ket{\nu'}\sim\prod_{\nu\neq\nu'}\alpha_\nu\ket{0}_F$ corresponds to a one quasiparticle state, while $\ket{\nu'\tilde\nu'}\sim\prod_{\nu>0,\nu\neq\tilde\nu}\alpha_\nu^\dagger\alpha_{\tilde\nu}^\dagger\ket{0}_F$ is proportional to a two-quasiparticle state. Both states display an energy (pairing) gap with respect to the ground state. In the first case of the order of $\Delta$, in the second $2\Delta$ . A property closely connected with the fact that in BCS condensation as described by $\ket{BCS}$, a macroscopic number of Cooper pairs are phase coherent, each pair being made out of fermions entangled in time reversal states. Implying an energy gap for both pair dissociation and single pair breaking. Property referred to as off-diagonal-long-range-order (ODLRO)\footnote{\cite{Yang:62}.} (see App. \ref{App4.B.3}). Within this scenario it is useful in discussing BCS condensation in nuclei to go back to its origin, namely superconductivity in metals. It is found that for superconductors displaying a strong electron-phonon coupling\footnote{\cite{Bohr:58}.}, the behaviour of the BCS occupation probability $V^2_k$ resembles very much that of the same quantity but for the metal at temperature close but above the critical temperature, i.e. the metal in the normal state. A consequence of processes like those shown in graphs (b) and (c) of Fig. \ref{fig1.0.8} where, in this case the fermions are electrons (or electron holes), and the bosons are phonons\footnote{See e.g. \cite{Tinkham:96} p.28.}. Such similitude is also observed in the nuclear case (Fig. \ref{fig0.4.3} (a), thin smooth curve around $\lambda$), and Fig. \ref{fig1.2.5}. Returning to superconductors, it is found that the changes observed in the behaviour of the system upon cooling below the critical temperature, cannot be described at profit just in terms of changes in the occupation probability of one-electron momentum eigenstate (from the sharp line to the smooth curve around $\lambda$, Fig. \ref{fig0.4.3} (a)), in keeping with the fact that in the process, no gap opens up in $E_k$-space. In fact, the partial occupation $V^2_\nu$ of the normal state carrying random (gauge) phases is, by lowering the temperature through $T_c$, progressively replaced by a single quantum state. In it, about the same set of many-body states, in which fermions are dressed and interact pairwise by coupling to bosons (aside than through the bare interaction), become coherently superposed with a fixed phase relation leading to ODLRO. It is then not so much, or better not only, the superconducting (superfluid) that is special, but also the so called normal state.
 
In what follows we  discuss the structure of $H_{fluct}$ and single out the term responsible for restoring gauge invariance to the BCS mean field solution giving thus rise to pairing rotational bands for discrete values of the angular momentum in gauge space, namely $N$, differing in two units from each other. In terms of quasiparticles, $H_{fluct}$ can be expressed  as
\begin{align}\label{eq0.1.91}
H_{fluct}=H'_p+H''_p+C
\end{align}
where 
\begin{align}\label{eq0.1.92}
H'_p=-\frac{G}{4}\left(\sum_{\nu>0}\left(U^2_\nu-V^2_\nu\right)\left(\Gamma^\dagger_\nu+\Gamma_\nu\right)\right)^2
\end{align}
and
\begin{align}\label{eq0.1.93}
H''_p=\frac{G}{4}\left(\sum_{\nu}\left(\Gamma^\dagger_\nu-\Gamma_\nu\right)\right)^2,
\end{align}
with
\begin{align}\label{eq0.1.94}
\Gamma^\dagger_\nu=\alpha^\dagger_\nu\alpha_{\tilde \nu}^\dagger.
\end{align}
The term $C$ stands for constant terms, as well as for terms proportional to the number of quasiparticles, which consequently vanish when acting on $\ket{BCS}$. The term $H'_p$ gives rise to two-quasiparticle pairing vibrations with energies $\gtrsim2\Delta$. It can be shown that it is the term $H''_p$ which restores gauge invariance\footnote{\cite{Hogassen:61,Bes:66}. For details see \cite{Brink:05}.}. In fact,
\begin{align}\label{eq0.1.95}
\left[H_{MF}+H''_p,\hat N\right]=0.
\end{align}
We now diagonalize $H_{MF}+H''_p$ in the quasiparticle RPA (QRPA),
\begin{align}\label{eq0.1.96}
\left[H_{MF}+H''_p,\Gamma^\dagger_n\right]=\hbar\omega_n\Gamma^\dagger_n,\quad \left[\Gamma_n,\Gamma^\dagger_{n'}\right]=\delta(n,n'),
\end{align}
where
\begin{align}\label{eq0.1.97}
\Gamma^\dagger_n=\sum_{\nu}\left(a_{n\nu}\Gamma^\dagger_n+b_{n\nu}\Gamma_\nu\right),\quad \Gamma^\dagger_\nu=\alpha^\dagger_\nu\alpha^\dagger_{\tilde\nu},	
\end{align}
is the creation operator of the $n$th vibrational mode. In the case of the $n=1$, lowest energy root, it can be written as
\begin{align}\label{eq0.1.98}
\ket{1''}=\Gamma''^\dagger_1\ket{0''}=\frac{\Lambda''_1}{2\Delta}(\hat N-N_0)\ket{0''},	
\end{align}
where $\hat N$ is the particle number operator written in terms of $\Gamma^\dagger_\nu$ and $\Gamma_\nu$, and $\Lambda''_1$ is the strength of the quasiparticle-mode coupling. The prefactor is the zero point fluctuation (ZPF) of the mode, that is (see Eq. (\ref{eq1.0.9}) for the case of surface vibration),
\begin{align}\label{eq0.1.99}
\sqrt{\frac{\hbar\omega''_1}{2C''_1}}=\sqrt{\frac{\hbar^2}{2D''_1\hbar\omega''_1}}.
\end{align}
Because the lowest frequency is $\omega''_1=0$, the associated ZPF diverge $(\Lambda''_1\sim(\hbar\omega''_1)^{-1/2})$. It can be seen that this is because $C''_1\to0$, while $D''_1$ remains finite. In fact,
\begin{align}\label{eq0.1.100}
\frac{D''_1}{\hbar^2}=\frac{2\Delta^2}{\Lambda''^2_1\hbar\omega''_1}=4\sum_{\nu>0}\frac{U^2_\nu V^2_\nu}{2E_\nu}.
\end{align}
 A rigid rotation in gauge space can be generated by a series of infinitesimal operations of type $\mathcal G(\delta\phi)=e^{i(\hat N-N_0)\delta\phi}$, the one phonon state $\ket{1''}=\Gamma^\dagger_1\ket{0''}$ is obtained from rotations in gauge space of divergent amplitude. That is, fluctuations of $\phi$ over the whole $0-2\pi$ range. By proper inclusion of these fluctuations one can restore gauge invariance violated by $\ket{BCS}_{\mathcal K'}$. The resulting states\footnote{Where use has been made of Eq. (\ref{eq0.1.90}), (\ref{eq0.1.80}) and (\ref{eq3.7.27}).}
\begin{align}\label{eq0.1.101}
\ket{N}\sim\int^{2\pi}_0d\phi\,e^{iN\phi}\ket{BCS(\phi)}_{\mathcal K'}\sim\left(\sum_{\nu>0}c'_\nu P^\dagger_\nu\right)^{N/2}\ket{0}_F
\end{align}
have a definite number of particles and constitute the members of a pairing rotational band. Making use of a simplified model (single $j$-shell) it can be shown that the energies of these states can be written as,
\begin{align}\label{eq0.1.102}
E_N=\lambda(N-N_0)+\frac{G}{4}\left(N-N_0\right)^2,
\end{align}
where 
\begin{align}\label{eq0.1.103}
\frac{G}{4}=\frac{\hbar^2}{2D''_1}.
\end{align}
An example of pairing rotational bands is provided by the ground state of the single open closed shell superfluid isotopes of the $^A_{50}$Sn$_N$-isotopes (Fig. \ref{fig0.4.5}), $N_0=68$ having been found to minimize $E_0=\braket{BCS|H|BCS}$ (see Eq. (\ref{eq0.1.62})), and thus has been used in the solution of the BCS number equation (\ref{eq0.1.72}). Theory provides an overall account of the experimental findings. Making use of the BCS pair transfer amplitudes,
\begin{align}\label{eq0.1.104}
\braket{BCS|P^\dagger_\nu|BCS}=U_\nu V_\nu
\end{align}
in combination with a computer code and of global optical parameters, one can account for the  absolute value of the Cooper pair transfer differential cross section, within experimental errors (see Figs. \ref{fig1.5} and   \ref{fig8_2_4}). The fact that projecting out the different Sn-isotopes from the intrinsic BCS state describing $^{118}$Sn one obtains a quantitative description of observations carried out with the help of the specific probe of pairing correlations (Cooper pair transfer), testifies to the fact that pairing rotational bands can be considered elementary modes of nuclear excitation, emergent properties of spontaneous symmetry breaking of  gauge invariance. 

Furthermore, the fact that these results follow the use of QRPA\footnote{Using the Tamm-Dancoff approximation, i.e. setting $Y\equiv0$ (and thus $\sum X^2=1$) in the QRPA approximation does not lead to particle number conservation, in keeping with the fact that the amplitudes $Y$ are closely connected to ZPF.}in the calculation of the ZPF of the collective solutions of the pairing Hamiltonian indicates the importance of conserving approximations\footnote{That is approximations respecting sum rules.} to describe quantum many-body problems in general, and the finite size quantum many-body problem (FSQMB), of which the nuclear case represents a paradigmatic example, in particular. 

Aside from low-lying collective states, that is rotations and low-energy vibrations, nuclei also display high-lying collective modes known as giant resonances.
\section{Giant dipole resonance}\label{S1.5}
If one shines a beam of photons on a nucleus it is observed that the system 
absorbs energy resonantly essentially at a single frequency, of the order of\footnote{Making use of $h=4.1357\times10^{-15}$ eVs one obtains for $h\nu=1$ eV the frequency $\nu=2.42\times10^{14}$ Hz and thus $\nu=4.8\times10^{21}$ Hz for $h\nu=20$ MeV. The wavelength of a photon of energy $E$ is $\lambda=hc/E\approx2\pi\times200$ MeV fm/$E$, which for $E=20$ MeV leads to $\lambda\approx63$ fm.} $\nu=5\times10^{21}$ Hz, corresponding to an energy of $h\nu\approx20$ MeV.

It is not difficult to understand how $\gamma$-rays excite a nuclear dipole vibration. A photon carries with it an oscillating electric field. Although the wavelength of a 20 MeV $\gamma$-ray is smaller than that of other forms of electromagnetic radiation such as visible light, it is still large ($\lambda\approx63$ fm) compared to the dimensions of e.g. $^{40}$Ca ($R_0\approx4.1$ fm). As a result the photon electric field is nearly uniform across the nucleus at any time. The field exerts a force on the positively charged protons. Consequently, it can set the center of mass into an antenna like, dipole oscillation (Thompson scattering), in which case no photon is absorbed. Another possibility is that it leads to an internal excitation of the system. In this case because the center of mass of the system does not move, the neutrons have to oscillate against the protons, again in an antenna-like motion. The restoring force of the vibration, known as the giant dipole resonance (GDR), is provided by the attractive force between protons and neutrons.

The connotation of giant is in keeping with the fact that it essentially carries the full photo absorption cross section (energy weighted sum rule, see below), and resonance because it displays a Lorentzian-like shape with a full width at half maximum of few MeV $(\lesssim5$ MeV), considerably  smaller than the energy centroid\footnote{Within this context we refer to the discussion concerning the renormalization of collective modes carried out in the text before Eq. (\ref{eq1.0.26}), in particular regarding the cancellation between self-energy and vertex corrections (see Fig. \ref{fig1.0.9} and footnote \ref{footnote7}). This is a basic result of NFT --as discussed in later chapters-- being connected with sum rule arguments in general, and particle conservation in particular. Argument which has been extended to the case of finite temperature as well as to include relativistic effects \cite{Ward:50,Nambu:60,Bortignon:81,Bertsch:83,Bortignon:98,Litvinova:18,Wibowo:19}.} $\hbar\omega_{GDR}\approx80/A^{1/3}$ MeV. Microscopically, the GDR can be viewed as a correlated particle-hole excitation, that is a state made out of a coherent linear combination of proton and neutron particle-hole excitations with essentially $\Delta N=1$, as well as small $\Delta N=3,5,\dots$ components (Fig. \ref{fig1.0.7}). Because the difference in energy between major shells is $\hbar\omega_0\approx41A^{1/3}$ MeV, one expects that about half of the contribution to the energy of the GDR arises from the neutron-proton interaction. More precisely, from the so called  symmetry potential $V_1$ (see Eq.  (\ref{eq1.0.4bis})), which measures the energy price the system has to pay to separate protons from neutrons. Theoretical estimates lead to 
\begin{align}\label{eq0.1.106}
(\hbar\omega_{GDR})^2=(\hbar\omega_0)^2+\frac{3\hbar^2V_1}{m\braket{r^2}}=\frac{1}{A^{2/3}}\left[(41)^2+(60)^2\right]\text{ MeV}^2,
\end{align}
resulting in

\begin{align}\label{eq0.1.107}
\hbar\omega_{GDR}\approx\frac{73}{A^{1/3}}\approx\frac{87}{R}\text{ MeV,}
\end{align}
where $R=1.2A^{1/3}$ is the numerical value of the nuclear radius when measured in fm. The above quantity is to be compared with the empirical value $\hbar\omega_{GDR}\approx(80/A^{1/3})$ MeV $\approx(95/R)$ MeV.

It is of notice that the elastic vibrational frequency of a spherical solid made out of particles of mass $m$ can be written as $\omega_{el}^2\sim\mu/(m\rho R^2)\sim v_l^2/R^2$, where $R$ is the radius, $\rho$ the density and $v_t$ the transverse sound velocity proportional to the Lam\'e shear modulus of elasticity $\mu$.
In other words, giant resonances in general, and the GDR in particular, are embodiments of the elastic response\footnote{\cite{Bertsch:05}.} of the nucleus to an impulsive external fields, like that provided by a photon. The nuclear rigidity to sudden solicitations is provided by the shell structure, quantitatively measured by the energy separation between major shells.
\section{Giant pairing vibrations}\label{Sect2.6}
Due to spatial quantization, in particular to the existence of major shells of pair degeneracy $\Omega(\equiv(2j+1)/2)$ separated by an energy $\hbar\omega_0\approx41/a^{1/3}$ MeV, the Cooper pair model can be extended to encompass pair addition and pair substraction modes across major shells\footnote{\cite{Broglia:77}.}. Assuming that both the single-particle energies $\epsilon_k$ and $\epsilon_i$ appearing in Fig. \ref{fig0.3.1} are both at an energy    $\hbar\omega_0$ away from the Fermi energy, one obtains the dispersion relation 
\begin{align}\label{eq0.1.108}
-\frac{1}{G}=\frac{\Omega}{W-2\hbar\omega_0}-\frac{\Omega}{W+2\hbar\omega_0},
\end{align}
leading to
\begin{align}\label{eq0.1.109}
(2\hbar\omega_0)^2-W^2=4\hbar\omega_0G\Omega,
\end{align}
and implying a high lying pair addition mode of energy
\begin{align}\label{eq0.1.110}
W=2\hbar\omega_0\left(1-\frac{G\Omega}{\hbar\omega_0}\right)^{1/2}.
\end{align}
The forwards (backwards) going RPA amplitudes are, in the present case
\begin{align}\label{eq0.1.111}
X=\frac{\Lambda_0\Omega^{1/2}}{2\hbar\omega_0-W}\quad\text{ and      }\quad Y=\frac{\Lambda_0\Omega^{1/2}}{2\hbar\omega_0+W}\
\end{align}
normalized according to the relation 
\begin{align}\label{eq0.1.112}
1=X^2-Y^2=\Lambda^2_0\Omega\frac{8\hbar\omega_0W}{\left((2\hbar\omega_0)^2-W^2\right)^2},
\end{align}
where $\Lambda_0$ stands for the particle-pair vibration coupling vertex. Making use of (\ref{eq0.1.109}) one obtains
\begin{align}\label{eq0.1.113}
\left(\frac{\Lambda_0}{G}\right)^2=\Omega\left(1-\frac{G\Omega}{\hbar\omega_0}\right)^{-1/2}
\end{align}
quantity corresponding, within the framework of the simplified model used, to the two-nucleon transfer cross section. Summing up, the monopole giant pairing vibration has an energy close to $2\hbar\omega_0$, and is expected to be populated in two-particle transfer processes with a cross section of the order of that associated with the low-lying pair addition mode, being this one of the order of $\Omega$. Simple estimates of (\ref{eq0.1.110}) and (\ref{eq0.1.113}) can be obtained making use of $\Omega\approx\frac{2}{3}A^{2/3}$
and $G\approx17/A$ MeV, namely 
\begin{align}\label{eq0.1.114}
W=0.85\times2\hbar\omega_0,\quad\left(\frac{\Lambda_0}{G}\right)^2\approx1.2 \Omega.
\end{align}
Experimental evidence of GPV in light nuclei have been reported\footnote{\cite{Cappuzzello:15}; \cite{Cavallaro:19},  \cite{Bortignon:16}. See also \cite{Laskin:16,Betan:02,Dussel:09,Mouginot:11,Khan:04}.}
\section{Sum rules}\label{Sect1.7}
There are important operator identities which restrict the possible matrix elements in a physical system. Let us calculate the double commutator of the Hamiltonian describing the system and a single-particle operator $F$. That is
\begin{align}\label{eq0.1.115}
\left[\hat F,\left[H,\hat F\right]\right]=\left(2\hat FH\hat F-\hat F^2H-H\hat F^2\right)
\end{align}
Let us assume that $\hat F=\sum_k F(\mathbf r_k)$ and $H=T+v(\mathbf r, \mathbf r')$, where $v(\mathbf r,\mathbf r')=-\kappa_1 \hat F(\mathbf r)\hat F(\mathbf r')$. Thus 
\begin{align}\label{eq0.1.116}
\left[\hat F,\left[H,\hat F\right]\right]=\sum_k\frac{\hbar^2}{m}\left(\pmb \nabla_kF(\mathbf r_k)\right)^2
\end{align}
Let us take the average value on the correlated ground state
\begin{align}\label{eq0.1.117}
S(F)=\sum_{\alpha'}|\braket{\alpha'|F|\tilde 0}|^2(E_{\alpha'}-E_0)=\frac{\hbar^2}{2m}\int d\mathbf r\, |\pmb\nabla F|^2\rho(\mathbf r),
\end{align}
where we have used $H\ket{\alpha}=E_\alpha\ket{\alpha}$ and $H\ket{\tilde 0}=E_0\ket{\tilde 0}$, and the sum $\sum_{\alpha'}$ is over the complete set of eigenstates of the system. The above result describes the reaction of a system at equilibrium to which one applies an impulsive field, which gives the particles a momentum $\pmb \nabla F$. On the average, the particles started at rest so their average energy after the sudden impulse is $\hbar^2|\pmb \nabla F|^2/2m$, a result which does not depend on the interaction acting among the nucleons, the energy being absorbed from the (instantaneous) external field before the system is disturbed from equilibrium. The result (\ref{eq0.1.117}) is known as the energy weighted sum rule (EWSR). 

An important application of (\ref{eq0.1.117}) implies a situation where $F$ has a constant gradient. Inserting $\mathbf F=z\times\mathbf z$ in (\ref{eq0.1.117}), the integral simplifies because $\pmb \nabla F=1$, and the integral leads just to the number of particles,
\begin{align}\label{eq0.1.118}
\sum_{\alpha}|\braket{\alpha|F|\tilde 0}|^2(E_{\alpha}-E_0)=\frac{\hbar^2N}{2m}.
\end{align}
The electric field of a photon is of this form in the dipole approximation, which is valid when the size of the system is small compared to the wavelength of the photon, the single-particle field being
\begin{align}\label{eq0.1.119}
F(\mathbf r)=e\sum_{k}\left[\frac{N-Z}{A}-t_z(k)\right]r_kY_{1\mu}(\hat r_k),
\end{align}
with $t_z=-1/2$ for protons and $+1/2$ for neutrons. For the dipole operator referred to the nuclear center of mass one obtains
\begin{align}\label{eq0.1.120}
\sum_{\alpha'}|\braket{\alpha'|F|\tilde 0}|^2(E_{\alpha'}-E_0)=\frac{9}{4\pi}\frac{\hbar^2e^2}{2m}\frac{NZ}{A}.
\end{align}
 The above relation is known as the Thomas-Reiche-Kuhn (TRK) sum rule, and is equal to the maximum energy a system can absorb from the dipole field. The RPA solution respects the EWSR, while the Tamm-Dancoff approximation (TDA), resulting by setting $Y^\alpha_{ki}=0$ and normalizing the $X$-components $(\sum_{ki}{X^{\alpha}_{ki}}^2=1)$ fulfill the non-energy weighted sum rule. A fact which testifies to the important role ZPF play in nuclei.
\section{Ground state correlations}\label{S1.8}
The zero point fluctuations associated with collective nuclear vibrations affect, among other thing, the nuclear mean field properties\footnote{\cite{Gogny:78,Esbensen:83,Reinhard:79,Khodel:82,Barranco:87a,Barranco:85}. See also \cite{Brown:63,Anderson:62} (it is likely a coincidence in connection with this inaugural issue of Phys. Lett. that short of hundred pages after, one finds the paper of B. D. Josephson, Possible new effects in superconducting tunneling, Phys. Lett. \textbf{1}, 251 (1962)) .}. Within the harmonic aproximation discussed in Sections \ref{S1.1.1} and \ref{S1.1.2} in connection with ($\beta=0$) surface vibrational modes, the ground state $\Psi_0$ of (\ref{eq1.0.7}) has an energy $\frac{1}{2}\hbar\left(\frac{C_\lambda}{D_\lambda}\right)^{1/2}=\hbar\omega_\lambda/2$ for each degree of freedom (e.g. ($2\lambda+1$) in the case of a vibration of multipolarity $\lambda$). The associated coordinate-momentum indeterminacy relation assumes its minimum value\footnote{The same result is found for $\Psi_n$ describing a state with $n$-quanta, and the basis that solutions with $n\gg1$ behaves as ``quasiclassical'' or ``coherent'' states of the harmonic oscillator (\cite{Glauber:07}) in keeping with the fact that the contribution of the zero point energy is negligible in such case $((n+1/2)\hbar\omega\approx n\hbar\omega)$ and that the many quanta wavepacket always attain the lower limit of (\ref{eq0.1.121}) (\cite{Basdevant:05} pp. 153,465) (discussions with Pier Francesco Bortignon in March 2018 concerning coherent states are gratefully acknowledged). Schr\"odinger used this result  in a paper (\cite{Schrodinger:26}) to suggest that waves (material waves) described by his wave function are the only reality, particles being only derivative things. In support of his view he considered a superposition of linear harmonic oscillator wavefunctions and showed that the wave group holds permanently together in the course of time. And he adds that the same will be true for the electron as it moves in high orbits of the hydrogen atom, hoping that wave mechanics would turn out to be a branch of classical physics (\cite{Pais:00}). It was Born who first provided the correct interpretation of Scrh\"odinger's wavefunction (modulus square) in his paper ``Quantum mechanical collision phenomena'' (\cite{Born:26}). In it it is stated that the result of solving with wave mechanics the process of elastic scattering of a beam of particles by a static potential is not what the state after the collision is, but how probable is a given effect of the collision.}
\begin{align}\label{eq0.1.121}
\Delta\alpha_{\lambda\mu}{(n)}\Delta\pi_{\lambda\mu}{(n)}=\frac{\hbar}{2},
\end{align}
as
\begin{align}\label{eq1.8.2}
\Delta\alpha_{\lambda\mu}{(n)}=\sqrt{\frac{\hbar\omega_\lambda(n)}{2C_\lambda(n)}},
\end{align}
and $\Delta\pi_{\lambda\mu}=\sqrt{\hbar D\omega_\lambda/2}$, and in keeping with the fact that $|\Psi_0|^2$ is mathematically a Poisson distribution. The above expression implies that the mean square radius will be modified from its mean field value $R_0$  and thus also the nuclear density distribution. The value of $\hbar\omega_\lambda(n)/2C_\lambda(n)$ can be determined by calculating the collective mode $\ket{n_\lambda(n)=1}=\Gamma_{\lambda\mu}^\dagger(n)\ket{\tilde 0}$ in RPA. As seen from the caption to Fig. \ref{fig1.0.7} the zero point fluctuation of the vibration enter the definition of the $Y$-amplitudes of the mode. Let us start by calculating the effect of the zero point fluctuations on the nuclear density distribution. The corresponding operator can be written as
\begin{align}\label{eq0.1.123}
\hat\rho(\mathbf r)=a^\dagger(\mathbf r)a(\mathbf r),
\end{align}
where $a^\dagger(\mathbf r)$ is the creation operator of a nucleon at point $\mathbf r$. It can be expressed in terms of the phase space creation operators $a^\dagger_\nu(\nu\equiv n,l,j,m)$ as
\begin{align}\label{eq0.1.124}
a^\dagger(\mathbf r)=\sum_\nu\varphi^*_\nu(\mathbf r) a^\dagger_\nu,
\end{align}
where $\varphi_\nu(\mathbf r)$ are the single-particle wavefunctions. Thus
\begin{align}\label{eq0.1.125}
\hat\rho(\mathbf r)=\sum_{\nu\nu'}\varphi^*_\nu(\mathbf r)\varphi_{\nu'}(\mathbf r)a^\dagger_\nu a_{\nu'}.
\end{align}
The matrix element in the HF ground state is (Fig. \ref{fig1.0.6})
\begin{align}\label{eq0.1.126}
\rho_0(\mathbf r)=_F\braket{0|\hat\rho(\mathbf r)|0}_F=\sum_{i, (\epsilon_i\leq\epsilon_F)}|\varphi_i(\mathbf r)|^2.
\end{align}
To lowest order of perturbation theory in the particle-vibration coupling vertex, the NFT diagrams associated with the change of $\rho_0$ due to ZPF are shown in Fig. \ref{fig0.5.1}. Graphs (a) and (b) and (c) and (d) describe the changes in the density operator and in the single-particle potential respectively. This can be seen from the insets (I) and (II). The dashed horizontal line starting with a cross and ending at a hatched circle represents the renormalized density operator. This phenomenon is similar to that encountered in connection with vertex renormalization in Fig. \ref{fig1.0.9}, that is the renormalization of the particle-vibration coupling (insets (I) and (I')). Concerning potential renormalization, the bold face arrowed line shown in inset (II) of Fig. \ref{fig0.5.1} represents the motion of a renormalized nucleon due to the self-energy process induced by the coupling to vibrational modes. A phenomenon which can be described at profit through an effective mass, the so called $\omega$-mass $m_\omega$, in which case particle motion is described by the Hamiltonian\footnote{The ratio $(m/m_\omega)$ gives a measure of the single-particle energy content (\cite{Mahaux:85}, Eq. (3.5.18)), while $m_\omega$ is proportional to the (energy-) slope of the (single-particle) self-energy dispersion relation (Sect. \ref{C4AppA1}; see also  \cite{Brink:05} and refs. therein).}$\left(\hbar^2/2m_\omega\right)\nabla^2+\left(\frac{m}{m_\omega}\right)U(r)$. The $\omega$-mass can be written as $m_\omega=(1+\lambda)m$, where $m$ is the nucleon mass and $\lambda$ is the so called mass enhancement factor $\lambda=N(0)\Lambda$, where $N(0)$ is the density of levels at the Fermi energy, and $\Lambda$ the PVC vertex strength, typical values being $\lambda=0.4$.

 The fact that in calculating $\delta\rho$, that is, the correction to the nuclear density distribution (renormalization of the density operator), one finds to the same order of perturbation a correction to the potential, is in keeping with the self consistency existing between the two quantities (Eq. (\ref{eq1.0.18})). \textit{Now, what changes is not only the single-particle energy, but also the single-particle content measured by $Z_\omega=m/m_\omega$, as well as the radial dependence of the wavefunctions of the states}. It is of notice that the effective mass approximation, although being quite useful, cannot take care of the energy dependence of the renormalization process which leads, in the case of single-particle motion to renormalized energies, spectroscopic amplitudes and radial wavefunctions. 
\begin{figure}
	\centerline {
		\includegraphics*[width=15cm, angle=0.]{introduccion/figs/fig0_8_1}
	}
	\caption{Lowest-order corrections in the particle-vibration coupling vertex of the nuclear density due to the presence of zero-point fluctuations associated with density vibrations. An arrowed line pointing upwards denotes a particle, while one pointing downward a hole. A wavy line represents a surface phonon. The density operator is described through a dotted horizontal line starting with a cross. Graphs (\textbf{a}) and (\textbf{b}) are typical examples of density contributions to $\delta\rho$ (see inset (\textbf{I}); the dashed horizontal line starting with a cross and ending at a hatched circle in the diagram to the right, represents the renormalized density operator, resulting from the processes displayed to the left); (\textbf{c}) and (\textbf{d}) are of potential contributions (see inset (\textbf{II}); the bold face arrowed line represents the renormalized single-particle state due to the coupling to the vibrations leading to the self energy process shown to the left).}
	\label{fig0.5.1}
\end{figure}
\begin{figure}
	\centerline {
		\includegraphics*[width=10cm, angle=0.]{introduccion/figs/fig0_8_2}
	}
	\caption{Modification in the charge density of $^{40}$Ca induced by the zero-point fluctuations associated with vibrations of the surface modes. The results labeled HF, HF+(a)+(b), and HF+(a)+(b)+(c)+(d) are the Hartree-Fock density, and that resulting from adding to it the corrections $\delta\rho$ associated with the processes (a)+(b) and (a)+(b)+(c)+(d) displayed in Fig. \ref{fig0.5.1}, respectively. In the lower part of the figure the  quantities $\delta\rho$ are displayed.}
	\label{fig0.5.2}
\end{figure}
\begin{figure}
	\centerline {
		\includegraphics*[width=8cm, angle=0.]{introduccion/figs/fig0_5_3}
	}
	\caption{(\textbf{a}) Example of zero point fluctuation of the ground state of the double-magic nucleus $^{208}_{82}$Pb$_{126}$ associated with the low-lying octupole vibration of this system, observed at an energy of 2.615 MeV and displaying a collective electromagnetic decay to the ground state. The \textbf{proton} particle-hole component $(h_{9/2},d^{-1}_{3/2})_{3^-}$ displayed carries a large amplitude in the octupole vibration wavefunction. (\textbf{b}) Diagram representing the transfer of one proton to $^{208}$Pb, which fills the $d_{3/2}^{-1}$ hole state leading to a $3/2^+$ in $^{209}_{83}$Bi$_{126}$, member of the septuplet of states $\ket{(3^-\otimes h_{9/2}J^\pi)}$ with $J^\pi=3/2^+,5/2^+,\dots,15/2^+$. The horizontal dashed line starting with a cross stands for the stripping process $(^3\text{He},d)$.}
	\label{fig0.5.3}
\end{figure}
The analytic expressions associated with diagrams (a) and (c) of Fig. \ref{fig0.5.1} are 
\begin{align}\label{eq0.1.127}
\delta\rho(r)_{(a)}=\frac{(2\lambda+1)}{4\pi}\sum_{\nu_1\nu_2 n}\left[Y_n(\nu_1\nu_2;\lambda)\right]^2 R_{\nu_1}(r)R_{\nu_2}(r),
\end{align}
and
\begin{align}\label{eq0.1.128}
\nonumber \delta\rho(r)_{(c)}&=(2\lambda+1)\Lambda_n(\lambda)\sum_{\nu_1\nu_2\nu_3}\frac{M(\nu_1,\nu_3;\lambda)}{\epsilon_{\nu_1}-\epsilon_{\nu_2}}(2j_1+1)^{-1/2}\\&\times Y_n(\nu_3\nu_2;\lambda)\times R_{\nu_1}(r)R_{\nu_2}(r),
\end{align}
where $M$ is the matrix element of $\frac{R_0}{\kappa}\frac{\partial U}{\partial r}Y_{\lambda\mu}(\hat r)$ and $n=1,2\dots$ the first, second, etc vibrational modes as a function of increasing energy, and $\Lambda_n$ is the strength of the particle-vibration coupling associated with the $n$-mode of multipolarity $\lambda$. The functions $R(r)$ are the radial wavefunctions associated with the states $\nu_1$ and $\nu_2$. While $\delta\rho_{(a)}$ can be written in terms of the RPA $Y$-amplitudes which are directly associated with the zero point fluctuations of harmonic motion (Fig. \ref{fig1.0.7} (c)), $\delta\rho_{(c)}$ contains a scattering vertex not found in RPA -- that is going beyond the harmonic approximation-- and essential to describe renormalization processes of the different degrees of freedom, namely single-particle (energy, single-particle content and radial dependence of the wavefunction) and collective motion, as well as interactions. In particular the pairing interaction. 

In Fig. \ref{fig0.5.2}  results of calculations of $\delta\rho$ carried out for the closed shell nucleus $^{40}$Ca are shown. The vibrations were calculated by diagonalizing separable interactions of multipolarity $\lambda$ in the RPA. All the roots of multipolarity and parity $\lambda^\pi=2^+,3^-,4^+$ and $5^-$ which exhaust the EWSR were included in the calculations. Both isoscalar and isovector degrees of freedom were considered, as well  low-lying and giant resonances.

From the point of view of the single-particle motion the vibrations associated with low-lying modes display very low frequency ($\hbar\omega_\lambda/\epsilon_F\approx0.1$) and lead to an ensemble of adiabatic deformed shapes. Nucleons can thus reach to distances from the nuclear center which are considerably larger than the radius $R$ of the static spherical potential. Because the frequency of the giant resonances are of similar magnitude to those corresponding to the single-particle motion, the associated surface deformations average out. 
Said it differently,  the low-lying vibrational modes  account for most of the contributions to the changes in the density distribution\footnote{\label{f52C1} Another example of the  recurrent central role played by low-frequency modes in determining the properties and behavior of systems at all levels of organization, from the atomic nucleus to the Casimir effect in QED (see Fig. \ref{fig6G3}), to phonons in superconductors as well as  to the folding of proteins and brain  activity ($\nu<0.1$ Hz) (\cite{Mitra:18,Vyazovskiy:13}).}. Making use of the corresponding ($\delta\rho)_{low-lying}$, the mean square radius of $^{40}$Ca was calculated\footnote{\cite{Barranco:87a}.}, leading to $\braket{r^2}=(3/5)R_0^2=10.11$ fm$^2 (R_0=1.2A^{1/3}\text{ fm}=4.1\text{ fm})$, in overall agreement with the experimental findings. 

Similar calculations to the ones discussed above, but in this case taking into account only the contributions of the low-lying octupole vibration\footnote{\cite{Brown:63}. See also \cite{Anderson:62}.} indicate that nucleons are to be found a reasonable part of the time in higher shells than those assigned to them by the shell model. The average number of ``excited'' particles being $\approx2.4$. 
%If these are present, pickup reactions such as $(p,d)$ and $(d,t)$ will show them . From the nature of the correlations, the pickup of such a particle will leave the residual nucleus in an excited state composed of a hole and a vibration (Fig. \ref{fig0.5.3}). 
Correspondingly, hole states will be left behind. Because of the presence of hole states in the closed shell nucleus, one can transfer a nucleon to states below the Fermi energy in, for example, $(d,p)$ or ($^3$He,$d$) one-neutron or one-proton stripping reactions respectively, leaving the final nucleus with one-nucleon above closed shell coupled to the vibrations.


Systematic studies of such multiplets have been carried out throughout the mass table. In particular around the closed shell nucleus $^{208}_{82}$Pb$_{126}$ (Fig. \ref{fig0.5.3}). Within this context it is not only  quite natural but also necessary, to deal with structure and reactions on equal footing.  This is one of the main goals of the present monograph. 
\section{Interactions}\label{S1.9}
A number of subjects are not touched upon in the present monograph. In particular, the role temperature plays in the structure and decay of nuclei and that of bare (e.g. the Argonne $NN$-potential) and/or effective forces (e.g. Gogny, Skyrme, etc.). In what follows we briefly elaborate on this last point, referring to \cite{Bortignon:98} and refs. therein concerning the first one.

The Coulomb interaction resulting from the exchange of photons between charged particles defines the domain area of quantum electrodynamics (QED). Being the electron and the photon fields in interaction, what we call an electron is only partially to be associated with the electron field alone. It is also partially to be associated with the photon field which dresses the electron field. And conversely what is called the photon field can materialize itself in space in terms of an electron and a positron\footnote{\cite{Schwinger:01}.}.

The Coulomb interaction is the best known of all physical interactions, and QED constitutes the paradigm of theories which can be considered correct. In natural units in which the magnetic moment of the electron emerging from Dirac equation has the value of 1, the results of QED corrections agree --once divergent contributions have been renormalized by properly adjusting the value of the bare electron mass and charge\footnote{Changing indirectly the Coulomb interaction. Not that observed in the laboratory and acting between physical (dressed) electrons, but the ``bare'' Coulomb interaction acting between the ``naked'' electron fields, and used in the calculation of the variety of processes through which the photon field dresses the electrons.}--, within experimental errors, that is down to the eleventh decimal figure, with observation. That is 1.00115965221$\pm$0.00000000003 (exp.); 1.00115965246$\pm$0.00000000020 (QED)\footnote{\cite{Kinoshita:90}.}.


 The energy difference between the $2S_{1/2}$ and the $2P_{1/2}$ states of the hydrogen atom which according to Dirac's theory should be degenerate, emerges naturally, according to QED, from the dressing of the hydrogen's electron by the photon associated with the zero point fluctuation of the electromagnetic vacuum. The measured value of the Lamb shift\footnote{\cite{Lamb:51b}.}  is 1057.845 (9) MHz, an experimental value\footnote{\cite{Lundeen:81,Lundeen:86,Pipkin:90}.} whose accuracy is limited by the $\approx 100$ MHz natural line width of the $2P_{1/2}$ state. The best QED value, limited by uncertainties in the radius of the proton, is 1057.865 MHz\footnote{\cite{Sapirstein:90,Grotch:94}.}

The above two numbers (electron magnetic moment and Lamb shift) are results of \textit{ab initio} calculations within the framework of an effective field theory (QED). \textit{Ab initio} to the extent that quantum mechanics in general\footnote{\cite{Heisenberg:25,Born:25a,Born:25b,Dirac:25,Schrodinger:26,Born:26,Heisenberg:27,Pauli:25}.} and the Dirac equation\footnote{\cite{Dirac:28a,Dirac:28b}.} in particular can be considered, on grounds of their universal validity, to be so, including QED with its bare mass and charge parameters used to cure infinities (renormalization)\footnote{\cite{Feynman:49,Schwinger:48,Tomonaga:46}; see also \cite{Dyson:49,Schwinger:58,Schweber:94}.}.

Nuclear field theory (NFT) was developed following the (graphical) approach adopted by Feynman in his formulation of QED, making use of the correspondence electron$\to$nucleon, photons$\to$collective vibrations, fine structure constant$\to$particle-vibration coupling. In keeping with the  fact that in Feynman's QED \textit{nothing is really free\footnote{\cite{Feynman:75}.} (bare)} and of the very large value of the induced (retarded) nuclear pairing interaction --as large or larger than that of the bare $NN$-interaction in e.g. the case of the $^1S_0$ (pairing channel)\footnote{For example, concerning the contribution to the pairing gap of $^{120}$Sn ($\Delta\approx1.45$ MeV) arising from the bare $^1S_0$ (pairing) interaction $v_{14}$ (Argonne $NN$-potential) and that associated with the exchange of collective vibrational modes between nucleons moving in time reversal states close to the Fermi energy (\cite{Idini:15}). In the case of $^{11}$Li, essentially all of the pairing interaction which binds the two halo neutrons to the core $^9$Li, arise from the exchange between them, of the soft $E1$-mode (pygmy dipole resonance  of $^{11}$Li). (\cite{Barranco:01,Broglia:19}). }-- arising from the exchange of collective vibrations between nucleons, the role which specific (bare) nuclear forces (four-point vertices $v$) play in the results of NFT calculations, becomes somewhat blurred. 


Starting from the mean field (see Eq. (\ref{eq1.0.18})) at the basis of the non-observable bare single-particle energies, one employs a generic shape (Woods-Saxon), adjusting the depth, radius and diffusivity of the central potential, and the depth of the spin-orbit one in  connection with an $r$-dependent effective $k$-mass (exchange potential) so that the dressed single-particle states reproduce the experimental findings\footnote{See e.g. \cite{Barranco:17} and \cite{Barranco:20}.}. Parallel to the bare mass parameter entering renormalized QED.
In connection with the coupling between nucleons and surface vibrations, namely the strength $\Lambda_\alpha$, one adjusts $\kappa$ so as to reproduce the properties of the collective modes (see inset Fig. \ref{fig1.0.7}), a procedure which parallels the tuning of the bare charge, and thus the strength of the coupling between electrons and photons (and of the bare Coulomb interaction) in renormalized QED. In other words, renormalized nuclear field theory (NFT)$_{\text{ren}}$

The Mayer and Jensen sequence of levels around the  $N=8$ magic nuclei is $1p_{1/2},1d_{5/2},2s_{1/2}$ (Fig. \ref{fig1.0.3}), while experimentally $^{11}_{4}$Be$_{7}$ displays the sequence\footnote{\cite{Kwan:14}.} $1/2^+,1/2^-$ (bound), $5/2^+$ (resonance). A consequence of the dressing of the bare $1p_{1/2},1d_{5/2}$ and  $2s_{1/2}$ states by the quadrupole vibration\footnote{\cite{Barranco:17}.} of $^{10}$Be ($\beta_2\approx0.9$). This extremely
large value of the dynamical deformation parameter implies a Lamb-shift like mechanism which shifts the bare $1p_{1/2}$ and $2s_{1/2}$ orbitals by more than 3 MeV with respect to each other, the resulting dressed levels coinciding with the experimental ones ($\epsilon_{\widetilde {1/2}^+}=-0.5$ MeV, $\epsilon_{\widetilde {1/2}^-}=-0.18$ MeV, i.e. $\Delta\epsilon_{\widetilde {1/2}^+-\widetilde {1/2}^-}=0.32$ MeV, see Figs. \ref{fig6.2.1x} and \ref{fig6.3.1}). It is of notice that at the same time, the bare $d_{5/2}$ resonance is moved down, from its bare energy, by $\approx6$ MeV. The resulting centroid and width of the dressed $\widetilde{5/2}^+$ resonance is $\approx1.45$ MeV ($\Gamma=0.17$ MeV) to be compared with the experimental values of 1.28 MeV and $\Gamma=0.15$ MeV. The predicted $E1$-decay between the parity inverted levels is $B(E1;\widetilde{1/2}^-\to\widetilde{1/2}^+)=0.11e^2$ fm$^2$, a value to be compared to the experimental value $B(E1)=0.102\pm0.002e^2$ fm$^2$. The two (three) numbers ($\epsilon_{\widetilde{1/2}^+},\epsilon_{\widetilde {1/2}^-} (\Delta\epsilon)$, and $B(E1)$) can, arguably, be used to assess the accuracy of renormalized NFT results on something which one can call equal footing, with the two numbers (electron magnetic moment and Lamb shift) discussed above in connection with the assessment of the level of accuracy of QED results. 


It would be computationally useful and conceptually satisfying to have a bare $NN$-potential eventually derived from QCD, reproducing the $NN$-phase shifts and which, employed in nuclear structure calculations, lead to bare static (Eq. (\ref{eq1.0.18})) and  time-dependent (dynamic, Eq. (\ref{eq1.0.19})) mean fields whose solutions (bare single-particle states and bare vibrational modes), interweaved according to the rules of an (effective) field theory, e.g. NFT (Sect. \ref{Sect1.7.2}), accounted for the value of the observable resulting from a ``complete'' set of probes (see e.g. Figs. \ref{fig1.4.1} and \ref{fig6.3.1}). Concerning single-particle motion: energies, absolute one-particle transfer differential cross sections and thus insight into both single-particle content and renormalized single-particle wavefunctions (form factors), $\gamma$-decay and associated effective charges, etc. Concerning collective vibrations: energies, $\gamma$-decay, absolute differential cross sections for inelastic and Coulomb excitations (surface modes) and for two-nucleon transfer processes (pairing vibrations).
Until such desideratum becomes reality, the aim at shedding light onto the physics at the basis of new experimental results and that of providing guidance in the quest to achieve a deeper and unified picture of nuclear structure and reactions can be carried out in terms of empirical renormalization, i.e. (NFT)$_{\text{ren}}$.  

In keeping with the fact that the domain area of QED is all of chemistry and much of biology\footnote{\cite{Feynman:06}.}, the important \textit{renormalization effects} undergone by \textit{bare forces} when used to describe even the simplest (real) many-body system --i.e. the hydrogen atom in the case of the Coulomb interaction-- is quite universal. Because of the variety of facets and the importance this issue has in connection with the unified nuclear field theory of structure and reactions we discuss in the present monograph , taylored after Feynman's diagrammatic version of QED, we attempt a shedding light on bare force-renormalization process phenomena through a number of interdisciplinary examples. A crown example is provided by dispersive (retarded) forces like the van der Waals interaction (App. \ref{C2AppD}). Also by the Casimir effect and the hydrophobic force (App. \ref{C7AppG}), let alone in the renormalization the Coulomb force undergoes in metals (App. \ref{C3AppEx}). Dressing changing not only the value, but also the sign of the interaction and resulting in the phenomenon of superconductivity. Hallmarks of which are perfect conductivity and perfect diamagnetism, and the special effects found in terms of the renormalized Coulomb force acting across a Josephson junction (Sect. \ref{C3AppC}). Within this scenario one can mention that at the basis of the mechanism which inverts, in the odd $N=7$ isotones $^{10}_3$Li and $^{11}_4$Be, the standard Mayer-Jensen sequence of levels $1p_{1/2}$ and $2s_{1/2}$ and comes close to do so also concerning the $1d_{5/2}$ and $1p_{1/2}$ single-particle levels constitutes, arguably, the strongest Lamb shift-like phenomenon displayed by a physical system.
\section{Hindsight}\label{Sect1.10}
In order for a nucleon moving in a level close to the Fermi energy to display a mean free path larger than nuclear dimensions and to be reflected at the nuclear surface through an elastic process, all other nucleons must move in a rather ordered, correlated fashion. Within this context to posit that single-particle motion is the most collective of all nuclear motions\footnote{\cite{Mottelson:62},} seems natural. Associated with mean field and single-particle motion one finds the typical bunching of the corresponding levels closely connected with the major shells lying above and below the Fermi energy. A fact which determines the parity (and angular momentum) of the lowest excitation associated with promoting  a nucleon across the Fermi surface.  By correlating these excitations through the same components of the $NN$-interaction leading to the single-particle bunching within the mean field approximation, one obtains the collective multipole  vibrations of particle-hole type. Similar arguments result in the presence of multipole pairing vibrations in the low-energy spectrum, and in their coupling to single-particle motion.

It is then natural to consider single-particle motion  and collective states on equal footing, and as basis states of a physical description of the atomic nucleus in which to diagonalize both three point  (PVC) and four-point ($v$) vertices , eventually taking in this second case also 3$N$ terms into account\footnote{See footnote \ref{f9} of this Chapter.}. Such an approach also provides direct indication of the minimum set of experiments needed to obtain a ``complete'' picture (test)  of the theoretical description of the atomic nucleus. They are the specific probes of each of the basis states (elementary modes of excitation). Namely inelastic scattering and Coulomb excitation (particle-hole collective vibrations and rotations), one-particle transfer processes (independent-particle motion) and two-particle transfer reactions (pairing vibrations and rotations). 


Summing up, because of the interweaving existing between the variety of elementary modes of excitation, experimental probes associated with fields which carry transfer quantum number $\beta=0, \pm 1$ and $\pm2$, and different multipolarities, spins and isospins are needed to characterize the structure of nuclei. This is what we attempt at explaining and formulating in the following chapters, setting special emphasis in transfer processes, in which case the relative motion of the reacting nuclei and the intrinsic motion of the nucleons in target and projectile cannot be separated, and one is forced to treat structure and reactions in an unified way.









%\renewcommand{\bibname}{Bibliography Ch 1}
%\bibliographystyle{abbrvnat}

%\bibliography{/home/gregory/book/nuclear_bib.bib}
