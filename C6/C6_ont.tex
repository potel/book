\documentclass[a4paper,14pt]{book}
%\usepackage[sectionbib]{chapterbib}
\usepackage{chapterbib}
\usepackage{appendix}
%\documentclass[a4paper]{book}
% \linespread{2.}
%\documentclass[12pt]{article}
%\documentclass[12pt]{cmmp}

%%\usepackage{psfig}
%\usepackage{harvard}
\usepackage{epsfig}
%%\usepackage{amsmath}
\usepackage{amsfonts}
%%\usepackage{amssymb}
%%\usepackage{graphicx}
%%
%%\usepackage{txfonts}
%%%\usepackage{mathrsfs}
%
%\usepackage{feynmf}     %<------------ Obbligatorio
\unitlength=1mm         %<------------ Obbligatorio
%
\newcommand{\braket}[1]{\langle {#1} \rangle }
\newcommand{\ket}[1]{|{#1} \rangle }
\newcommand{\bra}[1]{\langle {#1}|}
\usepackage{latexsym}
\usepackage{amssymb}
\usepackage{amsmath}
\usepackage[varg]{txfonts}
\usepackage{mathrsfs}
\usepackage{upgreek}
\usepackage[round]{natbib}
%\usepackage [latin1]{inputenc}
\usepackage{verbatim}
\usepackage{array}
\usepackage{color}
%\pagestyle{plain}
\usepackage{graphicx}
\DeclareMathAlphabet{\mathpzc}{OT1}{pzc}{m}{it}



\begin{document}
 \setcounter{chapter}{5}
\chapter{One--particle transfer}
In what follows we present a derivation of the one--particle transfer differential cross section within the framework of the DWBA \citep{Satchler:80,Broglia:04a}. The structure input for the calculations are mean field potentials and single--particle states dressed through the coupling with the variety of collective, (quasi) bosonic vibrations, leading to modified formfactors\footnote{It is of notice that single--particle modified formfactors have their counterpart in the transition densities (\ref{...})  and in the modified two--nucleon formfactors (\ref{...}) associated with inelastic and pair transfer \cite{Broglia:73}, respectively} resulting from the interweaving of these vibrations and a number of orbitals with the original, unperturbed single--particle state \citep{Bohr:75,Bes:77}. With the help of these modified formfactors, and of optical potentials, one can calculate the absolute differential cross sections, quantities which can  be directly compared with the experimental findings.

In this way one avoids to introduce, let alone use spectroscopic factors, quantities which are rather elusive to define. This is in keeping with the fact that as a nucleon moves through the nucleus it feels the presence of the other nucleons whose configurations change as time proceeds. It takes time for this information to be fed back on the nucleon. This renders the average potential nonlocal in time (cf. \citet{Mahaux:85} and references therein). A time--dependent operator can always be transformed into an energy dependent operator, implying an $\omega$--dependence of the properties which are usually adscribed to particles like (effective) mass, charge, etc. Furthermore, due to the Pauli principle, the average potential is also non local in space (cf. App \ref{...})  Consequently, one is forced to deal with nucleons which carry around a cloud of (quasi) bosons, aside from  exchanging its position with that of the other nucleons. It is of notice that the above mentioned phenomena are not only found within the realm of nuclear physics, but are common within the framework of many--body systems as well as of field theories like quantum electrodynamic (QED). In fact, a basic result of such theories is that nothing is free \citep{Feynman:75}. A textbook example being provided by the Lamb shift, resulting from the dressing of the hydrogen atom electron, as a result of the exchange of this electron with those participating in the spontaneous, virtual excitation (zero point fluctuations (ZPF)) of the QED vacuum (cf. App \ref{,,,}).
Within this context see Sect. \ref{...} (Examples and Applications) concerning the phenomenon of parity inversion in $N$=6 (closed shell) exotic halo nuclei.



\section{General derivation}
We proceed now to derive the transition amplitude for the reaction (cf. Fig. \ref{....})
\begin{equation}\label{eq_onept1}
    A+a(=b+1)\longrightarrow B(=A+1)+b.
\end{equation}
Let us assume that the nucleon bound initially to the core $b$ is in a single--particle state with orbital and total angular momentum $l_i$ and  $j_i$ respectively, and that the nucleon in the final state (bound to core $A$)  is in the $l_f,j_f$ state. The total spin and magnetic quantum numbers of nuclei $A,a,B,b$ are $\{J_A,M_A\},\{J_a,M_a\},\{J_B,M_B\},\{J_b,M_b\}$ respectively. Denoting $\xi_A$ and $\xi_b$ the intrinsic coordinates of the wavefunctions describing the structure of nuclei $A$ and $b$ respectively, and $\mathbf{r}_{An}$ and $\mathbf{r}_{bn}$ the relative coordinates of the transferred nucleon with respect to the CM of nuclei $A$ and $b$ respectively, one can write the ``intrinsic''  wavefunctions of the colliding nuclei $A,a$ as
\begin{equation}\label{eq_onept2}
    \begin{split}
    &\phi_{M_A}^{J_A}(\xi_A),\\
    &\Psi(\xi_b,\mathbf{r}_{b1})=\sum_{m_i}\langle J_b\;j_i\;M_b\;m_i|J_a\;M_a\rangle\phi_{M_b}^{J_b}(\xi_b)\psi_{m_i}^{j_i}(\mathbf{r}_{bn},\sigma),
    \end{split}
\end{equation}
while the ``intrinsic'' wavefunctions describing the structure of nuclei $B$ and $b$ are
\begin{equation}\label{eq_onept3}
    \begin{split}
    &\phi_{M_b}^{J_b}(\xi_b),\\
    &\Psi(\xi_A,\mathbf{r}_{A1})=\sum_{m_f}\langle J_A\;j_f\;M_A\;m_f|J_B\;M_B\rangle\phi_{M_A}^{J_A}(\xi_A)\psi_{m_f}^{j_f}(\mathbf{r}_{An},\sigma).
    \end{split}
\end{equation}
For an unpolarized incident beam (sum over $M_A,M_a$ and divide  by $(2J_A+1),(2J_a+1)$) and assuming that  one does not detect the final polarization (sum over $M_B,M_b$), the differential cross section in the DWBA can be written as
\begin{equation}\label{eq_onept4}
    \begin{split}
\frac{d\sigma}{d\Omega}&=\frac{k_f}{k_i}\frac{\mu_i\mu_f}{4\pi^2\hbar^4}\frac{1}{(2J_A+1)(2J_a+1)}\\
&\times\sum_{\substack{M_A,M_a\\M_B,M_b}}\left|\sum_{m_i,m_f}\langle J_b\;j_i\;M_b\;m_i|J_a\;M_a\rangle\langle J_A\;j_f\;M_A\;m_f|J_B\;M_B\rangle T_{m_i,m_f}\right|^2.
    \end{split}
\end{equation}
The transition amplitude $T_{m_i,m_f}$ is
\begin{equation}\label{eq_onept5}
T_{m_i,m_f}=\sum_\sigma\int d\mathbf{r}_fd\mathbf{r}_{bn}\chi^{(-)*}(\mathbf{r}_f)
\psi_{m_f}^{j_f*}(\mathbf{r}_{An},\sigma)V(r_{bn})\psi_{m_i}^{j_i}(\mathbf{r}_{bn},\sigma)\chi^{(+)}(\mathbf{r}_i),
\end{equation}
where
\begin{equation}\label{eq_onept12}
\psi_{m_i}^{j_i}(\mathbf{r}_{An},\sigma)=u_{j_i}(r_{bn})\left[ Y^{l_i} (\hat r_i)\chi(\sigma)\right]_{j_i m_i};
\end{equation}
is the single--particle wavefunction describing the motion of the nucleon $t$ o be transferred, in the initial state. Similarly for $\psi_{m_f}^{j_f}$. 
The incoming and outgoing distorted waves are
 \begin{equation}\label{eq_onept6}
\chi^{(+)}(\mathbf{k}_i,\mathbf{r}_i)= \frac{ 4\pi }{k_i r_i}\sum_{l'} i^{l'}
e^{i\sigma_i^{l'}} g_{l'}(\hat r_i) \left[ Y^{l'} (\hat r_i) Y^{l'} (\hat k_i)\right]^0_0,
\end{equation}
and
 \begin{equation}\label{eq_onept7}
\chi^{(-)*}(\mathbf{k}_f,\mathbf{r}_f)= \frac{ 4\pi }{k_f r_f}\sum_{l} i^{-l}
e^{i\sigma_f^{l}} f_{l}  (\hat r_f) \left[ Y^{l} (\hat r_f) Y^{l} (\hat k_f)\right]^0_0,
\end{equation}
respectively. Now,
\begin{equation}\label{eq_onept8}
    \begin{split}
\left[ Y^{l} (\hat r_f) Y^{l} (\hat k_f)\right]^0_0&\left[ Y^{l'} (\hat r_i) Y^{l'} (\hat k_i)\right]^0_0=\sum_K \bigl((l l)_0(l' l')_0|(l l')_K(l l')_K\bigr)_0\\
&\times \left\{\left[ Y^{l} (\hat r_f) Y^{l'} (\hat r_i)\right]^K\left[ Y^{l} (\hat k_f) Y^{l'} (\hat k_i)\right]^K\right\}^0_0.
    \end{split}
\end{equation}
The $9j$--symbol can be explicitly evaluated to give,
\begin{equation}\label{eq_onept9}
\bigl((l l)_0(l' l')_0|(l l')_K(l l')_K\bigr)_0=\sqrt{\frac{2K+1}{(2l+1)(2l'+1)}},
\end{equation}
and the angular momenta coupling is
\begin{equation}\label{eq_onept10}
    \begin{split}
\left\{\vphantom{\left[ Y^{l} (\hat r_f) Y^{l'} (\hat r_i)\right]^K}\right.&\left.\left[ Y^{l} (\hat r_f) Y^{l'} (\hat r_i)\right]^K\left[ Y^{l} (\hat k_f) Y^{l'} (\hat k_i)\right]^K\right\}^0_0=\sum_M \langle K\;K\;M\;-M|0\;0\rangle
 \left[ Y^{l} (\hat r_f) Y^{l'} (\hat r_i)\right]^K_M\\
 &\times\left[ Y^{l} (\hat k_f) Y^{l'} (\hat k_i)\right]^K_{-M}=\sum_M\frac{(-1)^{K+M}}{\sqrt{2K+1}}\left[ Y^{l} (\hat r_f) Y^{l'} (\hat r_i)\right]^K_M
\left[ Y^{l} (\hat k_f) Y^{l'} (\hat k_i)\right]^K_{-M}.
    \end{split}
\end{equation}
Thus,
\begin{equation}\label{eq_onept11}
    \begin{split}
\left[ Y^{l} (\hat r_f) Y^{l} (\hat k_f)\right]^0_0&\left[ Y^{l'} (\hat r_i) Y^{l'} (\hat k_i)\right]^0_0=\\
&\sum_{K,M}\frac{(-1)^{K+M}}{\sqrt{(2l+1)(2l'+1)}}\left[ Y^{l} (\hat r_f) Y^{l'} (\hat r_i)\right]^K_M
\left[ Y^{l} (\hat k_f) Y^{l'} (\hat k_i)\right]^K_{-M}.
    \end{split}
\end{equation}
For the angular integral to be different from zero, the integrand must be coupled to zero angular momentum (scalar). Noting that the only  variables over which one integrates in the above expression  are $\hat r_i,\hat r_f$, we have to couple the remaining functions of the angular variables, namely the wavefunctions $\psi_{m_f}^{j_f*}(\mathbf{r}_{An},\sigma)=(-1)^{j_f-m_f}\psi_{-m_f}^{j_f}(\mathbf{r}_{An},-\sigma)$ and $\psi_{m_i}^{j_i}(\mathbf{r}_{bn},\sigma)$ to angular momentum $K$, as well as to fulfill $M=m_f-m_i$. Let us then consider
\begin{multline}\label{eq_onept35}
(-1)^{j_f-m_f}\psi_{-m_f}^{j_f}(\mathbf{r}_{An},-\sigma)\psi_{m_i}^{j_i}(\mathbf{r}_{bn},\sigma)=
(-1)^{j_f-m_f}u_{j_f}(r_{An})u_{j_i}(r_{bn})\\
\times \sum_P \langle j_f\;j_i\;-m_f\;m_i|P\;m_i-m_f\rangle \left\{\left[ Y^{l_f}(\hat r_{An}) \chi^{1/2}(-\sigma)\right]^{j_f}\left[ Y^{l_i}(\hat r_{bn}) \chi^{1/2}(\sigma)\right]^{j_i}\right\}^P_{m_i-m_f}.
\end{multline}
Recoupling the spherical harmonics to angular momentum $K$ and the spinors to $S=0$, only one term survives the angular integral in \ref{eq_onept5}, namely

\begin{multline}\label{eq_onept34}
(-1)^{j_f-m_f}u_{j_f}(r_{An})u_{j_i}(r_{bn})\bigl((l_f \tfrac{1}{2})_{j_f}(l_i \tfrac{1}{2})_{j_i}|(l_f l_i)_K(\tfrac{1}{2} \tfrac{1}{2})_0\bigr)_K\\
 \times\langle j_f\;j_i\;-m_f\;m_i|K\;m_i-m_f\rangle\left[Y^{l_f}(\hat r_{An})  Y^{l_i}(\hat r_{bn}) \right]^{K}_{m_i-m_f}\left[ \chi(-\sigma)\chi(\sigma)\right]^0_0.
\end{multline}

Making use of the fact that the sum over spins yields a factor $-\sqrt{2}$, and in keeping with the fact that $M=m_f-m_i$, one obtains,
\begin{multline}\label{eq_onept14}
T_{m_i,m_f}=(-1)^{j_f-m_f}\frac{-16\sqrt{2}\pi^2}{k_fk_i}\sum_{ll'}i^{l'-l}e^{\sigma_f^l+\sigma_i^{l'}}\sum_K\bigl((l_f \tfrac{1}{2})_{j_f}(l_i \tfrac{1}{2})_{j_i}|(l_f l_i)_K(\tfrac{1}{2} \tfrac{1}{2})_0\bigr)_K\\
\times\langle j_f\;j_i\;-m_f\;m_i|K\;m_i-m_f\rangle\left[ Y^{l} (\hat k_f) Y^{l'} (\hat k_i)\right]^K_{m_i-m_f}\int d\mathbf{r}_fd\mathbf{r}_{bn}\frac{f_l(r_f)g_{l'}(r_i)}{r_fr_i}\\
\times u_{j_f}(r_{An})u_{j_i}(r_{bn})V(r_{bn})
(-1)^{K+m_f-m_i}\left[ Y^{l} (\hat r_f) Y^{l'} (\hat r_i)\right]^K_{m_f-m_i}\left[ Y^{l_f}(\hat r_{An}) Y^{l_i}(\hat r_{bn})\right]^K_{m_i-m_f}.
\end{multline}
Again, the only term of the expression
\begin{multline*}
(-1)^{K+m_f-m_i}\left[ Y^{l} (\hat r_f) Y^{l'} (\hat r_i)\right]^K_{m_f-m_i}\left[ Y^{l_f}(\hat r_{An}) Y^{l_i}(\hat r_{bn})\right]^K_{m_i-m_f}=\\
(-1)^{K+m_f-m_i}\sum_P \langle K\;K\;m_f-m_i\;m_i-m_f|P\;0\rangle\left\{\left[ Y^{l} (\hat r_f) Y^{l'} (\hat r_i)\right]^K\left[ Y^{l_f}(\hat r_{An}) Y^{l_i}(\hat r_{bn})\right]^K\right\}^P_0
\end{multline*}
which survives after angular integration is the one with $P=0$, that is,
\begin{multline*}
\frac{1}{\sqrt{(2K+1)}}\left\{\left[ Y^{l} (\hat r_f) Y^{l'} (\hat r_i)\right]^K\left[ Y^{l_f}(\hat r_{An}) Y^{l_i}(\hat r_{bn})\right]^K\right\}^0_0=\\
\frac{1}{\sqrt{(2K+1)}}\sum_{M_K}\langle K\;K\;M_K\;-M_K|0\;0\rangle\left[ Y^{l} (\hat r_f) Y^{l'} (\hat r_i)\right]^K_{M_K}\\
\times\left[ Y^{l_f}(\hat r_{An}) Y^{l_i}(\hat r_{bn})\right]^K_{-M_K}=\frac{1}{\sqrt{(2K+1)}}\sum_{M_K}\frac{(-1)^{K+M_K}}{\sqrt{(2K+1)}}\left[ Y^{l} (\hat r_f) Y^{l'} (\hat r_i)\right]^K_{M_K}\\
\times\left[ Y^{l_f}(\hat r_{An}) Y^{l_i}(\hat r_{bn})\right]^K_{-M_K}=\\
\frac{1}{2K+1}\sum_{M_K}(-1)^{K+M_K}\left[ Y^{l} (\hat r_f) Y^{l'} (\hat r_i)\right]^K_{M_K}\left[ Y^{l_f}(\hat r_{An}) Y^{l_i}(\hat r_{bn})\right]^K_{-M_K},
\end{multline*}
an expression which is spherically symmetric. One can evaluate it for a particular configuration, in particular setting $\hat r_f=\hat z$ and the center of mass $A,b,n$  in the $x-z$ plane (see Fig. \ref{fig1}). Once the orientation in space of this ``standard'' configuration is specified (with, for example, a rotation $0\leq\alpha\leq 2\pi$ around $\hat z$, a rotation $0\leq\beta\leq \pi$ around the new $x$ axis and a rotation $0\leq\gamma\leq 2\pi$ around $\hat r_{bB}$), the only remaining angular coordinate is $\theta$, while the integral over the other three angles yields a  $8\pi^2$. Setting $\hat r_f=\hat z$ one obtains
\begin{equation}\label{eq_onept18}
\left[ Y^{l} (\hat r_f) Y^{l'} (\hat r_i)\right]^K_{M_K}=\langle l\;l'\;0\;M_K|K\;M_K\rangle\sqrt{\frac{2l+1}{4\pi}}Y_{M_K}^{l'}(\hat r_i).
\end{equation}
Because of $M=m_i-m_f$ and $m=m_f$  $T_{m_i,m_f}\equiv T_{m,M} $ where
\begin{multline}\label{eq_onept19}
T_{m,M}=(-1)^{j_f-m}\frac{-64\sqrt{2}\pi^{7/2}}{k_f k_i}\sum_{ll'}i^{l'-l}e^{\sigma_f^l+\sigma_i^{l'}}\sqrt{2l+1}\sum_K\frac{(-1)^{K}}{2K+1}\bigl((l_f \tfrac{1}{2})_{j_f}(l_i \tfrac{1}{2})_{j_i}|(l_f l_i)_K(\tfrac{1}{2} \tfrac{1}{2})_0\bigr)_K\\
\times\langle j_f\;j_i\;-m\;M+m|K\;M\rangle\left[ Y^{l} (\hat k_f) Y^{l'} (\hat k_i)\right]^K_{M}\int d\mathbf{r}_fd\mathbf{r}_{bn}\frac{f_l(r_f)g_{l'}(r_i)}{r_fr_i}\\
\times u_{j_f}(r_{An})u_{j_i}(r_{bn})V(r_{bn})
\sum_{M_K}(-1)^{M_K}\langle l\;l'\;0\;M_K|K\;M_K\rangle \left[ Y^{l_f}(\hat r_{An}) Y^{l_i}(\hat r_{bn})\right]^K_{-M_K}Y_{M_K}^{l'}(\hat r_i).
\end{multline}



We now turn our attention to the sum
\begin{equation}\label{eq_onept20}
\sum_{\substack{M_A,M_a\\M_B,M_b}}\left|\sum_{m,M}\langle J_b\;j_i\;M_b\;m_i|J_a\;M_a\rangle\langle J_A\;j_f\;M_A\;m_f|J_B\;M_B\rangle T_{m,M}\right|^2,
\end{equation}
found in the expression for the differential cross section (\ref{eq_onept4}). For any given value $m',M'$ of $m,M$, the sum will be
\begin{multline}\label{eq_onept21}
\sum_{M_a,M_b}\left|\langle J_b\;j_i\;M_b\;m_i|J_a\;M_a\rangle\right|^2\sum_{M_A,M_B}\left|\langle J_A\;j_f\;M_A\;m_f|J_B\;M_B\rangle\right|^2 \left|T_{m',M'}\right|^2=\\
\frac{(2J_a+1)(2J_B+1)}{(2j_i+1)(2j_f+1)}\sum_{M_a,M_b}\left|\langle J_b\;J_a\;M_b\;-M_a|j_i\;m_i\rangle\right|^2\\
\times\sum_{M_A,M_B}\left|\langle J_A\;J_B\;M_A\;-M_B|j_f\;m_f\rangle\right|^2 \left|T_{m',M'}\right|^2,
\end{multline}
by virtue of the symmetry property of Clebsch--Gordan coefficients
\begin{equation}\label{eq_onept22}
\langle J_b\;j_i\;M_b\;m_i|J_a\;M_a\rangle=(-1)^{J_b-M_b}\sqrt{\frac{(2J_a+1)}{(2j_i+1)}}\langle J_b\;J_a\;M_b\;-M_a|j_i\;m_i\rangle.
\end{equation}
The sum over the Clebsch--Gordan coefficients in (\ref{eq_onept21})is one, so (\ref{eq_onept20}) is just
 \begin{equation}\label{eq_onept23}
\frac{(2J_a+1)(2J_B+1)}{(2j_i+1)(2j_f+1)}\sum_{m,M}\left|T_{m,M}\right|^2,
\end{equation}
and the differential cross section turns out to be
\begin{equation}\label{eq_onept24}
    \begin{split}
\frac{d\sigma}{d\Omega}&=\frac{k_f}{k_i}\frac{\mu_i\mu_f}{4\pi^2\hbar^4}
\frac{(2J_B+1)}{(2j_i+1)(2j_f+1)(2J_A+1)}\sum_{m,M}\left| T_{m,M}\right|^2.
    \end{split}
\end{equation}
where
\begin{equation}\label{eq_onept25}
T_{m,M}=\sum_{Kll'}(-1)^{-m}\langle j_f\;j_i\;-m\;M+m|K\;M\rangle\left[ Y^{l} (\hat k_f) Y^{l'} (\hat k_i)\right]^K_{M}t_{ll'}^K.
\end{equation}
Orienting $\hat k_i$ along the incident $z$ direction,
\begin{equation}\label{eq_onept26}
\left[ Y^{l} (\hat k_f) Y^{l'} (\hat k_i)\right]^K_{M}=\langle l\;l'\;M\;0|K\;M\rangle\sqrt{\frac{2l'+1}{4\pi}}Y_{M}^{l}(\hat k_f),
\end{equation}
and
\begin{equation}\label{eq_onept27}
\begin{split}
T_{m,M}=\sum_{Kll'}(-1)^{-m}\langle l\;l'\;M\;0|K\;M\rangle \langle j_f\;j_i\;-m\;M+m|K\;M\rangle Y_{M}^{l}(\hat k_f)\,t_{ll'}^K,
\end{split}
\end{equation}
with
\begin{multline}\label{eq_onept28}
t_{ll'}^K=(-1)^{K+j_f}\frac{-32\sqrt{2}\pi^3}{k_f k_i} i^{l'-l}e^{\sigma_f^l+\sigma_i^{l'}}\frac{\sqrt{(2l+1)(2l'+1)}}{2K+1}\bigl((l_f \tfrac{1}{2})_{j_f}(l_i \tfrac{1}{2})_{j_i}|(l_f l_i)_K(\tfrac{1}{2} \tfrac{1}{2})_0\bigr)_K\\
\times\int dr_f\,dr_{bn}d\theta r_{bn}^2 \sin \theta \,r_f \frac{f_l(r_f)g_{l'}(r_i)}{r_i}u_{j_f}(r_{An})u_{j_i}(r_{bn})V(r_{bn})\\
\times\sum_{M_K}(-1)^{M_K}\langle l\;l'\;0\;M_K|K\;M_K\rangle \left[ Y^{l_f}(\hat r_{An}) Y^{l_i}(\hat r_{bn})\right]^K_{-M_K}Y_{M_K}^{l'}(\hat r_i).
\end{multline}
 \begin{figure}
\centerline{\includegraphics*[width=10cm,angle=0]{figs_C6/coord.png}}
\caption{Coordinate system in the ``standard'' configuration. Note that $\mathbf{r}_f\equiv\mathbf{r}_{Bb}$, and $\mathbf{r}_i\equiv\mathbf{r}_{Aa}$.}\label{fig1}
\end{figure}
\subsection{Coordinates}
To perform the integral in (\ref{eq_onept28}), one needs the expression of $r_i,r_{An},\hat r_{An},\hat r_{bn},\hat r_i$ in term of the integration variables $r_f,r_{bn},\theta$.Because one is interested in evaluating these quantities in the particular configuration depicted in Fig. \ref{fig1}, one has
\begin{align}
\mathbf{r}_f&=r_f \,\hat z,\\
\mathbf{r}_{bn}&=-r_{bn}(\sin \theta \,\hat x+ \cos \theta  \,\hat z),\\
\mathbf{r}_{Bn}&=\mathbf{r}_f+\mathbf{r}_{bn}=-r_{bn}\sin \theta \,\hat x+(r_f-r_{bn}\cos \theta)\,\hat z.
\end{align}
One can then write
\begin{align}
\mathbf{r}_{An}&=\frac{A+1}{A}\mathbf{r}_{Bn}=-\frac{A+1}{A}r_{bn}\sin \theta \,\hat x+\frac{A+1}{A}(r_f-r_{bn}\cos \theta)\,\hat z,\\
\mathbf{r}_{an}&=\frac{b}{b+1}\mathbf{r}_{bn}=-\frac{b}{b+1}r_{bn}(\sin \theta \,\hat x+ \cos \theta  \,\hat z),
\end{align}
and
\begin{multline}
\mathbf{r}_i=\mathbf{r}_{An}-\mathbf{r}_{an}=-\frac{2A+1}{(A+1)A}r_{bn}\sin \theta \,\hat x
+\left(\frac{A+1}{A}r_f-\frac{2A+1}{(A+1)A}r_{bn}\cos \theta\right)\,\hat z,
\end{multline}
where $A,b$ are the number of nucleons of nuclei $A$ and $b$ respectively.


\subsection{Zero range approximation}
In the zero range approximation,
\begin{equation}\label{eq_onept29}
\int dr_{bn} r_{bn}^2 u_{j_i}(r_{bn})V(r_{bn})=D_0;\quad u_{j_i}(r_{bn})V(r_{bn})=\delta(r_{bn})/r_{bn}^2.
\end{equation}
It can be shown (see Fig. \ref{fig1}) that for $r_{bn}=0$
\begin{equation}\label{eq_onept30}
\begin{split}
&\mathbf{r}_{An}=\frac{m_A+1}{m_A}\mathbf{r}_f\\
&\mathbf{r}_i=\frac{m_A+1}{m_A}\mathbf{r}_f.\\
\end{split}
\end{equation}
One then obtains
\begin{equation}\label{eq_onept31}
\begin{split}
t_{ll'}^K=&\frac{-16\sqrt{2}\pi^2}{k_f k_i}(-1)^K \frac{D_0}{\alpha} i^{l'-l}e^{\sigma_f^l+\sigma_i^{l'}}\frac{\sqrt{(2l+1)(2l'+1)(2l_i+1)(2l_f+1)}}{2K+1}\bigl((l_f \tfrac{1}{2})_{j_f}(l_i \tfrac{1}{2})_{j_i}|(l_f l_i)_K(\tfrac{1}{2} \tfrac{1}{2})_0\bigr)_K\\
&\times\langle l\;l'\;0\;0|K\;0\rangle\langle l_f\;l_i\;0\;0|K\;0\rangle\int dr_f\ f_l(r_f)g_{l'}(\alpha r_f)u_{j_f}(\alpha r_f),
\end{split}
\end{equation}
with
\begin{equation}\label{eq_onept32}
\alpha=\frac{A+1}{A}.
\end{equation}

\section{Examples and Applications}
In this section we discuss some examples which illustrate the workings of single--particle transfer processes at large, and in particular the flavour of the limitations by which nuclear structure studies suffer, when this specific probe to study quasiparticle properties is not operative. Let us in fact start with such an example.
\subsection{Dressing of single--particle states: parity inversion in $^{11}$Li}
The $N = 6$ isotope of $^9_3$Li displays quite ordinary structural properties and can, at first
glance, be thought of a two–neutron hole system in the $N = 8$ closed shell. That this is not the case emerges clearly from the fact that $^{10}$Li is not bound, the lowest virtual ($1/2^+$) and
resonant ($1/2^-$) states testify to the fact that, in the present case, the $N = 6$ is a far better
magic neutron number  than $N = 8$. Furthermore, that the unbound $s_{1/2}$
state lies lower than the unbound $p_{1/2}$ state, in plain contradiction with static mean field
theory. Dressing the (standard) mean field single–particle state with vibrations , mostly with the core quadrupole vibration, through
polarization (effective mass–like) and correlation diagrams (vacuum zero point fluctuations
(ZPF)) diagrams, similar to those associated with the (lowest order) Lamb shift Feynman
diagrams), move the $s_{1/2}$ and $p_{1/2}$ mean field levels around. In particular the $p_{1/2}$ from a
bound state ($\approx-1.2$MeV) to a resonant state lying at $\approx0.5$ MeV (Pauli principle, vacuum
ZPF process)


How can one check that CO and PO like processes as the ones shown in Fig. \ref{...} (cf. also Fig. \ref{...}) are the basic processes dressing the odd neutron of $^{10}$Li, and thus the mechanism at the basis of parity inversion? The answer is, forcing these virtual processes to become real. While this is not easy to accomplish in one--particle transfer processes involving the unbound $s_{1/2}$ and $p_{1/2}$, $^{10}$Li virtual and resonant states, it can done with the help of two--particle transfer processes, namely that associated with inverse kinematics ($p,t$) reaction $^1\text{H}(^{11}\text{Li},^{9}\text{Li}(2.69\text{MeV};1/2^-))^3$H (cf. Fig. \ref{...} and Chs. \ref{.....}). Such a reaction is feasible, in keeping with the fact that, adding a neutron to $^{10}$Li leads to a bound state (see Fig. \ref{...}). In fact, $^{11}_3$Li$_8$ displays a two--neutron separation energy $S_{2n}\approx$ 400 keV (for further details we refer to Ch. \ref{..}, Sect. \ref{..}). The price to pay for not being able to use the specific probe for single--particle modes (one--particle transfer), is that of adding to the self--energy contributions in question those corresponding to vertex corrections (cf. Fig. \ref{..}; for details cf. Sect. \ref{...} (application $2n$--transfer)). Within the present context, it is difficult if not impossible to talk about single--particle motion without also referring to collective vibrational states (cf. e.g. Fig. \ref{...}) both in structure and reactions as well as to talk about pair addition and pair substraction correlations, without at the same time talking about vibrations and dressed quasiparticle motion (see e.g. Fig. \ref{...}). And this again in structure and reactions. Within the framework of the present monograph, the above facts imply that Chapters \ref{...} (inelastic), \ref{,,} (one--particle transfer) and \ref{..} and \ref{ç} (two--particle transfer and applications) form a higher unity. The unity extending also to Ch. \ref{..} (knock--out reactions), if one also considers the question of final state interactions, and thus of the possibility that the population of the state $1/2^-$ depicted in Fig. \ref{...} receives contributions other, and more involved, than those associated with the direct two--nucleon pick--up depicted (for details cf. Ch. \ref{...}).

\section{The Lamb Shift}
In Fig. \ref{...} we display a schematic summary of the electron--photon processes, associated with Pauli principle corrections, leading to the splitting of the lowest $s,p$ states of the hydrogen atom known as the Lamb shift.


In the upper part of the figure the predicted position of the electronic single--particle levels of the hydrogen atom as resulting from the solution of the Scr\"{o}dinger equation (Coulomb field). In the lowest part of the figure one displays the electron of an hydrogen atom (upwards going arrowed line) in presence of vacuum ZPF (electron--positron pair plus photon, oyster--like diagram). Because the associate electron virtually occupies states already occupied by the hydrogen's electron, thus violating Pauli principle, one has to antisymmetrize the corresponding two--electron state. Such process gives rise to the exchange of the corresponding fermionic lines and thus to CO--like digrams as well as, through time ordering, to PO--like diagrams. The results provide a quantitative account of the experimental findings.  




\section{No--recoil, local, plane wave limit}
In this Appendix we discuss some aspects of th relation existing between nuclear structure and one--particle transfer cross sections. To do so, we repeat some of the steps carried out in the text but this time in the most simple and straightforward way, ignoring the complications associated with the spin carried out by the particles, the spin--orbit dependence of the optical model potential, the recoil effect, etc.
We consider the case of $A(d,p)A+1$ reaction, namely of neutron stripping. The intrinsic wave functions $\psi_\alpha$ and $\psi_\beta$, where $\alpha=(A,d)$ and $\beta=((A+1),p)$,
\begin{subequations}
\begin{align}\label{eq422}
&\psi_\alpha=\psi_{M_{A}}^{I_A}(\xi_A) \phi_d(\vec r_{np}),\\
\begin{split}
&\psi_\beta=\psi_{M_{A+1}}^{I_{A+1}}(\xi_{A+1})\\
& \;\;\;\;=\sum_{l,I'_A} (I'_A;l \vert \} I_{A+1})
[\psi^{I'_A}(\xi_A)\phi^l(\vec r_{n})]_{M_{A+1}-M_A}^{I_{A+1}},
\end{split}
\end{align}
\end{subequations}
where $(I'_A;l \vert \} I_{A+1})$ is a generalized fractional parentage coefficient. It is of notice that this fractional parentage expansion is not well defined. In fact, as a rule,\\  \mbox{$(I'_A;l \vert \} I_{A+1})\phi^l(\vec r_{n})_{M_{A+1}-M_A}$} is an involved, dressed quasiparticle state containing only a fraction of the ``pure'' single particle strength (cf. \ref{ff}). For simplicity we assume the expansion is operative.	
To further simplify the derivation we assume we are dealing with spinless particles. The variable $\vec r_{np}$ is the relative coordinate of the proton and the neutron (see Fig. \ref{f}).


The transition matrix element can now be written as
\begin{equation}\label{eq423}
 \begin{split}
T_{d,p}&= \langle \psi_{M_{A+1}}^{I_{A+1}}(\xi_{A+1}) \chi^{(-)}_p(k_p,\vec{r}_p),
V'_\beta \, \psi_{M_{A}}^{I_{A}}(\xi_{A}) \chi^{(+)}_d(k_d,\vec{r}_d)\rangle= \\
& \sum_{\substack{l,I'_A\\M'_A}} (I'_A;l \vert \} I_{A+1}) (I'_A \,M'_A\, l\, M_{A+1}-M'\,_A \vert I_{A+1}\,M_{A+1})\\
& \times\int d\vec{r}_n d \vec{r}_p \chi^{* (-)}_p(k_p,\vec{r}_p) \phi_{M_{A+1}-M'_A}^{*l}(\vec{r}_n)
(\psi_{M_{A}}^{I_{A}}(\xi_{A}),V'_\beta \psi_{M'_{A}}^{I'_{A}}(\xi_{A}))\\
& \times\phi_d(\vec r_{np})
\chi^{(+)}_d(k_d,\vec{r}_d) \; \delta_{I'_A,I_A} \; \delta_{M'_A,M_A}.
\end{split}
\end{equation}
In the stripping approximation
\begin{equation}\label{eq424}
 \begin{split}
V'_\beta & = V_\beta(\xi,\vec r_\beta)- \bar U_\beta (r_\beta)\\
&=V_\beta(\xi_A,\vec r_{pA})+V_\beta(\vec r_{pn})-\bar U_\beta (r_{pA})
\end{split}
\end{equation}
Then
\begin{equation}\label{eq425}
 \begin{split}
(\psi_{M_{A}}^{I_{A}}(\xi_{A}) & ,V'_\beta \psi_{M_{A}}^{I_{A}}(\xi_{A}))=
(\psi_{M_{A}}^{I_{A}}(\xi_{A}), V_\beta(\xi_A,\vec r_{pA}) \psi_{M_{A}}^{I_{A}}(\xi_{A}))\\
&+(\psi_{M_{A}}^{I_{A}}(\xi_{A}), V_\beta(\vec r_{pn})
\psi_{M_{A}}^{I_{A}}(\xi_{A}))- \bar U_\beta (r_{pA}).
\end{split}
\end{equation}
We assume
\begin{equation}\label{eq426}
  U_\beta (r_{pA})=(\psi_{M_{A}}^{I_{A}}(\xi_{A}), V_\beta(\xi_A,\vec r_{pA}) \psi_{M_{A}}^{I_{A}}(\xi_{A})).
\end{equation}
Then
\begin{equation}\label{eq427}
(\psi_{M_{A}}^{I_{A}}(\xi_{A}), V'_\beta\, \psi_{M_{A}}^{I_{A}}(\xi_{A}))= V_{np}(\vec r_{pn})
\end{equation}


Inserting eq. (\ref{eq427}) into eq. (\ref{eq423}) we obtain
 \begin{equation}\label{eq428}
 \begin{split}
T_{d,p}&= \sum_l (I_A;l \vert \} I_{A+1}) (I_A\, M_A\, l\, M_{A+1}-M_A \vert I_{A+1}\,M_{A+1})  \\
&\times\int d\vec{r}_n d \vec{r}_p \chi^{* (-)}_p(k_p,\vec{r}_p) \phi_{M_{A+1}-M_A}^{*l}(\vec{r}_n)
V(\vec r_{pn}) \phi_d(\vec r_{np})
\chi^{(+)}_d(k_d,\vec{r}_d)
\end{split}
\end{equation}
The differential cross section is then equal to
\begin{equation}\label{eq429}
\frac{d \sigma}{d \Omega} = \frac{2}{3} \frac{\mu_p \mu_d}{(2\pi \hbar^2)^2}\frac{(2I_{A+1}+1)}{(2I_A+1)}
\frac{k_p}{k_d}\sum_{l,m_l}\frac{(I_A;l \vert \} I_{A+1})^2}{2l+1} \vert B_{m_l}^l\vert ^2,
\end{equation}
where
\begin{equation}\label{eq430}
B_{m_l}^l(\theta)=\int d\vec{r}_n d \vec{r}_p \chi^{* (-)}_p(k_p,\vec{r}_p) Y_m^{*l}(\hat r_n) u_{nl}(r_n)
V(\vec r_{pn}) \phi_d(\vec r_{np})
\chi^{(+)}_d(k_d,\vec{r}_d)
\end{equation}
and
\begin{equation}\label{eq431}
\phi_m^{l}(\vec{r}_n)=u_{nl}(r_n) Y_m^{l}(\hat r_n)
\end{equation}
is the single-particle wave function of a neutron moving in the core A. For simplicity, the radial wave function $u_{nl}(r_n)$ can be assumed to be a solution of a Saxon-Woods potential of parameters $V_0\approx 50$ MeV, $a=0.65$ fm and $r_0=1.25$ fm.



Equation (\ref{eq429}) gives the cross section for the stripping from the projectile of a neutron that would correspond to the n$^{\mathrm{th}}$ valence neutron in the nucleus ($A+1$). If we now want the cross section for stripping any of the valence nutrons of the final nucleus from the projectile, we must multiply eq. (\ref{eq429}) by $n$. A more careful treatment of the antisymmetry with respect to the neutrons shoes this to be the correct answer.


Finally we get
\begin{equation}\label{eq432}
\frac{d \sigma}{d \Omega}=\frac{(2I_{A+1}+1)}{(2I_A+1)} \sum_l S_l \sigma_l(\theta)
\end{equation}
where
\begin{equation}\label{eq433}
S_l= n (I_A;l \vert \} I_{A+1})^2
\end{equation}
and
\begin{equation}\label{eq434}
\sigma_l(\theta)=\frac{2}{3} \frac{\mu_p \mu_d}{(2\pi \hbar^2)^2}
\frac{k_p}{k_d}\frac{1}{2l+1}\sum_{m} \vert B_{m}^l\vert^2
\end{equation}


The distorted wave programs numerically evaluate the quantity $B_{m_l}^l(\theta)$, using for the wave functions $\chi^{(-)}$ and $\chi^{(+)}$ the solution of the optical potentials that fit the elastic scattering, i.e.
\begin{equation}\label{eq435}
(-\nabla ^2+\bar U-k^2) \chi=0
\end{equation}
(see eq. (\ref{eq11})).
Note that if the target nucleus is even, $I_A=0$,  $l=I_{A+1}$. That is, only one $l$ value contributes in eq. (\ref{eq429}), and the angular distribution is uniquely given by $\sum_{m} \vert B_{m}^l\vert^2$. The $l$-dependence of the angular distributions helps to identify $l=I_{A+1}$. The factor $S_l$ needed to normalize the calculated function to the data yields (assuming a good fit to the angular distribution), is known in the literature as the spectroscopic factor. It was assumed not only that it could be defined, but also that it contained all the nuclear structure information (aside from that associated with the angular distribution) which could be extracted from single--particle transfer. In other words, that it was the bridge directly connecting theory with experiment. Because nucleons are never bare, but are dressed by the coupling to collective modes (cf. \ref{ff}), the spectroscopic factor approximation is at best a helpful tool to get order of magnitude information from one particle transfer data.
There is a fundamental problem which makes the handling of integrals like that of (\ref{eq430}) difficult to handle, if not numerically at least conceptually. This difficulty is connected with the so called recoil effect \footnote{While this effect could be treated in a cavalier fashion in the case of light ion reactions ($m_a/m_A\ll 1$), this was not possible in the case of heavy ion reactions, as the change in momenta involved were always sizeable.}, namely the fact that the center of mass of the two interacting particles in entrance ($\mathbf r_{\alpha}: \alpha=a+A$) and exit  ($\mathbf r_{\beta}: \beta=b+B$) channels is different. This is at variance with what one is accustomed to deal with in nuclear structure calculations, in which the Hartree potential depends on a single coordinate, as well as in the case of elastic and inelastic reactions, situations in which $\mathbf r_{\alpha}=\mathbf r_{\beta}$. When $\mathbf r_{\alpha}\neq\mathbf r_{\beta}$ we enter a rather more complex many--body problem, in particular if continuum states are to be considered, than nuclear structure practitioners were accustomed to.
 
 
 Of notice that similar difficulties have been faced in connection with the non--local Fock (exchange) potential. As a rule, the corresponding (HF) mean field equations are rendered local making use of the $k$--mass approximation or within the framework of Local Density Functional Theory (DFT), in particular with the help of the Kohn--Sham equations (see e.g.  \cite{Mahaux:85}, \cite{Broglia:04b}). Although much of the work in this field is connected with the correlation potential  (interweaving of single--particle and collective motion), an important fraction is connected with the exchange potential.
 
 
 In any case, and returning to the subject of the present appendix, it is always useful to be able to introduce approximations which can help the physics which is at the basis of the phenomenon under discussion (single--particle motion) emerge in a natural way, if not to compare in detail with the experimental data. 
Within this context, to reduce the integral \ref{eq430} it is customary to assume that the proton-neutron interaction $V_{np}$ has zero-range, i.e.
\begin{equation}\label{eq436}
 V_{np}(\vec r_{np})\phi_d(\vec r_{np})=D_0 \delta(\vec r_{np})
\end{equation}
so that  $B_{m}^l$ becomes equal to
\begin{equation}\label{eq430a}
B_{m_l}^l(\theta)=D_0 \int d\vec r \chi^{* (-)}_p(k_p,\vec r) Y_{m_l}^{*l}(\hat r) u_{l}(r)
\chi^{(+)}_d(k_d,\vec r),
\end{equation}
which is  a three dimensional integral, but in fact essentially a one--dimensional integral, as the integration over the angles is simple to carry out.


\section{Plane-wave limit}


If in eq. (\ref{eq435}) we set $\bar U=0$ the distorted waves becomes plane waves i.e.
 \begin{subequations}
\begin{align}\label{eq437}
&\chi^{(+)}_d(k_d,\vec r)=e^{i \vec k_d \cdot \vec r},\\
&\chi^{*(-)}_d(k_p,\vec r)=e^{-i \vec k_p \cdot \vec r}.
\end{align}
\end{subequations}
Equation (\ref{eq430a}) can now be written as
\begin{equation}\label{eq438}
B_{m}^l=D_0 \int d\vec r e^{i (\vec k_d-\vec k_p) \cdot \vec r} Y_m^{*l}(\hat r) u_{l}(r).
\end{equation}
The linear momentum transferred to the nucleus is $\vec k_d-\vec k_p=\vec q$.
Let us expand $e^{i \vec q \cdot \vec r}$ in spherical harmonics, i.e.
\begin{equation}\label{eq439}
\begin{split}
 e^{i \vec q \cdot \vec r}&=\sum_l i^l j_l(qr)(2l+1)P_l(\hat q \cdot \hat r)\\
& =4 \pi \sum_l i^l j_l(qr)\sum_m Y_m^{*l}(\hat q) Y_m^{l}(\hat r),
\end{split}
\end{equation}
so
\begin{equation}\label{eq440}
 \int d\hat r e^{i \vec q \cdot \vec r} Y_m^{l}(\hat r)= 4 \pi i^l j_l(qr) Y_m^{*l}(\hat q).
\end{equation}
Then
\begin{equation}\label{eq441}
\begin{split}
 \sum_{m} \vert B_{m}^l\vert^2 & = \sum_{m} \vert Y_m^{l}(\hat q)\vert^2 D_0^2 16 \pi^2 \times \\
& \left \vert \int r^2 dr j_l(qr) u_l(r) \right \vert ^2=\\
& \frac{2l+1}{4 \pi} D_0^2 16 \pi^2 \left \vert \int r^2 dr j_l(qr) u_l(r) \right \vert ^2.
\end{split}
\end{equation}
Thus, the angular distribution is given by the integral $\left \vert \int r^2 dr j_l(qr) u_l(r) \right \vert ^2$ If we assume that the process takes place mostly on the surface, the angular distribution will be given by $ \vert j_l(qR_0) \vert ^2 $ where $R_0$ is the nuclear radius.

\begin{figure}
\centerline{\includegraphics*[width=18cm,angle=0]{figs_C6/pwl}}
\caption{}\label{fig3}
\end{figure}

We then have
\begin{equation}\label{eq442}
 \begin{split}
  q^2&= k_d^2+k_p^2- 2 k_d k_p \cos(\theta)\\
& =(k_d^2+k_p^2- 2 k_d k_p) + 2 k_d k_p (1-\cos(\theta))\\
& =(k_d-k_p)^2+ 4 k_d k_p \left(\sin (\theta/2)\right) ^2  \\
& \approx 4 k_d k_p \left(\sin (\theta/2)\right) ^2
\end{split}
\end{equation}
since $ k_d \approx k_p $ for stripping reactions at typical energies. Thus the angular distribution has a diffraction-like structure given by
\begin{equation}\label{eq443}
\vert j_l(qR_0) \vert ^2= j_l^2 (2R_0 \sqrt{k_d k_p} \sin (\theta/2)).
\end{equation}
The function $j_l(x)$ has its first maximum at $x=l$, i.e. where
\begin{equation}\label{eq444}
\sin (\theta/2)=\frac{l}{2 R_0 k},\quad \quad (k_p \approx  k_d=k),
\end{equation}
Examples of the above relation are provided in Fig. \ref{ñ}
\begin{figure}
\centerline{\includegraphics*[width=10cm,angle=0]{figs_C6/Reaction3}}
\caption{One--particle reaction $a(=b+1)+A\rightarrow b+B(=A+1)$. The time arrow is assumed to point upwards. The quantum numbers characterizing the states in which the transferred nucleon moves in projectile and target are denoted $a'_1$ and $a_1$ respectively. The interaction inducing the nucleon to be transferred can act either in the entrance channel ($(a,A);v_{1A}$, prior representation) or in the exit channel ($(b,B);v_{1b}$, post representation), in keeping with energy conservation. In the transfer process, the nucleon changes orbital at the same time that a change in the mass partition takes place. The corresponding relative motion mismatch is known as the recoil process, and is represented by a jagged line which provides information on the evolution of $r_{1A}$ ($r_{1b}$). In other words on the coupling of reaction and transfer modes.}\label{fig1}
\end{figure}

\begin{subappendices}
\section{Minimal requirements for a consistent mean field theory}
In what follows th question of why, rigorously speaking, one cannot talk about single--particle motion, let alone spectroscopic factor, not even within the framework of Hartree--Fock theory, is briefly touched upon.


As can be seen from Fig. \ref{kk} the minimum requirements of selfconsistency to be imposed upon single particle motion requires both non--locality in space (HF) and in time (TDHF)
\begin{equation}
i\hbar \frac{\partial \rho_\nu}{\partial t}=-\frac{\hbar^2}{2 m}\nabla^2 \varphi_\nu(x,t)+\int dx'dt'U(x-x',t-t')\varphi_\nu(x',t')
\end{equation}
and consequently also of collective vibrations and, consequently, from their interweaving to dressed single--particles (quasiparticles), let alone renormalized collective modes. Assuming for simplicity infinite nuclear matter ( confined by a constant potential of depth $V_0$), and thus plane wave solutions, the above time--dependent Schr\"{o}dinger equation leads to the quasiparticle dispersion relation (cf. e.g. \ref{ll})
\begin{equation}
\hbar\omega=\frac{\hbar^2k^2}{2m^*}+\frac{m}{m^*}V_0,
\end{equation}
where the effective mass
\begin{equation}
m^*=\frac{m_k m_\omega}{m},
\end{equation}
in the product of the $k$--mass
\begin{equation}
m_k=m\left(1+\frac{m}{\hbar^2k}\frac{\partial U}{\partial k}\right)^{-1}
\end{equation}
closely connected with the Pauli principle $\left(\frac{\partial U}{\partial k}\approx \frac{\partial U_x}{\partial k}\right)$, while the $\omega$--mass
\begin{equation}
m_\omega=m\left(1-\frac{\partial U}{\partial \hbar \omega}\right)
\end{equation}
results from the dressing of the nucleon through the coupling with the (quasi) bosons. Because typically $m_k\approx 0.7 m$ and $m_\omega \approx 1.4 m$ $m^*\approx m$, one could be tempted to conclude that the results embodied in the dispersion relation \ref{ñññ} reflects that the distribution of levels around the Fermi energy can be described in terms of the solutions of a Schr\"{o}dinger equation in which nucleons of mass equal to the bare nucleon mass $m$ move in a Saxon--Woods potential of depth $V_0$.


Now, it can be shown that the occupancy of levels around $\varepsilon_F$ is given by $Z_\omega$ (cf. Fig. \ref{gg}) a quantity which is equal to $m/m_\omega\approx 0.7$. This, in keeping with the fact that the time the nucleon is coupled to the vibrations it cannot behave as a single--particle and can thus not contribute to e.g. the single--particle pickup cross section.

It is of notice that the selfconsistence requirements for the iterative solution of the Kohn--Sham equations
 \begin{equation}
 H^{KS}\varphi_\gamma(\mathbf{r})=\lambda_\gamma\varphi_\gamma(\mathbf{r}),
 \end{equation}
where
 \begin{equation}
 H^{KS}=-\frac{\hbar^2}{2 m_e}\nabla^2+U_H(\mathbf{r})+V_{ext}(\mathbf{r})+U_{xc}(\mathbf{r}),
 \end{equation}
$H^{KS}$ being known as the Kohn--Sham Hamiltonian, $V_{ext}(\mathbf{r})$ being the field created by the ions and acting on the electrons. Both the Hartree and the exchange--correlation potentials $U_H(\mathbf{r})$ and $U_{xc}(\mathbf{r})$ depend on the (local) density, hence on the whole set of wavefunctions $\varphi_\gamma(\mathbf{r})$. Thus, the set of $KS$--equations must be solve selfconsistently. (\ref{gg}).
\section{Model for single--particle strength function: Dyson equation}
In the previous section we introduce the argument of the impossibility of defining a ``bona fide'' single--particle spectroscopic factor. It was done with the help of Feynman (NFT) diagrams. In what follows we essentially repeat the arguments, but this time in terms of Dyson's (Schwinger) language.

For simplicity, we consider a two--level model where the pure single--particle state $|a\rangle$ couples to a more complicated state 
$|\alpha\rangle$, made out of a fermion (particle or hole), couple to a particle--hole excitation which, if iterated to all orders can give rise to a collective state (cf. Fig.\ref{l}). The Hamiltonian describing the system is
 \begin{equation}
 H=H_0+U
 \end{equation}
 where
  \begin{equation}
  H_0|a\rangle=E_a|a\rangle,
  \end{equation}
  and
    \begin{equation}
    H_0|\alpha\rangle=E_\alpha|\alpha\rangle.
    \end{equation}
  Let us call $\langle a|U|\alpha\rangle=U_{a\alpha}$ and assume $\langle a|U|a\rangle=\langle \alpha|U|\alpha \rangle=0$.
  
  From the secular equation associated with $H$, namely
\begin{equation}
\left(
\begin{matrix}
E_\alpha & U_{a\alpha}\\  
U_{a \alpha} & E_a-E_i \\ 
\end{matrix}
\right)
\left(
\begin{matrix}
C_\alpha(i)\\  
C_a(i)\\ 
\end{matrix}
\right)=0,
\end{equation}
and associated normalization condition
\begin{equation}
C_a^2(i)+C_\alpha^2(i)=0,
 \end{equation}
one obtains
\begin{equation}\label{eqAp6B1}
C_a^2(i)=\left(1+\frac{U_{a\alpha}^2}{(E_\alpha-E_i)^2}\right)^{-1},
 \end{equation}
and 
\begin{equation}\label{eqAp6B2}
\Delta E_a(E)=E_a-E=\frac{U_{a\alpha}^2}{E_a-E}.
 \end{equation}
The relations \ref{eqAp6B1} and \ref{eqAp6B2} allows one to write the correlated state
\begin{equation}\label{eqAp6B3}
|\tilde a\rangle=C_a(i)| a\rangle+C_\alpha (i)| \alpha\rangle,
 \end{equation}
the corresponding energy being provided by the (iterative) solution of the Dyson equation \ref{eqAp6B2}, which propagate the bubble diagrams shown in Figs \ref{t} (a) and (b) to infinite order leading to collective vibrations (see Fig. \ref{t})

With the help of the definition \ref{4}, and making use of the fact that in the present case, $U\equiv\Delta E_a(E)$, one obtains
\begin{equation}\label{eqAp6B4}
Z_\omega=C_a^2(i)=\frac{m_\omega}{m}
 \end{equation}
the solution of \ref{eqAp6B2} together with the relations \ref{r} and \ref{r} lead to the quasiparticle state \ref{3}, to be employed in the calculation of the one--particle transfer transition amplitudes (cf. e.g. \ref{d} and \ref{t}) 
\section{Self--energy and vertex corrections}
In Fig. \ref{...} an example of the fact that in field theories (e.g. QED or NFT), nothing is free and that e.g. the bare masss of a fermion (electron or nucleon), is the parameter one adjusts ($n_k$) so that the result of a measurement (cf. Fig. \ref{...}) gives the observed mass (single particle energy). In Fig. \ref{...}, lowest order diagrams associated with the renormalization of the fermion--boson interaction (vertex corrections) are given. The sum of contributions (a) and (b) can, in principle, be represented by a renormalized vertex (cf. digram (c) of Fig. \ref{...}). It is of notice, however, that there is, as a rule, conspicuous interference (e.g. cancellation) in the nuclear case between vertex nd self--energy contributions (cf. diagram (e) and (f) of Fig. \ref{,,,} a phenomenon closely related with conservation laws (generalized Ward identities)). In particular, cancellation in the case in which the bosonic modes are isoscalar consequently, one has to sum explicitly the different amplitudes with the corresponding phases and eventually take the modulus squared of the result to eventually obtain the quantities to be compared with the data, a fact that precludes the use of an effective (renormalized) vertex (cf. Fig. \ref{...}).


Within the framework of QED the above mentioned cancellations are exact implying that the interaction between one and two photon states vanishes (Furry theorem). The physics at the basis of the cancellation found in the nuclear case can be exemplified by looking at a spherical nucleus displaying a low--lying collective quadrupole zero point vibration. The associated zero point fluctuations (ZPF) lead to time dependent shapes with varied instantaneous values of the quadrupole moment,  and of its orientation (dynamical spontaneous breaking of rotational invariance). In other words, a component of the ground state wavefunction ($|(j_p \otimes j_h^{-1})_{2^+} \otimes 2^+; 0^+\rangle $) can be viewed as a  gas of quadrupole (quasi) bosons promoting a nucleon across the Fermi energy (particle--hole excitation) will lead to fermionic states which behave as having a positive (particle) and a negative (hole) effective quadrupole moment, in keeping with the fact that the closed shell system is spherical, thus carrying zero quadrupole moment. 
\section{One--particle knockout within DWBA}
\subsection{Spinless particles}
We are going to consider the reaction $A+a \rightarrow a+b+c$, in which the cluster $b$ is knocked out from the nucleus $A(=c+b)$. Cluster $b$ is thus initially bounded, while the final states of $a,b$ and the initial state of $a$ are all in the continuum, and can be described with distorted waves defined as scattering solutions of a (as a rule, complex) suitable optical potential. A schematic depiction of the situation is shown in Fig. \ref{fig1}. We will begin by considering the simplified case in which the clusters $a,b,c$ are spinless.
\subsubsection{Transition amplitude}
We consider optical potentials $U(r_{aA}),U(r_{cb}),U(r_{ac})$ which will be central potentials without a spin--orbit term. In addition, the interaction $v(r_{ab})$ between $a$ and $b$ is taken to be an arbitrary function of the distance $r_{ab}$. Then, the transition amplitude which is at the basis of the evaluation of the multi--differential cross section is the 6--dimensional integral
\begin{equation}\label{eq1}
\begin{split}
T_{m_b}=\int d\mathbf{r}_{aA}d \mathbf{r}_{bc}\chi^{(-)*}(\mathbf{r}_{ac})\chi^{(-)*}(\mathbf{r}_{bc})v(r_{ab})\chi^{(+)}(\mathbf{r}_{aA})u_{l_b}(r_{bc})Y^{l_b}_{m_b}(\hat{\mathbf{r}}_{bc}).
\end{split}
\end{equation}
\subsubsection{Coordinates}
The vectors $\mathbf{r}_{ab},\mathbf{r}_{ac}$ can easily be written in function of the integration variables $\mathbf{r}_{aA},\mathbf{r}_{bc}$ (see Fig. \ref{fig1}), namely
\begin{equation}\label{eq18}
\begin{split}
\mathbf{r}_{ac}&=\mathbf{r}_{aA}+\frac{b}{A}\mathbf{r}_{bc},\\
\mathbf{r}_{ab}&=\mathbf{r}_{aA}-\frac{c}{A}\mathbf{r}_{bc},
\end{split}
\end{equation}
where $b,c,A$ stand for the number of nucleons of the species $b,c$ and $A$ respectively.
\subsubsection{Distorted waves in the continuum}
A standard way to reduce the dimensionality of the integral \ref{eq1} consists in expanding the continuum wave functions $\chi^{(+)}(\mathbf{r}_{aA}),\chi^{(-)*}(\mathbf{r}_{ac}),\chi^{(-)*}(\mathbf{r}_{bc})$ in a basis of eigenstates of the angular momentum operator (partial waves). Then we can exploit the transformation properties of these eigenstates under rotations to perform the angular integrations. With time--reversed phase convention, that is
\begin{equation}\label{eq19}
Y_m^l(\theta,\phi)=i^l \sqrt{\frac{2l+1}{4\pi}\frac{(l-m)!}{(l+m)!}}P_l^m(\cos \theta)e^{im\phi},
\end{equation}
 the general form of these expansions is
 \begin{equation}\label{eq2}
\chi^{(+)}(\mathbf{k},\mathbf{r})= \sum_{l}\frac{4\pi}{k r} i^{l}\sqrt{2l+1}
e^{i\sigma^{l}} F_{l}(r) \left[ Y^{l} (\hat {\mathbf{r}}) Y^{l} (\hat {\mathbf{k}})\right]^0_0,
\end{equation}
 \begin{equation}\label{eq3}
\chi^{(-)*}(\mathbf{k},\mathbf{r})= \sum_{l}\frac{4\pi}{k r} i^{-l}\sqrt{2l+1}
e^{i\sigma^{l}} F_{l}(r) \left[ Y^{l} (\hat {\mathbf{r}}) Y^{l} (\hat {\mathbf{k}})\right]^0_0,
\end{equation}
where $\sigma_l$ is the Coulomb phase shift. The radial functions $F_{l}(r)$ are regular (finite at $r=0$) solutions of the one--dimensional Schr\"{o}dinger equation with an effective potential $U(r)+\tfrac{\hbar^2 l(l+1)}{2\mu r^2}$ and suitable asymptotic behaviour at $r\rightarrow\infty$ as boundary conditions. 
So the distorted waves appearing in \ref{eq1} are
 \begin{equation}\label{eq4}
\chi^{(+)}(\mathbf{k_{a}},\mathbf{r}_{aA})= \sum_{l_a}\frac{4\pi}{k_a r_{aA}} i^{l_a}\sqrt{2l_a+1}
e^{i\sigma^{l_a}} F_{l_a}(r_{aA}) \left[ Y^{l_a} (\hat{\mathbf r}_{aA}) Y^{l_a} (\hat{ \mathbf k}_{a})\right]^0_0,
\end{equation}
(initial relative motion between $A$ and $a$, defined from the complex optical potential $U(r_{Aa})$)
 \begin{figure}
\centerline{\includegraphics*[width=10cm,angle=0]{figs_C6/knock1.pdf}}
\vspace{-4cm}
\caption{Sketch of the system considered to describe the reaction $A+a \rightarrow a+b+c$. The nucleus $A$ is viewed as an inert cluster $b$ bounded to an inert core $c$.}\label{fig1}
\end{figure}
 \begin{equation}\label{eq5}
\chi^{(-)*}(\mathbf{k'_{a}},\mathbf{r}_{ac})= \sum_{l'_a}\frac{4\pi}{k'_a r_{ac}} i^{-l'_a}\sqrt{2l'_a+1}
e^{i\sigma^{l'_a}} F_{l'_a}(r_{ac}) \left[ Y^{l'_a} (\hat{\mathbf r}_{ac}) Y^{l'_a} (\hat{ \mathbf k}'_{a})\right]^0_0,
\end{equation}
(final relative motion between $c$ and $a$, defined from the complex optical potential $U(r_{ac})$)
 \begin{equation}\label{eq6}
\chi^{(-)*}(\mathbf{k'_{b}},\mathbf{r}_{bc})= \sum_{l'_b}\frac{4\pi}{k'_b r_{bc}} i^{-l'_b}\sqrt{2l'_b+1}
e^{i\sigma^{l'_b}} F_{l'_b}(r_{bc}) \left[ Y^{l'_b} (\hat{\mathbf r}_{bc}) Y^{l'_b} (\hat{ \mathbf k}'_{b})\right]^0_0,
\end{equation}
(final relative motion between $b$ and $c$, defined from the complex optical potential $U(r_{bc})$).
\subsubsection{Recoupling of angular momenta}
We thus need to evaluate the 6--dimensional integral
\begin{equation}\label{eq14}
\begin{split}
\frac{64\pi^3}{k_ak'_ak'_b}&\int d\mathbf{r}_{aA}d \mathbf{r}_{bc}u_{l_b}(r_{cb})v(r_{ab})\sum_{l_a,l'_a,l'_b}\sqrt{(2l_a+1)(2l'_a+1)(2l'_b+1)}\\
&\times e^{i(\sigma^{l_a}+\sigma^{l'_a}+\sigma^{l'_b})} \frac{F_{l_a}(r_{aA})  F_{l'_a}(r_{ac})F_{l'_b}(r_{bc})}{r_{ac}r_{aA}r_{bc}}\\
&\times \left[ Y^{l_a} (\hat{\mathbf r}_{aA}) Y^{l_a} (\hat{ \mathbf k}_{a})\right]^0_0\left[ Y^{l'_a} (\hat{\mathbf r}_{ac}) Y^{l'_a} (\hat{ \mathbf k}'_{a})\right]^0_0\left[ Y^{l'_b} (\hat{\mathbf r}_{bc}) Y^{l'_b} (\hat{ \mathbf k}'_{b})\right]^0_0Y^{l_b}_{m_b}(\hat{\mathbf{r}}_{bc}).
\end{split}
\end{equation}
Note that this expression depends explicitly on the asymptotic kinetic energies ($k_a,k'_a,k'_b$) and scattering angles  ($\hat{ \mathbf k}_{a},\hat{ \mathbf k}'_{a},\hat{ \mathbf k}'_{b}$) of $a,b$.
Now we will take advantage of the partial wave expansion to reduce the dimensions of the integral from 6 to 3. A possible strategy to deal with \ref{eq14} is to recouple together all the terms that depend on the integration variables to a global angular momentum and retain  only the term coupled to 0 as the only one surviving the integration.
Let us couple separately the terms corresponding to particle $a$ and particle $b$. For particle $a$
\begin{equation}\label{eq7}
\begin{split}
\left[ Y^{l_a} (\hat{\mathbf r}_{aA}) Y^{l_a} (\hat{ \mathbf k}_{a})\right]^0_0 & \left[ Y^{l'_a} (\hat{\mathbf r}_{ac}) Y^{l'_a} (\hat{ \mathbf k}'_{a})\right]^0_0=\sum_K \bigl((l_a l_a)_0(l'_a l'_a)_0|(l_a l'_a)_K(l_a l'_a)_K\bigr)_0\\
& \times \left\{\left[ Y^{l_a} (\hat{\mathbf r}_{aA}) Y^{l'_a} (\hat{ \mathbf r}_{ac})\right]^K \left[Y^{l_a} (\hat{\mathbf k}_{a}) Y^{l'_a} (\hat{ \mathbf k}'_{a})\right]^K\right\}^0_0.
\end{split}
\end{equation}
We can evaluate the $9j$ symbol,
\begin{equation}\label{eq8}
\bigl((l_a l_a)_0(l'_a l'_a)_0|(l_a l'_a)_K(l_a l'_a)_K\bigr)_0=\sqrt{\frac{2K+1}{(2l'_a+1)(2l_a+1)}},
\end{equation}
and expand the coupling,
\begin{equation}\label{eq9}
\begin{split}
&\left\{\left[ Y^{l_a}(\hat{\mathbf r}_{aA})  Y^{l'_a} (\hat{ \mathbf r}_{ac})\right]^K \left[Y^{l_a} (\hat{\mathbf k}_{a}) Y^{l'_a} (\hat{ \mathbf k}'_{a})\right]^K\right\}^0_0=\sum_M \langle K\;K\;M\;-M|0\;0\rangle\\
&\times \left[ Y^{l_a} (\hat{\mathbf r}_{aA}) Y^{l'_a} (\hat{ \mathbf r}_{ac})\right]^K_M \left[Y^{l_a} (\hat{\mathbf k}_{a}) Y^{l'_a} (\hat{ \mathbf k}'_{a})\right]^K_{-M}=\sum_M\frac{(-1)^{K+M}}{\sqrt{2K+1}}\\
&\times \left[ Y^{l_a} (\hat{\mathbf r}_{aA}) Y^{l'_a} (\hat{ \mathbf r}_{ac})\right]^K_M \left[Y^{l_a} (\hat{\mathbf k}_{a}) Y^{l'_a} (\hat{ \mathbf k}'_{a})\right]^K_{-M}.
\end{split}
\end{equation}
Thus,
\begin{equation}\label{eq20}
\begin{split}
\left[ Y^{l_a} (\hat{\mathbf r}_{aA}) Y^{l_a} (\hat{ \mathbf k}_{a})\right]^0_0 & \left[ Y^{l'_a} (\hat{\mathbf r}_{ac}) Y^{l'_a} (\hat{ \mathbf k}'_{a})\right]^0_0=\sqrt{\frac{1}{(2l'_a+1)(2l_a+1)}}\\
&\times\sum_{KM}(-1)^{K+M}\left[ Y^{l_a} (\hat{\mathbf r}_{aA}) Y^{l'_a} (\hat{ \mathbf r}_{ac})\right]^K_M \left[Y^{l_a} (\hat{\mathbf k}_{a}) Y^{l'_a} (\hat{ \mathbf k}'_{a})\right]^K_{-M}.
\end{split}
\end{equation}
We can further simplify the above expression if we take the direction of the initial momentum to be parallel to the $z$ axis, so $Y^{l_a}_m (\hat{\mathbf k}_{a})=Y^{l_a}_m (\hat{\mathbf z})=\sqrt{\frac{2l_a+1}{4\pi}}\delta_{m,0}$. Then,
\begin{equation}\label{eq10}
\begin{split}
\left[ Y^{l_a} (\hat{\mathbf r}_{aA}) Y^{l_a} (\hat{ \mathbf k}_{a})\right]^0_0 & \left[ Y^{l'_a} (\hat{\mathbf r}_{ac}) Y^{l'_a} (\hat{ \mathbf k}'_{a})\right]^0_0=\sqrt{\frac{1}{4\pi(2l'_a+1)}}\sum_{KM}(-1)^{K+M}\\
&\times\langle l_a\;0\;l'_a\;-M|K\;-M\rangle\left[ Y^{l_a} (\hat{\mathbf r}_{aA}) Y^{l'_a} (\hat{ \mathbf r}_{ac})\right]^K_M   Y^{l'_a}_{-M} (\hat{ \mathbf k}'_{a}).
\end{split}
\end{equation}
For the particle $b$ we have
\begin{equation}\label{eq11}
\begin{split}
Y^{l_b}_{m_b}(\hat{\mathbf{r}}_{bc})\left[ Y^{l'_b} (\hat{\mathbf r}_{bc}) Y^{l'_b} (\hat{ \mathbf k}'_{b})\right]^0_0=Y^{l_b}_{m_b}(\hat{\mathbf{r}}_{cb})\sum_m \frac{(-1)^{l'_b+m}}{\sqrt{2l'_b+1}}Y^{l'_b}_m (\hat{\mathbf r}_{bc})Y^{l'_b}_{-m} (\hat{ \mathbf k}'_{b}),
\end{split}
\end{equation}
but we can write
\begin{equation}\label{eq12}
\begin{split}
Y^{l_b}_{m_b}(\hat{\mathbf{r}}_{bc})Y^{l'_b}_m (\hat{\mathbf r}_{bc})=\sum_{K'}\langle l_b\;m_b\;l'_b\;m|K'\;m_b+m\rangle \left[ Y^{l_b} (\hat{\mathbf r}_{bc}) Y^{l'_b} (\hat{\mathbf r}_{bc})\right]^{K'}_{m_b+m}.
\end{split}
\end{equation}
In order to couple to 0 angular momentum with \ref{eq10} we must only keep the term with  $K'=K,\;m=-M-m_b$ so
\begin{equation}\label{eq13}
\begin{split}
Y^{l_b}_{m_b}(\hat{\mathbf{r}}_{bc})&\left[ Y^{l'_b} (\hat{\mathbf r}_{bc}) Y^{l'_b} (\hat{ \mathbf k}'_{b})\right]^0_0=\frac{(-1)^{l'_b-M-m_b}}{\sqrt{2l'_b+1}}\langle l_b\;m_b\;l'_b\;-M-m_b|K\;-M\rangle\\
&\times \left[ Y^{l_b} (\hat{\mathbf r}_{bc}) Y^{l'_b} (\hat{\mathbf r}_{bc})\right]^{K}_{-M}Y^{l'_b}_{-M-m_b} (\hat{ \mathbf k}'_{b}),
\end{split}
\end{equation}
and \ref{eq14} becomes
\begin{equation}\label{eq15}
\begin{split}
\frac{32\pi^2}{k_ak'_ak'_b}&\sum_{KM}(-1)^{K+l'_b-m_b}\langle l_a\;0\;l'_a\;-M|K\;-M\rangle\langle l_b\;m_b\;l'_b\;-M-m_b|K\;-M\rangle\\
&\times \sum_{l_a,l'_a,l'_b}\sqrt{(2l_a+1)} e^{i(\sigma^{l_a}+\sigma^{l'_a}+\sigma^{l'_b})}Y^{l'_b}_{-M-m_b} (\hat{ \mathbf k}'_{b}) Y^{l'_a}_{-M} (\hat{ \mathbf k}'_{a})\int d\mathbf{r}_{aA	}d \mathbf{r}_{bc}u_{l_b}(r_{bc})v(r_{ab}) \\
&\times\frac{F_{l_a}(r_{aA})  F_{l'_a}(r_{ac})F_{l'_b}(r_{bc})}{r_{ac}r_{aA}r_{bc}}\left[ Y^{l_a} (\hat{\mathbf r}_{aA}) Y^{l'_a} (\hat{ \mathbf r}_{ac})\right]^K_M   \left[ Y^{l_b} (\hat{\mathbf r}_{bc}) Y^{l'_b} (\hat{\mathbf r}_{bc})\right]^{K}_{-M}.
\end{split}
\end{equation}
Note that
\begin{equation}\label{eq16}
\begin{split}
\left[ Y^{l_a} (\hat{\mathbf r}_{aA}) Y^{l'_a} (\hat{ \mathbf r}_{ac})\right]^K_M &   \left[ Y^{l_b} (\hat{\mathbf r}_{bc}) Y^{l'_b} (\hat{\mathbf r}_{bc})\right]^{K}_{-M}=\sum_P \langle K\;M\;K\;-M|P\;0\rangle\\
&\times \left\{\left[ Y^{l_a} (\hat{\mathbf r}_{aA}) Y^{l'_a} (\hat{ \mathbf r}_{ac})\right]^K\left[ Y^{l_b} (\hat{\mathbf r}_{bc}) Y^{l'_b} (\hat{\mathbf r}_{bc})\right]^{K} \right\}^P_0,
\end{split}
\end{equation}
and that to survive the integration the rotational tensors must be coupled to $P=0$. Keeping only this term in the sum over $P$, we get
\begin{equation}\label{eq17}
\begin{split}
\left[ Y^{l_a} (\hat{\mathbf r}_{aA}) Y^{l'_a} (\hat{ \mathbf r}_{ac})\right]^K_M &   \left[ Y^{l_b} (\hat{\mathbf r}_{bc}) Y^{l'_b} (\hat{\mathbf r}_{bc})\right]^{K}_{-M}=\\
&\frac{(-1)^{K+M}}{\sqrt{2K+1}}\left\{\left[ Y^{l_a} (\hat{\mathbf r}_{aA}) Y^{l'_a} (\hat{ \mathbf r}_{ac})\right]^K\left[ Y^{l_b} (\hat{\mathbf r}_{bc}) Y^{l'_b} (\hat{\mathbf r}_{bc})\right]^{K} \right\}^0_0.
\end{split}
\end{equation}
 \begin{figure}
\centerline{\includegraphics*[width=10cm,angle=0]{figs_C6/coords2.pdf}}
\vspace{-3cm}
\caption{Coordinates in the ``standard'' configuration.}\label{fig2}
\end{figure}
The coordinate--dependent part of the latter expression is  a rotationally invariant scalar, so it can be evaluated in any conventional ``standard'' configuration such as the one depicted in Fig. \ref{fig2}. It must then be multiplied by a factor resulting of the integration of the remaining angular variables, which accounts for the rigid rotations needed to connect any arbitrary configuration to one of this type. This factor turns out to be $8\pi^2$ (a $4\pi$ factor for all possible orientations of, say, $\mathbf r_{aA}$ and a $2\pi$ factor for a complete rotation around its direction). According to Fig. \ref{fig2},
\begin{equation}\label{eq22}
\begin{split}
\mathbf{r}_{bc}&=r_{bc}\left(\sin \theta\, \hat x+\cos \theta\,\hat z \right),\\
\mathbf{r}_{aA}&=-r_{aA}\,\hat z,\\
\mathbf{r}_{ac}&=\frac{b}{A}r_{bc}\sin \theta\,\hat x+\left(\frac{b}{A}r_{bc}\cos \theta-r_{aA}\right)\,\hat z.
\end{split}
\end{equation}
As $\mathbf{r}_{aA}$ lies parallel to the $z$ axis, $Y^{l_a}_{M_K} (\hat{\mathbf r}_{aA})=\sqrt{\frac{2l_a+1}{4\pi}}\delta_{M_K,0}$ and
\begin{equation}\label{eq21}
\begin{split}
\left[ Y^{l_a} (\hat{\mathbf r}_{aA})\right.&\left. Y^{l'_a} (\hat{ \mathbf r}_{ac})\right]^K_{M_K}=\sum_{m}\langle l_a\;m\;l'_a\;M_K-m|K\;M_K\rangle Y^{l_a}_{m} (\hat{ \mathbf r}_{aA})Y^{l'_a}_{M_K-m} (\hat{ \mathbf r}_{ac})=\\
&\sqrt{\frac{2l_a+1}{4\pi}} \langle l_a\;0\;l'_a\;M_K|K\;M_K\rangle Y^{l'_a}_{M_K} (\hat{ \mathbf r}_{ac}).
\end{split}
\end{equation}
Then
\begin{equation}\label{eq23}
\begin{split}
&\left\{\left[ Y^{l_a} (\hat{\mathbf r}_{aA}) Y^{l'_a} (\hat{ \mathbf r}_{ac})\right]^K\left[ Y^{l_b} (\hat{\mathbf r}_{bc}) Y^{l'_b} (\hat{\mathbf r}_{bc})\right]^{K} \right\}^0_0=\\
&\sum_{M_K}\langle K\;M_K\;K\;-M_K|0\;0\rangle \left[ Y^{l_a} (\hat{\mathbf r}_{aA}) Y^{l'_a} (\hat{ \mathbf r}_{ac})\right]^K_{M_K}\left[ Y^{l_b} (\hat{\mathbf r}_{bc}) Y^{l'_b} (\hat{\mathbf r}_{bc})\right]^{K}_{-M_K}=\\
&\sqrt{\frac{2l_a+1}{4\pi}}\sum_{M_K}\frac{(-1)^{K+M_K}}{\sqrt{2K+1}} \langle l_a\;0\;l'_a\;M_K|K\;M_K\rangle\\
&\times \left[ Y^{l_b} (\hat{\mathbf r}_{bc}) Y^{l'_b} (\hat{\mathbf r}_{bc})\right]^{K}_{-M_K} Y^{l'_a}_{M_K} (\hat{ \mathbf r}_{ac}).
\end{split}
\end{equation}
Remembering the $8\pi^2$ factor, the term arising from \ref{eq17} to be considered in the integral is
\begin{equation}\label{eq24}
\begin{split}
4\pi^{3/2}\frac{\sqrt{2l_a+1}}{2K+1}&(-1)^K\sum_{M_K}(-1)^{M_K} \langle l_a\;0\;l'_a\;M_K|K\;M_K\rangle\\
&\times \left[ Y^{l_b} (\cos \theta,0) Y^{l'_b} (\cos \theta,0)\right]^{K}_{-M_K} Y^{l'_a}_{M_K} (\cos \theta_{ac},0),
\end{split}
\end{equation}
with
\begin{equation}\label{eq25}
\cos \theta_{ac}=\frac{\frac{b}{A}r_{bc}\cos \theta-r_{aA}}{\sqrt{\left(\frac{b}{A}r_{bc}\sin \theta\right)^2+\left(\frac{b}{A}r_{bc}\cos \theta-r_{aA}\right)^2}},
\end{equation}
(see \ref{eq22}). The final expression of the transition amplitude is
\begin{equation}\label{eq26}
\begin{split}
T_{m_b}(\mathbf{k}'_a,\mathbf{k}'_b)=\frac{128\pi^{7/2}}{k_ak'_ak'_b}&\sum_{KM}\frac{(-1)^{l'_b+m_b}}{2K+1}\langle l_a\;0\;l'_a\;-M|K\;-M\rangle\langle l_b\;m_b\;l'_b\;-M-m_b|K\;-M\rangle\\
&\times \sum_{l_a,l'_a,l'_b}(2l_a+1) e^{i(\sigma^{l_a}+\sigma^{l'_a}+\sigma^{l'_b})}Y^{l'_b}_{-M-m_b} (\hat{ \mathbf k}'_{b}) Y^{l'_a}_{-M} (\hat{ \mathbf k}'_{a})\,\mathcal I(l_a,l_a',l_b',K),
\end{split}
\end{equation}
where
\begin{equation}\label{eq27}
\begin{split}
\mathcal I(l_a,l_a'&,l_b',K)=\int dr_{aA} dr_{bc}d\theta r_{aA}r_{bc} \frac{\sin \theta}{r_{ac}} u_{l_b}(r_{bc})v(r_{ab})F_{l_a}(r_{aA})  F_{l'_a}(r_{ac})F_{l'_b}(r_{bc}) \\
&\times \sum_{M_K} (-1)^{M_K}\langle l_a\;0\;l'_a\;M_K|K\;M_K\rangle \left[ Y^{l_b} (\cos \theta,0) Y^{l'_b} (\cos \theta,0)\right]^{K}_{-M_K} Y^{l'_a}_{M_K} (\cos \theta_{ac},0)
\end{split}
\end{equation}
is a 3--dimensional integral that can be numerically evaluated with, e.g., Gaussian integration.
\subsection{Particles with spin}
 \begin{figure}
\centerline{\includegraphics*[width=10cm,angle=0]{figs_C6/knock2.pdf}}
\vspace{-4cm}
\caption{Now all three clusters $a,b,c$ have definite spins and projections. The nucleus $A$ is coupled to total spin $J_A,M_A$.}\label{fig3}
\end{figure}
We will now turn to the case in which the clusters have a definite spin (see Fig. \ref{fig3}),  and the optical potentials $U(r_{aA}),U(r_{cb}),U(r_{ac})$ are now central potentials with a spin--orbit term proportional to the usual product $\mathbf l \cdot \mathbf s=1/2(j(j+1)-l(l+1)-3/4)$ for particles with spin 1/2. In addition, the interaction $V(r_{ab},\boldsymbol\sigma_a,\boldsymbol\sigma_b)$ between $a$ and $b$ is taken to be a separable function of the distance $r_{ab}$ and of the spin orientations, $V(r_{ab},\boldsymbol\sigma_a,\boldsymbol\sigma_b)=v(r_{ab})v_\sigma(\boldsymbol\sigma_a,\boldsymbol\sigma_b)$. Note that this rules out a spin--orbit term and terms proportional to $\mathbf{r}\cdot \boldsymbol\sigma$, such as the tensor term! For the moment we will assume that the spin--dependent interaction is rotationally invariant (scalar with respect to rotations), such as, e.g., $v_\sigma(\boldsymbol\sigma_a,\boldsymbol\sigma_b)\propto\boldsymbol\sigma_a \cdot\boldsymbol\sigma_b$. Again, that excludes from our formalism tensor terms in the interaction. The transition amplitude is
\begin{equation}\label{eq28}
\begin{split}
T_{m_a,m_b}^{m'_a,m'_b}=\sum_{\sigma_a,\sigma_b}\int d\mathbf{r}_{aA}d \mathbf{r}_{bc}&\chi^{(-)*}_{m'_a}(\mathbf{r}_{ac},\sigma_a)\chi^{(-)*}_{m'_b}(\mathbf{r}_{bc},\sigma_b)\\
&\times v(r_{ab})v_\sigma(\sigma_a,\sigma_b)\chi^{(+)}_{m_a}(\mathbf{r}_{aA},\sigma_a)\psi_{m_b}^{l_b,j_b}(\mathbf{r}_{bc},\sigma_b).
\end{split}
\end{equation}
\subsubsection{Distorted waves}
The distorted waves in \ref{eq28} $\chi_{m}(\mathbf{r},\sigma)=\chi(\mathbf{r})\phi^{1/2}_m(\sigma)$ have a spin dependence contained in the spinor $\phi^{1/2}_m(\sigma)$, where $\sigma$ is the spin degree of freedom and $m$ the projection of the spin along the quantization axis. The superscript $1/2$ reminds us that we are considering spin $1/2$ particles, which have important consequences when dealing with the spin--orbit term of the optical potentials. As for the spin--dependent term $v_\sigma(\boldsymbol\sigma_a,\boldsymbol\sigma_b)$, the value of the spin of the involved particles does not make much difference \emph{as long as this term is rotationally invariant}. After \ref{eq2},
 \begin{equation}\label{eq29}
\chi^{(+)}(\mathbf{k},\mathbf{r})\phi_m(\sigma)= \sum_{l,j}\frac{4\pi}{k r} i^{l}\sqrt{2l+1}
e^{i\sigma^{l}} F_{l,j}(r) \left[ Y^{l} (\hat {\mathbf{r}}) Y^{l} (\hat {\mathbf{k}})\right]^0_0\phi^{1/2}_m(\sigma).
\end{equation}
Note that now the sum is also over the total angular momentum $j$, because the radial functions $F_{l,j}(r)$ depend now on $j$ as well as on $l$, being solutions of an optical potential with a spin--orbit term proportional to $1/2\left(j(j+1)-l(l+1)-3/4\right)$. We must then couple the radial and spin functions to total angular momentum $j$, noting that 
 \begin{equation}\label{eq30}
 \begin{split}
\left[ Y^{l} (\hat {\mathbf{r}}) \right. & \left. Y^{l} (\hat {\mathbf{k}})\right]^0_0\phi^{1/2}_m(\sigma)=\sum_{m_l} \langle l\;m_l\;l\;-m_l|0\;0\rangle Y^{l}_{m_l} (\hat {\mathbf{r}})Y^{l}_{-m_l} (\hat {\mathbf{k}})\phi^{1/2}_m(\sigma)=\\
&\sum_{m_l} \frac{(-1)^{l-m_l}}{\sqrt{2l+1}} Y^{l}_{m_l} (\hat {\mathbf{r}})Y^{l}_{-m_l} (\hat {\mathbf{k}})\phi^{1/2}_m(\sigma),
 \end{split}
\end{equation}
and
 \begin{equation}\label{eq31}
 Y^{l}_{m_l} (\hat {\mathbf{r}})\phi^{1/2}_m(\sigma)=\sum_j \langle l\;m_l\;1/2\;m|j\;m_l+m\rangle \left[ Y^{l} (\hat {\mathbf{r}})\phi^{1/2}(\sigma)\right]^j_{m_l+m},
\end{equation}
we can write
 \begin{equation}\label{eq32}
 \begin{split}
\left[ Y^{l} (\hat {\mathbf{r}}) \right. & \left. Y^{l} (\hat {\mathbf{k}})\right]^0_0\phi^{1/2}_m(\sigma)=\sum_{m_l,j} \frac{(-1)^{l+m_l}}{\sqrt{2l+1}} \langle l\;m_l\;1/2\;m|j\;m_l+m\rangle \\
&\times \left[ Y^{l} (\hat {\mathbf{r}})\phi^{1/2}(\sigma)\right]^j_{m_l+m}Y^{l}_{-m_l} (\hat {\mathbf{k}}),
 \end{split}
\end{equation}
and the distorted waves in \ref{eq28} are
 \begin{equation}\label{eq33}
\begin{split} 
\chi^{(+)}_{m_a}(\mathbf{r}_{aA},&\mathbf{k}_{a},\sigma_a)= \sum_{l_a,m_{l_a},j_a}\frac{4\pi}{k_a r_{aA}} i^{l_a}(-1)^{l_a+m_{l_a}}
e^{i\sigma^{l_a}} F_{l_a,j_a}(r_{aA})\\
 &\times\langle l_a\;m_{l_a}\;1/2\;m_a|j_a\;m_{l_a}+m_a\rangle
 \left[ Y^{l_a} (\hat {\mathbf{r}}_{aA})\phi^{1/2}(\sigma_a)\right]^{j_a}_{m_{l_a}+m_a}Y^{l_a}_{-m_{l_a}} (\hat {\mathbf{k}}_a),
\end{split} 
\end{equation}
 \begin{equation}\label{eq34}
\begin{split} 
\chi^{(-)*}_{m'_b}(\mathbf{r}_{bc},&\mathbf{k}'_{b},\sigma_b)= \sum_{l'_b,m_{l'_b},j'_b}\frac{4\pi}{k'_b r_{bc}} i^{-l'_b}(-1)^{l'_b+m_{l'_b}}
e^{i\sigma^{l'_b}} F_{l'_b,j'_b}(r_{bc})\\
 &\times\langle l'_b\;m_{l'_b}\;1/2\;m'_b|j'_b\;m_{l'_b}+m'_b\rangle
 \left[ Y^{l'_b} (\hat {\mathbf{r}}_{bc})\phi^{1/2}(\sigma_b)\right]^{j'_b*}_{m_{l'_b}+m'_b}Y^{l'_b*}_{-m_{l'_b}} (\hat {\mathbf{k}}'_b),
\end{split} 
\end{equation}
 \begin{equation}\label{eq35}
\begin{split} 
\chi^{(-)*}_{m'_a}(\mathbf{r}_{ac},&\mathbf{k}'_{a},\sigma_a)= \sum_{l'_a,m_{l'_a},j'_a}\frac{4\pi}{k'_a r_{ac}} i^{-l'_a}(-1)^{l'_a+m_{l'_a}}
e^{i\sigma^{l'_a}} F_{l'_a,j'_a}(r_{ac})\\
 &\times\langle l'_a\;m_{l'_a}\;1/2\;m'_a|j'_a\;m_{l'_a}+m'_a\rangle
 \left[ Y^{l'_a} (\hat {\mathbf{r}}_{ac})\phi^{1/2}(\sigma_a)\right]^{j'_a*}_{m_{l'_a}+m'_a}Y^{l'_a*}_{-m_{l'_a}} (\hat {\mathbf{k}}'_a).
\end{split} 
\end{equation}
The initial bounded wavefunction of particle $b$ is
 \begin{equation}\label{eq36}
\psi_{m_b}^{l_b,j_b}(\mathbf{r}_{bc},\sigma_b)=u_{l_b,j_b}(r_{bc})\left[ Y^{l_b} (\hat {\mathbf{r}}_{bc})\phi^{1/2}(\sigma_b)\right]^{j_b}_{m_b},
\end{equation}
substituting in \ref{eq28},

\begin{multline}\label{eq37}
T_{m_a,m_b}^{m'_a,m'_b}(\mathbf{k}'_a,\mathbf{k}'_b)=\frac{64\pi^3}{k_ak'_ak'_b}\sum_{\sigma_a,\sigma_b}\sum_{l_a,m_{l_a},j_a}\sum_{l'_a,m_{l'_a},j'_a}\sum_{l'_b,m_{l'_b},j'_b}
 e^{i(\sigma^{l_a}+\sigma^{l'_a}+\sigma^{l'_b})}i^{l_a-l'_a-l'_b}(-1)^{l_a-m_{l_a}+l'_a-j'_a+l'_b-j'_b}\\
\times \langle l'_a\;m_{l'_a}\;1/2\;m'_a|j'_a\;m_{l'_a}+m'_a\rangle \langle l_a\;m_{l_a}\;1/2\;m_a|j_a\;m_{l_a}+m_a\rangle\langle l'_b\;m_{l'_b}\;1/2\;m'_b|j'_b\;m_{l'_b}+m'_b\rangle\\
\times Y^{l_a}_{-m_{l_a}} (\hat {\mathbf{k}}_a)Y^{l'_b}_{-m_{l'_b}} (\hat {\mathbf{k}}'_b)Y^{l'_a}_{-m_{l'_a}} (\hat {\mathbf{k}}'_a)
\int d\mathbf{r}_{aA}d \mathbf{r}_{bc}\left[ Y^{l'_a} (\hat {\mathbf{r}}_{ac})\phi^{1/2}(\sigma_a)\right]^{j'_a}_{-m_{l'_a}-m'_a}\left[ Y^{l'_b} (\hat {\mathbf{r}}_{bc})\phi^{1/2}(\sigma_b)\right]^{j'_b}_{-m_{l'_b}-m'_b}\\
\times \frac{F_{l_a,j_a}(r_{aA})  F_{l'_a,j'_a}(r_{ac})F_{l'_b,j'_b}(r_{bc})}{r_{ac}r_{aA}r_{bc}}u_{l_b,j_b}(r_{bc})v(r_{ab})v_\sigma(\sigma_a,\sigma_b)\\
\times\left[ Y^{l_a} (\hat {\mathbf{r}}_{aA})\phi^{1/2}(\sigma_a)\right]^{j_a}_{m_{l_a}+m_a}\left[ Y^{l_b} (\hat {\mathbf{r}}_{bc})\phi^{1/2}(\sigma_b)\right]^{j_b}_{m_b},
\end{multline}
where we have used 
 \begin{equation}\label{eq59}
\left[ Y^{l} (\hat {\mathbf{r}})\phi^{1/2}(\sigma)\right]^{j*}_{m}=(-1)^{j-m}\left[Y^{l} (\hat {\mathbf{r}})\phi^{1/2}(\sigma)\right]^{j}_{-m}.
\end{equation}
\subsubsection{Recoupling of angular momenta}
Let us now separate spatial and spin coordinates, noting that the spin functions must be coupled to 0 (this is a consequence of the  interaction $v_\sigma(\sigma_a,\sigma_b)$ being rotationally invariant). Starting with particle $a$,
\begin{multline}\label{eq38}
\left[ Y^{l'_a} (\hat {\mathbf{r}}_{ac})\phi^{1/2^*}(\sigma_a)\right]^{j'_a}_{-m_{l'_a}-m'_a}\left[ Y^{l_a} (\hat {\mathbf{r}}_{aA})\phi^{1/2}(\sigma_a)\right]^{j_a}_{m_{l_a}+m_a}=\\
\sum_K \bigl((l'_a \tfrac{1}{2})_{j'_a}(l_a \tfrac{1}{2})_{j_a}|(l_a l'_a)_K(\tfrac{1}{2} \tfrac{1}{2})_0\bigr)_K\\
\times \left[ Y^{l'_a} (\hat {\mathbf{r}}_{ac})Y^{l_a} (\hat {\mathbf{r}}_{aA})\right]^{K}_{-m_{l'_a}-m'_a+m_{l_a}+m_a}\left[\phi^{1/2^*}(\sigma_a)\phi^{1/2}(\sigma_a)\right]^0_0.
\end{multline}
For particle $b$,
\begin{multline}\label{eq39}
\left[ Y^{l'_b} (\hat {\mathbf{r}}_{bc})\phi^{1/2^*}(\sigma_b)\right]^{j'_b}_{-m_{l'_b}-m'_b}\left[ Y^{l_b} (\hat {\mathbf{r}}_{bc})\phi^{1/2}(\sigma_b)\right]^{j_b}_{m_b}=\\
\sum_{K'} \bigl((l'_b \tfrac{1}{2})_{j'_b}(l_b \tfrac{1}{2})_{j_b}|(l_b l'_b)_{K'}(\tfrac{1}{2} \tfrac{1}{2})_0\bigr)_{K'}\\
\times \left[ Y^{l'_b} (\hat {\mathbf{r}}_{bc})Y^{l_b} (\hat {\mathbf{r}}_{bc})\right]^{K'}_{-m_{l'_b}-m'_b+m_b}\left[\phi^{1/2^*}(\sigma_b)\phi^{1/2}(\sigma_b)\right]^0_0.
\end{multline}
The spin summation yields a constant factor,
 \begin{equation}\label{eq40}
\sum_{\sigma_a,\sigma_b}\left[\phi^{1/2^*}(\sigma_a)\phi^{1/2}(\sigma_a)\right]^0_0\left[\phi^{1/2^*}(\sigma_b)\phi^{1/2}(\sigma_b)\right]^0_0v_\sigma(\sigma_a,\sigma_b)\equiv T_\sigma,
\end{equation}
and what we have yet to do is very similar to what we have done for spinless particles. First of all note that the necessity to couple all angular momenta to 0 imposes $K'=K$ and $m_{l_a}+m_a-m_{l'_a}-m'_a=m_{l'_b}+m'_b-m_b$ (see \ref{eq38} and \ref{eq39}). If we set $M=m_{l_a}+m_a-m_{l'_a}-m'_a$ and take, as before, $\hat {\mathbf{k}}_a\equiv \hat z$
\begin{multline}\label{eq41}
T_{m_a,m_b}^{m'_a,m'_b}(\mathbf{k}'_a,\mathbf{k}'_b)=\frac{32\pi^{5/2}}{k_ak'_ak'_b}T_\sigma\sum_{l_a,j_a}\sum_{l'_a,j'_a}\sum_{l'_b,j'_b}\sum_{K,M}
 e^{i(\sigma^{l_a}+\sigma^{l'_a}+\sigma^{l'_b})}i^{l_a-l'_a-l'_b}(-1)^{l_a+l'_a+l'_b-j'_a-j'_b}\\
 \times \sqrt{2l_a+1}\bigl((l'_a \tfrac{1}{2})_{j'_a}(l_a \tfrac{1}{2})_{j_a}|(l_a l'_a)_K(\tfrac{1}{2} \tfrac{1}{2})_0\bigr)_K\,\bigl((l'_b \tfrac{1}{2})_{j'_b}(l_b \tfrac{1}{2})_{j_b}|(l_b l'_b)_{K}(\tfrac{1}{2} \tfrac{1}{2})_0\bigr)_{K}\\
\times \langle l'_a\;m_a-m'_a-M\;1/2\;m'_a|j'_a\;m_a-M\rangle \langle l_a\;0\;1/2\;m_a|j_a\;m_a\rangle\langle l'_b\;m_b-m'_b+M\;1/2\;m'_b|j'_b\;M+m_b\rangle\\
\times Y^{l'_b}_{m'_b-m_b-M} (\hat {\mathbf{k}}'_b)Y^{l'_a}_{m'_a-m_a+M} (\hat {\mathbf{k}}'_a)
\int d\mathbf{r}_{aA}d \mathbf{r}_{bc}\frac{F_{l_a,j_a}(r_{aA})  F_{l'_a,j'_a}(r_{ac})F_{l'_b,j'_b}(r_{bc})}{r_{ac}r_{aA}r_{bc}}\\
\times u_{l_b,j_b}(r_{bc})v(r_{ab})\left[ Y^{l_a} (\hat{\mathbf r}_{aA}) Y^{l'_a} (\hat{ \mathbf r}_{ac})\right]^K_M   \left[ Y^{l_b} (\hat{\mathbf r}_{bc}) Y^{l'_b} (\hat{\mathbf r}_{bc})\right]^{K}_{-M}.
\end{multline}
The integral of the above expression is similar to the one in \ref{eq15}, so we obtain
\begin{multline}\label{eq42}
T_{m_a,m_b}^{m'_a,m'_b}(\mathbf{k}'_a,\mathbf{k}'_b)=\frac{128\pi^{4}}{k_ak'_ak'_b}T_\sigma\sum_{l_a,j_a}\sum_{l'_a,j'_a}\sum_{l'_b,j'_b}\sum_{K,M}
 e^{i(\sigma^{l_a}+\sigma^{l'_a}+\sigma^{l'_b})}i^{l_a-l'_a-l'_b}(-1)^{l_a+l'_a+l'_b-j'_a-j'_b}\\
 \times \frac{2l_a+1}{2K+1}\bigl((l'_a \tfrac{1}{2})_{j'_a}(l_a \tfrac{1}{2})_{j_a}|(l_a l'_a)_K(\tfrac{1}{2} \tfrac{1}{2})_0\bigr)_K\,\bigl((l'_b \tfrac{1}{2})_{j'_b}(l_b \tfrac{1}{2})_{j_b}|(l_b l'_b)_{K}(\tfrac{1}{2} \tfrac{1}{2})_0\bigr)_{K}\\
\times \langle l'_a\;m_a-m'_a-M\;1/2\;m'_a|j'_a\;m_a-M\rangle \langle l'_b\;m_b-m'_b+M\;1/2\;m'_b|j'_b\;M+m_b\rangle\\
\times \langle l_a\;0\;1/2\;m_a|j_a\;m_a\rangle Y^{l'_b}_{m'_b-m_b-M} (\hat {\mathbf{k}}'_b)Y^{l'_a}_{m_a-m'_a+M} (\hat {\mathbf{k}}'_a)
\mathcal I(l_a,l'_a,l'_b,j_a,j'_a,j'_b,K),
\end{multline}
with
\begin{multline}\label{eq43}
\mathcal I(l_a,l'_a,l'_b,j_a,j'_a,j'_b,K)=\int dr_{aA} dr_{bc}d\theta r_{aA}r_{bc} \frac{\sin \theta}{r_{ac}} u_{l_b}(r_{bc})v(r_{ab})\\
\times F_{l_a,j_a}(r_{aA})  F_{l'_a,j'_a}(r_{ac})F_{l'_b,j'_b}(r_{bc}) \\
\times \sum_{M_K} \langle l_a\;0\;l'_a\;M_K|K\;M_K\rangle \left[ Y^{l_b} (\cos \theta,0) Y^{l'_b} (\cos \theta,0)\right]^{K}_{-M_K} Y^{l'_a}_{M_K} (\cos \theta_{ac},0).
\end{multline}
Again, this is a 3--dimensional integral that can be evaluated with the method of Gaussian quadratures. The transition amplitude $T_{m_a,m_b}^{m'_a,m'_b}(\mathbf{k}'_a,\mathbf{k}'_b)$ depends explicitly on the initial ($m_a,m'_a$) and final ($m'_a,m'_b$) polarizations of $a,b$. If the particle $b$ is initially coupled to core $c$ to total angular momentum $J_A,M_A$, the amplitude to be considered is rather
\begin{equation}\label{eq45}
T_{m_a}^{m'_a,m'_b}(\mathbf{k}'_a,\mathbf{k}'_b)=\sum_{m_b}\langle j_b\;m_b\;j_c\;M_A-m_b|J_A\;M_A\rangle\, T_{m_a,m_b}^{m'_a,m'_b}(\mathbf{k}'_a,\mathbf{k}'_b),
\end{equation}
and the multi--differential cross section for detecting particle $c$ (or $a$) is
\begin{equation}\label{eq46}
\left.\frac{d\sigma}{d\mathbf{k}'_ad\mathbf{k}'_b}\right]_{m_a}^{m'_a,m'_b}=\frac{k'_a}{k_a}\frac{\mu_{aA}\mu_{ac}}{4\pi^2\hbar^4}\left|\sum_{m_b}\langle j_b\;m_b\;j_c\;M_A-m_b|J_A\;M_A\rangle\, T_{m_a,m_b}^{m'_a,m'_b}(\mathbf{k}'_a,\mathbf{k}'_b)\right|^2.
\end{equation}
All spin--polarization observables (analysing powers, etc.,) can be derived from this expression. But let us now work out the expression of the cross section for an unpolarized beam (sum over initial spin orientations divided by the number of such orientations) and when we do not detect the final polarizations (sum over final spin orientations), 
\begin{equation}\label{eq47}
\begin{split}
\frac{d\sigma}{d\mathbf{k}'_ad\mathbf{k}'_b}&=\frac{k'_a}{k_a}\frac{\mu_{aA}\mu_{ac}}{4\pi^2\hbar^4}\frac{1}{(2J_A+1)(2j_a+1)}\\
&\times \sum_{\substack{m_a,m'_a\\M_A,m'_b}}\left|\sum_{m_b}\langle j_b\;m_b\;j_c\;M_A-m_b|J_A\;M_A\rangle\, T_{m_a,m_b}^{m'_a,m'_b}(\mathbf{k}'_a,\mathbf{k}'_b)\right|^2.
\end{split}
\end{equation}
The sum above can be simplified a bit. Let us consider a single particular value of $m_b$ in the sum over $m_b$,
\begin{equation}\label{eq48}
\begin{split}
\sum_{m_a,m'_a,m'_b}&\left|T_{m_a,m_b}^{m'_a,m'_b}(\mathbf{k}'_a,\mathbf{k}'_b)\right|^2\sum_{M_A}\Big|\langle j_b\;m_b\;j_c\;M_A-m_b|J_A\;M_A\rangle\Big|^2=\\
&\frac{2J_A+1}{2j_b+1}\sum_{m_a,m_a,m'_b}\left|T_{m_a,m_b}^{m'_a,m'_b}(\mathbf{k}'_a,\mathbf{k}'_b)\right|^2\\
&\times\sum_{M_A}\Big|\langle J_A\;-M_A\;j_c\;M_A-m_b|j_b\;m_b\rangle\Big|^2,
\end{split}
\end{equation}
where we have used
\begin{equation}\label{eq49}
\langle j_b\;m_b\;j_c\;M_A-m_b|J_A\;M_A\rangle=(-1)^{j_c-M_A+m_b}\sqrt{\frac{2J_A+1}{2j_b+1}}\langle J_A\;-M_A\;j_c\;M_A-m_b|j_b\;m_b\rangle.
\end{equation}
As
\begin{equation}\label{eq50}
\sum_{M_A}\Big|\langle J_A\;-M_A\;j_c\;M_A-m_b|j_b\;m_b\rangle\Big|^2=1,
\end{equation}
we finally have
\begin{equation}\label{eq51}
\begin{split}
\frac{d\sigma}{d\mathbf{k}'_ad\mathbf{k}'_b}&=\frac{k'_a}{k_a}\frac{\mu_{aA}\mu_{ac}}{4\pi^2\hbar^4}\frac{1}{(2j_b+1)(2j_a+1)}\sum_{\substack{m_a,m'_a,m'_b}}\left|\sum_{m_b} T_{m_a,m_b}^{m'_a,m'_b}(\mathbf{k}'_a,\mathbf{k}'_b)\right|^2.
\end{split}
\end{equation}
\subsubsection{Zero range approximation.}
The zero range approximation consists in taking $v(r_{ab})=D_0\delta(r_{ab})$. Then, (see \ref{eq18})
\begin{equation}\label{eq52}
\begin{split}
\mathbf{r}_{aA}&=\frac{c}{A}\mathbf{r}_{bc},\\
\mathbf{r}_{ac}&=\mathbf{r}_{bc}.
\end{split} 
\end{equation}
The angular dependence of the integral can be readily evaluated. From \ref{eq17}, noting that $\hat{\mathbf r}_{aA}=\hat{\mathbf r}_{ac}=\hat{\mathbf r}_{bc}\equiv \hat{\mathbf r}$,
\begin{equation}\label{eq53}
\begin{split}
\left[ Y^{l_a} (\hat{\mathbf r}) Y^{l'_a} (\hat{ \mathbf r})\right]^K_M &   \left[ Y^{l_b} (\hat{\mathbf r}) Y^{l'_b} (\hat{\mathbf r})\right]^{K}_{-M}=\\
&\frac{(-1)^{K-M}}{\sqrt{2K+1}}\left\{\left[ Y^{l_a} (\hat{\mathbf r}) Y^{l'_a} (\hat{ \mathbf r})\right]^K\left[ Y^{l_b} (\hat{\mathbf r}) Y^{l'_b} (\hat{\mathbf r})\right]^{K} \right\}^0_0.
\end{split}
\end{equation}
We can as before evaluate this expression in the configuration shown in Fig. \ref{fig2} ($\hat{\mathbf r}=\hat z$), but now the multiplicative factor is $4\pi$. The corresponding contribution to the integral is
\begin{equation}\label{eq54}
\frac{(-1)^K}{4\pi(2K+1)}\langle l_a\;0\;l'_a\;0|K\;0\rangle\sqrt{(2l_a+1)(2l'_a+1)(2l_b+1)(2l'_b+1)},
\end{equation}
and
\begin{multline}\label{eq55}
T_{m_a,m_b}^{m'_a,m'_b}(\mathbf{k}'_a,\mathbf{k}'_b)=\frac{16\pi^{2}}{k_ak'_ak'_b}\frac{c}{A}D_0T_\sigma\sum_{l_a,j_a}\sum_{l'_a,j'_a}\sum_{l'_b,j'_b}\sum_{K,M}
 e^{i(\sigma^{l_a}+\sigma^{l'_a}+\sigma^{l'_b})}i^{l_a-l'_a-l'_b}(-1)^{l_a+l'_a+l'_b-j'_a-j'_b}\\
 \times\sqrt{(2l_a+1)(2l'_a+1)(2l_b+1)(2l'_b+1)}\,\langle l_a\;0\;l'_a\;0|K\;0\rangle\\
 \times \frac{2l_a+1}{2K+1}\bigl((l'_a \tfrac{1}{2})_{j'_a}(l_a \tfrac{1}{2})_{j_a}|(l_a l'_a)_K(\tfrac{1}{2} \tfrac{1}{2})_0\bigr)_K\,\bigl((l'_b \tfrac{1}{2})_{j'_b}(l_b \tfrac{1}{2})_{j_b}|(l_b l'_b)_{K}(\tfrac{1}{2} \tfrac{1}{2})_0\bigr)_{K}\\
\times \langle l'_a\;m_a-m'_a-M\;1/2\;m'_a|j'_a\;m_a-M\rangle \langle l'_b\;m_b-m'_b+M\;1/2\;m'_b|j'_b\;M+m_b\rangle\\
\times \langle l\;0\;1/2\;m_a|j\;m_a\rangle Y^{l'_b}_{M+m_b+m'_b} (\hat {\mathbf{k}}'_b)Y^{l'_a}_{m_a+m'_a-M} (\hat {\mathbf{k}}'_a)
\mathcal I_{ZR}(l_a,l'_a,l'_b,j_a,j'_a,j'_b),
\end{multline}
where now the 1--dimensional integral to solve is
\begin{equation}\label{eq56}
\mathcal I_{ZR}(l_a,l'_a,l'_b,j_a,j'_a,j'_b)=\int dr u_{l_b}(r)F_{l_a,j_a}(\tfrac{c}{A}r)  F_{l'_a,j'_a}(r)F_{l'_b,j'_b}(r)/r.
\end{equation}
 \begin{figure}
\centerline{\includegraphics*[width=10cm,angle=0]{figs_C6/onept.pdf}}
\vspace{-1cm}
\caption{One particle transfer reaction $A(=c+b)+a\rightarrow B(=a+b)+c$.}\label{fig4}
\end{figure}
\subsection{One particle transfer}
It may be interesting to state the expression for the one particle transfer reaction within the same context and using the same elements, in order to better compare these two type of experiments. In particle transfer, the final state of $b$ is a bounded state of the $B(=a+b)$ nucleus, and we can carry on in a similar way as done previously just by substituting the distorted wave (continuum) wave function \ref{eq34} with
 \begin{equation}\label{eq57}
\psi_{m'_b}^{l'_b,j'_b*}(\mathbf{r}_{ab},\sigma_b)=u^*_{l'_b,j'_b}(r_{ab})\left[ Y^{l'_b} (\hat {\mathbf{r}}_{ab})\phi^{1/2}(\sigma_b)\right]^{j'_b*}_{m'_b},
\end{equation}
so the transition amplitude is now
\begin{multline}\label{eq58}
T_{m_a,m_b}^{m'_a,m'_b}(\mathbf{k}'_a)=\frac{8\pi^{3/2}}{k_ak'_a}\sum_{\sigma_a,\sigma_b}\sum_{l_a,j_a}\sum_{l'_a,m_{l'_a},j'_a}
 e^{i(\sigma^{l_a}+\sigma^{l'_a})}i^{l_a-l'_a}(-1)^{l_a+l'_a-j'_a-j'_b}\\
\times \sqrt{2l_a+1}\,\langle l'_a\;m_{l'_a}\;1/2\;m'_a|j'_a\;m_{l'_a}+m'_a\rangle \langle l_a\;0\;1/2\;m_a|j_a\;m_a\rangle\\
\times Y^{l'_a}_{-m_{l'_a}} (\hat {\mathbf{k}}'_a)
\int d\mathbf{r}_{aA}d \mathbf{r}_{bc}\left[ Y^{l'_a} (\hat {\mathbf{r}}_{Bc})\phi^{1/2}(\sigma_a)\right]^{j'_a}_{-m_{l'_a}-m'_a}\left[ Y^{l'_b} (\hat {\mathbf{r}}_{ab})\phi^{1/2}(\sigma_b)\right]^{j'_b}_{-m'_b}\\
\times \frac{F_{l_a,j_a}(r_{aA})  F_{l'_a,j'_a}(r_{Bc})}{r_{Bc}r_{aA}}u^*_{l'_b,j'_b}(r_{ab})u_{l_b,j_b}(r_{bc})v(r_{ab})v_\sigma(\sigma_a,\sigma_b)\\
\times\left[ Y^{l_a} (\hat {\mathbf{r}}_{aA})\phi^{1/2}(\sigma_a)\right]^{j_a}_{m_a}\left[ Y^{l_b} (\hat {\mathbf{r}}_{bc})\phi^{1/2}(\sigma_b)\right]^{j_b}_{m_b}.
\end{multline}
Using \ref{eq38}, \ref{eq39}, \ref{eq40}, and setting $M=m_a-m'_a-m_{l'_a}$
\begin{multline}\label{eq60}
T_{m_a,m_b}^{m'_a,m'_b}(\mathbf{k}'_a)=\frac{8\pi^{3/2}}{k_ak'_a}T_{\sigma}\sum_{l_a,j_a}\sum_{l'_a,j'_a}\sum_{K,M}
 e^{i(\sigma^{l_a}+\sigma^{l'_a})}i^{l_a-l'_a}(-1)^{l_a+l'_a-j'_a-j'_b}\\
 \times\bigl((l'_a \tfrac{1}{2})_{j'_a}(l_a \tfrac{1}{2})_{j_a}|(l_a l'_a)_K(\tfrac{1}{2} \tfrac{1}{2})_0\bigr)_K\,\bigl((l'_b \tfrac{1}{2})_{j'_b}(l_b \tfrac{1}{2})_{j_b}|(l_b l'_b)_{K}(\tfrac{1}{2} \tfrac{1}{2})_0\bigr)_{K}\\
\times \sqrt{2l_a+1}\,\langle l'_a\;m_a-m'_a-M\;1/2\;m'_a|j'_a\;m_a-M\rangle \langle l_a\;0\;1/2\;m_a|j_a\;m_a\rangle\\
\times Y^{l'_a}_{m_a-m'_a-M} (\hat {\mathbf{k}}'_a)
\int d\mathbf{r}_{aA}d \mathbf{r}_{bc}\frac{F_{l_a,j_a}(r_{aA})  F_{l'_a,j'_a}(r_{Bc})}{r_{Bc}r_{aA}}u^*_{l'_b,j'_b}(r_{ab})u_{l_b,j_b}(r_{bc})v(r_{ab})\\
\times\left[ Y^{l_a} (\hat{\mathbf r}_{aA}) Y^{l'_a} (\hat{ \mathbf r}_{Bc})\right]^K_{M}   \left[ Y^{l_b} (\hat{\mathbf r}_{bc}) Y^{l'_b} (\hat{\mathbf r}_{ab})\right]^{K}_{-M}.
\end{multline}
Aside from \ref{eq18}, we also need 
\begin{equation}\label{eq63}
\mathbf{r}_{Bc}=\frac{a+B}{B}\mathbf{r}_{aA}+\frac{b}{A}\mathbf{r}_{bc}.
\end{equation}
From \ref{eq17}, \ref{eq22}, \ref{eq21}, \ref{eq23}, \ref{eq24}, we get
\begin{multline}\label{eq61}
T_{m_a,m_b}^{m'_a,m'_b}(\mathbf{k}'_a)=\frac{32\pi^{3}}{k_ak'_a}T_{\sigma}\sum_{l_a,j_a}\sum_{l'_a,j'_a}\sum_{K,M}
 e^{i(\sigma^{l_a}+\sigma^{l'_a})}i^{l_a-l'_a}(-1)^{l_a+l'_a-j'_a-j'_b}\\
 \times\bigl((l'_a \tfrac{1}{2})_{j'_a}(l_a \tfrac{1}{2})_{j_a}|(l_a l'_a)_K(\tfrac{1}{2} \tfrac{1}{2})_0\bigr)_K\,\bigl((l'_b \tfrac{1}{2})_{j'_b}(l_b \tfrac{1}{2})_{j_b}|(l_b l'_b)_{K}(\tfrac{1}{2} \tfrac{1}{2})_0\bigr)_{K}\\
\times \frac{2l_a+1}{2K+1}\,\langle l'_a\;m_a-m'_a-M\;1/2\;m'_a|j'_a\;m_a-M\rangle\\ \times\langle l_a\;0\;1/2\;m_a|j_a\;m_a\rangle
 Y^{l'_a}_{m_a-m'_a-M} (\hat {\mathbf{k}}'_a)
\mathcal I(l_a,l'_a,j_a,j'_a,j'_b,K),
\end{multline}
with
\begin{multline}\label{eq62}
I(l_a,l'_a,j_a,j'_a,K)=\int dr_{aA} dr_{bc}d\theta r_{aA}r^2_{bc} \frac{\sin \theta}{r_{Bc}}\\
\times F_{l_a,j_a}(r_{aA})  F_{l'_a,j'_a}(r_{ac})u^*_{l'_b,j'_b}(r_{ab}) u_{l_b,j_b}(r_{bc})v(r_{ab}) \\
\times \sum_{M_K} \langle l_a\;0\;l'_a\;M_K|K\;M_K\rangle \left[ Y^{l_b} (\cos \theta,0) Y^{l'_b} (\cos \theta_{ab},0)\right]^{K}_{-M_K} Y^{l'_a}_{M_K} (\cos \theta_{Bc},0),
\end{multline}
where (see \ref{eq18}, \ref{eq63} and Fig. \ref{fig2})

\begin{equation}\label{eq64}
\cos \theta_{ab}=\frac{-r_{aA}-\frac{c}{A}r_{bc}\cos \theta}{\sqrt{\left(\frac{c}{A}r_{bc}\sin \theta\right)^2+\left(r_{aA}+\frac{c}{A}r_{bc}\cos \theta\right)^2}},
\end{equation}
\begin{equation}\label{eq65}
\cos \theta_{Bc}=\frac{\frac{a+B}{B}r_{aA}+\frac{b}{A}r_{bc}\cos \theta}{\sqrt{\left(\frac{b}{A}r_{bc}\sin \theta\right)^2+\left(\frac{a+B}{B}r_{aA}+\frac{b}{A}r_{bc}\cos \theta\right)^2}},
\end{equation}
and
\begin{equation}\label{eq66}
r_{Bc}=\sqrt{\left(\frac{b}{A}r_{bc}\sin \theta\right)^2+\left(\frac{a+B}{B}r_{aA}+\frac{b}{A}r_{bc}\cos \theta\right)^2}.
\end{equation}
Again, this is nothing new as many codes exist which deal with one particle transfer within the same DWBA formalism we have used here, but it may be useful to have our own code to better compare transfer and knock--out experiments. By the way, \ref{eq61} can also be used when particle $b$ populates a resonant state in the continuum of nucleus $B$.  
\end{subappendices}

 \begin{figure}
\centerline{\includegraphics*[width=10cm,angle=0]{figs_C6/fig6_3}}
\caption{(I) Single-particle neutron resonances in $^{10}$Li. In (a)
the position of the levels $s_{1/2}$ and $p_{1/2}$ calculated making use
of mean-field theory is shown (hatched area and thin horizontal
line, respectively). The coupling of a single-neutron (upward
pointing arrowed line) to a vibration (wavy line) calculated
making use of the Feynman diagrams displayed in (b)
(schematically depicted also in terms of either solid dots (neutron)
or open circles (neutron hole) moving in a single-particle
level around or in the $^{9}$Li core (hatched area)), leads to conspicuous
shifts in the energy centroid of the $s_{1/2}$ and $p_{1/2}$ resonances
(shown by thick horizontal lines) and eventually to
an inversion in their sequence. In (c) we show the calculated
partial cross-section σl for neutron elastic scattering off $^9$Li.
(II) The two-neutron system  $^{11}$Li. We show in (a) the meanfield
picture of $^{11}$Li, where two neutrons (solid dots) move in
time-reversal states around the core $^9$Li (hatched area) in the
$s_{1/2}$ resonance leading to an unbound $s^2_{1/2}(0)$ state where the
two neutrons are coupled to zero angular momentum. The exchange
of vibrations between the two neutrons shown in the upper
part of the figure leads to a density-dependent interaction
which, added to the nucleon-nucleon interaction, correlates the
two-neutron system leading to a bound state $|0^+\rangle$, where the
two neutrons move with probability 0.40, 0.58 and 0.02 in the
two-particle configurations $s^2_{1/2}(0)$, $p^2_{1/2}(0)$  and $d^2_{5/2}(0)$, respectively.}\label{fig6_3}
\end{figure}
 \begin{figure}
\centerline{\includegraphics*[width=10cm,angle=0]{figs_C6/fig6_4}}
\caption{(a)Self--energy (see boxed process) and (b) vertex (pairing induced interaction boxed process) renormalization process, both associate with (c) a (two--particle)--(quadrupole vibration) intermediate (virtual state) which can be forced to became real in a $(p,t)$ reaction $^1$H($^{11}$Li,$^9$Li)$^3$H exciting the first excited state $|$12.69MeV;$1/2^-\rangle$ of $^9$Li (see Ch.\ref{dd}.)}\label{fig6_4}
\end{figure}
 \begin{figure}
\centerline{\includegraphics*[width=10cm,angle=0]{figs_C6/fig6_A1}}
\caption{(a)Scattering of two nucleons through the bare $NN$ interaction $v|\mathbf{r}-\mathbf{r}'|$, (b) contribution to the direct ($U$, Hartree) and (c) to the exchange ($U_x$, Fock) potential, resulting in (d) the (static) self consistent relation between potential and density, which (e) uncouples occupied ($\varepsilon_\nu\leq\varepsilon_F$) from empty states ($\varepsilon_\nu>\varepsilon_F$), (f) multiple scattering of two nucleons lead, through processes like the one depicted in (g), eventually propagated to all orders, to: (h) softening of the discontinuity of the occupancy of levels at $\varepsilon_F$, as well as to: (i) generalization of the static selfconsistency into a dynamic relation encompassing also collective vibrations (Time--dependent HF solutions of the nuclear Hamiltonian, conserving energy weighted...).}\label{fig6_A1}
\end{figure}
 \begin{figure}
\centerline{\includegraphics*[width=10cm,angle=0]{figs_C6/fig6_B1}}
\caption{Two state schematic model describing the breaking of the strength of the pure single--particle state $|a\rangle$, through the coupling to collective vibrations (wavy line) associated with polarization (PO) and correlation (CO) processes.}\label{fig6_B1}
\end{figure}
 \begin{figure}
\centerline{\includegraphics*[width=10cm,angle=0]{figs_C6/fig6_C1}}
\caption{The result of the probing with an external field (dotted line started with a cross) of he properties (mass, single--particle energy, etc) of a fermion (e.g. an electron or a nucleon, arrowed line) dressed through the coupling of (quasi) bosons (photons or collective vibrations, wavy line), corresponds to the modulus squared of the sum of the amplitudes associated with each of the four diagrams (a)--(d) (cf. Feynman, Theory of fundamental processes).}\label{fig6_C1}
\end{figure}
 \begin{figure}
\centerline{\includegraphics*[width=10cm,angle=0]{figs_C6/fig6_C2}}
\caption{These are triple--interaction vertex diagrams in which none of the incoming lines can be detached from either of the other two by cutting one line. Migdal's (1958) theorem states that, for phonons and electrons (Bardeen--Pines--Fr\"{o}lich mechanism to break gauge invariance), vertex corrections can be neglected, but usually they are not negligible, in any case not in nuclei (cancellation) (cf. e.g. P:W: Anderson, Basic notions of condensed matter physics). The solid circle in (c) represents the effective, renormalized vertex. }\label{fig6_C2}
\end{figure}
 \begin{figure}
\centerline{\includegraphics*[width=10cm,angle=0]{figs_C6/fig6_C3}}
\caption{Schematic representation of the processes associated with the Lamb shift.}\label{fig6_C3}
\end{figure}


\bibliographystyle{abbrvnat}
\bibliography{C:/Gregory/Broglia/notas_ricardo/nuclear_bib}










\end{document} 