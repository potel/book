\subsection{The structure of ``observable'' Cooper pairs}
In his Waynflete lectures on Cause and Chance, Max Born\footnote{\cite{Born:54,Pais:86}} states ``\dots quantum mechanics does not describe an objective state in an independent external world, but the aspect of this world gained by considering it from a certain subjective standpoint, or with certain experimental means and arrangements''\footnote{\cite{Born:48}}. It is within this context that we tried in previous sections to get insight concerning the structure of nuclear Cooper pairs. Specifically, in terms of two--nucleon transfer reactions. Being even more \textit{subjective} (concrete), we were interested in shedding light on the structure of one of the 5-6 Cooper pairs participating in the condensate (intrinsic state in gauge space) of the Sn--isotopes (ground state rotational band\footnote{\cite{Potel:13b,Potel:17}}) through pair transfer processes. That is $^{A+2}$Sn$(p,t)^{A}$Sn(gs) processes in general, and $^{120}$Sn$(p,t)^{118}$Sn in particular. From a strict observational perspective, concerning Cooper pairs, one can only refer to the information two--nucleon transfer absolute differential cross sections carry on these entities. On the other hand, leaving the discussion regarding the microscopic calculation of the optical potential, the carriers mediating information between structure and differential cross sections, e.g. between target and outgoing particle in a standard laboratory setup, are the distorted waves. These functions can be studied independently of the transfer processes under consideration, in elastic scattering experiments. Consequently, the non--local, correlated formfactors,
\begin{align}\label{eq6.6.1}
F(\mathbf r_1,\mathbf r_2,\mathbf r_{Ap})=F_{succ}+F_{sim}+F_{NO},
\end{align}
sum of the successive and simultaneous transfer processes and of the non--orthogonality correction, calculated with different stes of two--nucleon spectroscopic amplitudes can be compared at profit to each other. This is in keeping with the fact that they can be related, in an homogeneous fashion, with the absolute cross sections or, better, with the square root of these quantities. Examples of (\ref{eq6.6.1}) for the reaction $^{120}$Sn$(p,t)^{118}$Sn(gs) calculated making use of $B$--coefficients associated with BCS, HFB and NFT+(NG) theoretical descriptions are displayed in Fig. \ref{fig6.6.5}, for a representative value of the incident proton relative coordinate $\mathbf r_{pA}$. A quantitative assessment of the differences between the predictions of the different models can be made with the help of the root mean square deviation (RMSD) taken as reference the NFT+(NG) results (see table \ref{tab6.6.2}). The differences, normalized with respect to the mean square radius $\langle r^2\rangle^{1/2}\approx\sqrt{\frac{3}{5}}R_0/;(\approx 4.6$ fm) amount to few permil (\textperthousand)   (percent (\%) Table \ref{tab6.6.3}). As expected much smaller than those related to $\alpha_0$ (Table \ref{tab6.6.4}) and closely connected with the estimates provided by the square root differences of the absolute cross sections (Table \ref{tab6.4.5}).

\begin{table}
\begin{center}
\begin{tabular}{|c|c|c|c|}
\hline
    & $\sqrt{\sigma}(\mu\text{b})^{1/2}$  & $\left|\sqrt{\sigma}-\sqrt{\sigma_i}\right|\times10^{-2}$ fm & $\frac{\left|\sqrt{\sigma}-\sqrt{\sigma_i}\right|}{\langle r^2\rangle^{1/2}}$ (\%) \\ 
 \hline 
 exp&47.43&--&--\\
 \hline
 BCS&49.55&2.12&0.46\\
 \hline
 HFB&44.22&3.21&0.70\\
 \hline
 NFT+(NG)&45.79&1.64&0.36\\
 \hline
\end{tabular}
\end{center}
\caption{}\label{tab6.6.2}
\end{table} 

\begin{table}
\begin{center}
\begin{tabular}{|c|c|c|c|c|}
\hline
   i& $\sigma (\mu\text{b})$& $\left|{\sigma_{exp}}-{\sigma_i}\right|(\mu\text{b})$ & $\sqrt{\left|{\sigma_{exp}}-{\sigma_i}\right|}$ fm& $\frac{\left|\sqrt{\sigma}-\sqrt{\sigma_i}\right|}{\langle r^2\rangle^{1/2}}$ (\%) \\ 
 \hline 
 exp&2250&--&--&--\\
 \hline
 BCS&2455&205&0.14&3\\
 \hline
 HFB&1955&295&0.17&4\\
 \hline
 NFT+(NG)&2097&153&0.12&3\\
 \hline
\end{tabular}
\end{center}
\caption{}\label{tab6.6.3}
\end{table} 

\begin{table}
\begin{center}
\begin{tabular}{|c|c|c|}
\hline
i&$\left[(\alpha_0)_i\pm(\Delta(\alpha_0^2)_i)^{1/2}\right]$ & $(\Delta(\alpha_0^2)_i)^{1/2}/(\alpha_0)_i$\\
\hline
BCS&5.74$\pm 1.98$& 34\\
\hline
HFB&5.09$\pm 1.77$ &35\\
\hline
NFT+(NG)&5.21$\pm 1.38$& 26\\
\hline
\end{tabular}
\end{center}
\caption{}\label{tab6.6.4}
\end{table} 
