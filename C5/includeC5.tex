
\chapter{Fourth Lecture}\label{C5}
In this lecture we discuss very briefly some applications of the DW method, in the most simple and straightforward way, ignoring all the complications associated with the spin carried by the particles, the spin orbit dependence of the optical model potential $\bar{U}(r_\beta)$ etc. In following lectures we take again the problem in more detail.
\section{Inelastic $\alpha$-scattering}
We start assuming that the interaction $V'_\beta$ is equal to $V'_\beta=V'_\beta(\xi_\beta,r_\beta)$, which is usually called the stripping approximation.

We can then write eq. (\ref{eq31}) in the DW approximation as
\begin{equation}\label{eq4l1}
 \frac{d\sigma}{d\Omega}=\frac{k_\beta}{k_\alpha}\frac{\mu_\alpha \mu_\beta}{(2 \pi \hbar^2)}\vert\langle
\psi_\beta(\xi_\beta)\chi^{(-)}(k_\beta,\vec{r}_\beta),V'_\beta(\xi_\beta,r_\beta) \psi_\alpha(\xi_\alpha)\chi^{(+)}(k_\alpha,\vec{r}_\alpha)\rangle\vert^2.
\end{equation}
For the case of inelastic scattering $\xi_\alpha=\xi_\beta=\xi$, thus
\begin{subequations}
\begin{align}\label{eq4l2}
\psi_\beta(\xi_\beta)=\psi_{M_{I\beta}}^{I_\beta}(\xi)\\
\psi_\alpha(\xi_\alpha)=\psi_{M_{I\alpha}}^{I_\alpha}(\xi)\\
\vec{r}_\alpha=\vec{r}_\beta\;\;\mu_\alpha =\mu_\beta,
\end{align}
\end{subequations}
i.e we are always in the entrance channel.

Equation (\ref{eq4l1}) can now be rewritten as
\begin{equation}\label{eq4l3}
 \frac{d\sigma}{d\Omega}=\frac{k_\beta}{k_\alpha}\frac{m_\alpha^2}{(2 \pi \hbar^2)^2}\frac{1}{2 I_\alpha+1}\sum_{M_\alpha M_\beta}
\vert\langle
\chi^{(-)}(k_\beta,\vec{r}_\beta),V_{eff}(\vec{r}) \chi^{(+)}(k_\alpha,\vec{r}_\alpha)\rangle\vert^2,
\end{equation}
where
\begin{equation}\label{eq4l4}
\begin{split}
 V_{eff}=\int d\xi \,\psi_{M_{I\beta}}^{I_\beta^*}(\xi) V'_\beta(\xi,\vec{r}) \psi_{M_{I\alpha}}^{I_\alpha}(\xi)\\
= \int d\xi\, \psi_{M_{I\beta}}^{I_\beta^*}(\xi) V_\beta(\xi,\vec{r}) \psi_{M_{I\alpha}}^{I_\alpha}(\xi)
\end{split}
\end{equation}
 as $\psi^{I_\beta}$ and $\psi^{I_\alpha}$ are orthogonal (remember $V'_\beta=V_\beta-\bar{U}(r)$).
We expand the interaction in spherical harmonics, i.e.
\begin{equation}\label{eq4l5}
\begin{split}
V_\beta(\xi,\vec{r}) =\sum_{LM} V_M^L(\xi,r) Y_M^L(\hat{r})\\
= \sum_{LM} V_M^L(\xi,\vec{r}).
\end{split}
\end{equation}
Defining
\begin{equation}\label{eq4l6}
 \int d\xi \,\psi_{M_{I\beta}}^{I_\beta^*}(\xi) [V_M^L(\xi,r) \psi^{I^\alpha}(\xi)]_{M_{I\beta}}^{I_\beta}
= F_L(r),
\end{equation}
we can write eq.(\ref{eq4l4}) as
\begin{equation}\label{eq4l7}
V_{eff}(\vec{r})= \sum_{LM} (LMI_\alpha M_\alpha \vert I_\beta M_\beta) F_L(r) Y_M^L(\hat{r}).
\end{equation}
Inserting (\ref{eq4l7}) into (\ref{eq4l3}) we obtain



\begin{equation}\label{eq4l8}
\begin{split}
\frac{d\sigma}{d\Omega}&=\frac{k_\beta}{k_\alpha}\frac{m_\alpha^2}{(2 \pi \hbar^2)^2}\frac{1}{2 I_\alpha+1}
\sum_{M_\alpha M_\beta}
\left\vert
\sum_{LM} (LMI_\alpha M_\alpha \vert I_\beta M_\beta) \times\right.\\
&\left.\int d\vec r \chi^{(-)*}(k_\beta,\vec{r}_\beta)
F_L(r) Y_M^{L*}(\hat{r})\chi^{(+)}(k_\beta,\vec{r}_\beta)\right\vert^2=\\
&\frac{k_\beta}{k_\alpha}\frac{m_\alpha^2}{(2 \pi \hbar^2)^2}\frac{2 I_\beta+1}{2 I_\alpha+1}\times\\
&\sum_{LM}\frac{1}{2 L+1} \left\vert \int d\vec r \chi^{(-)*}(k_\beta,\vec{r}_\beta)
F_L(r) Y_M^{L*}(\hat{r})\chi^{(+)}(k_\beta,\vec{r}_\beta)\right\vert^2,
\end{split}
\end{equation}
where we have used he orthogonality relation between Clebsch-Gordan coefficients

\begin{equation}\label{eq4l10}
\begin{split}
\sum_{M_\alpha M_\beta}&(LMI_\alpha M_\alpha \vert I_\beta M_\beta)
 (L' M I_\alpha M_\alpha \vert I_\beta M_\beta)=\\
& \sqrt{\frac{(2 I_\beta+1)^2}{(2L+1)(2L'+1)}} \sum_{M_\alpha M_\beta}
(I_\beta -M_\beta  I_\alpha M_\alpha \vert L -M)\times\\
&(I_\beta -M_\beta  I_\alpha M_\alpha \vert L' -M)=\frac{2 I_\beta+1}{2 L+1} \delta_{L L'}\\
\text{(fixed $M$)}
\end{split}
\end{equation}


Let us now discuss the case of inelastic scattering of even spherical nuclei.


The macroscopic Hamiltonian describing the dynamics of the multipole surface vibrations in such nuclei can be written, in the harmonic approximation as

\begin{equation}\label{eq4l11}
\mathrm{H}=\sum_{L,M} \left\{ \frac{B_L}{2} \vert \dot{\alpha}_M^L \vert^2+
\frac{C_L}{2} \vert \alpha_M^L \vert^2 \right\},
\end{equation}
where the collective coordinate $\alpha_M^L$ is defined through the equation of the radius

\begin{equation}\label{eq4l12}
R(\hat r)= R_0 \left[1+\sum_{L,M} \alpha_M^L Y_M^{L*}(\hat r) \right],
\end{equation}
and where $R_0=r_0 A^{1/3}$ fm.



The collective mode is generated from the interaction of the multipole field carried by each particle and the field of the rest of the particles. In turn this coupling modifies the single-particle motion. In particular the incoming prjectile would feel this coupling. The potential $V'_\beta$ is equal to

\begin{align}\label{eq4l13}
\begin{split}
 V'_\beta & (\xi,\vec r)= U(r-R(\hat r))\\
& = U(r-R_0 -R_0\sum_{L,M} \alpha_M^L Y_M^{L*}(\hat r))\\
& = U(r-R_0) -R_0\sum_{L,M} \alpha_M^L Y_M^{L*}(\hat r) \frac{d U (r-R_0)}{d r}\\
& = V_\beta (\xi, r)- \bar{U}_\beta (r)
\end{split}\\\label{eq4l14}
\begin{split}
& \bar{U}_\beta (r)= -U(r-R_0)\\
& V_\beta (\xi,\vec r)=R_0 \frac{d \bar U_\beta (r)}{d r}\sum_{L,M} \alpha_M^L Y_M^{L*}(\hat r) \\
\end{split}
\end{align}
Comparing with eq. (\ref{eq4l5}) we obtain
\begin{equation}\label{eq4l15}
 V^L_M(\alpha,r)=R_0 \frac{d \bar U_\beta (r)}{d r} \alpha_{+M}^L
\end{equation}


Note that the Hamiltonian (\ref{eq4l11}) is the Hamiltonian of a five-dimensional harmonic oscillator, and that $\alpha_M^L$ is a classical variable. One can quantize this Hamiltonian in the standard way (see for example Messiah ``Quantum Mechanics'' Chapter 12)
\begin{equation}\label{eq4l16}
\alpha_M^L=\sqrt{\frac{\hbar \omega_L}{2 C_L}}(a_M^L-a^{+L}_{-M})
\end{equation}
where $\hbar \omega_L$ is the energy of the vibration, and $a^{+L}_{M}$ is the creation operator of a phonon. For an even nucleus
\begin{align}\label{eq4l17}
& \vert \Psi^{I_\alpha}_{M_\alpha}\rangle=\vert 0\rangle \quad (I_\alpha=M_\alpha=0)\\
& \vert 0\rangle:\quad \text{ground state} \notag
\end{align}
The one-phonon state is equal to
\begin{align}\label{eq4l18}
& \vert \Psi^{I_\alpha}_{M_\alpha}\rangle=\vert I;LM \rangle =a^{+L}_{M} \vert 0\rangle\\
& (I_\beta=L;M_{I_\beta}=M) \notag
\end{align}
We can now calculate the matrix element of the operator (\ref{eq4l15}), which connects states which differ in one phonon. Starting from the ground state we obtain
\begin{equation}\label{eq4l19}
\begin{split}
\langle  I;LM & \vert V^L_M (\alpha,r)\vert 0\rangle=\\
&(-1)^{L-M}R_0 \frac{d \bar U_\beta (r)}{d r} \sqrt{\frac{\hbar \omega_L}{2 C_L}}
\langle  0\vert(a_M^L-a^{+L}_{-M})\vert 0\rangle=\\
& R_0 \frac{d \bar U_\beta (r)}{d r} \sqrt{\frac{\hbar \omega_L}{2 C_L}}=
-\frac{R_0}{\sqrt{2L+1}} \frac{d \bar U_\beta (r)}{d r} \beta_L
\end{split}
\end{equation}
where
\begin{equation}\label{eq4l20}
 \beta_L=\sqrt{\frac{(2L+1)\hbar \omega_L}{2 C_L}}
\end{equation}


Substituting (\ref{eq4l19}) into eq. (\ref{eq4l8}) and making use of eqs. (\ref{eq4l17}) and (\ref{eq4l18}) we get
\begin{equation}\label{eq4l21}
\begin{split}
\frac{d\sigma}{d\Omega}&=\frac{k_\beta}{k_\alpha}\frac{\mu_\alpha^2}{(2 \pi \hbar^2)^2}(\beta_L R_0)^2\times\\
&\sum_{M}\frac{1}{2L+1}
\left\vert
\int d\vec r \chi^{(-)*}(k_\beta,\vec{r})
\frac{d  U (r)}{d r} Y_M^{L*}(\hat{r})\chi^{(+)}(k_\alpha,\vec{r}_\beta)\right\vert^2
\end{split}
\end{equation}


Suppose now that the nucleus has a permanent axially-symmetric deformation. For a $K=0$ band, the nuclear wave function has the form
\begin{equation}\label{eq4l22}
\Psi_{I M K=0} = \sqrt{\frac{2I+1}{8\pi^2}} \mathcal{D}_{M0}^I(\omega) \chi_{K=0} \quad \text{(intrinsic)}
\end{equation}
where we have used $(\omega)=(\theta,\phi,\psi)$ to label the Eulerian angles which serve as orientation parameters.


In the intrinsic frame (which we take to coincide with the space-fixed axis when $\theta=\phi=\psi=0$) the nuclear surface has the shape
\begin{equation}\label{eq4l23}
R(\hat r)= R_0 \left[ 1+ \sum_L b_L  Y^L_0(\hat r)\right]
\end{equation}
where the $b_L$ introduced here is $\alpha_0^L$ in the intrinsic frame. When the nucleus has orientation $\omega$, this shape is rotated into
\begin{equation}\label{eq4l24}
\hat{P}_\omega R(\hat r)= R_0 \left[ 1+ \sum_L b_L \mathcal D_{M0}^L(\omega) Y^L_0(\hat r)\right]
\end{equation}
we then have that
\begin{equation}\label{eq4l25}
W(r-R(\hat r))=W(r-R_0)-R_0\frac{d  W(r-R_0)}{d r}
\sum_L b_L \mathcal D_{M0}^L(\omega) Y^L_0(\hat r)
\end{equation}
which is the equivalent to eq. (\ref{eq4l13}) for the case of deformed nuclei. Then
\begin{equation}
 V_M^L(b,r;\omega)=-\frac{d
\bar U_\beta (r-R_0)}{d r}b_L \mathcal D_{M0}^L(\omega)
\end{equation}
The effective interaction is now equal to
\begin{equation}
\begin{split}
 \langle  \Psi&_{I M K=0} , V_M^L(b,r;\omega) \Psi_{000}\rangle =\\
& -R_0\frac{d
\bar U (r-R_0)}{d r}b_L \sqrt{\frac{(2L+1)^2}{8\pi^2}} \int \mathrm{d} \omega \mathcal D_{M0}^{L*}(\omega)
\mathcal D_{M0}^L(\omega)=\\
& -R_0\frac{d
\bar U (r-R_0)}{d r}b_L=-\frac{R_0}{\sqrt{(2L+1)}}\frac{d
\bar U (r-R_0)}{d r}\beta_L=F_L(r)\\
& \quad (\beta_L=\sqrt{(2L+1)} b_L)
\end{split}
\end{equation}
in complete analogy to (\ref{eq4l19}). Thus the same formfactor is used for both types of collective excitation.


The normalization factor $(\beta_L R_0)^2$ which is the only free parameter what is obtained from the comparison of the experimental and theoretical (DWBA) differential cross section. The quantity $\beta_L$ is known as the multipole deformation (dynamic or static) parameter, and gives a direct measure of the coupling of the projectile to the vibrational field.


The value of $\beta_L$ can also be obtained from the $B(E_L)$ reduced transition probability, in whichcase one has a measure of the electric moment, instead of the mass moment.


\section{One nucleon stripping reaction}


We discuss now thw case of $A(d,p)A+1$, namely of neutron stripping. The intrinsic wave functions $\psi_\alpha$ and $\psi_\beta$ are equal to

\begin{subequations}
\begin{align}\label{eq422}
&\psi_\alpha=\psi_{M_{A}}^{I_A}(\xi_A) \phi_d(\vec r_{np})\\
\begin{split}
&\psi_\beta=\psi_{M_{A+1}}^{I_{A+1}}(\xi_{A+1})=\\
& \;\;\;\;\sum_{l,I'_A} (I'_A;l \vert \} I_{A+1})
[\psi^{I'_A}(\xi_A)\phi^l(\vec r_{n})]_{M_{A+1}}^{I_{A+1}}
\end{split}
\end{align}
\end{subequations}
where $(I'_A;l \vert \} I_{A+1})$ is the generalized fractional parentage. To make simpler the derivation we assume we are dealing with spinless particles. The magnitude $\vec r_{np}$ is the relative coordinate of the proton and the neutron.


The transition matrix element can now be written as
\begin{equation}\label{eq423}
 \begin{split}
T_{d,p}&= \langle \psi_{M_{A+1}}^{I_{A+1}}(\xi_{A+1}) \chi^{(-)}_p(k_p,\vec{r}_p),
V'_\beta \psi_{M_{A}}^{I_{A}}(\xi_{A}) \chi^{(+)}_d(k_d,\vec{r}_d)\rangle= \\
& \sum_{\substack{l,I'_A\\M'_A}} (I'_A;l \vert \} I_{A+1}) (I'_A M'_A l M_{A+1}-M'_A \vert I_{A+1}M_{A+1})\times\\
& \int d\vec{r}_n d \vec{r}_p \chi^{* (-)}_p(k_p,\vec{r}_p) \phi_{M_{A+1}-M'_A}^{*l}(\vec{r}_n)
(\psi_{M_{A}}^{I_{A}}(\xi_{A}),V'_\beta \psi_{M'_{A}}^{I'_{A}}(\xi_{A}))\times\\
& \phi_d(\vec r_{np})
\chi^{(+)}_d(k_d,\vec{r}_d) \; \delta_{I'_A,I_A} \; \delta_{M'_A,M_A}
\end{split}
\end{equation}
In the stripping approximation
\begin{equation}\label{eq424}
 \begin{split}
V'_\beta & = V_\beta(\xi,\vec r_\beta)- U_\beta (r_\beta)=\\
& V_\beta(\xi_A,\vec r_{pA})+V_\beta(\vec r_{pn})- U_\beta (r_{pA})
\end{split}
\end{equation}
Then
\begin{equation}\label{eq425}
 \begin{split}
(\psi_{M_{A}}^{I_{A}}(\xi_{A}) & ,V'_\beta \psi_{M_{A}}^{I_{A}}(\xi_{A}))=
(\psi_{M_{A}}^{I_{A}}(\xi_{A}), V_\beta(\xi_A,\vec r_{pA}) \psi_{M_{A}}^{I_{A}}(\xi_{A}))+\\
&(\psi_{M_{A}}^{I_{A}}(\xi_{A}), V_\beta(\vec r_{pn})
\psi_{M_{A}}^{I_{A}}(\xi_{A}))- U_\beta (r_{pA})
\end{split}
\end{equation}
We assume
\begin{equation}\label{eq426}
  U_\beta (r_{pA})=(\psi_{M_{A}}^{I_{A}}(\xi_{A}), V_\beta(\xi_A,\vec r_{pA}) \psi_{M_{A}}^{I_{A}}(\xi_{A}))
\end{equation}
Then
\begin{equation}\label{eq427}
(\psi_{M_{A}}^{I_{A}}(\xi_{A}), V'_\beta\, \psi_{M_{A}}^{I_{A}}(\xi_{A}))= V_{np}(\vec r_{pn})
\end{equation}


Inserting eq. (\ref{eq427}) into eq. (\ref{eq423}) we obtain
 \begin{equation}\label{eq428}
 \begin{split}
T_{d,p}&= \sum_l (I_A;l \vert \} I_{A+1}) (I_A M_A l M_{A+1}-M_A \vert I_{A+1}M_{A+1}) \times \\
&\int d\vec{r}_n d \vec{r}_p \chi^{* (-)}_p(k_p,\vec{r}_p) \phi_{M_{A+1}-M'_A}^{*l}(\vec{r}_n)
V(\vec r_{pn}) \phi_d(\vec r_{np})
\chi^{(+)}_d(k_d,\vec{r}_d)
\end{split}
\end{equation}
The differential cross section is then equal to
\begin{equation}\label{eq429}
\frac{d \sigma}{d \Omega} = \frac{2}{3} \frac{\mu_p \mu_d}{(2\pi \hbar^2)^2}\frac{(2I_{A+1}+1)}{(2I_A+1)}
\frac{k_p}{k_d}\sum_{l,m_l}\frac{(I_A;l \vert \} I_{A+1})^2}{2l+1} \vert B_{m_l}^l\vert ^2
\end{equation}
where
\begin{equation}\label{eq430}
B_{m_l}^l(\theta)=\int d\vec{r}_n d \vec{r}_p \chi^{* (-)}_p(k_p,\vec{r}_p) Y_m^{*l}(\hat r_n) u_{nl}(r_n)
V(\vec r_{pn}) \phi_d(\vec r_{np})
\chi^{(+)}_d(k_d,\vec{r}_d)
\end{equation}
where
\begin{equation}\label{eq431}
\phi_m^{l}(\vec{r}_n)=u_{nl}(r_n) Y_m^{l}(\hat r_n)
\end{equation}
is the single-particle wave function of a neutron moving in the core A. Usually the radial wave function $u_{nl}(r_n)$ is the solution of a Saxon-Woods potential of parameters $V_0\approx 50$ MeV, $a=0.65$ fm and $r_0=1.25$ fm.



Equation (\ref{eq429}) gives the cross section for the stripping from the projectile of a neutron that would correspond to the n$^{\mathrm{th}}$ valence neutron in the nucleus ($A+1$). If we now want the cross section for stripping any of the valence nutrons of the final nucleus from the projectile, we must multiply eq. (\ref{eq429}) by $n$. A more careful treatment of the antisymmetry with respect to the neutrons shoes this to be the correct answer.


Finally we get
\begin{equation}\label{eq432}
\frac{d \sigma}{d \Omega}=\frac{(2I_{A+1}+1)}{(2I_A+1)} \sum_l S_l \sigma_l(\theta)
\end{equation}
where
\begin{equation}\label{eq433}
S_l= n (I_A;l \vert \} I_{A+1})^2
\end{equation}
and
\begin{equation}\label{eq434}
\sigma_l(\theta)=\frac{2}{3} \frac{\mu_p \mu_d}{(2\pi \hbar^2)^2}
\frac{k_p}{k_d}\frac{1}{2l+1}\sum_{m} \vert B_{m}^l\vert^2
\end{equation}


The distorted wave programs numerically evaluate the quantity $B_{m_l}^l(\theta)$, using for the wave functions $\chi^{(-)}$ and $\chi^{(+)}$ the solution of the optical potentials that fit the elastic scattering, i.e.
\begin{equation}\label{eq435}
(-\nabla ^2+\bar U-k^2) \chi=0
\end{equation}
(see eq. (\ref{eq11})).
Note that if the target nucleus is even, $I_A=0$ so $l=I_{A+1}$. That is, only one $l$ value contributes in eq. (\ref{eq429}), and the angular distribution is uniquely given by $\sum_{m} \vert B_{m}^l\vert^2$. The $l$-dependence of the angular distributions helps to identify $l=I_{A+1}$. The factor $S_l$ needed to normalize the calculated function to the data yields (assuming a good fit to the angular distribution), is the spectrocopic factor and can be compared directly to nuclear model predictions. It contains all the nuclear structure information.


The integral (\ref{eq430}) is a six-dimensional integral which is rather expensive to evaluate (in terms of the man power needed to write down the corresponding program and in terms of computing machine time). To reduce this integral it is customary to assume that the proton-neutron interaction $V_{np}$ has zero-range, i.e.
\begin{equation}\label{eq436}
 V_{np}(\vec r_{np})\phi_d(\vec r_{np})=D_0 \delta(\vec r_{np})
\end{equation}
so that  $B_{m}^l$ becomes equal to
\begin{equation}\label{eq430a}
B_{m_l}^l(\theta)=D_0 \int d\vec r \chi^{* (-)}_p(k_p,\vec r) Y_{m_l}^{*l}(\hat r) u_{l}(r)
\chi^{(+)}_d(k_d,\vec r)
\end{equation}
which is now a three dimensional integral. The integration over the angles is easy to carry out.


\section{Plane-wave limit}


If in eq. (\ref{eq435}) we set $\bar U=0$ the distorted waves becomes plane waves i.e.
 \begin{subequations}
\begin{align}\label{eq437}
&\chi^{(+)}_d(k_d,\vec r)=e^{i \vec k_d \cdot \vec r}\\
&\chi^{*(-)}_d(k_p,\vec r)=e^{-i \vec k_p \cdot \vec r}
\end{align}
\end{subequations}
Equation (\ref{eq430a}) can now be written as
\begin{equation}\label{eq438}
B_{m}^l=D_0 \int d\vec r e^{i (\vec k_d-\vec k_p) \cdot \vec r} Y_m^{*l}(\hat r) u_{l}(r)
\end{equation}
The linear momentum transferred to the nucleus is $\vec k_d-\vec k_p=\vec q$.
Let us expand $e^{i \vec q \cdot \vec r}$ in spherical harmonics, i.e.
\begin{equation}\label{eq439}
\begin{split}
 e^{i \vec q \cdot \vec r}&=\sum_l i^l j_l(qr)(2l+1)P_l(\hat q \cdot \hat r)=\\
& 4 \pi \sum_l i^l j_l(qr)\sum_m Y_m^{*l}(\hat q) Y_m^{l}(\hat r)
\end{split}
\end{equation}
so
\begin{equation}\label{eq440}
 \int d\hat r e^{i \vec q \cdot \vec r} Y_m^{l}(\hat r)= 4 \pi i^l j_l(qr) Y_m^{*l}(\hat q)
\end{equation}
Then
\begin{equation}\label{eq441}
\begin{split}
 \sum_{m} \vert B_{m}^l\vert^2 & = \sum_{m} \vert Y_m^{l}(\hat q)\vert^2 D_0^2 16 \pi^2 \times \\
& \left \vert \int r^2 dr j_l(qr) u_l(r) \right \vert ^2=\\
& \frac{2l+1}{4 \pi} D_0^2 16 \pi^2 \left \vert \int r^2 dr j_l(qr) u_l(r) \right \vert ^2
\end{split}
\end{equation}
Thus, the angular distribution is given by the integral $\left \vert \int r^2 dr j_l(qr) u_l(r) \right \vert ^2$ . If we assume that the process takes place mostly on the surface, the angular distribution will be given by $ \vert j_l(qR_0) \vert ^2 $ where $R_0$ is the nuclear radius.

%\begin{figure}
%\centerline{\includegraphics*[width=13cm,angle=0]{C:/Gregory/Broglia/notas_ricardo/Figures/ricardo_091105/4_1.eps}}
%\caption{}\label{fig4th_1}
%\end{figure}

We then have
\begin{equation}\label{eq442}
 \begin{split}
  q^2&= k_d^2+k_p^2- 2 k_d k_p \cos(\theta)=\\
& (k_d^2+k_p^2- 2 k_d k_p) + 2 k_d k_p (1-\cos(\theta))=\\
& (k_d-k_p)^2+ 4 k_d k_p \left(\sin (\theta/2)\right) ^2 \approx \\
& 4 k_d k_p \left(\sin (\theta/2)\right) ^2
\end{split}
\end{equation}
since $ k_d \approx k_p $ for stripping reactions at typical energies. Thus the angular distribution has a diffraction-like structure given by
\begin{equation}\label{eq443}
\vert j_l(qR_0) \vert ^2= j_l^2 (2R_0 \sqrt{k_d k_p} \sin (\theta/2))
\end{equation}
The function $j_l(x)$ has its first maximum at $x=l$, i.e. where
\begin{equation}\label{eq444}
\sin (\theta/2)=\frac{l}{2 R_0 k}\quad \quad (k_p \approx  k_d=k)
\end{equation}














