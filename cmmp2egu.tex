% This is file CMMP2egu.tex
%      released v1.02, 29th August 1997
%      released v1.01, 10th July 1997
% first release v1.0 beta, 5th March 1997
%   (based on CMMPguide.tex v0.35 for LaTeX2.09)
% Copyright (C) 1997 Cambridge University Press

\NeedsTeXFormat{LaTeX2e}

\documentclass{cmmp}

%%% Macros for the guide only %%%
% The following adds 6pt of space around verbatim environments.
\let\realverbatim=\verbatim
\let\realendverbatim=\endverbatim
\renewcommand\verbatim{\par\addvspace{6pt plus 2pt minus 1pt}\realverbatim}
\renewcommand\endverbatim{\realendverbatim\addvspace{6pt plus 2pt minus 1pt}}

\renewcommand\thesection{\arabic{section}}
%%% End of macros for the guide %%%


\begin{document}
\setcounter{chapter}{1}

\chapter*{\LaTeXe\ Style Guide for Authors}

\section{Introduction to \LaTeXe}

The \LaTeXe\ document preparation system is the latest version of \LaTeX\ which
is a special version of the \TeX\ typesetting program. \LaTeX\ adds to \TeX\ a
collection of commands which simplify typesetting by allowing the author to
concentrate on the logical structure of the document rather that its visual
layout.

\LaTeX\ provides a consistent and comprehensive document preparation
interface. There are simple-to-use commands for generating a table of
contents, lists of figures and/or tables, and indexes. \LaTeX\ can
automatically number list entries, equations, figures, tables, and
footnotes, as well as parts, chapters, sections and subsections. Using
this numbering system, bibliographic citations, page references and cross
references to any other numbered entity (\textit{e.g.} section, equation,
figure, list entry) are quite straightforward.

\LaTeX\ is a powerful tool for managing long and complex
documents. In particular, partial processing enables long documents
to be produced chapter by chapter without losing sequential information.

\section{The \textmd{\textsc{cmmp}} document class}

The \textsc{cmmp} document class is based on the book document class discussed
in the \LaTeXe\ manual. Commands which differ from regular \LaTeX\ or which are
in addition to regular \LaTeX\ are explained in this manual.  This short guide
is not a substitute for dedicated study of \textit{The \TeX\ book} or the
\LaTeX\ manual.

\subsection{Document class options}

In general, the standard document class options can be used, except the
following:
\begin{itemize}
\item \verb|10pt, 11pt, 12pt| --- unavailable.
\item \verb|twoside| --- \verb|twoside| is the default.
\item \verb|twocolumn| --- unavailable.
\end{itemize}
It is not advisable to use any document class options which change the way the
output looks, although most document class options can still be used without
problems.

\subsection{General}

\textbf{Warning:} There is no point spending lots of time sorting out pagination
problems with your text, as the output from CUP's \LaTeX\ may be different from
yours. Therefore it will be counter productive for the author to force lots of
pagebreaks. However this does not apply if you are producing the final camera
ready copy.

\subsection{A note on typography in this guide}

Throughout this guide, typewriter-style type will be used to indicate the
commands that you type and other characters you might see on your computer's
display screen. Braces ({\verb|{ }|}) are used to indicate \TeX's grouping
(and whenever groups are given in the definition of a command, they must be
given, even if empty [i.e., \verb|{} |], when the command is used).

\section{Installing the \textmd{\textsc{cmmp}} document class}

First, copy the \verb|cmmp.cls| file into the correct subdirectory on your hard
disk. Most implementations of \LaTeXe\ will search the current directory and
then the \verb|texinputs| directory for a class file. Therefore the best place
to store the class file is in the \verb|texinputs| directory, as this will avoid
the wrong class file being read by accident (as only one copy will exist). This
also has the advantage that any old versions of the \verb|cmmp.cls| file are
destroyed, when a new one is installed (as long as the old copy hasn't been
renamed).

\section{Starting work}

Think about the structure of the book and organise the files into manageable
pieces such as chapters. Next, create a main input file, for example,
\verb|book.tex|. This file will be set up to read in each of your input files
(\verb|chap1.tex|, \verb|chap2.tex|, \ldots, \verb|chap|\textit{n}\verb|.tex|).
Using a main input file in this way will enable you to cross-reference between
chapters and build a table of contents and index automatically. Your main input
file should look something like this:
\begin{verbatim}
\documentclass[draft]{cmmp}
\usepackage{makeidx}
\makeindex
\begin{document}
\tableofcontents

\documentclass[a4paper,11pt]{book}
%\usepackage{chapterbib}
\usepackage{dsfont}
\usepackage[title]{appendix}
\usepackage{slashbox}
\usepackage{enumerate}
\usepackage{footmisc}
%\documentclass[a4paper]{book}
% \linespread{2.}
%\numberwithin{section}
%\documentclass[12pt]{article}
%\documentclass[12pt]{cmmp}
\usepackage{textcomp}
%%\usepackage{psfig}
%\usepackage{harvard}
\usepackage{epsfig}
\usepackage{amsmath}   
\usepackage{amsfonts}
%\counterwithin{figure}{section}
\usepackage{amssymb}
\usepackage{bbold}
\usepackage{bbm}
\numberwithin{equation}{section}
\numberwithin{figure}{section}
\numberwithin{table}{section}
%%\usepackage{graphicx}
%%
%%\usepackage{txfonts}
%%%\usepackage{mathrsfs}
%
%%\usepackage{feynmf}     %<------------ Obbligatorio
\unitlength=1mm         %<------------ Obbligatorio
%
\newsavebox{\fmbox}
\newenvironment{fmpage}[1]
{\begin{lrbox}{\fmbox}\begin{minipage}{#1}}
{\end{minipage}\end{lrbox}\fbox{\usebox{\fmbox}}}
\newcommand{\braket}[1]{\langle {#1} \rangle }
\newcommand{\ket}[1]{|{#1} \rangle }
\newcommand{\bra}[1]{\langle {#1}|}
\newcommand\idop{\mathds 1}
\usepackage{dsfont}
\usepackage{latexsym}
\usepackage[varg]{txfonts}
\usepackage{mathrsfs}
\usepackage{upgreek}
\usepackage[round]{natbib}
%\usepackage [latin1]{inputenc}
\usepackage{verbatim}
\usepackage{array}
\usepackage{color}
%\pagestyle{plain}
\usepackage{graphicx}
\usepackage{caption}
\DeclareMathAlphabet{\mathpzc}{OT1}{pzc}{m}{it}
\begin{document}
\section{Views of the nucleus}
In the atom, the nucleus provides the Coulomb field in which negatively charged electrons $(-1\times e)$ move indenpendently of each other in single--particle orbitals. The filling of these orbitals explains Mendeleev's periodic table. Thus the valence of the chemical elements as well as the particular stability of the noble gases (He, N, Ar, Kr, Xe and Ra) associated with the closing of shells (Fig. \ref{fig1.0.1}). The dimension of the atom is measured in angstroms (\AA=$10^{-8}$cm), and typical energies in eV, the electron mass being $m_e\approx 0.5$ MeV (MeV=$10^6$eV).


The atomic nucleus is made out of positively charged protons ($1\times e$) and of (uncharged) neutrons, nucleons, of mass $\approx 10^3$ MeV ($m_p=938.3$ MeV, $m_n=939.6$ MeV). Nuclear dimensions are of the order of few fermis (fm$=10^{-13}$ cm). While the stability of the atom is provided by a source external to the electrons, namely the atomic nucleus, this system is  self--bound as a result of the strong interaction of range $a_0\approx 0.9$ fm and strength $v_0\approx -100$ MeV acting among nucleons. Carrying with the parallel, while most of the atom is empty space, the density of the atomic nucleus is conspicuous ($\rho=0.17$ nucleon/fm$^3$). The ``closed packed'' nature of this system implies a short mean free path as compared to nuclear dimensions. This can be estimated from classical kinetic theory $\lambda\approx(\rho\sigma)^{-1}\approx1$ fm, where $\sigma\approx 2\pi a_0^2$ is the nucleon--nucleon cross section. It seems then natural to liken the atomic nucleus to a liquid drop (Bohr and Kalckar).
This picture of the nucleus provided the framework to describe the basic features of the fission process (\cite{Meitner:39,Bohr:39}). 



The leptodermic properties of the atomic nucleus are closely connected with the semi--empirical mass formula (\cite{Weizsacker:35})
\begin{align}
m(N,Z)=(Nm_n+Zm_p)-\frac{1}{c^2}B(N,Z),
\end{align}
the binding energy being
\begin{align}\label{eq1.0.2}
B(N,Z)=\left(b_{vol}A-b_{surf}A^{2/3}-\frac{1}{2} b_{sym}\frac{(N-Z)^2}{A}-\frac{3}{5}\frac{Z^2e^2}{R_C}\right).
\end{align}
The second term in (\ref{eq1.0.2}) represents the surface energy, while
\begin{align}\label{eq1.0.3}
b_{surf}=4\pi r_0^2\gamma.
\end{align}
The nuclear radius is written as $R=r_0A^{1/3}$, with $r_0=1.2$ fm, the surface tension energy being $\gamma\approx 0.95$ MeV/fm$^2$.


When, in a heavy--ion reaction, the two nuclei come within the range of the nuclear forces, the trajectory of relative motion will be changed by the attraction which will act between the nuclear surfaces. This surface interaction is a fundamental quantity in all heavy ion reactions. Assuming two spherical nuclei at a relative distance $r_{aA}=R_a+R_A$, where $R_a$ and $R_A$ are the corresponding half--density radii, the force acting between the two surfaces is
\begin{align}\label{eq1.0.4}
\left(\frac{\partial U_{aA}^N}{\partial r}\right)_{r_{aA}}=4\pi \gamma\frac{R_aR_A}{R_a+R_A}
\end{align}
This result allows for the calculation of the ion--ion (proximity) potential which, supplemented with a position dependent absorption, can be used to accurately describe heavy ion reactions.


In such reactions, not only elastic processes are observed, but also anelastic ones in which one, or both of the nuclear surfaces is set into vibration (Fig. \ref{fig1.0.2}). The restoring force parameter associated with oscillations of multipolarity $\lambda$ is 
\begin{align}\label{eq1.0.4b}
C_\lambda=(\lambda-1)(\lambda+2)R^2\gamma-\frac{3}{2\pi}\frac{\lambda-1}{2\lambda+1}\frac{Z^2e^2}{R}
\end{align}
where the second term corresponds to the contribution of the Coulomb energy to $C_\lambda$. Assuming the flow associated with surface vibration to be irrotational, the associated inertia for small amplitude oscillations is, 
\begin{align}\label{eq1.0.5}
D_{\lambda}=\frac{3}{4\pi}\frac{1}{\lambda}AMR^2,
\end{align}
the energy of the corresponding mode being
\begin{align}\label{eq1.0.6}
\hbar\omega_\lambda=\hbar\sqrt{\frac{C_\lambda}{D_\lambda}}.
\end{align}

Experimental information associated with low--energy quadrupole vibrations, namely $\hbar\omega_{2}$ and the electromagnetic transition probabilities $B(E2)$, allow to determine $C_2$ and $D_2$. The resulting $C_2$ values exhibit variations by more than a factor of 10 both above and below the liquid--drop estimate. The observed values of $D_2$ are large as compared with the mass parameter for irrotational flow.

A picture apparently antithetic to that of the liquid drop, the shell model, emerged from the study of experimental data, plotting them against either the number of protons (atomic number), or the number of neutrons in the nuclei, rather than against the mass number.
One of the main nuclear features which led to the development of the shell model was the study of the stability and abundance of nuclear species and the discovery of what are usually called magic numbers (\cite{Elsasser:33,Mayer:48,Haxel:49}). What makes a number magic is that a configuration of a magic number of neutrons, or of protons, is unusually stable whatever the associated number of other nucleons (\cite{Mayer:49,Mayer:49b}).


The strong binding of a magic number of nucleons and weak binding for one more reminds, only relatively much weaker, the results displayed in Fig.  \ref{fig1.0.1}  concerning the atomic stability o\bibliographystyle{abbrvnat}f rare gases. In the nuclear case, at variance with the atomic case, the spin--orbit coupling play an important role, as can be seen from the level scheme shown in Fig. \ref{fig1.0.3}, obtained by assuming that nucleons move independently of each other in an average potential  of  spherical symmetry.


A closed shell, or a filled level, has angular momentum zero. Thus, nuclei with one nucleon outside (missing from) closed, should have the spin and parity of the orbital associated with the odd nucleon (--hole), a prediction confirmed by the data (available at that time) throughout the mass table. Such a picture implies that the nucleon mean free path is large compared to nuclear dimensions.


The systematic studies of the binding energies leading to the shell model found also that formula (\ref{eq1.0.2}) has to be supplemented to take into account the fact that nuclei with both odd number of protons and of neutrons are energetically unfavored compared with even--even ones (inset Fig. \ref{fig1.0.1}) by a quantity of the order of $\delta\approx33MeV/A^{3/4}$ called the pairing energy\footnote{Connecting with further developments associated with the BCS theory of superconductivity (\cite{Bardeen:57a,Bardeen:57b}) and its extension to the atomic nucleus (\cite{Bohr:58}), the quantity $\delta$ is identified with the pairing gap $\Delta$ parametrized according to $\Delta=12 $MeV$/\sqrt{A}$ (\cite{Bohr:69}). It is of notice that for typical superfluid nuclei like $^{120}$Sn, the expression of $\delta$ leads to $\delta\approx10$ MeV$/\sqrt{A}$.}.

The low--lying state of closed shell nuclei can be interpreted as harmonic quadrupole or octupole collective vibrations (Fig. \ref{fig1.0.4}) described by the Hamiltonian
\begin{align}\label{eq1.0.7}
H_{coll}=\sum_{\lambda\mu}\left(\frac{1}{2D_{\lambda}}|\Pi_{\lambda\mu}|^2+\frac{C_\lambda}{2}|\alpha_{\lambda\mu}|^2\right)
\end{align}
Following \cite{Dirac:26} one can describe the oscillatory motion introducing boson creation (annihilation) operator $\Gamma_{\lambda\mu}^\dagger$ ($\Gamma_{\lambda\mu}$) obeying
\begin{align}\label{eq1.0.8}
\left[\Gamma_{\alpha},\Gamma_{\alpha'}^\dagger\right]=\delta(\alpha,\alpha'),
\end{align}
leading to 
\begin{align}\label{eq1.0.9}
\hat\alpha_{\lambda\mu}=\sqrt{\frac{\hbar\omega}{2C}}\left(\Gamma_{\lambda\mu}^\dagger+(-1)^\mu\Gamma_{\lambda-\mu}\right),
\end{align}
and a similar expression for the conjugate momentum variable $\hat\Pi_{\lambda\mu}$, resulting in 
\begin{align}\label{eq1.0.9b}
\hat H_{coll}=\sum\hbar\omega\left((-1)^\mu\Gamma_{\lambda\mu}^\dagger\Gamma_{\lambda-\mu}+1/2\right).
\end{align}
The frequency is $\omega_\lambda=(C_\lambda/D_\lambda)^{1/2}$, while $(\hbar\omega_\lambda/2C_\lambda)^{1/2}$ is the amplitude of the zero--point fluctuation of the vacuum state $\ket{0}_B$, $\Gamma_{\lambda\mu}^\dagger \ket{0}_B$ being the one--phonon state.

The ground and low--lying states of nuclei with one nucleon outside closed shell can be described by the Hamiltonian
\begin{align}\label{eq1.0.10}
H_{sp}=\sum_{\nu}\epsilon_\nu a_\nu^\dagger a_\nu,
\end{align}
where $a_\nu^\dagger (a_\nu)$ is the single--particle creation (annihilation) operator,
\begin{align}\label{eq1.0.11}
\ket{\nu}=a_\nu^\dagger\ket{0}_F,
\end{align}
being the single--particle state of quantum numbers $\nu(\equiv nljm)$ and energy $\epsilon_\nu,\ket{0}_F$ being the Fermion vacuum. 

Both the existence of drops of nuclear matter displaying collective surface vibrations, and of independent--particle motion in a self--confining mean field are emergent properties not contained in the particles forming the system, neither in the $NN$--force, but on the fact that these particles behave according to the rules of quantum mechanics, move in a confined volume and that there are many of them.


Generalized rigidity as measured by the inertia parameter $D_\lambda$, as well as surface tension closely connected to the restoring force $C_\lambda$, implies that acting on the system with an external time--dependent (nuclear and/or Coulomb) field, the system reacts as a whole. This behavior is to be found nowhere in the properties of the nucleons, nor in the nucleon--nucleon scattering phase shifts at the basis of Yukawa prediction of the existence of a $\pi$--meson as the carrier of the strong force acting among nucleons.


Similarly, the fact that nuclei probed through fields which change in one unit particle number (e.g. $(d,p)$ and $(p,d)$ reactions) react in term of independent particle motion,  feeling the pushings and pullings of the other nucleons only when trying to leave the nucleus, is not apparent in the detailed properties of the $NN$--forces, not even in those carrying the quark--gluon input. Within this context, independent particle motion can be considered a \textit{bona fide} emergent property.


Collective surface vibrations and independent particle motion are examples of what are called elementary modes of excitation in many--body physics, and collective variables in soft--matter physics.


The oscillation of the nucleus under the influence of surface tension implies that the potential $U(R,r)$ in which nucleons move independently of each other change with time. For low--energy collective vibrations this change is slow as compared with single--particle motion. Within this scenario the nuclear radius can be written as  
\begin{align}\label{eq1.0.12}
R=R_0\left(1+\sum_{LM}\alpha_{\lambda\mu}Y_{\lambda\mu}^*\right)
\end{align}
Assuming small amplitude motion,
\begin{align}\label{eq1.0.13}
U(r,R)=U(r,R_0)+\delta U(r),
\end{align}
where
\begin{align}\label{eq1.0.14}
\delta U=-\kappa\hat \alpha \hat F,
\end{align}
and
\begin{align}\label{eq1.0.15}
\hat F=\sum_{\nu_1\nu_2}\bra{\nu_1}F\ket{\nu_2}a_{\nu_1}^\dagger a_{\nu_2},
\end{align}
while
\begin{align}\label{eq1.0.16}
F=\frac{R_0}{\kappa}\frac{\partial U}{\partial r}Y^*_{\lambda\mu}(\hat r).
\end{align}
The coupling between surface oscillation and single--particle motion, namely the particle vibration coupling (PVC) Hamiltonian $\delta U$ (Fig. \ref{fig1.0.5}) is a consequence of the overcompleteness of the basis. Diagonalizing $\delta U$ making use of the graphical (Feynman) rules of Nuclear Field Theory (NFT) to be discussed below, one obtains structure results which can be used in the calculation of transition probabilities and reaction cross sections which can be compared with experimental findings.

  
In fact, within the framework of NFT, single--particle are to be calculated as the Hartree--Fock solution of the $NN$--interaction $v(|\mathbf r-\mathbf r'|)$ (Fig. \ref{fig1.0.6}), in particular
\begin{align}\label{eq1.0.18}
U(r)=\int d\mathbf r' \rho(r')v(|\mathbf r-\mathbf r'|
\end{align}
being the Hartree field\footnote{To this potential one has to add the Fock potential resulting from the fact that nucleons are fermions. This exchange potential (Fig. \ref{fig1.0.6} (2 and 4)) is essential in the determination of single--particle energies and wavefunctions. Among other things, it takes care of eliminating the nucleon self interaction from the Hartree field.} expressing the selfconsistency between density $\rho$ and potential $U$ (Fig. \ref{fig1.0.6} (b) (1) and (3)), while vibrations are to be calculated in the Random Phase Approximation (RPA) making use of the same interaction\footnote{The sum of the so called ladder diagrams (see Fig. \ref{fig1.0.7}) are taken into account to infinite order in RPA. This is the reason why bubble contributions in the diagonalization of Eq. \ref{eq1.0.19b} are not allowed in NFT, being already contained in the basis states.} (Fig. \ref{fig1.0.7}), extending the selfconsistency to fluctuations $\delta\rho$ of the density and $\delta U$ of the mean field, that is,
\begin{align}\label{eq1.0.19}
\delta U(r)=\int d\mathbf r' \delta \rho(r')v(|\mathbf r-\mathbf r'|.
\end{align}
Making use of the selfconsistent solution of the relation (\ref{eq1.0.19}), one obtains the transition density $\delta\rho$. The matrix elements $\braket{\nu_i|\delta\rho|\nu_k}$ provide the  particle--vibration coupling strength to work out the variety of coupling processes between single--particle and collective motion (Fig. \ref{fig1.0.5}). That is, the matrix element of the PVC Hamiltonian $H_c$. Diagonalizing 
\begin{align}\label{eq1.0.19b}
H=H_{HF}+H_{RPA}+H_c+v,
\end{align}
making use of the rules of NFT to be discussed below, in the basis of single--particle and collective modes, that is solutions of $H_{HF}$ and of $H_{RPA}$ respectively, one obtains a solution of the total Hamiltonian. 
Because of quantal zero point fluctuations, a nucleon propagating in the nuclear medium moves through clouds of bosonic and fermionic virtual excitations to which it couple ($H_c+v$), becoming dressed and acquiring an effective mass, charge, etc. (Fig. \ref{fig1.0.8}). Vice versa, vibrational modes can become renormalized through the coupling to dressed nucleons which, in intermediate virtual states, can exchange the vibrational clothing with the second fermion (hole state) and renormalize the PVC vertex (Fig. \ref{fig1.0.9}) (\cite{Barranco:04}), as well as the bare $NN$--interaction. 


From being antithetic views of the nuclear structure a proper analysis of the experimental data testifies to the fact that the collective and the independent particle picture of the nuclear structure require and support each other (\cite{Bohr:75}). To obtain a quantitative description of nucleon  motion and nuclear phonons (vibrations), one needs a proper description of the $k$-- and $\omega$--dependent ``dielectric'' function of the nuclear medium, in a similar way in which a proper description of the reaction processes used as probes of the nuclear structure requires the use of the optical potential (continuum ``dielectric'' function). The NFT solution of (\ref{eq1.0.19b}) provide all the elements to calculate the nuclear structure properties of nuclei, and also  the optical potential needed to describe nucleon--nucleus scattering. It furthermore shows that both single--particle and vibrational elementary modes of excitation emerge from the same properties of the $NN$--interaction.


The development of experimental techniques and associated hardware has allowed for the identification of a rich variety of elementary modes of excitation aside from collective surface vibrations and of independent particle motion: quadrupole and octupole rotational bands, giant resonance of varied multipolarity and isospin, as well as pairing vibrations and rotation, together with giant pairing vibrations of transfer quantum number\footnote{A schematic separable force leading to surface vibrations can be written as $-\kappa \hat F^\dagger \hat F$, where $\hat F$ is defined in Eq. (\ref{eq1.0.15}). The resulting collective modes can thus be viewed as correlated particle--hole ($p-h$) excitations $(a^\dagger a)$, a process in which the number of nucleons (fermions) does not change. One speaks in this case of a mode with transfer quantum number $\beta=0$. In connection with the pairing energy mentioned in relation with the inset to Fig. \ref{fig1.0.1} and its connection to the theory of superconductivity, it is of notice that this theory is based on the concept of Cooper pairs, that is pairs of fermions moving in time reversal states which interact through $H_p=-G\hat P^\dagger \hat P$, where $\hat P^\dagger=\sum_{\nu>0}a_\nu^\dagger a_{\bar\nu}$. Consequently, in this case the concept of independent particle field $\hat F$ has to be generalized to include $\hat P^\dagger$ and $\hat P$. The resulting collective modes, pair addition and pair substraction modes (pairing vibrations), can be viewed as correlated ($p-p$) and ($h-h$) modes, changing the number of nucleons in $\pm 2$. One the speaks of vibrations with transfer quantum number $\beta = \pm 2$.} $\pm 2$. Modes which can be specifically excited in inelastic and Coulomb excitation processes, charge exchange, and one-- and two--particle transfer reactions.

One can choose to privilege one among this variety of elementary modes of excitation, for example, independent particle motion. Making use of the shell model, eventually the so called no core shell model, understood within this context as a full diagonalization of the $NN$--interaction in the single--particle basis,  attempt at describing the whole of structure and reactions. Another possibility is to use the elementary modes of excitation basis states to describe both structure and reactions  and nuclear field theory to deal with the overcompleteness and Pauli principle violations of the basis states.

From a systematic collaboration between the two approaches and of strong experimental input, it is likely that shell model calculations can help at individuating the proper interaction leading to realistic Hartree--Fock mean field and collective RPA particle--hole and pairing vibrational modes. As one possible return of such input, nuclear field theory will eventually be able to provide shell model practitioners,  friendly and accurate microscopic collective modes of excitation input.


The possible outcome of such collaboration and interplay could be that of being able to coin into few physical concepts the elements needed to accurately describe the atomic nucleus. In other words, carry out  calculations which are largely independent of the basis chosen. That is   truly predictive theories of structure and reactions, in which the physical content is simple to apprehend and visualize. 
\begin{figure}
\centerline {
\includegraphics*[width=12cm]{introduccion/figs/figpreface1}
}
\caption{The values of the atomic ionization potentials. The dots under the abscissa indicate closed shells, corresponding to electron numbers: 2(He), 10(Ne), 18(Ar), 36(Kr), 54(Xe), and 86(Ra). After \cite{Bohr:69}. In the inset, masses of nuclei with even $A$ are shown (after \cite{Mayer:55}).}
\label{fig1.0.1}
\end{figure}
\begin{figure}
\centerline {
\includegraphics*[width=12cm]{introduccion/figs/figpreface2x}
}
\caption{Emergent properties (collective nuclear models) \textbf{(a)} Nucleon--Nucleon ($NN$) interaction in a scattering experiment; \textbf{(b)} assembly of a swarm of nucleons condensing into drops of nuclear matter, examples shown in (c) and (e); \textbf{(c)} anelastic heavy ion reaction $a+A\to a+A^*$ setting the nucleus $A$ into an octupole surface oscillations \textbf{(d)}; in inset \textbf{(I)} the time--dependent nuclear plus Coulomb fields associated with the reaction (c) is represented by a cross followed by a dashed line, while the wavy line labeled $\lambda$ describes the propagation of the surface vibration shown in (d), time running upwards; \textbf{(e)} another possible outcome of nucleon condensation:the (weakly) quadrupole deformed nucleus $^{223}$Ra which can rotate as a whole with moment of inertia smaller than the rigid moment of inertia, but much larger than the irrotational one; \textbf{(f)} the surface of a quantal drop fluctuates (zero point fluctuations), with the variety of multipolarities with which the system reacts to time--dependent Coulomb/nuclear external fields (quadrupole ($\lambda=2$), octupole ($\lambda=3$), etc.), eventually producing a neck--in (saddle conformation) and the exotic decay $^{123}$Ra$\to^{209}$Pb+$^{14}$C as experimentally observed \textbf{(g)}.}
\label{fig1.0.2}
\end{figure}
\begin{figure}
\centerline {
\includegraphics*[width=12cm]{introduccion/figs/figpreface3}
}
\caption{To the left (first column), the sequence of levels of the harmonic oscillator potential labeled with the total oscillator quantum number and parity $\pi=(-1)^N$. The next column shows the splitting of major shell degeneracies obtained using a more realistic potential (Woods--Saxon), the quantum number being the number of radial nodes of the associated single--particle wave functions. The levels shown at the center result when a spin--orbit term is considered the quantum numbers $nlj$ characterizing the states of degeneracy $(2j+1)$ ($j=|l\pm1/2|$) (After \cite{Mayer:63}). In the inset, a schematic graphical representation of the reaction $^{208}_{82}$Pb$_{126}(d,p)^{209}$Pb(gs) is shown. A cross followed by a horizontal dashed line represents here the $(d,p)$ field, while a  single arrowed line describes the odd nucleon moving in the $g_{9/2}$ orbital above $N=126$ shell closure drawn as a bold line labeled $0^+$.}
\label{fig1.0.3}
\end{figure}
\begin{figure}
\centerline {
\includegraphics*[width=12cm]{introduccion/figs/figpreface4}
}
\caption{Harmonic quadrupole and octupole liquid drop collective surface vibrational modes.}
\label{fig1.0.4}
\end{figure}
\begin{figure}
\centerline {
\includegraphics*[width=10cm]{introduccion/figs/figpreface5}
}
\caption{Graphical representation of the process       by which a fermion, bouncing inelastically off the surface, sets it into vibration. Particles are represented by an arrowed line, while the vibration is shown by a wavy line. The black dot represents a nucleon moving in a spherical mean field of which it excites an octupole vibration after bouncing inelastically off the surface.}
\label{fig1.0.5}
\end{figure}
\begin{figure}
\centerline {
\includegraphics*[width=12cm]{introduccion/figs/figpreface6}
}
\caption{\textbf{(a)} Scattering of two nucleons through the bare $NN$--interaction; \textbf{(b)} (1) and (3): Contributions to the (direct) Hartree potential;(2) and (4): contributions to the (exchange) Fock potential.}
\label{fig1.0.6}
\end{figure}
\begin{figure}
\centerline {
\includegraphics*[width=12cm]{introduccion/figs/figpreface7}
}
\caption{(A) typical Feynman diagram diagonalizing the $NN$--interaction $v(|\mathbf r-\mathbf r'|)$ (horizontal dashed line) in a particle--hole basis provided by the Hartree--Fock solution of $v$, in the harmonic approximation (RPA). Bubbles going forward in time (inset (b)) are associated with configuration mixing of particle--hole states. Bubbles going backwards in time (inset (c)) are associated with zero point motion (fluctuations ZPF) of the ground state (term $1/2\hbar\omega$ for each degree of freedom in Eq. \ref{eq1.0.9b}). The self consistent solution of A is represented by a wavy line (inset (a)), that is a collective mode which can be viewed as a correlated particle hole excitation.}
\label{fig1.0.7}
\end{figure}
\begin{figure}
\centerline {
\includegraphics*[width=12cm]{introduccion/figs/figpreface8}
}
\caption{\textbf{(a)} a nucleon (single arrowed line) moving in presence of the zero point fluctuation of the nuclear ground state associated with a collective surface vibration; \textbf{(b)} Pauli principle leads to a dressing event of the nucleon; \textbf{(c)} time ordering gives rise to the second possible lowest order clothing process (time assumed to run upwards).}
\label{fig1.0.8}
\end{figure}
\begin{figure}
\centerline {
\includegraphics*[width=12cm]{introduccion/figs/figpreface9}
}
\caption{(Upper part) Examples of renormalization processes dressing a surface collective vibrational state. (Lower part) Intervening with an external electromagnetic field ($E\lambda$: cross followed by dashed horizontal line; bold wavy lines, vibration of multipolarity $\lambda$) the $B(E\lambda)$ transition strength can be measured.}
\label{fig1.0.9}
\end{figure}







\bibliographystyle{abbrvnat}

\bibliography{./nuclear_bib}
\end{document} 
% chap1.tex
% 2010/09/09, v2.10

\chapter{Introduction}
\label{intro}

This guide is for authors who are preparing a book for Cambridge University Press using the \LaTeX\ document preparation system, and the \cambridge\ class file.

The \LaTeX\ document preparation system is a special version of the \TeX\ typesetting program. \LaTeX\ adds to \TeX\ a collection of commands which simplify typesetting by allowing the author to concentrate on the logical structure of the document rather than its visual layout.

\LaTeX\ provides a consistent and comprehensive document preparation interface. There are simple-to-use commands for generating a table of contents (toc), lists of figures and/or tables, and indexes. \LaTeX\ can automatically number list entries, equations, figures, tables, and footnotes, as well as parts, chapters, sections and subsections. Using this numbering system, bibliographic citations, page references and cross references to any other numbered entity (e.g. chapter, section, equation, figure, list entry) are quite straightforward.

\LaTeX\ is a powerful tool for managing long and complex documents. In particular, partial processing enables long documents to be produced chapter by chapter without losing sequential information. The use of document classes allows a simple change of style to transform the appearance of your document.

\section{The \LaTeXe\ book document class}

The \cambridge\ class file preserves the standard \LaTeX\ interface such that any document which can be produced using the standard \LaTeXe\ book class can also be produced with the \cambridge\ class. However, the measure (i.e. width of text) is different from that for book, therefore linebreaks will change and long equations may need re-setting.

\section{The \cambridge\ document class}

The \cambridge\ design has been implemented as a \LaTeXe\ class file, and is based on the book class as discussed in the \LaTeX\ manual. Commands which differ from the standard \LaTeX\ interface, or which are provided in addition to the standard interface, are explained in this guide. This guide is \emph{not} a substitute for the \LaTeX\ manual itself.

\section{Implementing the \cambridge\ class file}
\label{usingcamb}

Copy \cambridge.cls into the correct subdirectory on your system. The \cambridge\ document class is implemented as a complete document class, \emph{not} a document class option. To run this guide through \LaTeX, you need to include the following class and style files:\\[0.5\baselineskip]
\verb"  \documentclass{"\texttt{\cambridge}\verb"}"\\
\verb"    \usepackage{natbib}"\\
\verb"    \usepackage{rotating}"\\
\verb"    \usepackage{floatpag}"\\
\verb"      \rotfloatpagestyle{empty}"\\
\verb"    \usepackage{amsthm}"\\
\verb"    \usepackage{graphicx}"\\
\verb"    \usepackage{multind}\ProvidesPackage{multind}"\\[0.5\baselineskip]
It may be that your book does not use references, rotation, theorems, graphics, or multiple indexes, in which case you simply need the first line. If you include \verb"multind.sty", you must also insert the command \verb"\ProvidesPackage{multind}". More recent style files include this information; it simply sends a message to the class file to re-style the index into the \cambridge\ style.

In general, the following standard document class options should \emph{not} be used:
 \begin{itemize}
  \item \texttt{10pt}, \texttt{11pt}, \texttt{12pt};
  \item \texttt{oneside}  (\texttt{twoside} is the default);
  \item \texttt{fleqn}, \texttt{leqno}, \texttt{titlepage}, \texttt{twocolumn}.
 \end{itemize}

\section{Implementing the multi-contributor option}

This option should be used where chapters have been written by different contributors. Please read Section~\ref{usingcamb} first; then implement the \verb"[multi]" option as follows:\\[0.5\baselineskip]
\verb"  \documentclass[multi]{"\texttt{\cambridge}\verb"}"\\[0.5\baselineskip]
Further details can be found in Section~\ref{multicontributor}.

\section{Fonts}

We suggest you use one of the following font options. The first is the default Computer Modern font; the other two are Times fonts (our preferred font for use in printed books):
\begin{enumerate}
\item Computer Modern
\item mathptmx, available from:\\
      http://www.ctan.org/tex-archive/fonts/psfonts/psnfss-source/mathptmx/
\item txfonts, available from:\\
      http://www.ctan.org/tex-archive/fonts/txfonts/
\end{enumerate}
If you deliver your manuscript in the default Computer Modern font, we will in most cases change the font to Times; however, if your book contains critical line and page breaks (e.g. in programming code) we will probably leave it in Computer Modern.

If you deliver your manuscript in either of the Times options, we are unlikely to change the font. Consult your editor for further information.

Mathptmx changes the default roman font to Adobe Times, but does not support bold math characters.

Txfonts does support bold math, but the kerning of subscripts and superscripts is not ideal. You must load txfonts \emph{after} amsthm.sty, otherwise you will get some `already defined' messages.\footnote{The reason we do not include times.sty as an option is because it mixes Computer Modern and Times fonts, and there is a clash
between math and italic characters.}

Please note that you must supply a pdf of your files so that the typesetters
can check characters such as bold math italic. If you are providing final pdf files
for printing, remember to embed all fonts as Type~1 fonts.

\section{Make-up}

This is a generic guide for many Cambridge designs. We have therefore not attempted to correct long lines, and there are occasions where pages may be a little long. The latter is due to the use of \verb"\begin{samepage}"\ldots \verb"\end{samepage}" where we are keeping text together for clarity. Authors should not include any page make-up commands, unless they are providing final PDFs for printing.

\endinput
% chap2.tex
% 2010/09/09, v2.10

% for multi-contributor books,  use \author
% for single-contributor books, though not required, use \chapterauthor

% uncomment \begin{abstract}...\end{abstract} for the Abstract to apppear

  \alphafootnotes
  \author[M\,M Magn\'usson and D\,A Tranah]
    {Magn\'us M\'ar Magn\'usson\footnotemark\
    and David Tranah\footnotemark}

  \chapterauthor{Magn\'us M\'ar Magn\'usson\footnotemark\
    and David Tranah\footnotemark}

  \chapter{The \cambridge\ class file in detail}

  \footnotetext[1]{Formerly of the Icelandic
    Meteorological Office, Reykjav\'\i k.}
  \footnotetext[2]{Supported by NSF Grant 43645.}
  \arabicfootnotes

  \contributor{Magn\'us M\'ar Magn\'usson
    \affiliation{International Glaciological Society,
      Scott Polar Research Institute,
      Lensfield Road, Cambridge CB2 1ER}}

  \contributor{David Tranah
    \affiliation{Cambridge University Press,
      The Edinburgh Building, Shaftesbury Road,
      Cambridge CB2 8RU}}

% \begin{abstract}
%   Thermal convection driven by centrifugal buoyancy in a rapidly rotating narrow annular channel is studied in the case of rigid cylindrical walls.
% \end{abstract}

The following notes may help you achieve the best effects with the \cambridge\ class file.

\section{Frenchspacing}

The \verb"\frenchspacing" option has been selected by default. This ensures that no extra space is inserted after full points, and is normal practice. If there is a strong reason for reversing this, you can key \verb"\nonfrenchspacing" in the preamble.

\section{Adding a subtitle to the front page}

The standard \verb"\title" command has been extended to take an optional argument which is then used as a subtitle on the main title page. For example, this document uses following title command:
\begin{verbatim}
  \title[Subtitle, If You Have One]
    {\LaTeXe\ GUIDE FOR AUTHORS USING THE \cambridge\ DESIGN}
\end{verbatim}


\section{Adding a blank page to your document}

Blank pages should not be numbered. If you require one, use the command \verb"\cleardoublepage", which has been redefined to start the next page on a recto, and if necessary, insert a totally blank verso page first.

\section{Adding a spanning rule to part and~chapter~openings}

If your editor has asked you to use the spanning rule option for your book, it is called in as follows:\\[0.5\baselineskip]
\verb"  \documentclass[spanningrule]{"\texttt{\cambridge}\verb"}"

\section{Chapter numbering}
If your book starts with an unnumbered chapter (e.g. \verb"\chapter*{Introduction}", then make all the numbered elements (e.g. section heads) unnumbered, by using \verb"\section*{...}". Otherwise, sections will be numbered 0.1, 0.2, etc.

\section{Section numbering}

\LaTeX\ provides five levels of section heads, and they are all defined in the \cambridge\ class file: \verb"\section", \verb"\subsection", \verb"\subsubsection", \verb"\paragraph", and \verb"\subparagraph". Numbers are given for the first three headings.

You can reduce the level of numbered section heads (it is not advisable to increase them). For instance, if you only want headings numbered down to subsections, add the following line to the preamble: \verb"\setcounter{secnumdepth}{2}". To number down to sections, make this \verb"\setcounter{secnumdepth}{1}", etc.


\section{Specifying running heads and toc entries}

\subsection{Single-contributor books}
\label{singlecontributor}

In \cambridge, chapter titles and section heads are used as running heads at the top of every page:
\begin{itemize}
\item chapter titles appear on even-numbered pages (versos), and
\item section heads appear on odd-numbered pages (rectos).
\end{itemize}
A problem with the standard version of \LaTeX\ has always been that the shortened versions of chapter and section titles, specified for running heads, have also been the entries for the toc. There are packages such as the memoir class which enable you to specify different toc entries, running head entries, and chapter titles. However, there is a simple way to add the verbose version of the chapter or section heads into the toc:
\begin{verbatim}
  \chapter[Toc entry]{Verbose chapter title}
  \chaptermark{Running head entry}

  \section[Toc entry]{Verbose section title
    \sectionmark{Running head entry}}
    \sectionmark{Running head entry}
\end{verbatim}
Note that for sections, you need the optional argument to \verb"\section", even if `Toc entry' is in fact the same text as `Verbose section title'. Also, the \verb"\sectionmark" has to be entered twice as shown, because the first \verb"\sectionmark" deals with the header of the page that the \verb"\section" command falls on, and the second deals with subsequent pages.

\subsection{Multi-contributor books}
\label{multicontributor}

Using the \cambridge\ multi-contributor option, author(s) name(s) and chapter titles are used as running heads at the top of every page:
\begin{itemize}
\item author(s) name(s) appear on even-numbered pages (versos), and
\item chapter titles appear on odd-numbered pages (rectos).
\end{itemize}
The author(s) names(s) may run to several lines, and contain new line commands (e.g. \verb"\\"), but the running head must be a single line. To enable you to specify an alternative short form of the author(s) name(s), the standard \verb"\author" command has been extended to take an optional argument to be used as the running head:
\begin{verbatim}
  \author[Author(s) name(s)]{The full author(s) name(s)}
\end{verbatim}
The following shows some coding for a chapter written by two authors, each of whom have footnotes. In this example, the authors' names will immediately follow the chapter title, and will read Magn\'us M\'ar Magn\'usson$^{a}$ and David Tranah$^{b}$. Their respective footnotes will be `$^{a}\enskip$Formerly of the Icelandic Meteorological Office, Reykjav\'\i k.' and `$^{b}\enskip$Supported by NSF~Grant 43645.' It is crucial that \verb"\author" precedes \verb"\chapter". If the authors have footnotes, you must start the chapter with \verb"\alphafootnotes", fill in the details for author(s), chapter title and author footnotes, then key \verb"\arabicfootnotes" to revert to arabic footnotes:
\begin{verbatim}
  \alphafootnotes
  \author[M\,M Magn\'usson and D\,A Tranah]
    {Magn\'us M\'ar Magn\'usson\footnotemark\
    and David Tranah\footnotemark}

  \chapter[Running head entry]
    {The \cambridge\ class file in detail}

  \footnotetext[1]{Formerly of the Icelandic
    Meteorological Office, Reykjav\'\i k.}
  \footnotetext[2]{Supported by NSF Grant 43645.}
  \arabicfootnotes
\end{verbatim}
Note that for multi-contributor books, the long version of the chapter title will always appear in the table of contents.


\section{Adding author(s) name(s) in single-contributor books}
Sometimes, chapters in single-contributor books are written by different people. If you wish the authors to appear beneath the chapter opening, as demonstrated in this chapter, key your chapter head as follows; note that \verb"\chapterauthor" must precede \verb"\chapter":
\begin{verbatim}
  \alphafootnotes
  \chapterauthor{Magn\'us M\'ar Magn\'usson\footnotemark\
    and David Tranah\footnotemark}

  \chapter{The \cambridge\ class file in detail}

  \footnotetext[1]{Formerly of the Icelandic
    Meteorological Office, Reykjav\'\i k.}
  \footnotetext[2]{Supported by NSF Grant 43645.}
  \arabicfootnotes
\end{verbatim}
If you have footnotes associated with the authors, you will also need to insert \verb"\alphafootnotes" and \verb"\arabicfootnotes".

\section{List of contributors}
\label{contrib}
The code for generating an automatic list of contributors should be entered after the author and chapter titles, as follows:
\begin{verbatim}
  \contributor{Magn\'us M\'ar Magn\'usson
    \affiliation{International Glaciological Society,
      Scott Polar Research Institute,
      Lensfield Road, Cambridge CB2 1ER}}

  \contributor{David Tranah
    \affiliation{Cambridge University Press,
      The Edinburgh Building, Shaftesbury Road,
      Cambridge CB2 8RU}}
\end{verbatim}
You then simply need to add the \verb"\listofcontributors" command after the table of contents (or after the artwork lists, if included) in the preamble, as follows:
\begin{verbatim}
  \tableofcontents
  \listoffigures
  \listoftables
  \listofcontributors
\end{verbatim}

\subsection{Note to editors regarding the list of contributors}

The contributors will appear in the same order as they are called in, since the list is generated in the same way as the table of contents. This means that at the final stage, the file will require editing to make the entries alphabetic.

Once you have a complete list of contributors, comment out the line which is generating them, and replace it as shown below:
\begin{verbatim}
  \tableofcontents
  \listoffigures
  \listoftables
 %\listofcontributors
  \editedlistofcontributors
\end{verbatim}
Next, rename the file with the extension \verb".loc" to \verb"editedloc.tex" (in the case of this guide, you would rename \texttt{\cambridge guide.loc} to \verb"editedloc.tex"). Edit this file as required, then run the file through \LaTeX\ once more, and the edited version will appear.

\section{Adding an Abstract}
The following code will give you an unnumbered section head `Abstract'. Keep the Abstract to one paragraph:
\begin{verbatim}
  \begin{abstract}
    Thermal convection driven by centrifugal...
  \end{abstract}
\end{verbatim}

\section{Adding a `copyright' line to a chapter opening~page}
If you are publishing a single chapter, with permission from Cambridge University Press, you may be required to add a copyright line (and/or other information) to the footer of the chapter opening page. This may be achieved using:
\begin{verbatim}
  \copyrightline{Reprinted from \textit{Mathematical
    Methods for Physics and Engineering} by Riley,
    Hobson and Bence \copyright\ 2009 Cambridge
    University Press.}
\end{verbatim}
Should the following chapter not require the copyright line, reverse this immediately before the next \verb"\chapter" command by using:
\begin{verbatim}
  \copyrightline{}
\end{verbatim}


\section{Changing the level of entries in the table of~contents}
\label{changingentries}
The \cambridge\ design will, by default, list parts, chapters and sections in the table of contents. However, to improve the usefulness of this guide, we have used the command:
\begin{verbatim}
  \setcounter{tocdepth}{2}
\end{verbatim}
to increase this by one level, so the table of contents in this document also shows subsections.


\section{Lists}
\label{lists}

The \cambridge\ class provides the following standard list environments:
\begin{enumerate}
 \item numbered lists, created using the \verb"enumerate" environment;
 \item bulleted lists, created using the \verb"itemize" environment;
 \item labelled lists, created using the \verb"description" environment.
\end{enumerate}
The \verb"enumerate" environment numbers each list item with an arabic numeral followed by a full point; alternative styles can be achieved by inserting a redefinition of the number labelling command after the \verb"\begin{enumerate}". For example, a list numbered with lower-case letters inside parentheses can be produced. Because `(a)' is wider than a standard arabic digit, the label width has to be increased. This is achieved by specifying the widest label in the list inside square braces:
\begin{verbatim}
  \begin{enumerate}[(a)]
    \renewcommand{\theenumi}{(\alph{enumi})}
    \item estimate the fluctuations in the near-wall region\ldots
    \item subdue these near-wall fluctuations\ldots
  \end{enumerate}
\end{verbatim}
This produces the following list:
  \begin{enumerate}[(a)]
    \renewcommand{\theenumi}{(\alph{enumi})}
    \item estimate the fluctuations in the near-wall region\ldots
    \item subdue these near-wall fluctuations\ldots
  \end{enumerate}


\section{Endnotes}

In addition to footnotes,\footnote{The footnote counter will be reset on chapters.} the \cambridge\ class provides a similar facility for endnotes. Their appearance depends on which option you are using:
\begin{enumerate}
\item for single-contributor books, the endnotes will be produced in the form of an unnumbered chapter at the end of the book;
\item for multi-contributor books, they are an unnumbered section at the end of each chapter.
\end{enumerate}
Endnotes are inserted into the text in a similar way to footnotes, but using the \verb"\endnote" command; for example,
\begin{verbatim}
  When the Richardson number\endnote{Lewis Fry Richardson
  (1881--1953).\label{richardson}} increases\ldots
\end{verbatim}
will produce `When the Richardson number\endnote{Lewis Fry Richardson (1881--1953).\label{richardson}} increases\ldots' in the text. Authors must choose between using footnotes and endnotes; do not use both.

\subsection{Single-contributor books}
Endnotes should be printed at the end of the book, after the appendices but before the bibliography and/or references.
\begin{verbatim}
    :
  \theendnotes
  \begin{thebibliography}{99}
    :
\end{verbatim}
The \verb"\theendnotes" command generates an unnumbered chapter which appears in the table of contents (see page~\pageref{richardson} for style).

\subsection{Multi-contributor books}

Endnotes should be printed at the end of the chapter using the same \verb"\theendnotes" command.

\section{Exercise environments}

\subsection{Exercises at the end of sections}
\label{exendofsections}

Authors using \verb"amsthm.sty" can define an \verb"{exer}" environment within the\linebreak \verb"\theoremstyle{definition}" -- see Appendix~\ref{amsthmcommands} for details. Alternatively, authors may use the \verb"exerciselist" environment which will typeset exercises at the end of each section. There is an option to add some useful text, such as `Exercise'; this is shown in the following example:
\begin{verbatim}
  \begin{exerciselist}[Exercise]
    \item Show that the link between shock formation and
          film rupture is invoked here because of the\ldots
    \item Show that the physical interpretation of\ldots
          \label{physi}
  \end{exerciselist}
\end{verbatim}
which will produce:
  \begin{exerciselist}[Exercise]
    \item Show that the link between shock formation and
          film rupture is invoked here because of the\ldots
    \item Show that the physical interpretation of\ldots
          \label{physi}
  \end{exerciselist}
As with all numbered environments, individual exercises (e.g. Exercise~\ref{physi}) can be cross-referenced.


\subsection{Exercises at the end of chapters}

If you would prefer to have the exercises at the end of each chapter, use the \verb"exercises" environment. This generates an entry in the table of contents and starts a new unnumbered section. For example,
\begin{verbatim}
  \begin{exercises}
    \item Let the film thickness be $h_0$,
          \begin{equation}
            h=h_0 H{\xi}.
          \label{exerciseeq}
          \end{equation}
          Integrating once\ldots
    \item Assuming the flow far away from\ldots
  \end{exercises}
\end{verbatim}
will produce:
  \begin{exercises}
    \item Let the film thickness be $h_0$,
          \begin{equation}
            h=h_0 H{\xi}.
          \label{exerciseeq}
          \end{equation}
          Integrating once\ldots
    \item Assuming the flow far away from\ldots
  \end{exercises}

\section{Figures}

The \cambridge\ class will cope with most positioning of your figures. As captions fall below figures, the figure must be included first, then the caption, then the label. This is illustrated in Figure~\ref{cantor}. The \verb"cantor1.eps" file has been called in by using \verb"\usepackage{graphicx}" in the preamble. Note that if you are producing a list of illustrations (using \verb"\listoffigures"), you need to repeat the caption in square braces, but without the full point.
  \begin{figure}
    \includegraphics[scale=0.55]{cantor1.eps}
    %  note that the square brace option below is only required
    %  if you intend to produce a list of illustrations
    \caption[Shortened figure caption for the list of illustrations]
      {A Cantor repeller. Long figure captions will be indented left
      and right; short ones will be centred by default.}
    \label{cantor}
\rule[-20pt]{\textwidth}{0.6pt}
\begin{verbatim}
  \begin{figure}
    \includegraphics[scale=0.55]{cantor1.eps}
    %  note that the square brace option below is only required
    %  if you intend to produce a list of illustrations
    \caption[Shortened figure caption for the list of illustrations]
      {A Cantor repeller. Long figure captions will be indented left
      and right; short ones will be centred by default.}
    \label{cantor}
  \end{figure}
\end{verbatim}
\rule[20pt]{\textwidth}{0.5pt}
  \end{figure}

\section{Tables}

The \cambridge\ class will cope with most positioning of your tables. Table captions must be included first, the the label, then the body of the table. This is illustrated in Table~\ref{sample-table}.
  \begin{table}
    \begin{minipage}{188pt}
      %  note that the square brace option below is only required
      %  if you intend to produce a list of tables
    \caption[Shortened table caption for the list of tables]
      {Longer table captions have to be placed inside
      a minipage, otherwise they overhang the table rules.}
    \label{sample-table}
    \addtolength\tabcolsep{2pt}% to stretch columns, if required
      \begin{tabular}{@{}c@{\hspace{25pt}}ccc@{}}
        \hline \hline
        Figure\footnote{\textit{Note:} You must also use a minipage
          environment if you have footnotes.} & $hA$ & $hB$ & $hC$\\
        \hline
        1 & $\exp\left(\pi i\frac58\right)$
          & $\exp\left(\pi i\frac18\right)$ & $0$\\[3pt]
        2 & $-1$    & $\exp\left(\pi i\frac34\right)$ & $1$\\[11pt]
        3 & $-4+3i$ & $-4+3i$ & $\frac74$\\[3pt]
        4 & $-2$    & $-2$    & $\frac54 i$ \\
        \hline \hline
      \end{tabular}
    \end{minipage}
    \rule[-20pt]{\textwidth}{0.5pt}
\begin{verbatim}
  \begin{table}
    \begin{minipage}{188pt}
      %  note that the square brace option below is only required
      %  if you intend to produce a list of tables
    \caption[Shortened table caption for the list of tables]
      {Longer table captions have to be placed inside
      a minipage, otherwise they overhang the table rules.}
    \label{sample-table}
    \addtolength\tabcolsep{2pt}% to stretch columns, if required
      \begin{tabular}{@{}c@{\hspace{25pt}}ccc@{}}
        \hline \hline
        Figure\footnote{\textit{Note:} You must also use a minipage
          environment if you have footnotes.} & $hA$ & $hB$ & $hC$\\
        \hline
        1 & $\exp\left(\pi i\frac58\right)$
          & $\exp\left(\pi i\frac18\right)$ & $0$\\[3pt]
        2 & $-1$    & $\exp\left(\pi i\frac34\right)$ & $1$\\[11pt]
        3 & $-4+3i$ & $-4+3i$ & $\frac74$\\[3pt]
        4 & $-2$    & $-2$    & $\frac54 i$ \\
        \hline \hline
      \end{tabular}
    \end{minipage}
  \end{table}
\end{verbatim}
\rule[20pt]{\textwidth}{0.5pt}
  \end{table}

\subsection{My vertical rules have disappeared}

Vertical rules in tables are not \cambridge\ style, and have been automatically removed; this gives your document a truly professional look. Instead of vertical rules, we recommend the use of extra horizontal space, see Section~\ref{addhoriz}. The rules have been removed by redefining the \verb"tabular" environment. The amended definition also inserts extra vertical space above and below the horizontal rules (produced by \verb"\hline").

If you really must have them reinstated, read Section~\ref{reinstate}.

\subsection{Reinstating the vertical rules}
\label{reinstate}
Authors can revert to the standard \LaTeX\ style, if necessary. Tables will take on a rather squashed appearance, as in the \LaTeX\ book, whereby there is no added space around horizontal rules. Add the command \verb"\reinstaterules" in the preamble, and re-run your files through \LaTeX.

\subsection{There is very little space around the rules in my~table}
Tables revert to the standard, rather squashed look of standard \LaTeX\ tables for two reasons:
\begin{enumerate}
  \item you are using \verb"array.sty"; or
  \item you have chosen to reinstate vertical rules (see Section~\ref{reinstate})
\end{enumerate}
In both cases, the tabular environment is redefined.


\subsection{Adding space between columns}
\label{addhoriz}
You can add space (2pt in this example) between every column using\linebreak \verb"\addtolength\tabcolsep{2pt}". However, if you only wanted to expand the space between columns~1 and~2 to~25pt, you would do this using\linebreak  \verb"\begin{tabular}{@{}c@{\hspace{25pt}}ccc@{}}" (see Table~\ref{sample-table}).

\subsection{Adding space between rows}
If you need some form of separation between rows (for example, between rows~2 and~3 in the body of Table~\ref{sample-table}), adding \verb"[11pt]" immediately after the double backslash at the end of row~2 will add an 11pt vertical space (the equivalent of a blank line at this typesize). This is neater than adding another horizontal line.


\section{Landscape figures and tables, using rotating.sty}

Landscape figures and tables (floats) may be typeset using the \verb"rotating.sty" package. Note that the direction of rotation depends on the page number -- which requires at least two passes through \LaTeX. If we are going to know whether pages are odd or even, we need to use the \verb"\pageref" mechanism, and labels. But labels won't work unless the user has put in a caption. \textit{Beware!}

In addition to \verb"rotating.sty", you should also include \verb"floatpag.sty" and the command \verb"\rotfloatpagestyle{empty}". This combination ensures that headers and footers are removed from the float page:
\begin{verbatim}
  \usepackage{rotating}
  \usepackage{floatpag}
  \rotfloatpagestyle{empty}
\end{verbatim}
In some DVI previewers, floats may not appear rotated. If this happens, you need to convert the DVI file to PostScript or PDF.

Occasionally, when you convert a PostScript file to a PDF file, you may find that the page comes out upside-down. There will be a setting to change this. For instance, if you are using PDFCreator 0.9.7, choose the following options in this sequence:
\begin{description}
  \item Options -- Program -- PDF -- Auto-Rotate Pages: Change to `None'.
\end{description}
Other programs will have similar\vadjust{\pagebreak} procedures.


\subsection{Coding for landscape figures}

The landscape figure (Figure~\ref{sidecantor}) was typeset using the following coding:
\begin{verbatim}
  \begin{sidewaysfigure}
    \centering
    \includegraphics[scale=0.95]{cantor1.eps}
    %  note that the square brace option below is only required
    %  if you intend to produce a list of illustrations
    \caption[Landscape figure]{A Cantor repeller. Figure captions
      will be centred by default.}
    \label{sidecantor}
  \end{sidewaysfigure}
\end{verbatim}
  \begin{sidewaysfigure}
    \centering
    \includegraphics[scale=0.95]{cantor1.eps}
    %  note that the square brace option below is only required
    %  if you intend to produce a list of illustrations
    \caption[Landscape figure]{A Cantor repeller. Figure captions
      will be centred by default.}
    \label{sidecantor}
  \end{sidewaysfigure}



\subsection{Coding for landscape tables}

Table~\ref{sideways} has been produced using the following coding:
%
\begin{smallverbatim}
\begin{sidewaystable}
  \caption[Landscape table]{Grooved ware and beaker features, their finds and
    radiocarbon dates. For a breakdown of the pottery assemblages see
    Tables~I and~III; for the flints see Tables~II and~IV; for the animal
    bones see Table~V.}
  \label{sideways}
  \addtolength\tabcolsep{-2pt}
  \begin{tabular}{@{}lcccllccccc@{}}
  \hline\hline
  Context & Length & Breadth/  & Depth & Profile & Pottery & Flint & Animal
                                                   & Stone & Other & C14 Dates\\
  && Diameter &&&&& Bones\\[5pt]
  & m & m & m\\
  \hline\\[-5pt]
  \multicolumn{10}{@{}l}{\textbf{Grooved Ware}}\\
  784 & --   & 0.9$\phantom{0}$ &0.18  & Sloping U & P1      & $\times$46
        & $\phantom{0}$$\times$8 && $\times$2 bone & 2150 $\pm$100\,\textsc{bc}\\
  785 & --   & 1.00             &0.12   & Sloping U & P2--4  & $\times$23
                                           & $\times$21 & Hammerstone & -- & --\\
  962 & --   & 1.37             &0.20   & Sloping U & P5--6  & $\times$48
                     & $\times$57 & --& --& 1990 $\pm$80\,\textsc{bc} (Layer 4)\\
  &&&&&&&&&& 1870 $\pm$90\,\textsc{bc} (Layer 1)\\
  983 & 0.83 & 0.73             &0.25   & Stepped U & --     & $\times$18
                                & $\phantom{0}$$\times$8 & -- & Fired clay & --\\
  &&&&&&&&&&\\
  \multicolumn{10}{@{}l}{\textbf{Beaker}}\\
  552 & --   & 0.68             & 0.12  & Saucer    & P7--14 & --           & --
                                                                   & -- &-- &--\\
  790 & --   & 0.60             & 0.25  & U         & P15    & $\times$12   & --
                                                      & Quartzite-lump & -- &--\\
  794 & 2.89 & 0.75             & 0.25  & Irreg.    & P16    & $\phantom{0}$$\times$3
                                                              & -- & -- &-- &--\\
  \hline\hline
  \end{tabular}
\end{sidewaystable}
\end{smallverbatim}
%
\begin{sidewaystable}
  \caption[Landscape table]{Grooved ware and beaker features, their finds and
    radiocarbon dates. For a breakdown of the pottery assemblages see
    Tables~I and~III; for the flints see Tables~II and~IV; for the animal
    bones see Table~V.}
  \label{sideways}
  \addtolength\tabcolsep{-2pt}
  \begin{tabular}{@{}lcccllccccc@{}}
  \hline\hline
  Context & Length & Breadth/  & Depth & Profile & Pottery & Flint & Animal
                                                   & Stone & Other & C14 Dates\\
  && Diameter &&&&& Bones\\[5pt]
  & m & m & m\\
  \hline\\[-5pt]
  \multicolumn{10}{@{}l}{\textbf{Grooved Ware}}\\
  784 & --   & 0.9$\phantom{0}$ &0.18  & Sloping U & P1      & $\times$46
        & $\phantom{0}$$\times$8 && $\times$2 bone & 2150 $\pm$100\,\textsc{bc}\\
  785 & --   & 1.00             &0.12   & Sloping U & P2--4  & $\times$23
                                           & $\times$21 & Hammerstone & -- & --\\
  962 & --   & 1.37             &0.20   & Sloping U & P5--6  & $\times$48
                     & $\times$57 & --& --& 1990 $\pm$80\,\textsc{bc} (Layer 4)\\
  &&&&&&&&&& 1870 $\pm$90\,\textsc{bc} (Layer 1)\\
  983 & 0.83 & 0.73             &0.25   & Stepped U & --     & $\times$18
                                & $\phantom{0}$$\times$8 & -- & Fired clay & --\\
  &&&&&&&&&&\\
  \multicolumn{10}{@{}l}{\textbf{Beaker}}\\
  552 & --   & 0.68             & 0.12  & Saucer    & P7--14 & --           & --
                                                                   & -- &-- &--\\
  790 & --   & 0.60             & 0.25  & U         & P15    & $\times$12   & --
                                                      & Quartzite-lump & -- &--\\
  794 & 2.89 & 0.75             & 0.25  & Irreg.    & P16    & $\phantom{0}$$\times$3
                                                              & -- & -- &-- &--\\
  \hline\hline
  \end{tabular}%
\end{sidewaystable}

\endinput
\include {....}

\printindex
\end{document}
\end{verbatim}
Using the \verb|\include| command will mean that each file will have its own
\verb|.aux| file, enabling the files to be processed together or separately, by
the use of the \verb|\includeonly| command. See the \LaTeXe\ manual (pages 72--74,
210) for more information.

\section{Sectioning commands}

All chapters begin with a title and/or a number. The Cambridge University Press
style which \textsc{cmmp} implements requires minimum capitalization for all
headings; that is, only the first word and proper names take initial capital
letters -- all other words are lowercase.

For numbered chapters, use \verb|\chapter{}|. For unnumbered chapters, use
\verb|\chapter*{}|. To obtain a section head, use \verb|\section{}| (numbered)
or \verb|\section*{}| (unnumbered),  and for subsections use
\verb|\subsection{}|. In the \textsc{cmmp} style, subsections are not numbered.

\medskip
\noindent \textbf{N.B.}: the \verb|\\| command doesn't work within a
\verb|\chapter| command.

\section{Special environments}

Special environments include theorem-like environments and proofs where
the text formatting distinguishes them from the main text.

\subsection*{Theorems}

Each time you introduce a new theorem-like environment you must define it
with \verb|\newtheorem{}{}|; each environment is defined only once. All new
theorems should be defined in an external macro file which should be included
using \verb|\usepackage|.

To define a theorem-like environment, for example a corollary, type the
following:
\begin{verbatim}
\newtheorem{corollary}{Corollary}
\end{verbatim}
Here is an example of its use:
\begin{verbatim}
\begin{corollary}
This is the first corollary.
\end{corollary}
Some intervening text.
\begin{corollary}
This is the second corollary. They number automatically.
\end{corollary}
\end{verbatim}
When typeset, the above code will produce:
\newtheorem{corollary}{Corollary}
%
\begin{corollary}
This is the first corollary.
\end{corollary}
Some intervening text.
\begin{corollary}
This is the second corollary. They number automatically.
\end{corollary}
The first argument of \verb|\newtheorem| is the name of the environment as
you will refer to it with \verb|\begin{}| and \verb|\end{}|. The second
argument is the name of the environment as it will appear on the printed page.
A detailed description of the \verb|\newtheorem| command can be found in the
\LaTeXe\ manual (pages 56--57, 193--194).

\subsection*{Proofs}

For proofs, use \verb|\begin{proof}| and \verb|\end{proof}|.
\begin{verbatim}
\begin{proof}
This is the proof of the above corollary.
\end{proof}
\end{verbatim}
which looks like this in print:
\begin{proof}
This is the proof of the above corollary.
\end{proof}
\noindent A box is inserted at the end of each proof for clarity.

\section{Footnotes}

Footnotes in the \textsc{cmmp} style are numbered symbolically (with $^*$,
$^\dagger$, etc.). The footnote\footnote{An example footnote} at the bottom
of this page was keyed thus:
\begin{verbatim}
The footnote\footnote{An example footnote} at the bottom ...
\end{verbatim}
Ensure there is no space before the \verb|\| of \verb|\footnote{}|, and at
least one space after it.

\LaTeX\ imposes a limit of nine footnote symbols. If you exceed this limit
in one chapter \LaTeX\ will complain. There are two solutions to this problem:
either reduce the number of footnotes per chapter, or redefine the footnote
counter to use Arabic numbers.

\noindent e.g.
\begin{verbatim}
\renewcommand{\thefootnote}{\arabic{footnote}}
\end{verbatim}
The above code needs to be placed in the main input file, before the
\verb|\begin{document}|, or at the top of each chapter file. You should not
mix the style of footnotes within your book. Use one style or the other.

\section{Displayed equations}

There are two types of single line displayed equations, numbered and
unnumbered. Unnumbered equations can be obtained by the use of
\verb|\begin| and \verb|\end{displaymath}|. For numbered equations use
\verb|\begin| and \verb|\end{equation}|.

\noindent e.g.
\begin{verbatim}
\begin{displaymath}
x + 4 = 24
\end{displaymath}
and
\begin{equation}
x + 2 - 5 = 36
\end{equation}
\end{verbatim}
which produces
\begin{displaymath}
x + 4 = 24
\end{displaymath}
and
\begin{equation}
x + 2 - 5 = 36
\end{equation}
For unnumbered multi-line equations, use \verb|\begin| and
\verb|\end{eqnarray*}| and for numbered multi-line
equations use \verb|\begin| and \verb|\end{eqnarray}|

\noindent e.g.
\begin{verbatim}
\begin{eqnarray*}
x & = & 4 + 2 + y\\
  & = & 3 - 8 - z
\end{eqnarray*}
\end{verbatim}

\begin{verbatim}
\begin{eqnarray}
x & = & 4 + 2 + y\\
  & = & 3 - 8 - z
\end{eqnarray}
\end{verbatim}
which produces
\begin{eqnarray*}
x & = & 4 + 2 + y\\
  & = & 3 - 8 - z
\end{eqnarray*}

\begin{eqnarray}
x & = & 4 + 2 + y\nonumber\\
  & = & 3 - 8 - z
\end{eqnarray}
In addition, a new environment called \verb|ceqnarray| centres the rows of
a multi-line formula. It is used in the following way:
\begin{verbatim}
\begin{ceqnarray}
u,v  =  -v,u\\
fu,gv  =  fg u,v + fu(g)v - gv(f)u\\
o  =  u,[v,w] + w,[u,v] + v,[w,u]\\
w,u+v  =  w,u + w,v
\end{ceqnarray}
\end{verbatim}
which produces
\begin{ceqnarray}
u,v  =  -v,u\\
fu,gv  =  fg u,v + fu(g)v - gv(f)u\\
o  =  u,[v,w] + w,[u,v] + v,[w,u]\\
w,u+v  =  w,u + w,v
\end{ceqnarray}
Note that no ampersands (\verb|&|) are used in the \verb|ceqnarray| environment.
See the \LaTeXe\ manual for detailed instructions on how to format mathematical
equations with \LaTeX\ (pages 39-52, 187--191).

\section{Lists}

It is occasionally convenient to list a number of items in a format different
from that of the surrounding text. \textsc{cmmp} supplies several methods:
\begin{itemize}
\item Numbered lists, created using \verb|\begin| and \verb|\end{enumerate}|;
\item Unnumbered lists, created with \verb|\begin| and \verb|\end{unnum}|;
\item Bulletted lists, created with \verb|\begin| and \verb|\end{itemize}|;
\item Labelled lists, created with \verb|\begin| and \verb|\end{description}|.
\end{itemize}
The items in the lists are introduced with the \verb|\item| command.
Sublists can be created using the same commands; for example,
\begin{verbatim}
\begin{enumerate}
\item This is item one.
\item This is item two.
  \begin{enumerate}
  \item This is sub-item one.
  \end{enumerate}
\item This is item three.
  \begin{enumerate}
  \item This is another sub-item.
  \item This is another.
    \begin{enumerate}
    \item A subsub-list item.
    \end{enumerate}
  \end{enumerate}
\end{enumerate}
\end{verbatim}
which creates the following list:
\begin{enumerate}
\item This is item one.
\item This is item two.
\begin{enumerate}
\item This is sub-item one.
\end{enumerate}
\item This is item three.
\begin{enumerate}
\item This is another sub-item.
\item This is another.
\begin{enumerate}
\item A subsub-list item.
\end{enumerate}
\end{enumerate}
\end{enumerate}
It is important that the pairs of commands are nested properly. If you
misplace or forget a \verb|\end{}|, \LaTeX\ will complain.

\section{Illustrations}

The \verb|figure| environment allows you to integrate the artwork of a figure,
the caption, and its position on the page. Electronic integration is achieved by
using \verb"epsf.sty" or \verb"psfig.sty", both of which are freely available.
The \verb|figure| environment is used in the following way:
\begin{verbatim}
\begin{figure}[t]
\vspace{100pt}
\caption{This is the caption of my figure.}
\end{figure}
\end{verbatim}
%
\begin{figure}[t]
\vspace{100pt}
\caption{This is the caption of my figure.}
\end{figure}

The optional \verb|[t]| is a `location argument'. It tells \LaTeX\ to place
the figure at the top of a page. Other choices are \verb|[b]| for
bottom of a page; \verb|[p]| for a separate page containing no text; and
\verb|[h]| for here, the position in the text where the environment appears.

If a location argument is missing (e.g. \verb|\begin{figure}|), the default
is \verb|[tbp]|. Figures are placed at the earliest place after the point
in the text where the \verb|figure| environment occurs; a figure may not be
printed before an earlier figure and cannot violate its location argument.
Allow \LaTeX\ the greatest number of options for placing your figures, or
they might all appear at the end of your chapter. The same rules apply for
placement of tables.

Books are easier to read when the illustrations (and tables; see below) are
positioned at the tops and bottoms of the pages. This desideratum, together
with a few others, leads naturally to a fairly simple, though often
contradictory, set of rules for the positioning of floating insertions.
\begin{enumerate}
\item An illustration should be positioned after its first citation
(or call-out): preferably either at the  bottom of the page in which
it is cited on or at the top of the next page.
\item When two illustrations will fit on one page, the earlier one should
be positioned at the top of the page and the latter at the bottom.
\item A minimum of four lines of text should appear on a page with
illustrations. If less than four lines will fit on the page, it should
contain only figures (and their captions).
\item Illustrations should be positioned before the end of the section in
which they are cited and must never overlap into supplementary sections such
as bibliographies, indices, or appendices.
\item Illustrations should never be positioned on the first page of a
chapter or other major section (that is, one that begins a new right-hand
page).
\end{enumerate}
\verb|cmmp.cls| will obey most of these rules most of the time unless extreme
demands are made upon it. Badly placed figures can usually be fixed by moving
the \verb|figure| environment within the text.

Remember to leave no blank lines or space between the text and a
\verb|\begin{figure}|; it can create extra vertical space in your output.

If you have a long paragraph, you can place the \verb|figure| environment
in the centre of a paragraph; be sure to leave a blank space between
\verb|\end{figure}| and the rest of the paragraph to avoid two words printing
together with no intervening space.

Placing figures and tables correctly in a manuscript is one of the most
difficult jobs of preparing the final camera-ready copy. Before spending
lots of time positioning figures and tables, consult your editor for advice.

\section{Tables}

Tables follow the same placement rules as figures. A sample table is
inserted following this paragraph.
\begin{table}[h]
\centering
\caption{This is my table caption.}
\begin{tabular}{ccc}
\hline\hline
Zeros & Ones & Twos\\
\hline
0 & 1 & 2\\
0 & 1 & 2\\
0 & 1 & 2\\
0 & 1 & 2\\
0 & 1 & 2\\
\hline\hline
\end{tabular}
\end{table}
\newpage %%% req'd

\noindent This is how it was typed:
\begin{verbatim}
\begin{table}
\centering
\caption{This is my table caption.}
\begin{tabular}{ccc}
\hline\hline
Zeros & Ones & Twos\\
\hline
0 & 1 & 2\\
0 & 1 & 2\\
0 & 1 & 2\\
0 & 1 & 2\\
0 & 1 & 2\\
\hline\hline
\end{tabular}
\end{table}
\end{verbatim}
The CUP house style for tables is as follows:
\begin{itemize}
\item The table caption should always be above the table.
\item There should always be a horizontal double rule at the top of the
table (after the caption), and a double rule at the very end (but before
any table footnotes).
\item There should be no vertical rules.
\end{itemize}
Consult the \LaTeXe\ manual for detailed instructions regarding formatting
tables (pages 58--59, 197--200).

\section{Making a bibliography}

The bibliography begins with \verb|\begin{thebibliography}|. Each
bibliographic item should start with a \verb|\bibitem| command and should
start on a new line of its own. There are four main formats for bibliographic
items.
\begin{enumerate}
\item References to articles in journals
\item References to articles in books
\item References to books
\item References to theses and dissertations
\end{enumerate}
Examples of each are contained in the bibliography (see page~\pageref{bib}).
%%%\vadjust{\eject}

\noindent The bibliography was coded in the following way:
\begin{verbatim}
\begin{thebibliography}{4}

\bibitem{abbott}
Abbott, L.F. and Deser, S. (1982). Stability of gravity with
a cosmological constant, \textit{Nucl. Phys.} \textbf{B195},
76--96.

\bibitem{adams}
Adams, J.F. (1981). Spin (8), triality, $F_4$ and all that,
in \textit{Superspace and Supergravity}, ed. S.W.~Hawking
and M.~R\"ocek (Cambridge University Press, Cambridge).

\bibitem{arnold}
Arnol'd, V.I. (1978). \textit{Mathematical Methods of
Classical Mechanics} (Springer, New York).

\bibitem{buch}
Buchdahl, N.P. (1982). Applications of Several Complex
Variables to Twistor Theory, Oxford University
D. Phil. thesis.

\end{thebibliography}
\end{verbatim}
The standard \LaTeX\ interface has been preserved, so that Bib\TeX\ can
be used in the standard way.

\section{Making an index}

The \textsc{cmmp} document class supports the standard \LaTeX\ indexing scheme.
A detailed description on how to build an index can be found in the
\LaTeXe\ manual (pages 74--75, 211-212).

\section{Miscellaneous}

\subsection{Marginal notes}

Notes can be placed in the margin for your editor, coauthor, or yourself
with the \verb|\query| command, which works\query{This is a query} like
this:
\begin{verbatim}
works\query{This is a query} like this
\end{verbatim}
It is important that there should be no space before the \verb|\query|
command. At least one space must follow it.

\subsection{Getting the best pagebreaks}

In order to achieve the best pagebreaks, you may want to make certain
paragraphs one line longer or shorter. To make a paragraph one line longer,
use \verb|\looseness=1|; to make a paragraph one line shorter, use
\verb|\looseness=-1|. See \textit{The \TeX\ book} for details. Again before
spending lots of time on this, consult your editor for advice.

\section{Creating new commands}

To save time in keyboarding, you may define short commands for often-used
command strings. For instance, rather than typing the long
\verb|\begin| \verb|{equation}| and \verb|\end{equation}| you can type
\verb|\be| and \verb|\ee| instead, if you define them in the following way:
\begin{verbatim}
\newcommand{\be}{\begin{equation}}
\newcommand{\ee}{\end{equation}}
\end{verbatim}
Similarly, \verb|\mathcal{C}| can be replaced by \verb|\cC| (or any command
which is easy for you to remember) and \verb|\mathbf{V}| by \verb|\bV|.

If you want to reduce the number of keystrokes you have to make,
you can use  the \verb|\let}| command to abbreviate long commands.
For instance, by typing \verb|\let\bigtriangledown=\btd| you allow
either \verb|\btd| or \verb|\bigtriangledown| to be used to produce
$\bigtriangledown$.

\begin{thebibliography}{4}\label{bib}

\bibitem{abbott}
Abbott, L.F. and Deser, S. (1982). Stability of gravity with
a cosmological constant, \textit{Nucl. Phys.} \textbf{B195},
76--96.

\bibitem{adams}
Adams, J.F. (1981). Spin (8), triality, $F_4$ and all that,
in \textit{Superspace and Supergravity}, ed. S.W.~Hawking
and M.~R\"ocek (Cambridge University Press, Cambridge).

\bibitem{arnold}
Arnol'd, V.I. (1978). \textit{Mathematical Methods of
Classical Mechanics} (Springer, New York).

\bibitem{buch}
Buchdahl, N.P. (1982). Applications of Several Complex
Variables to Twistor Theory, Oxford University
D. Phil. thesis.

\end{thebibliography}

\vspace{20pt}
\hrule width 2in
\vspace{5pt}
\noindent \LaTeXe\ \textsc{cmmp} guide v1.02

\end{document}
