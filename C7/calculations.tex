\documentclass[a4paper,14pt]{book}
% \linespread{2.}
\usepackage{latexsym}
\usepackage{amssymb}
\usepackage{amsmath}
\usepackage{txfonts}
\usepackage{mathrsfs}
\usepackage{upgreek}
%\usepackage [latin1]{inputenc}
\usepackage{verbatim}
\usepackage{array}
\usepackage{color}
\pagestyle{plain}
\usepackage{graphicx}
\begin{document}
\section{Details of the Calculation}
Let us illustrate the calculation with a $A+t \rightarrow B(\equiv A+2)+p$ reaction, in which $A+2$ and $A$ are even nuclei in their $0^+$ ground state. The extension of the following expressions to the transfer of pairs coupled to arbitrary angular momentum is straightforward. The wavefunction of the nucleus $A+2$ is 
\begin{equation}\label{eq1}
\Psi_{A+2}(\xi_A,\mathbf r_{A1},\sigma_1,\mathbf r_{A2},\sigma_2)=\psi_A(\xi_A)\sum_{l_i,j_i}[\phi^{A+2}_{l_i,j_i}(\mathbf r_{A1},\sigma_1,\mathbf r_{A2},\sigma_2)]^0_0,
\end{equation} 
where 
\begin{equation}\label{eq2}
\phi^{A+2}_{l_i,j_i}(\mathbf r_{A1},\sigma_1,\mathbf r_{A2},\sigma_2)=\sum_{nm}a_{nm}\left[\varphi^{A+2}_{n,l_i,j_i}(\mathbf r_{A1},\sigma_1)\varphi^{A+2}_{m,l_i,j_i}(\mathbf r_{A2},\sigma_2)\right]^0_0,
\end{equation} 
and the wavefunctions $\varphi^{A+2}_{n,l_i,j_i}(\mathbf r)$ are eigenfunctions of a Woods--Saxon potential
\begin{equation}\label{Eq17}
U(r)=-\frac{V_0}
{1+\exp\left[\frac{r-R_0}{a}\right]},\quad\quad R_0=r_0 A^{1/3}.
\end{equation}
The depth $V_0$ is adjusted to reproduce the experimental single--particles energies.
 The spatial part of the  wavefunction of the two neutrons in the tritium is $\phi_t(\mathbf r_{p1},\mathbf r_{p2})=\rho(r_{p1})\rho(r_{p2})\rho(r_{12})$, where $r_{p1},r_{p2},r_{12}$ are the distances between neutron 1 and the proton, neutron 2 and the proton and between neutrons 1 and 2 respectively, and $\rho(r)$ is a wavefunction with hard core at $r=0.45$ fm, as depicted in Fig \ref{fig1}.
 
\begin{figure}
\centerline{\includegraphics*[width=.55\textwidth,angle=0]{C:/Gregory/workspace/coops/figuras/tritium.pdf}}
\caption{Tritium wavefunction}\label{fig1}
\end{figure}
\begin{figure}
\centerline{\includegraphics*[width=.55\textwidth,angle=0]{C:/Gregory/workspace/coops/figuras/deuteron.pdf}}
\caption{Deuteron wavefunction}\label{fig2}
\end{figure}
The differential cross section is
\begin{equation}
\frac{d\sigma}{d\Omega}=\frac{\mu_i\mu_f}{(4\pi\hbar^2)^2}\frac{k_f}{k_i}\left|T^{(1)}+T^{(2)}_{succ}-T^{(2)}_{NO}\right|^2,
\end{equation}

where the three amplitudes contributing to the transfer are (see also \cite{Bayman:82}):
\begin{subequations}
\begin{multline}\label{eq1_40}
T^{(1)}=2\sum_{l_i,j_i}\sum_{\sigma_1 \sigma_2}\int d\mathbf{r}_{tA}d\mathbf{r}_{p1}d\mathbf{r}_{A2}
  [\phi^{A+2}_{l_i,j_i}(\mathbf r_{A1},\sigma_1,\mathbf r_{A2},\sigma_2)]^{0*}_0\chi^{(-)*}_{pB}(\mathbf{r}_{pB})\\
 \times v(\mathbf{r}_{p1}) \phi_t(\mathbf r_{p1},\mathbf r_{p2})\chi^{(+)}_{tA}(\mathbf{r}_{tA}),
\end{multline}
\begin{multline}\label{eq1_41}
T^{(2)}_{succ}=2\sum_{l_i,j_i}\sum_{l_f,j_f,m_f}\sum_{\substack{\sigma_1 \sigma_2\\\sigma'_1 \sigma'_2}}
\int d\mathbf{r}_{dF}d\mathbf{r}_{p1}d\mathbf{r}_{A2}
[\phi^{A+2}_{l_i,j_i}(\mathbf r_{A1},\sigma_{1},\mathbf r_{A2},\sigma_2)]^{0*}_0\chi^{(-)*}_{pB}(\mathbf{r}_{pB})
 v(\mathbf{r}_{p1})\\
 \times\phi_d(\mathbf r_{p1})\varphi^{A+1}_{l_f,j_f,m_f}(\mathbf r_{A2}) \int d\mathbf{r}'_{dF}d\mathbf{r}'_{p1}d\mathbf{r}'_{A2}G(\mathbf{r}_{dF},\mathbf{r}'_{dF})\\
 \times \phi_d(\mathbf r'_{p1})^*\varphi^{A+1*}_{l_f,j_f,m_f}(\mathbf r'_{A2}) \frac{2\mu_{dF}}{\hbar^2}v(\mathbf{r}'_{p2})
 \phi_d(\mathbf r'_{p1})\phi_d(\mathbf r'_{p2}) \chi^{(+)}_{tA}(\mathbf{r}'_{tA}),
\end{multline}
\begin{multline}\label{eq1_42}
T^{(2)}_{NO}=2\sum_{l_i,j_i}\sum_{l_f,j_f,m_f}\sum_{\substack{\sigma_1 \sigma_2\\\sigma'_1 \sigma'_2}}
\int d\mathbf{r}_{dF}d\mathbf{r}_{p1}d\mathbf{r}_{A2}
[\phi^{A+2}_{l_i,j_i}(\mathbf r_{A1},\sigma_1,\mathbf r_{A2},\sigma_2)]^{0*}_0\chi^{(-)*}_{pB}(\mathbf{r}_{pB})
 v(\mathbf{r}_{p1})\\
 \times \phi_d(\mathbf r_{p1})\varphi^{A+1}_{l_f,j_f,m_f}(\mathbf r_{A2})\int d\mathbf{r}'_{p1}d\mathbf{r}'_{A2}d\mathbf{r}'_{dF}\\
 \times\phi_d(\mathbf r'_{p1})^*\varphi^{A+1*}_{l_f,j_f,m_f}(\mathbf r'_{A2}) 
  \phi_d(\mathbf r'_{p1})\phi_d(\mathbf r'_{p2})\chi^{(+)}_{tA}(\mathbf{r}'_{tA}).
\end{multline}
\end{subequations}
In these expressions, $\varphi^{A+1}_{l_f,j_f,m_f}(\mathbf r_{A1})$ are the wavefunctions describing the intermediate states of the nucleus $F(\equiv A+1)$, generated as solutions of a Woods--Saxon potential, and $\phi_d(\mathbf r_{p2})$ is the wavefunction of the deuteron bound state (see Fig. \ref{fig2}). Note that some or all of the $\varphi^{A+1}_{l_f,j_f,m_f}(\mathbf r_{A1})$ may be in the continuum for unbound or loosely bound $F$, and some discretization procedure is required in order to deal with these states. In this case, they are generated by embedding the Woods--Saxon potential in a spherical box of large enough radius. In actual calculations we got convergence with less than 20 continuum states in a 30 fm radius box. As for the wavefunction of the neutrons in the tritium, it is generated with the $p-n$ Tang--Herndon interaction
\begin{align}\label{eq8}
v(r)&=-v_0\exp\left(-k(r-r_c)\right) \quad r>r_c\\
v(r)&=\infty \quad r<r_c,
\end{align}
where $k=2.5$ fm$^{-1}$ and $r_c=0.45$ fm, and the depth $v_0$ is adjusted to reproduce the experimental separation energies.
The positive--energy wavefunctions  $\chi^{(+)}_{tA}(\mathbf{r}_{tA})$ and $\chi^{(-)}_{pB}(\mathbf{r}_{pB})$ are the ingoing distorted wave in the initial channel and the outgoing distorted wave in the final channel respectively. They are continuum solutions of the Schr\"{o}dinger equation associated with the corresponding optical potentials.


The transition potential responsible for the transfer of the pair is, in the \emph{post} representation,
\begin{equation}\label{eq1_43}
    V_\beta=v_{pB}-U_{\beta},
\end{equation}

where $v_{pB}$ is the interaction between the proton and nucleus $B$, and $U_{\beta}$ is the optical potential in the final channel. We make the assumption that $v_{pB}$ can be decomposed into a term containing the interaction between $A$ and $p$ and the potential describing the interaction between the proton and each of the transferred nucleons, namely
\begin{equation}\label{eq1_44}
    v_{pB}=v_{pA}+v_{p1}+v_{p2},
\end{equation}
where $v_{p1}$ and $v_{p2}$ is the hard--core potential (\ref{eq8}). The transition potential is
\begin{equation}\label{eq1_45}
    V_\beta=v_{pA}+v_{p1}+v_{p2}-U_{\beta}.
\end{equation}

Assuming that $\langle \beta |v_{pA}|\alpha \rangle \simeq \langle \beta |U_{\beta}|\alpha \rangle $ (i.e, assuming that the matrix element of the core--core interaction between the initial and final states is very similar to the matrix element of the real part of the optical potential), one obtains the final expression of the transfer potential in the \emph{post} representation,
\begin{equation}\label{eq1_45x}
    V_\beta\simeq v_{p1}+v_{p2}=v(\mathbf{r}_{p1})+v(\mathbf{r}_{p2}).
\end{equation}
We make the further approximation of using the same interaction potential in all (e.g. initial, intermediate and final) the channels.


The extension to a heavy--ion reaction $A+a(\equiv b+2) \longrightarrow B(\equiv A+2)+b$ imply no essential modifications in the formalism. The deuteron and triton states in (\ref{eq1_40},\ref{eq1_41},\ref{eq1_42}) must be substituted with the corresponding wavefunctions $\Psi_{b+2}(\xi_b,\mathbf r_{b1},\sigma_1,\mathbf r_{b2},\sigma_2)$, constructed in a similar way as in (\ref{eq1},\ref{eq2}). The interaction potential used in (\ref{eq1_40},\ref{eq1_41},\ref{eq1_42})  will now be the Woods--Saxon used to define the initial (final) state in the post (prior) representation, instead of the proton--neutron interaction (\ref{eq8}).

The Green function $G(\mathbf{r}_{dF},\mathbf{r}'_{dF})$ propagates the intermediate channel $d,F$, and can be expanded in partial waves
\begin{equation}\label{eq7}
G(\mathbf{r}_{dF},\mathbf{r}_{dF}')=i\sum_{l}\sqrt{2l+1}
\frac{f_{l}(k_{dF},r_<)P_{l}(k_{dF},r_>)}{k_{dF}r_{dF}r_{dF}'}
\left[  Y^{l} (\hat r_{dF}) Y^{l} (\hat r_{dF}')\right]_0^0.
\end{equation}
The $f_{l}(k_{dF},r)$ and $P_l(k_{dF},r)$ are the regular and the irregular solutions of a Schr\"{o}dinger equation with  a suitable optical potential and an energy equal to the kinetic energy in the intermediate state. In most cases of interest, the result is hardly altered if we use the same energy of the relative motion between nuclei for all the intermediate states. This representative energy is calculated when both intermediate nuclei are in their corresponding ground states. However, the validity of this approximation can break down in some particular cases. If, for example, some relevant intermediate state become off shell, its contribution is significantly quenched. An interesting situation can arise when this happens to all possible intermediate states, so they can only be virtually populated.  

\bibliographystyle{unsrt}
\bibliography{C:/Gregory/Broglia/notas_ricardo/nuclear_bib}


\end{document} 
