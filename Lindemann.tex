\documentclass[a4paper,11pt]{book}
\usepackage[sectionbib]{chapterbib}
%\usepackage{chapterbib}
\usepackage{dsfont}
\usepackage[title]{appendix}
\usepackage{slashbox}
\usepackage{enumerate}
%\documentclass[a4paper]{book}
% \linespread{2.}
%\numberwithin{section}
%\documentclass[12pt]{article}
%\documentclass[12pt]{cmmp}

%%\usepackage{psfig}
%\usepackage{harvard}
\usepackage{epsfig}
\usepackage{amsmath}   
\usepackage{amsfonts}
%\counterwithin{figure}{section}
\usepackage{amssymb}
\usepackage{bbold}
\usepackage{bbm}

%%\usepackage{graphicx}
%%
%%\usepackage{txfonts}
%%%\usepackage{mathrsfs}
%
%\usepackage{feynmf}     %<------------ Obbligatorio
\unitlength=1mm         %<------------ Obbligatorio
%
\newsavebox{\fmbox}
\newenvironment{fmpage}[1]
{\begin{lrbox}{\fmbox}\begin{minipage}{#1}}
{\end{minipage}\end{lrbox}\fbox{\usebox{\fmbox}}}
\newcommand{\braket}[1]{\langle {#1} \rangle }
\newcommand{\ket}[1]{|{#1} \rangle }
\newcommand{\bra}[1]{\langle {#1}|}
\newcommand\idop{\mathds 1}
\usepackage{dsfont}
\usepackage{latexsym}
\usepackage[varg]{txfonts}
\usepackage{mathrsfs}
\usepackage{upgreek}
\usepackage[round]{natbib}
%\usepackage [latin1]{inputenc}
\usepackage{verbatim}
\usepackage{array}
\usepackage{color}
%\pagestyle{plain}
\usepackage{graphicx}
\usepackage{caption}
\DeclareMathAlphabet{\mathpzc}{OT1}{pzc}{m}{it}
\title{\large{Nuclear Structure and Reactions}\\superfluidity in nuclei with Cooper pair transfer\\}
\author{G. Potel and R. A. Broglia}

\begin{document}
\section*{Lindemann criterion and connection with quantality parameter}\label{C2AppC}
The original Lindemann criterion (\cite{Lindemann:10}) compares the atomic fluctuation amplitude $\langle\Delta r^2\rangle^{1/2}$ with the lattice constant $a$ of a crystal. If this ratio, which is defined as the disorder parameter $\Delta L$, reaches a certain value, fluctuations cannot increase without damaging or destroying the crystal lattice. The results of experiments and simulations show that the critical value of $\Delta_L$ for simple solids is in the range of 0.10 to 0.15, relatively independent of the type of substance, the nature of the interaction potential, and the crystal structure (\cite{Bilgram:87,Lowen:94,Stillinger:95}). Applications of this criterion to an inhomogeneous finite system like a protein in its native state (aperiodic crystal, \cite{Schrodinger:44}) requires evaluation of the generalized Lindemann parameter (\cite{Stillinger:90})
\begin{align}
\Delta_L=\frac{\sqrt{\sum_i\langle \Delta r_i^2\rangle/N}}{a'},
\end{align}  
where $N$ is the number of atoms and $a'$ the most probable non--bonded near--neighbor distance, $\mathbf r_i$ is the position of atom $i$, $\Delta r_i^2=(\mathbf r_i-\langle \mathbf r_i^2\rangle)$, and $\langle\rangle$ denotes configurational averages at the conditions of measurement or simulations (e.g. biological, in which case $T\approx 310$ K, PH$\approx 7$, etc.\footnote{Fluctuations, classical (thermal) or quantal imply a probabilistic description. While one can only predict the odds for a given outcome of an experiment, probabilities themselves evolve in a deterministic fashion.}). The dynamics as a function of the distance from the geometric center of the protein is characterized by defining an interior ($int$) Lindemann parameter, 
\begin{align}
\Delta^{int}_L(r_{cut})=\frac{\sqrt{\sum_{i,r_i<r_{cut}}\langle \Delta r_i^2\rangle/N}}{a'},
\end{align}  
which is obtained by averaging over the atoms that are within a chosen cutoff distance, $r_{cut}$, from the center of mass of the protein.

Simulations and experimental data for a number of proteins, in particular Barnase, Myoglobin, Crambin and Ribonuclease A indicate 0.14 as the critical value distinguishing between solid--like and liquid--like behaviour, and $r_{cut}\approx 6$ \AA. As can be seen from Table \ref{tab2C1}, the interior of a protein, under physiological conditions, is solid--like  ($\Delta_L<0.14$), while its surface is liquid--like ($\Delta_L>0.14$). The beginning of thermal denaturation in the simulations appears to be related to the melting of its interior (i.e. $\Delta^{int}_L>0.14$), so that the entire protein becomes liquid--like. This is also the situation of the denatured state of a protein under physiological conditions (see e.g. \cite{Rosner:17}) 



\begin{table}[h]
 \begin{tabular}{|c|c|c|c|c|}
 \hline
 &\multicolumn{4}{|c|}{$\Delta_L(\Delta_L^{int}(6\;\text{\AA}))(300$ K)}\\
 \cline{2-5}
 &\multicolumn{3}{|c|}{MD simulations}&X--ray data\\
 \hline
 Proteins&Barnase&Myoglobin&Crambin&Ribonuclease A\\
 \hline
 all atoms&0.21(0.12)&0.16(0.11)&0.16(0.09)&0.16(0.12)\\
 backbone atoms only&0.16(0.10)&0.12(0.09)&0.12(0.08)&0.13(0.10)\\
 side--chain atoms only&0.25(0.14)&0.18(0.12)&0.19(0.10)&0.19(0.13)\\
 \hline
 \end{tabular}
 \caption{The heavy--atom $\Delta_L(\Delta_L^{int})$ value, for four proteins at 300 K. After \cite{Zhou:99}.}\label{tab2C1}
 \end{table}

\subsection*{Lindemann (``disorder'') parameter for a nucleus}
An estimate of  $\sqrt{\sum_i\langle \Delta r_i^2\rangle/A}$ in the case of nuclei considered as a sphere of nuclear matter of radius $R_0$, is provided by the ``spill out'' of nucleons due to quantal effects. That is\footnote{\cite{Bertsch:05}, see e.g. Ch. 5.} $\sqrt{\quad}\approx 0.69\times a_0$, where $a_0$ is of the order of the range of nuclear forces ($\approx 1$ fm).


The average internucleon distance can be determined from the relation (\cite{Brink:05}, App. C)
\begin{align}
a'=\left(\frac{V}{A}\right)^{1/3}=\left(\frac{\frac{4\pi}{3}R^3}{A}\right)^{1/3}=\left(\frac{4\pi}{3}\right)^{1/3}\times 1.2\; \text{fm}\approx 2\;\text{fm}
\end{align} 
Thus,
\begin{align}
\Delta_L=\frac{0.69 a_0}{2\;\text{fm}}\approx0.35.
\end{align} 
While it is difficult to compare among them crystals, aperiodic finite crystals and atomic nuclei, arguably, the above value indicates that a nucleus is liquid--like. More precisely, it is made out of a non--Newtonian fluid, which reacts elastically to sudden so\-li\-ci\-ta\-tions $(\lesssim 10^{-22}$ s),  and plastically to long lasting strain $(\gtrsim 10^{-21}$ s). In any case, one expects from $\Delta_L\approx 0.35$ that the nucleon mean free path is long, larger than nuclear dimensions.
\subsection*{Quantality parameter}
In quantum mechanics, the zero--point kinetic energy, $\sim\hbar^2/Ma_0^2$, associated with  the localization of a particle within a volume of radius $a_0$ implies that in the lowest energy state the particles is delocalized. This is because the potential energy gain of the single classical configuration of fixed particles which minimize the mean field (HF) solution, is overwhelmed by the quantal kinetic energy. Such delocalized quantal fluids provide the basis for discussing the state of electrons in atoms and in metals, of the he atoms in the ground state of He liquids (both fermionic $^3$He and bosonic $^4$He), and the state of nucleons in the ground state of atomic nuclei.

\begin{table}[h]
 \begin{tabular}{|c|c|c|c|c|c|c|}
 \hline  \multicolumn{2}{|c|}{constituents} & $M^{a)}$  & $a_0$(cm) &$v_0$(eV)  &q&phase($T=0$)    \\ 
 \hline  \multicolumn{2}{|c|}{ $^{3}$He}  &3& 2.9(-8)  &8.6(-4)  &0.19  &liquid    \\ 
 \hline  \multicolumn{2}{|c|}{$^{4}$He}  &4&  2.9(-8)&  8.6(-4)&  0.14& liquid   \\ 
 \hline  \multicolumn{2}{|c|}{  H$_2$}&2&  3.3(-8)&  32(-4)&  0.06&solid   \\ 
 \hline    \multicolumn{2}{|c|}{$^{20}$Ne}&20& 3.1(-8) &  31(-4)&  0.007&solid    \\ 
 \hline    nucleons&bare&1&  9(-14)& 100(+6)$^{b)}$ &  0.4&liquid  \\ 
 \cline{2-7} &ind.&1&60(-14)&0.5(
 +6)&2.0&liquid\\
 \hline
 \end{tabular}
 \caption{Quantality parameter. After \cite{Mottelson:02}.$^{a)}$ units of nucleon mass,  nuclear $^{b)}$ $^1S_0\, NN$--Argonne potential $v_{14}$.}\label{tab2C2}
 \end{table}
The relative magnitude of the quantal kinetic energy of localization compared with the potential energy can be qualitatively characterized by the quantality parameter (\cite{Mottelson:02})
\begin{align}
q=\frac{\hbar^2}{Ma_0^2}\frac{1}{|v_0|},
\end{align} 
where $M$ is the mass of the individual particles, while $v_0$ and $a_0$ measure  the strength of the attraction and the range corresponding to the minimum of the potential, respectively. When $q$ is small, quantal effects are small and the lowest state of the system is expected to have a crystalline structure, while for sufficiently large values of $q$, the system will remain a quantum fluid even in its ground state.


The values of the force parameters and the resulting quantality parameters for several condensed matter systems are collected in Table \ref{tab2C2}. For nuclei we have two sets. One associated with the bare $NN$--interaction ($^1S_0$ channel),
\begin{align}
a_0\approx 1\;\text{fm};\quad v_0=-100\;\text{MeV},
\end{align}  
and another with the induced pairing interaction
\begin{align}
a_0\approx R(=1.2 A^{1/3}\;\text{fm})\quad v_0=-0.5\;\text{MeV}.
\end{align}  
It is seen that the transition between quantum liquid and crystalline solid occurs at $q\approx0.1$ (between He and H$_2$). Thus nuclei are expected to be in a (non--Newtonian) quantum liquid phase.


In keeping with the fact that $q$ is of the order of unity 
 in the nuclear case, it is likely that mean field theory is applicable to the description of the nucleons in the ground state of the system. The marked variation of the binding energy per particle as a function of mass number $A=N+Z$ for specific values of $N$ and $Z$ (magic numbers), testifies to the fact that nucleons display, in states lying close to the Fermi energy, a long mean free path as compared with nuclear dimensions ($R\approx1.2 A^{1/3}$ fm $\approx 6-7$ fm).
 
 
 The results discussed above, namely that $q\ll1$ implies localization, that is fixed relations between the constituents, and thus spontaneous symmetry breaking, while $q\gtrsim0.14$ implies delocalization and thus homogeneity, is an example of the fact that while potential energy always prefer spatial arrangements, fluctuations, classical or quantal, favour symmetry (\cite{Anderson:84}). 
 
\bibliographystyle{abbrvnat}
\bibliography{./nuclear_bib}
\end{document} 